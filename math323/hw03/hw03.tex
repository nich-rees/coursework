\documentclass{article}
\usepackage{amsmath, amsfonts, amsthm, amssymb}
\usepackage{geometry}
\geometry{letterpaper, margin=2.0cm, includefoot, footskip=30pt}

\usepackage{fancyhdr}
\pagestyle{fancy}

\lhead{Math 323}
\chead{Homework 3}
\rhead{Nicholas Rees, 11848363}
\cfoot{Page \thepage}

\newcommand{\N}{{\mathbb N}}
\newcommand{\Z}{{\mathbb Z}}
\newcommand{\Q}{{\mathbb Q}}
\newcommand{\R}{{\mathbb R}}
\newcommand{\C}{{\mathbb C}}
\newcommand{\ep}{{\varepsilon}}

\newcommand{\problem}[1]{
	\begin{center}\fbox{
		\begin{minipage}{17.0 cm}
			\setlength{\parindent}{1.5em}
			{\it \noindent#1}
		\end{minipage}}
	\end{center}}

\newtheorem{lemma}{Lemma}

\renewcommand{\theenumi}{(\alph{enumi})}

\begin{document}
\begin{center}
	{\bf Math 323 Homework 3}
\end{center}

\subsection*{Problem 2 (Chapter 2.9)}
\problem{
	Show that if $D$ is a domain and $F_1$ and $F_2$ are fields such that
	$D$ is a subring of each and each is generated by $D$,
	then there is a unique isomorphism of $F_1$ onto
	$F_2$ that is the identity map on $D$.
}
\begin{proof}[Solution]\let\qed\relax
	ff
	Use the universal property.

	Theorem 2.9: Let $D$ be a commutative domain,
	$F$ its field of fractions.
	Then any monomorphism $\eta_D$ of $D$ into a field $F'$
	has a unique extension to a monomorphism of $\eta_F$ of $F$ into $F'$.

	Since $D$ is a subring of $F_1$ and $F_2$, we have
	the natural embedding of $D$ in $F_1$, $\eta_1 \colon D \hookrightarrow F_1$,
	and the natural embedding of $D$ in $F_2$, $\eta_2 \colon D \hookrightarrow F_2$.
	Since $\eta_1,\eta_2$ are injective, their inverses are

	Some Oakley wisdom:
	when we say the fields such that $\{\text{Fields }F \supset D\}$,
	we can't say a smallest field because there is no order on all fields.
	But if we consider subfields of $F_1$, $S = \{K \colon F_1 \supset K \supset D\}$,
	then saying $F_1$ is the smallest is like saying $S = \{F_1\}$.

	So we have the commutative diagram,
	$D \to F, D \to K(D), K(D) \to F$ (last arrow unique).
	But we must have that $K(D) = F_1$, so $F_1$ is isomorphic
	to the field of fractions.
	Both of them are, so we are done.

	Also to use theorem, we need $D$ commutative.
	But since $D$ is in a field,
	we have $a,b \in D$, and so in $F$, $ab = ba$, but in $D$.
\end{proof}


\subsection*{Problem 5 (Chapter 2.9)}
\problem{
	Let $R$ be a commutative ring, and $S$ a submonoid of the multiplicative monoid of $R$.
	In $R \times S$ define $(a,s) \sim (b,t)$ if there exists a
	$u \in S$ such that $u(at - bs) = 0$.
	Show that this is an equivalence relation in $R \times S$.
	Denote the equivalence class of $(a,s)$ as $a/s$ and the quotient set
	consisting of these classes as $RS^{-1}$.
	Show that $RS^{-1}$ becomes a ring relative to
	\begin{align*}
		a/s + b/t &= (at+bs)/st\\
		(a/s)(b/t) &= ab/st\\
		0 &= 0/1\\
		1 &= 1/1
	\end{align*}
	Show that $a \to a/1$ is a homomorphism of $R$ into $RS^{-1}$
	and that this is a monomorphism if and only if no element of $S$
	is a zero divisor in $R$.
	Show that the elements $s/1$, $s \in S$, are units in $RS^{-1}$.
}
\begin{proof}[Solution]\let\qed\relax
	ff
\end{proof}


\subsection*{Problem 2 (Chapter 2.10)}
\problem{
	Show that $\sqrt{3}\not\in \Q[\sqrt{2}]$ and that the real numbers
	$1,\sqrt{2},\sqrt{3},\sqrt{6}$ are linearly independent over $\Q$.
	Show that $u = \sqrt{2} + \sqrt{3}$ is algebraic and determine an ideal $I$
	such that $\Q[x]/I \cong \Q[u]$.
}

\begin{proof}[Solution]\let\qed\relax
	Assume there exists $a_0,a_1,\dots \in \Q$ such that
	$\sqrt{3} = a_0 + a_1\sqrt{2} + a_2(\sqrt{2})^2 + a_3(\sqrt{2})^3 + \cdots$,
	which clearly is equivalent to there being $a,b \in \Q$ such that
	$\sqrt{3} = a + b\sqrt{2}$
	(since each term is either an element in $\Q$,
	or an element in $\Q$ times $\sqrt{2}$).
	If we square both sides, we get $3 = a^2 + 2b^2 + 2ab\sqrt{2}
	\implies \sqrt{2} = \frac{3 - a^2 - 2b^2}{2ab} \in \Q$
	(since $\Q$ is a field),
	but by any standard proof, $\sqrt{2} \not\in \Q$, a contradiction.
	Hence, there do not exist $a,b \in \Q$,
	and so $\sqrt{3} \not\in \Q[\sqrt{2}]$.

	Recall from linear algebra it is sufficient to show,
	when $a_0,a_1,a_2,a_3 \in \Q$, that
	$a_0(1) + a_1\sqrt{2} + a_2\sqrt{3} + a_3\sqrt{6} = 0$
	only when $a_0 = a_1 = a_2 = a_3 = 0$.
	ff

	We want to show that there exists an $n \in \N^0$ and nonzero $a_i \in \Q$ such that
	$a_0 + a_1u + a_2u^2 + \cdots + a_nu^n = 0$.
	We have
	\[
		1 - 10u^2 + u^4 = 1 - 50 - 20\sqrt{6} + 49 + 20\sqrt{6} = 0
	\]
	Hence, $u = \sqrt{2} + \sqrt{3}$ is algebraic.

	We want to find $I$ such that $u = x + I$.
	I'm pretty sure this is $1 - 10x^2 + x^4$.
	ff
\end{proof}


\subsection*{Problem 4 (Chapter 2.10)}
\problem{
	Let $\Delta = \prod_{i > j}(x_i - x_j)$ in $\Z[x_1,\dots,x_r]$
	and let $\zeta(\pi)$ be the automorphism of $\Z[x_1,\dots,x_r]$
	which maps $x_i \to x_{\pi(i)}, 1 \leq i \leq r$.
	(Every automorphism of the ring $\Z[x_1,\dots,x_r]$ is the identity on $\Z$. Why?)
	Verify that if $\tau$ is a transposition then
	$\Delta \to -\Delta$ under $\zeta(\tau)$.
	Use this to prove the result given in section 1.6 that if
	$\pi$ is a product of an even number of transpositions,
	then every factorization of $\pi$ as a product of transpositions contains
	an even number of transposititions.
	Show that $\Delta^2 \to \Delta^2$ under every $\zeta(\pi)$.
}

\begin{proof}[Solution]\let\qed\relax
	First, note that every automorphism of $\Z[x_1,\dots,x_r]$ is the identity on $\Z$,
	since ff

	Let $\tau = (mn)$ be a transposition, where $1 \leq m,n \leq r$, $m\neq n$.
	Without loss of generality, let $n > m$.
	Since $\zeta$ is an automorphism, we have
	$\zeta(\tau)(\Delta) = \prod_{i>j} \zeta(\tau)(x_i - x_j)
	= \prod_{i>j} (x_{\tau(i)} - x_{\tau(j)})$.
	Consider each factor, $(x_{\tau(i)} - x_{\tau(j)})$.
	If $\tau(i) = i$ and $\tau(j) = j$,
	then $(x_{\tau(i)} - x_{\tau(j)}) = (x_i - x_j)$.
	If $\tau(i) = k$ and $\tau(j) = j$,
	we have that $(x_{\tau(i)} - x_{\tau(j)}) = (x_j - x_k)$,
	but we must have then $\tau(k) = i$ and there is some other factor
	$(x_{\tau(j)} - x_{\tau(k)}) = (x_j - x_i)$,
	and multiplication of polynomials is commutative,
	so we can swap these two terms and nothing changes.
	If $\tau(i) = j$ so $\tau(j) = i$, we have $(x_{\tau(j)} - x_{\tau(j)})
	= (x_i - x_j) = -(x_j - x_i)$.
	This covers all the possible case in the product of $\Delta$,
	and so, since there is only one factor that changes,
	specifically by picking up a negative, we have
	$\zeta(\tau)(\Delta) = -\Delta$.

	The result from 1.6 then follows,
	since ff (negative sign)

	Recall that every $\pi$ can be decomposed as a product of transpositions.
	Hence, if $\pi = \tau_n \circ \tau_{n-1} \circ \cdots \circ \tau_1$,
	we have $\zeta(\pi) = \zeta(\tau_n) \circ \zeta(\tau_{n-1}) \circ \cdots \circ \zeta(\tau_1)$.
	Since $\zeta(\pi)$ is an automorphism, we have
	$\zeta(\pi)(\Delta^2) = (\zeta(\pi)(\Delta))^2$. So
	\[
		\zeta(\pi)(\Delta^2) =
		((\zeta(\tau_n) \circ \zeta(\tau_{n-1}) \circ \cdots \zeta(\tau_1))(\Delta))^2
		= (\pm \Delta)^2 = \Delta
	\]
	where the second to last equality is from the fact $\zeta(\tau)(\Delta) = -\Delta$.
\end{proof}


\subsection*{Problem 7 (Chapter 2.10)}
\problem{
	Let $R[[x]]$ denote the set of unrestricted sequences
	$(a_0,a_1,\dots)$, $a_i \in R$.
	Show that one gets a ring from $R[[x]$ if one defines
	$+,\cdot,0,1$ as in the polynomial ring.
	This is called the ring of \emph{formal power series in one indeterminate}.
}

\begin{proof}[Solution]\let\qed\relax
	ff just copy book
\end{proof}


\subsection*{Problem 1 (Chapter 2.11)}
\problem{
	Let $f(x) = x^n + a_1 x^{n-1} + \cdots + a_n$, $a_i \in F$, a field,
	$n > 0$ and let $u = x + (f(x))$ in $F[x]/(f(x))$.
	Show that every element of $F[u]$ can be written in one and only one
	way in the form $b_0 + b_1u + \cdots + b_{n-1}u^{n-1}$, $b_j \in F$.
}

\begin{proof}[Solution]\let\qed\relax
	ff
\end{proof}


\subsection*{Problem 2 (Chapter 2.11)}
\problem{
	Take $F = \Q$, $f(x) = x^3 + 3x - 2$ in exercise $1$.
	Show that $F[u]$ is a field and express the elements
	\[
		(2u^2 + u - 3)(3u^2 - 4u + 1), \qquad (u^2 - u + 4)^{-1}
	\]
	as polynomials of degree $\leq 2$ in $u$.
}

\begin{proof}[Solution]\let\qed\relax
	ff
\end{proof}


\subsection*{Problem 3 (Chapter 2.11)}
\problem{
	\begin{enumerate}
		\item Show that $\Q[\sqrt{2}]$ and $\Q[\sqrt{3}]$ are not isomorhpic.
		\item Let $\mathbb{F}_p = \Z/(p)$, $p$ a prime,
			and let $R_1 = \mathbb{F}_p[x]/(x^2-2)$,
			$R_2 = \mathbb{F}_p[x]/(x^2-3)$.
			Determine whether $R_1 \cong R_2$ in each of the cases
			in which $p = 2$, $5$, or $11$.
	\end{enumerate}
}

\begin{enumerate}
	\item \begin{proof}[Solution]\let\qed\relax
		ff
	\end{proof}
	\item \begin{proof}[Solution]\let\qed\relax
		ff
	\end{proof}
\end{enumerate}


\subsection*{Problem 4 (Chapter 2.11)}
\problem{
	Show that $x^3 + x^2 + 1$ is irreducible in $(\Z/(2))[x]$
	and that $(\Z/(2))[x]/(x^3+x^2+1)$ is a field with eight elements.
}

\begin{proof}[Solution]\let\qed\relax
	Recall $I_1 + I_2 := (I_1 \cup I_2)$.
	If $a_1 \in I_1$ and $a_2 \in I_2$, then $a = 0$ works.
	If $a_1 \in I_1$ but $a_2 \not\in I_2$ (and so $a_2 \in I_1$
	since $a_2 = i_1 + i_2$ and ff hmm now this seems a lot more trivial,
	we have $a = a_2$ works, since $a_1 - a_2 \in I_1$ since $I$ is a subgroup
	with respect to addition,
	and $a_2 - a_2 = 0 \in I_2$ since $I_2$ must contain the zero
	since it is a group with respect to addition.
	The same works when $a_1 \not\in I_1$ and $a_2 \in I_2$, i.e. $a = a_1$.
	ff
\end{proof}
\end{document}
