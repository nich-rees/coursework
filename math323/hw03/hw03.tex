\documentclass{article}
\usepackage{amsmath, amsfonts, amsthm, amssymb}
\usepackage{geometry}
\geometry{letterpaper, margin=2.0cm, includefoot, footskip=30pt}

\usepackage{fancyhdr}
\pagestyle{fancy}

\lhead{Math 323}
\chead{Homework 3}
\rhead{Nicholas Rees, 11848363}
\cfoot{Page \thepage}

\newcommand{\N}{{\mathbb N}}
\newcommand{\Z}{{\mathbb Z}}
\newcommand{\Q}{{\mathbb Q}}
\newcommand{\R}{{\mathbb R}}
\newcommand{\C}{{\mathbb C}}
\newcommand{\ep}{{\varepsilon}}

\newcommand{\problem}[1]{
	\begin{center}\fbox{
		\begin{minipage}{17.0 cm}
			\setlength{\parindent}{1.5em}
			{\it \noindent#1}
		\end{minipage}}
	\end{center}}

\newtheorem{lemma}{Lemma}

\renewcommand{\theenumi}{(\alph{enumi})}

\begin{document}
\begin{center}
	{\bf Math 323 Homework 3}
\end{center}

\subsection*{Problem 2 (Chapter 2.9)}
\problem{
	Show that if $D$ is a domain and $F_1$ and $F_2$ are fields such that
	$D$ is a subring of each and each is generated by $D$,
	then there is a unique isomorphism of $F_1$ onto
	$F_2$ that is the identity map on $D$.
}
\begin{proof}[Solution]\let\qed\relax
	Note that $D$ commutative, since $D$ is in a field:
	since $a,b \in D$, and so $a,b \in F$, and since $F$ is a field, $ab = ba$.

	Let $\eta_1\colon D \to F_1$ be the identity homomorphism into $F_1$,
	which is a monomorphism since $D$ is embedded in $F_1$ by definition
	of $D$ generating $F_1$.
	Hence, if $K$ is the field of fractions of $D$,
	we have that there is a unique monomorphism $\eta_1'$ from $K$ into $F_1$.
	The image of $K$ in $F_1$ is a subring of $F_1$ that contains $D$,
	but by definition of $D$ generating $F_1$,
	$F_1$ contains no proper subring that contains $D$,
	hence since $\eta_1'$ must be surjective, so $\eta_1'$ is a bijection.
	So there is a unique isomorphism of $F_1$ onto $K$.
	An identical argument also shows that there is a unique isomorphism
	of $K$ onto $F_2$.
	Composing the two, we get that there is a unique isomorphism
	from $F_1$ onto $F_2$.

	ff show that it is identity on $D$.
\end{proof}


\subsection*{Problem 5 (Chapter 2.9)}
\problem{
	Let $R$ be a commutative ring, and $S$ a submonoid of the multiplicative monoid of $R$.
	In $R \times S$ define $(a,s) \sim (b,t)$ if there exists a
	$u \in S$ such that $u(at - bs) = 0$.
	Show that this is an equivalence relation in $R \times S$.
	Denote the equivalence class of $(a,s)$ as $a/s$ and the quotient set
	consisting of these classes as $RS^{-1}$.
	Show that $RS^{-1}$ becomes a ring relative to
	\begin{align*}
		a/s + b/t &= (at+bs)/st\\
		(a/s)(b/t) &= ab/st\\
		0 &= 0/1\\
		1 &= 1/1
	\end{align*}
	Show that $a \to a/1$ is a homomorphism of $R$ into $RS^{-1}$
	and that this is a monomorphism if and only if no element of $S$
	is a zero divisor in $R$.
	Show that the elements $s/1$, $s \in S$, are units in $RS^{-1}$.
}
\begin{proof}[Solution]\let\qed\relax
	ff
\end{proof}


\subsection*{Problem 2 (Chapter 2.10)}
\problem{
	Show that $\sqrt{3}\not\in \Q[\sqrt{2}]$ and that the real numbers
	$1,\sqrt{2},\sqrt{3},\sqrt{6}$ are linearly independent over $\Q$.
	Show that $u = \sqrt{2} + \sqrt{3}$ is algebraic and determine an ideal $I$
	such that $\Q[x]/I \cong \Q[u]$.
}

\begin{proof}[Solution]\let\qed\relax
	Assume there exists $a_0,a_1,\dots \in \Q$ such that
	$\sqrt{3} = a_0 + a_1\sqrt{2} + a_2(\sqrt{2})^2 + a_3(\sqrt{2})^3 + \cdots$,
	which clearly is equivalent to there being $a,b \in \Q$ such that
	$\sqrt{3} = a + b\sqrt{2}$
	(since each term is either an element in $\Q$,
	or an element in $\Q$ times $\sqrt{2}$).
	If we square both sides, we get $3 = a^2 + 2b^2 + 2ab\sqrt{2}
	\implies \sqrt{2} = \frac{3 - a^2 - 2b^2}{2ab} \in \Q$
	(since $\Q$ is a field),
	but by any standard proof, $\sqrt{2} \not\in \Q$, a contradiction.
	Hence, there do not exist $a,b \in \Q$,
	and so $\sqrt{3} \not\in \Q[\sqrt{2}]$.

	Recall from linear algebra it is sufficient to show,
	when $a_0,a_1,a_2,a_3 \in \Q$, that
	$a_0(1) + a_1\sqrt{2} + a_2\sqrt{3} + a_3\sqrt{6} = 0$
	only when $a_0 = a_1 = a_2 = a_3 = 0$.
	ff

	We want to show that there exists an $n \in \N^0$ and nonzero $a_i \in \Q$ such that
	$a_0 + a_1u + a_2u^2 + \cdots + a_nu^n = 0$.
	We have
	\[
		1 - 10u^2 + u^4 = 1 - 50 - 20\sqrt{6} + 49 + 20\sqrt{6} = 0
	\]
	Hence, $u = \sqrt{2} + \sqrt{3}$ is algebraic.

	We want to find $I$ such that $u = x + I$.
	I'm pretty sure this is $1 - 10x^2 + x^4$.
	ff
\end{proof}


\subsection*{Problem 4 (Chapter 2.10)}
\problem{
	Let $\Delta = \prod_{i > j}(x_i - x_j)$ in $\Z[x_1,\dots,x_r]$
	and let $\zeta(\pi)$ be the automorphism of $\Z[x_1,\dots,x_r]$
	which maps $x_i \to x_{\pi(i)}, 1 \leq i \leq r$.
	(Every automorphism of the ring $\Z[x_1,\dots,x_r]$ is the identity on $\Z$. Why?)
	Verify that if $\tau$ is a transposition then
	$\Delta \to -\Delta$ under $\zeta(\tau)$.
	Use this to prove the result given in section 1.6 that if
	$\pi$ is a product of an even number of transpositions,
	then every factorization of $\pi$ as a product of transpositions contains
	an even number of transposititions.
	Show that $\Delta^2 \to \Delta^2$ under every $\zeta(\pi)$.
}

\begin{proof}[Solution]\let\qed\relax
	First, note that every automorphism of $\Z[x_1,\dots,x_r]$ is the identity on $\Z$,
	since ff

	Let $\tau = (mn)$ be a transposition, where $1 \leq m,n \leq r$, $m\neq n$.
	Without loss of generality, let $n > m$.
	Since $\zeta$ is an automorphism, we have
	$\zeta(\tau)(\Delta) = \prod_{i>j} \zeta(\tau)(x_i - x_j)
	= \prod_{i>j} (x_{\tau(i)} - x_{\tau(j)})$.
	Consider each factor, $(x_{\tau(i)} - x_{\tau(j)})$.
	If $\tau(i) = i$ and $\tau(j) = j$,
	then $(x_{\tau(i)} - x_{\tau(j)}) = (x_i - x_j)$.
	If $\tau(i) = k$ and $\tau(j) = j$,
	we have that $(x_{\tau(i)} - x_{\tau(j)}) = (x_j - x_k)$,
	but we must have then $\tau(k) = i$ and there is some other factor
	$(x_{\tau(j)} - x_{\tau(k)}) = (x_j - x_i)$,
	and multiplication of polynomials is commutative,
	so we can swap these two terms and nothing changes.
	If $\tau(i) = j$ so $\tau(j) = i$, we have $(x_{\tau(j)} - x_{\tau(j)})
	= (x_i - x_j) = -(x_j - x_i)$.
	This covers all the possible case in the product of $\Delta$,
	and so, since there is only one factor that changes,
	specifically by picking up a negative, we have
	$\zeta(\tau)(\Delta) = -\Delta$.

	The result from 1.6 then follows,
	since ff (negative sign)

	Recall that every $\pi$ can be decomposed as a product of transpositions.
	Hence, if $\pi = \tau_n\tau_{n-1} \cdots\tau_1$,
	we have $\zeta(\pi) = \zeta(\tau_n) \circ \zeta(\tau_{n-1}) \circ \cdots \circ \zeta(\tau_1)$.
	Since $\zeta(\pi)$ is an automorphism, we have
	$\zeta(\pi)(\Delta^2) = (\zeta(\pi)(\Delta))^2$. So
	\[
		\zeta(\pi)(\Delta^2) =
		((\zeta(\tau_n) \circ \zeta(\tau_{n-1}) \circ \cdots \zeta(\tau_1))(\Delta))^2
		= (\pm \Delta)^2 = \Delta
	\]
	where the second to last equality is from the fact $\zeta(\tau)(\Delta) = -\Delta$.
\end{proof}


\subsection*{Problem 7 (Chapter 2.10)}
\problem{
	Let $R[[x]]$ denote the set of unrestricted sequences
	$(a_0,a_1,\dots)$, $a_i \in R$.
	Show that one gets a ring from $R[[x]$ if one defines
	$+,\cdot,0,1$ as in the polynomial ring.
	This is called the ring of \emph{formal power series in one indeterminate}.
}

\begin{proof}[Solution]\let\qed\relax
	ff just copy book
\end{proof}


\subsection*{Problem 1 (Chapter 2.11)}
\problem{
	Let $f(x) = x^n + a_1 x^{n-1} + \cdots + a_n$, $a_i \in F$, a field,
	$n > 0$ and let $u = x + (f(x))$ in $F[x]/(f(x))$.
	Show that every element of $F[u]$ can be written in one and only one
	way in the form $b_0 + b_1u + \cdots + b_{n-1}u^{n-1}$, $b_j \in F$.
}

\begin{proof}[Solution]\let\qed\relax
	We define the homomorphism $\eta$

	Consider an element $c_0 + c_1x + \cdots + c_mx^m \in F[x]$.
	Every element in $F[u]$ is of the form
	$b_0 + b_1u + b_2u^2 + \cdots + b_mu^m$, where $b_j \in F$.
	Plugging in $u = x + (f(x))$, we get
	$b_0 + b_1x + b_2x^2 + 2b_2x(f(x)) + b_2(f(x))^2 + \cdots + b_mx^m + \cdots
	= b_0 + b_1x + b_2x^2 + \cdots + b_mx^m + (f(x))$.
	ff

	Let $c_0 + c_1u + \cdots + c_mu^m$ be an arbitrary element of $F[u]$.
	Since $F[u] \cong F[x]/(f(x))$,
	we have the isomorphism $\eta \colon F[u] \to F[x]/(f(x))$,
	where $c_0 + c_1u + \cdots + c_m u^m \mapsto
	c_0 + c_1x + \cdots c_mx^m + (f(x))$.
	By the division algorithm, there exists some $q(x),r(x) \in F[x]$,
	both unique since $F$ is a field, such that
	\[
		c_0 + c_1x + \cdots c_mx^m = q(x)f(x) + r(x)
	\]
	where $\deg(r) < f(x)$.
	Since $f$ has degree $n$, we can write $r(x) = b_0 + b_1x + \cdots + b_{n-1}u^{n-1}$,
	and so, since $q(x)f(x) \in (f(x))$,
	\[
		c_0 + c_1x + \cdots c_mx^m + (f(x)) = b_0 + b_1x + \cdots + b_{n-1}u^{n-1} + (f(x))
	\]
	Putting this back through $\eta^{-1}$ (which we have because it is bijective)
	to $F[u]$, we get
	\[
		\eta(c_0 + c_1x + \cdots c_mx^m) = \eta(b_0 + b_1x + \cdots + b_{n-1}u^{n-1})
	\]
	and since $\eta$ is injective, we have that
	\[
		c_0 + c_1x + \cdots c_mx^m = b_0 + b_1x + \cdots + b_{n-1}u^{n-1}
	\]
	and this is unique by the uniqueness of our remainder.
\end{proof}


\subsection*{Problem 2 (Chapter 2.11)}
\problem{
	Take $F = \Q$, $f(x) = x^3 + 3x - 2$ in exercise $1$.
	Show that $F[u]$ is a field and express the elements
	\[
		(2u^2 + u - 3)(3u^2 - 4u + 1), \qquad (u^2 - u + 4)^{-1}
	\]
	as polynomials of degree $\leq 2$ in $u$.
}

\begin{proof}[Solution]\let\qed\relax
	To show that $F[u]$ is a field, by Theorem 2.16,
	it is sufficient to show that $f(x)$ is irreducible.
	For the sake of contradiction, assume that $f(x)$ is reducible.
	Then there exists $g(x),k(x) \in F[x]$ where $\deg(g),\deg(k)>0$, such that $f(x) = g(x)k(x)$.
	Since $\deg(gk) = \deg(g) + \deg(k)$, we must have, assuming $\deg(g) \geq \deg(k)$,
	that $\deg(g) = 2$ and $\deg(k) = 1$.
	So $k(x)$ is of the form $k(x) = a_0 + a_1x$ where $a_0,a_1 \in \Q$.
	Hence, we have $f(-a_0/a_1) = (a_0 + a_1(-a_0/a_1))g(-a_0/a_1) = 0$.
	So $f(x)$ has a root at some rational.
	We'll let $\frac{p}{q} = -\frac{a_0}{a_1}$ where $p \in \Z$, $q \in \N$,
	and $\gcd(p,q) = 1$.
	So we have $0 = f(\frac{p}{q}) = p^3/q^3 + 3p/q - 2
	= \frac{p^3 + 3pq^2 - 2q^3}{q^3}
	\implies p^3 + 3pq^2 - 2q^3 = 0$.
	Now consider this modulo $q$, then we have that
	$p^3 \equiv 0 \, (\mathrm{mod}\, q)$,
	so $p$ is a zero divisor in $\Z/(q)$.
	But from Theorem 2.4, since $p$ and $q$ are coprime,
	$p$ is a unit in $\Z/(q)$.
	But an element cannot be both a unit and zero divisor in a ring,
	else $pp^2 = 0 \implies p^{-1}p^{-1}pp^2 = 0 \implies p = 0$,
	and $0$ is not a unit, hence a contradiction.
	Thus, there is no linear polynomial factor of $g(x)$ in $\Q[x]$,
	and so $g(x)$ is irreducible.
	Thus $F[u]$ is a field.

	Now, for $(2u^2 + u - 3)(3u^2 - 4u + 1)$,
	we can compute
	\[
		(2u^2 + u - 3)(3u^2 - 4u + 1)
		= 6u^4 - 5u^3 - 11u^2 + 13u - 3
	\]
	We can map this through the isomorphism $\eta$ to $F[x]/I$ to get
	$6x^4 - 5x^3 - 11x^2 + 13x - 3 + I$.
	We can divide out by $f(x)$ to get an element in the same equivalence class:
	\[
		6x^4 - 5x^3 - 11x^2 + 13x - 3 + I
		= (6x - 5)(x^3 + 3x -2) + (- 18x^2 + 27x - 10) + I
		= - 18x^2 + 27x - 10 + I
	\]
	Hence, putting this back through $\eta^{-1}$ (which we have because it is bijective)
	to $F[u]$ we have
	\[
		\eta((2u^2 + u - 3)(3u^2 - 4u + 1))
		= 6x^4 - 5x^3 - 11x^2 + 13x - 3 + I = - 18x^2 + 27x - 10 + I
		= \eta(-18u^2 + 27u - 10)
	\]
	And since $\eta$ is injective, we have 
	\[
		(2u^2 + u - 3)(3u^2 - 4u + 1) = -18u^2 + 27u - 10
	\]

	For the second, we are looking for $g(u) = a_0 + a_1u + a_2u^2\in F[u]$
	such that $g(u)(u^2 - u + 4)$
	(we need not check the other side because it is a field, and so commutative).
	We can compute
	\begin{align*}
		g(u)(u^2 - u + 4) &= a_0u^2 - a_0u + 4a_0
		+ a_1u^3 - a_1u^2 + 4a_1u + a_2u^4 - a_2u^3 + 4a_2u^2\\
		&= a_2u^4 + (a_1 - a_2)u^3 + (a_0 - a_1 + 4a_2)u^2 + (4a_1 - a_0)u + 4a_0
	\end{align*}
	We can map this through the isomorphism $\eta$ to $F[x]/I$ to get
	$a_2x^4 + (a_1 - a_2)x^3 + (a_0 - a_1 + 4a_2)x^2 + (4a_1 - a_0)x + 4a_0 + I$.
	We can divide out by $f(x)$ to get an element in the same equivalence class:
	\begin{align*}
		&\big(a_2x^4 + (a_1 - a_2)x^3 + (a_0 - a_1 + 4a_2)x^2 + (4a_1 - a_0)x + 4a_0\big)
		- (a_2x + (a_1 - a_2))(x^3 + 3x - 2)\\
		= &(a_0 - a_1 + 4a_2)x^2 + (4a_1 - a_0)x + 4a_0
		- 3a_2x^2 - 3(a_1 - a_2)x + 2a_2x + 2(a_1 - a_2)\\
		= &(a_0 - a_1 + a_2)x^2 + (-a_0 + a_1 + 3a_2)x + (4a_0 + 2a_1 - 2a_2)
	\end{align*}
	Hence, $a_2x^4 + (a_1 - a_2)x^3 + (a_0 - a_1 + 4a_2)x^2 + (4a_1 - a_0)x + 4a_0 + I
	= (a_0 - a_1 + a_2)x^2 + (-a_0 + a_1 + 3a_2)x + (4a_0 + 2a_1 - 2a_2) + I$.
	If we want this to equal to $1 + I$, so we solve the system
	\[
		\begin{cases}
			0 = a_0 - a_1 + a_2\\
			0 = -a_0 + a_1 + 3a_2\\
			1 = 4a_0 + 2a_1 - 2a_2
		\end{cases}
	\]
	One can solve this system to find $a_0 = a_1 = \frac{1}{6}, a_2 = 0$.
	Given these assignments, then
	\[
		a_2x^4 + (a_1 - a_2)x^3 + (a_0 - a_1 + 4a_2)x^2 + (4a_1 - a_0)x + 4a_0 + I
		= 1 + I
	\]
	Hence, if $g(u) = \frac{1}{6} + \frac{1}{6}u$,
	then $\eta(g(u)(u^2-u+4)) = 1 + I = \eta(1_{F[u]})$.
	Since $\eta$ is a isomorphism, using injectivity, we get
	$g(u)(u^2 - u + 4) = 1_{F[u]}$, and so
	\[
		(u^2 - u + 4)^{-1} = \frac{1}{6} + \frac{1}{6}u
	\]
\end{proof}


\subsection*{Problem 3 (Chapter 2.11)}
\problem{
	\begin{enumerate}
		\item Show that $\Q[\sqrt{2}]$ and $\Q[\sqrt{3}]$ are not isomorhpic.
		\item Let $\mathbb{F}_p = \Z/(p)$, $p$ a prime,
			and let $R_1 = \mathbb{F}_p[x]/(x^2-2)$,
			$R_2 = \mathbb{F}_p[x]/(x^2-3)$.
			Determine whether $R_1 \cong R_2$ in each of the cases
			in which $p = 2$, $5$, or $11$.
	\end{enumerate}
}

\begin{enumerate}
	\item \begin{proof}[Solution]\let\qed\relax
		For the sake of contradiction, assume there exists some isomorphism
		$\phi \colon \Q[\sqrt{2}] \to \Q[\sqrt{3}]$.
		Note that any homomorphism between two fields that contain $\Q$
		must be the identity map on itself:
		$\phi(1) = 1$ and so $\phi(1+1) = \phi(1) + \phi(1) = 2$.
		A simple induction would give us then $\phi(n) = n$ for all $n \in \N$.
		Then $0 = \phi(0) = \phi(n-n) = \phi(n) + \phi(-n) = n + \phi(-n)
		\implies \phi(-n) = -n$.
		So $\phi(n) = n$ for all $n \in \Z$.
		We also have $\phi(\frac{1}{n}) = \phi(n)^{-1} = n^{-1} = \frac{1}{n}$ for all $n \in \Z^*$,
		Hence, since each element in $\Q$ can be written as $\frac{m}{n}$
		where $m \in \Z$ and $n \in \N$, we have
		$\phi(\frac{m}{n}) = \phi(m)\phi(\frac{1}{n}) = \frac{m}{n}$.

		Since $\phi$ must be surjective,
		there exists some $q = a + b\sqrt{2} \in \Q[\sqrt{2}]$
		such that $\phi(q) = \sqrt{3} \in \Q[\sqrt{3}]$.
		Thus,
		\[
			3 = \phi(q)\phi(q) = \phi(q^2)
			= \phi(a^2 + 2ab\sqrt{2} + b^2)
			= a^2 + 2ab\phi(\sqrt{2}) + b^2
		\]
		But then we have $\phi(\sqrt{2}) = \frac{3-a^2-b^2}{2ab} \in \Q$.
		Let this rational value be $q \in \Q$.
		So $\phi(\sqrt{2}) = q$.
		But we also have $q \in \Q[\sqrt{2}]$ and $\phi(q) = q$.
		And so $\phi(\sqrt{2}) = \phi(q)$.
		But since $\phi$ is an isomorphism, and so is injective,
		we have that $\sqrt{2} = q \implies \sqrt{2} \in \Q$,
		which is a contradiction.
	\end{proof}
	\item \begin{proof}[Solution]\let\qed\relax
		ff
	\end{proof}
\end{enumerate}


\subsection*{Problem 4 (Chapter 2.11)}
\problem{
	Show that $x^3 + x^2 + 1$ is irreducible in $(\Z/(2))[x]$
	and that $(\Z/(2))[x]/(x^3+x^2+1)$ is a field with eight elements.
}

\begin{proof}[Solution]\let\qed\relax
	ff
\end{proof}
\end{document}
