\documentclass{article}
\usepackage{amsmath, amsfonts, amsthm, amssymb}
\usepackage{geometry}
\geometry{letterpaper, margin=2.0cm, includefoot, footskip=30pt}

\usepackage{fancyhdr}
\pagestyle{fancy}

\lhead{Math 323}
\chead{Homework 1}
\rhead{Nicholas Rees, 11848363}
\cfoot{Page \thepage}

\newcommand{\N}{{\mathbb N}}
\newcommand{\Z}{{\mathbb Z}}
\newcommand{\Q}{{\mathbb Q}}
\newcommand{\R}{{\mathbb R}}
\newcommand{\C}{{\mathbb C}}
\newcommand{\ep}{{\varepsilon}}

\newtheorem{lemma}{Lemma}

\renewcommand{\theenumi}{(\alph{enumi})}

\begin{document}
\subsection*{Problem 1 (Ch. 2.1)}
{\it Let $C$ be the set of real-valued continuous functions on the real line $\R$.
Show that $C$ with the usual addition of functions and $0$ is an abelian group,
and that $C$ with product $(f\cdot g)(x) = f(g(x))$ and $1$ the identity map is a monoid.
Is $C$ with these compositions and $0$ and $1$ a ring?}
\begin{proof}[Solution]\let\qed\relax
	Let $f,g,h \in C$.

	We first show $(C,+,0)$ is an abelian group.
	We have that $f + g$ is also a real-valued continuous function,
	and so $f + g \in C$.
	The associativity and commutativity of real addition gives
	$(f(x_0) + g(x_0)) + h(x_0) = f(x_0) + (g(x_0) + h(x_0))$
	and $f(x_0) + g(x_0) = g(x_0) + f(x_0)$
	for all $x_0 \in \R$, hence $(f + g) + h = f + (g + h)$ and $f + g = g + h$.
	Furthermore, the zero function $0$ is in $C$,
	and $0 + f = f + 0 = f$.
	Finally, if we consider $F = -f$,
	multiplying by a scalar does not change if a function is continuous or not,
	so $F \in C$,
	and $f + F = F + f = 0$.
	This satisfies all the conditions for an abelian group.
	
	We now show that $(C, \circ, 1)$ is a monoid.
	Recall that the composition of two continuous functions is also continuous,
	so $f \circ g \in C$.
	Furthermore, $(f \circ g)\circ h(x) = f(g(h(x))) = f \circ (g \circ h)(x)$,
	which shows associativity.
	Finally, the identity map is continuous on $\R$,
	and $(1 \circ f)(x) = (f \circ 1)(x) = f(x)$.
	This satisfies all the conditions of a monoid.

	It remains to consider the distributive laws, which will show that $C$ is not a ring.
	Let $f(x) = x + 1$, $g(x) = 1$ and $h(x) = -1$.
	These are all obviously in $C$.
	We have $(f\circ(g+h))(x) = (f\circ0)(x) = 1$ for all $x$,
	however $(f \circ g)(x) + (f \circ h)(x) = 2 + 0 = 2$ for all $x$.
	Thus $(f\circ(g+h))(x) \neq (f \circ g)(x) + (f \circ h)(x)$,
	and so $C$ is not a ring.
\end{proof}

\subsection*{Problem 4 (Ch. 2.1)}
{\it Let $I$ be the set of complex numbers of the form
$m + n\sqrt{-3}$ where either $m,n\in\Z$ or both
$m$ and $n$ are halves of odd integers.
Show that $I$ is a subring of $\C$.}
\begin{proof}[Solution]\let\qed\relax
	We first show that $(I,+,0)$ form an abelian group.
	Since $\C$ is a ring, $+$ is associative and commutative,
	and $0 = 0 + 0\sqrt{-3} \in I$.
	Note that for any $m + n\sqrt{-3}$, $-m - n\sqrt{-3}$ is the additive inverse in $\C$,
	and if $m,n \in \Z$, so is $-m,-n$,
	or if $m$ and $n$ are halves of odd integers, say $2m$ and $2n$,
	then $-m,-n$ are halves of $-2m,-2n$ which are also odd integers;
	so additive inverses of elements in $I$ are also in $I$.
	Finally, $(m + n\sqrt{-3}) + (m' + n'\sqrt{-3}) = (m + m') + (n + n')\sqrt{-3}$.
	If $m,n$ and $m',n'$ were all integers,
	then $m + m' \in \Z$ and $n + n' \in \Z$.
	If one of $m,n$ and $m',n'$ were integers,
	and so the others were half of odd integers,
	then $m + m'$ and $n + n'$ are also half of odd integers,
	namely $2m+2m'$ and $2n+2n'$
	(which is odd, since WLOG $2m,2n$ are even and $2m',2n'$ are odd).
	If all of $m,n,m',n'$ were half of odd integers,
	then $m + m' \in \Z$ and $n + n' \in \Z$.
	Hence, $(m + n\sqrt{-3}) + (m' + n'\sqrt{-3}) \in I$.
	This shows that $(I, + , 0)$ is an abelian group.

	We now show that $(I,\cdot,1)$ is a monoid.
	Since $\C$ is a ring, $\cdot$ is associative.
	Note that the multiplicative identity in $\C$,
	$1 + 0\sqrt{-3}$, is in $I$ as well (both $m,n \in \Z$).
	Finally, we show that $I$ is closed under multiplication.
	Note $(m + n\sqrt{-3})\cdot(m' + n'\sqrt{3})
	= (mm' - 3nn') + (mn' + nm')\sqrt{-3}$.
	When $m,n,m',n' \in \Z$, then $mm'-3nn'$ and $mn'+nm'$
	are in $\Z$ as well.
	If one of the two, say WLOG $m,n\in\Z$ while $m',n'$ are halves of odd integers,
	then let $l = 2m'$, $k = 2n'$ where $l,k$ are odd,
	and we have $mm' - 3nn' = (ml - 3nk)/2$ and ff even or odd istg

	It now remains to show the distributive laws hold.
	See
	\begin{align*}
		(m + n\sqrt{-3})\big((m'+n'\sqrt{-3}) + (m'' + n''\sqrt{-3})\big)
		&= (m + n\sqrt{-3})\big((m'+m'') + (n'+n'')\sqrt{-3}\big)\\
		&= (m(m'+m'') - 3n(n'+n'')) + (m(n'+n'') + n(m'+m''))\sqrt{-3}\\
		&= mm' + mm'' - 3nn' -3nn'' + (mn' + mn'' + nm' + nm'')\sqrt{-3}\\
		&= mm' -3nn' + (mn' + nm')\sqrt{-3}\\
		&\quad+ mm'' -3nn'' + (mn'' + nm'')\sqrt{-3}\\
		&= (m + n\sqrt{-3})(m' + n'\sqrt{-3}) + (m + n\sqrt{-3})(m'' + n''\sqrt{-3})
	\end{align*}
	and
	\begin{align*}
		\big((m'+n'\sqrt{-3}) + (m'' + n''\sqrt{-3})\big)(m + n\sqrt{-3})
		&= \big((m'+m'') + (n'+n'')\sqrt{-3}\big)(m + n\sqrt{-3})\\
		&= ((m'+m'')m - 3(n'+n'')n) + ((n'+n'')m + (m'+m'')n)\sqrt{-3}\\
		&= m'm + m''m - 3n'n -3n''n + (n'm + n''m + m'n + m''n)\sqrt{-3}\\
		&= m'm -3n'n + (n'm + m'n)\sqrt{-3}\\
		&\quad+ m''m -3n''n + (n''m + m''n)\sqrt{-3}\\
		&= (m' + n'\sqrt{-3})(m + n\sqrt{-3}) + (m'' + n''\sqrt{-3})(m + n\sqrt{-3})
	\end{align*}
	Thus, we have proven that $I$ is a ring,
	and so is a subring of $\C$.
\end{proof}

\subsection*{Problem 1 (Ch. 2.2)}
{\it Show that any finite domain is a division ring.}
\begin{proof}[Solution]\let\qed\relax
	A domain is that $R^*$ is a monoid (no zero divisors),
	while a division ring is where $R^*$ is a group (everything invertible).
	
	Let $n$ be the number of elements in $R^*$.
	Since there are no zero divisors,
	$a,a^2,\dots,a^n \neq 0$, and by pigeonhole principle,
	there exists $j, 1 \leq j \leq n$ such that $a^j = a^{n+1}$.
	Then does $a^{n+1-j} = 1$?

	$a(1+0) = a$.

	$1 + 1 + \cdots + 1 = 0$ eventually because additive group.
	Hmm... I think exploit something about how we can
	get back to $0$ with adding
	(gauranteed with finite group, perhaps not with infinite)
	but we can't get to $0$ with multiplying.
\end{proof}

\subsection*{Problem 4 (Ch. 2.2)}
{\it Show that if $1 - ab$ is invertible in a ring then so is $1 - ba$.}
\begin{proof}[Solution]\let\qed\relax
	Assume there exists $c$ such that $c(1-ab) = (1-ab)c = 1$.
	Let $d = 1 + bca$.
	Using the distributive property of the ring, we see
	\[
		d(1-ba) = (1 - ba) + bca(1 - ba)
		= 1 - ba + bc(a - aba) =
		1 - ba + bc(1 - ab)a = 1 - ba + ba = 1
	\]
	and
	\[
		(1-ba)d = (1 - ba) + (1 - ba)bca
		= 1 - ba + (b - bab)ca =
		1 - ba + b(1 - ab)ca = 1 - ba + ba = 1
	\]
	hence, $d$ is an inverse of $1 - ba$,
	so $1- ba$ is invertible.
\end{proof}

\subsection*{Problem 6 (Ch. 2.2)}
{\it Let $u$ be an element of a ring that has a right inverse.
Prove that the following conditions on $u$ are equivalent:
(1) $u$ has more than one right inverse,
(2) $u$ is not a unit,
(3) $u$ is a left $0$ divisor.}
\begin{proof}[Solution]\let\qed\relax
	We first show (1) $\implies$ (2).
	We show the contrapositive.
	Let $u$ be a unit, that is $\exists v$ such that $vu = uv = 1$.
	Now let $v'$ be another right inverse of $u$.
	Then $uv' = 1$, so
	then $(vu)v' = v(uv') \implies v' = v$.
	Hence, any right inverse of $u$ is just $v$,
	so there cnanot be more than one right inverse.

	Now we show (2) $\implies$ (3).
	Let $v$ be the right inverse of $u$.
	Since $u$ is not a unit, $uv = 1$ but $vu \neq 1$.
	See
	\[
		0 = 1 - uv \implies 0u = (1 - uv)u \implies
		0 = u - uvu \implies 0 = u(1-vu)
	\]
	And since $1 \neq vu \implies 1 - vu \neq 0$, we have that $u$
	is a left $0$ divisor.

	Now we show (3) $\implies$ (1).
	We have $\exists v$ such that $uv = 1$ and $\exists w\neq0$ such that $uw = 0$.
	Then $uv + uw = 1 + 0 \implies u(v + w) = 1$.
	But since $w \neq 0 \implies v + w \neq v$,
	we have that $v + w$ is a distinct right inverse of $u$.
	Hence, $u$ has more than one right inverse, $v$ and $v + w$.
\end{proof}

\subsection*{Problem 7 (Ch. 2.2)}
{\it (Kaplansky.) Prove that if an element of a ring has more than one
right inverse then it has infinitely many.
Construct a counterexample to show that this does not hold for monoids.}
\begin{proof}[Solution]\let\qed\relax
	ff
\end{proof}
\end{document}
