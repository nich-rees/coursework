\documentclass{article}
\usepackage{amsmath, amsfonts, amsthm, amssymb}
\usepackage{geometry}
\geometry{letterpaper, margin=2.0cm, includefoot, footskip=30pt}

\usepackage{fancyhdr}
\pagestyle{fancy}

\lhead{Math 323}
\chead{Homework 1}
\rhead{Nicholas Rees, 11848363}
\cfoot{Page \thepage}

\newcommand{\N}{{\mathbb N}}
\newcommand{\Z}{{\mathbb Z}}
\newcommand{\Q}{{\mathbb Q}}
\newcommand{\R}{{\mathbb R}}
\newcommand{\C}{{\mathbb C}}
\newcommand{\ep}{{\varepsilon}}

\newtheorem{lemma}{Lemma}

\renewcommand{\theenumi}{(\alph{enumi})}

\begin{document}
\subsection*{Problem 1 (Ch. 2.1)}
{\it Let $C$ be the set of real-valued continuous functions on the real line $\R$.
Show that $C$ with the usual addition of functions and $0$ is an abelian group,
and that $C$ with product $(f\cdot g)(x) = f(g(x))$ and $1$ the identity map is a monoid.
Is $C$ with these compositions and $0$ and $1$ a ring?}
\begin{proof}[Solution]\let\qed\relax
	Let $f,g,h \in C$.

	We first show $(C,+,0)$ is an abelian group.
	We have that $f + g$ is also a real-valued continuous function,
	and so $f + g \in C$.
	The associativity and commutativity of real addition gives
	$(f(x_0) + g(x_0)) + h(x_0) = f(x_0) + (g(x_0) + h(x_0))$
	and $f(x_0) + g(x_0) = g(x_0) + f(x_0)$
	for all $x_0 \in \R$, hence $(f + g) + h = f + (g + h)$ and $f + g = g + h$.
	Furthermore, the zero function $0$ is in $C$,
	and $0 + f = f + 0 = f$.
	Finally, if we consider $F = -f$,
	multiplying by a scalar does not change if a function is continuous or not,
	so $F \in C$,
	and $f + F = F + f = 0$.
	This satisfies all the conditions for an abelian group.
	
	We now show that $(C, \circ, 1)$ is a monoid.
	Recall that the composition of two continuous functions is also continuous,
	so $f \circ g \in C$.
	Furthermore, $(f \circ g)\circ h(x) = f(g(h(x))) = f \circ (g \circ h)(x)$,
	which shows associativity.
	Finally, the identity map is continuous on $\R$,
	and $(1 \circ f)(x) = (f \circ 1)(x) = f(x)$.
	This satisfies all the conditions of a monoid.

	It remains to consider the distributive laws, which will show that $C$ is not a ring.
	Let $f(x) = x + 1$, $g(x) = 1$ and $h(x) = -1$.
	These are all obviously in $C$.
	We have $(f\circ(g+h))(x) = (f\circ0)(x) = 1$ for all $x$,
	however $(f \circ g)(x) + (f \circ h)(x) = 2 + 0 = 2$ for all $x$.
	Thus $(f\circ(g+h))(x) \neq (f \circ g)(x) + (f \circ h)(x)$,
	and so $C$ is not a ring.
\end{proof}

\subsection*{Problem 4 (Ch. 2.1)}
{\it Let $I$ be the set of complex numbers of the form
$m + n\sqrt{-3}$ where either $m,n\in\Z$ or both
$m$ and $n$ are halves of odd integers.
Show that $I$ is a subring of $\C$.}
\begin{proof}[Solution]\let\qed\relax
	We first show that $(I,+,0)$ form an abelian group.
	Since $\C$ is a ring, $+$ is associative and commutative,
	and $0 = 0 + 0\sqrt{-3} \in I$.
	Note that for any $m + n\sqrt{-3}$, $-m - n\sqrt{-3}$ is the additive inverse in $\C$,
	and if $m,n \in \Z$, so is $-m,-n$,
	or if $m$ and $n$ are halves of odd integers, say $2m$ and $2n$,
	then $-m,-n$ are halves of $-2m,-2n$ which are also odd integers;
	so additive inverses of elements in $I$ are also in $I$.
	Finally, $(m + n\sqrt{-3}) + (m' + n'\sqrt{-3}) = (m + m') + (n + n')\sqrt{-3}$.
	If $m,n$ and $m',n'$ were all integers,
	then $m + m' \in \Z$ and $n + n' \in \Z$.
	If one of $m,n$ and $m',n'$ were integers,
	and so the others were half of odd integers,
	then $m + m'$ and $n + n'$ are also half of odd integers,
	namely $2m+2m'$ and $2n+2n'$
	(which is odd, since WLOG $2m,2n$ are even and $2m',2n'$ are odd).
	If all of $m,n,m',n'$ were half of odd integers,
	then $m + m' \in \Z$ and $n + n' \in \Z$.
	Hence, $(m + n\sqrt{-3}) + (m' + n'\sqrt{-3}) \in I$.
	This shows that $(I, + , 0)$ is an abelian group.

	We now show that $(I,\cdot,1)$ is a monoid.
	Since $\C$ is a ring, $\cdot$ is associative.
	Note that the multiplicative identity in $\C$,
	$1 + 0\sqrt{-3}$, is in $I$ as well (both $m,n \in \Z$).
	Finally, we show that $I$ is closed under multiplication.
	Note
	\[
		(m + n\sqrt{-3})\cdot(m' + n'\sqrt{3})
		= (mm' - 3nn') + (mn' + nm')\sqrt{-3}
	\]
	When $m,n,m',n' \in \Z$, then $mm'-3nn'$ and $mn'+nm'$
	are in $\Z$ as well.
	If one of the two, say WLOG $m,n\in\Z$, while $m',n'$ are halves of odd integers,
	then let $l = 2m'$, $k = 2n'$ where $l,k$ are odd,
	and we have $mm' - 3nn' = (ml - 3nk)/2$ which is an integer
	when $ml - 3nk$ is even and half an odd integer when $ml - 3nk$ is odd
	(and one of the two always happens, since $ml - 3nk \in \Z$);
	we also have $mn' + nm' = (mk + nl)/2$ which is an integer
	when $mk+nl$ is even and half an odd integer when $mk + nl$ is odd;
	it remains to show that $ml - 3nk$ and $mk + nl$ have the same parity:
	since $l,k,3$ are odd, $ml \equiv mk \equiv m \,(\mathrm{mod}\,2)$
	and $3nk \equiv nl \equiv n \,(\mathrm{mod}\,2)$,
	so
	\[
		ml-3nk \equiv m - n \equiv m - n + 2n \equiv m + n
		\equiv mk + nl\,(\mathrm{mod}\,2)
	\]
	which confirms that they have the same parity.
	We now can turn to the final case, which is when $m,n,m',n'$
	are all halves of odd integers.
	Then denote $a = 2m, b = 2n, l = 2m', k = 2n'$ all of which are odd.
	See $mm'-3nn' = \frac{al-3bk}{2}$ and $al-3bk$ is even so $mm'-3nn'$ is an integer,
	and $mn' + nm' = \frac{ak + bl}{2}$ and $ak + bl$ is even so $mn' + nm'$ is an integer.
	This exhausts all possible cases of $m,n,m',n'$,
	showing that $I$ is closed under multiplication.

	It now remains to show the distributive laws hold.
	See
	\begin{align*}
		(m + n\sqrt{-3})\big((m'+n'\sqrt{-3}) + (m'' + n''\sqrt{-3})\big)
		&= (m + n\sqrt{-3})\big((m'+m'') + (n'+n'')\sqrt{-3}\big)\\
		&= (m(m'+m'') - 3n(n'+n'')) + (m(n'+n'') + n(m'+m''))\sqrt{-3}\\
		&= mm' + mm'' - 3nn' -3nn'' + (mn' + mn'' + nm' + nm'')\sqrt{-3}\\
		&= mm' -3nn' + (mn' + nm')\sqrt{-3}\\
		&\quad+ mm'' -3nn'' + (mn'' + nm'')\sqrt{-3}\\
		&= (m + n\sqrt{-3})(m' + n'\sqrt{-3}) + (m + n\sqrt{-3})(m'' + n''\sqrt{-3})
	\end{align*}
	and
	\begin{align*}
		\big((m'+n'\sqrt{-3}) + (m'' + n''\sqrt{-3})\big)(m + n\sqrt{-3})
		&= \big((m'+m'') + (n'+n'')\sqrt{-3}\big)(m + n\sqrt{-3})\\
		&= ((m'+m'')m - 3(n'+n'')n) + ((n'+n'')m + (m'+m'')n)\sqrt{-3}\\
		&= m'm + m''m - 3n'n -3n''n + (n'm + n''m + m'n + m''n)\sqrt{-3}\\
		&= m'm -3n'n + (n'm + m'n)\sqrt{-3}\\
		&\quad+ m''m -3n''n + (n''m + m''n)\sqrt{-3}\\
		&= (m' + n'\sqrt{-3})(m + n\sqrt{-3}) + (m'' + n''\sqrt{-3})(m + n\sqrt{-3})
	\end{align*}
	Thus, we have proven that $I$ is a ring,
	and so is a subring of $\C$.
\end{proof}

\subsection*{Problem 1 (Ch. 2.2)}
{\it Show that any finite domain is a division ring.}
\begin{proof}[Solution]\let\qed\relax
	For the sake of contradiction, assume that $R$ is a finite domain that is not a division ring.
	Then, there exists some element $a \in R$, $a\neq 0$ that is not invertible.
	Let $n$ denote the finite number of elements in $R^* = R \setminus \{0\}$.

	We claim that for every $x,y' \in R$, if $xa = x'a$, then $x=x'$,
	since $xa = x'a \implies xa - x'a = 0 \implies (x-x')a = 0$,
	and since $R$ is a domain and $a \neq 0$, $x - x' = 0 \implies x = x'$.
	Hence, $\{x_1a,x_2a,\dots,x_na\}$ are distinct, non-zero elements,
	where $x_i$ ranges over all the elements of $R^*$
	(the non-zeroness is because $x_i,a \neq 0 \implies x_ia\neq0$ in a domain).
	Since all of $x_ia \in R^*$
	(by the fact that $(R^*,\cdot)$ is a monoid when $R$ is a domain)
	so $\{x_1a,x_2a,\dots,x_na\} \subset R^*$,
	and there are the same number of elements ($n$)
	in both $\{x_1a,x_2a,\dots,x_na\}$ and $R^*$, we have
	$\{x_1a,x_2a,\dots,x_na\} = R^*$.
	Hence, there exists some $1 \leq j \leq n$ such that $x_ja = 1$.
	Thus, $a$ has a left inverse.
	From now on, denote $l = x_j$.
	
	We now do everything for right multiplication.
	So $ay = ay' \implies y = y'$ since
	$ay - ay' = 0 \implies a(y-y') = 0 \implies y - y' = 0 \implies y = y'$
	as before.
	Hence, $\{ay_1,ay_2,\dots,ay_n\}$ are distinct, non-zero elements,
	where $y_i$ rangers over all the elements of $R^*$.
	Since $ay_i \in R^*$ and there are $n$ elements in the set and $R^*$,
	we again have that there exists $1 \leq k \leq n$ such that $ay_k = 1$.
	Thus $a$ has a right inverse, denoted $r = y_k$.

	Now since we have $la = 1$, $ar = 1 \implies lar = l \implies r = l$.
	Hence, $l$ is an inverse of $a$, contradicting our assumption that
	$a$ was not invertible.
	Therefore, we have shown that any finite domain is also a division ring.
\end{proof}

\subsection*{Problem 4 (Ch. 2.2)}
{\it Show that if $1 - ab$ is invertible in a ring then so is $1 - ba$.}
\begin{proof}[Solution]\let\qed\relax
	Assume there exists $c$ such that $c(1-ab) = (1-ab)c = 1$.
	Let $d = 1 + bca$.
	Using the distributive property of the ring, we see
	\[
		d(1-ba) = (1 - ba) + bca(1 - ba)
		= 1 - ba + bc(a - aba) =
		1 - ba + bc(1 - ab)a = 1 - ba + ba = 1
	\]
	and
	\[
		(1-ba)d = (1 - ba) + (1 - ba)bca
		= 1 - ba + (b - bab)ca =
		1 - ba + b(1 - ab)ca = 1 - ba + ba = 1
	\]
	hence, $d$ is an inverse of $1 - ba$,
	so $1- ba$ is invertible.
\end{proof}

\subsection*{Problem 6 (Ch. 2.2)}
{\it Let $u$ be an element of a ring that has a right inverse.
Prove that the following conditions on $u$ are equivalent:
(1) $u$ has more than one right inverse,
(2) $u$ is not a unit,
(3) $u$ is a left $0$ divisor.}
\begin{proof}[Solution]\let\qed\relax
	We first show (1) $\implies$ (2).
	We show the contrapositive.
	Let $u$ be a unit, that is $\exists v$ such that $vu = uv = 1$.
	Now let $v'$ be another right inverse of $u$.
	Then $uv' = 1$, so
	then $(vu)v' = v(uv') \implies v' = v$.
	Hence, any right inverse of $u$ is just $v$,
	so there cannot be more than one right inverse.

	Now we show (2) $\implies$ (3).
	Let $v$ be the right inverse of $u$.
	Since $u$ is not a unit, $uv = 1$ but $vu \neq 1$.
	See
	\[
		0 = 1 - uv \implies 0u = (1 - uv)u \implies
		0 = u - uvu \implies 0 = u(1-vu)
	\]
	And since $1 \neq vu \implies 1 - vu \neq 0$, we have that $u$
	is a left $0$ divisor.

	Now we show (3) $\implies$ (1).
	We have $\exists v$ such that $uv = 1$ and $\exists w\neq0$ such that $uw = 0$.
	Then $uv + uw = 1 + 0 \implies u(v + w) = 1$.
	But since $w \neq 0 \implies v + w \neq v$,
	we have that $v + w$ is a distinct right inverse of $u$.
	Hence, $u$ has more than one right inverse, $v$ and $v + w$.
\end{proof}

\subsection*{Problem 7 (Ch. 2.2)}
{\it (Kaplansky.) Prove that if an element of a ring has more than one
right inverse then it has infinitely many.
Construct a counterexample to show that this does not hold for monoids.}
\begin{proof}[Solution]\let\qed\relax
	Let $u$ be an element of a ring $R$ that has more than one right inverse.
	So there is some $v \in R$ such that $uv = 1$.
	From Problem 6 above, $u$ is not a unit, so $vu \neq 1$.
	This also means that $u^n \neq 1$ when $n > 0$,
	otherwise $uu^{n-1} = u^{n-1}u = 1$ making $u$ a unit.
	Now, for all $n \in \N_0$, define $v_n = (1-vu)u^n + v$.
	Note that $v_n \in R$.
	These $v_n$ are all right inverses of $u$:
	$uv_n = u\left(1-vu)u^n + v\right) = u(1-vu)u^n + uv = (u - uvu)u^n + 1
	= (u - u)u^n + 1 = 0 + 1 = 1$.
	Furthermore, we claim that the map $\phi \colon \N_0 \to \{v_i\}_{i\in\N_0}$
	defined by $\phi \colon n \mapsto v_n$ is injective.
	So assume that $n \neq m$ and we will show that $\phi(n) \neq \phi(m)$,
	i.e. $v_n \neq v_m$.
	WLOG assume that $n > m$.
	Since $n-m-1 \geq 0$, $n-m-1 \in \N_0$
	so $uv_{n-m-1} = 1 \implies v_{n-m-1}u \neq 1$ (otherwise it would be a unit),
	hence $(1-vu)u^{n-m} + vu \neq 1 \implies (1-vu)u^{n-m}\neq 1-vu$.
	Thus $(1-vu)u^n \neq (1-vu)u^m$, i.e. $v_n \neq v_m$ i.e. $\phi(n) \neq \phi(m)$.
	So $\{v_i\}_{i\in\N_0}$ is at least countable in size.
	Thus, there are infinitely many right inverses of $u$.

	Counterexample: define the free monoid $M = \langle a,b,c\rangle$
	such that $ab = ac = 1$
	(operation is concatenation, $1$ is the unit, where $1a = a1 = a$, etc.).
	Both $b,c$ are right inverses of $a$,
	but by definition, no other distinct elements are right inverses of $a$.
\end{proof}
\end{document}
