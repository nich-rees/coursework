\documentclass{article}
\usepackage{amsmath, amsfonts, amsthm, amssymb}
\usepackage{geometry}
\geometry{letterpaper, margin=2.0cm, includefoot, footskip=30pt}

\usepackage{fancyhdr}
\pagestyle{fancy}

\lhead{Math 322}
\chead{Homework 6}
\rhead{Nicholas Rees, 11848363}
\cfoot{Page \thepage}

\newcommand{\N}{{\mathbb N}}
\newcommand{\Z}{{\mathbb Z}}
\newcommand{\Q}{{\mathbb Q}}
\newcommand{\R}{{\mathbb R}}
\newcommand{\C}{{\mathbb C}}
\newcommand{\ep}{{\varepsilon}}

\renewcommand{\theenumi}{(\alph{enumi})}

\begin{document}
\subsection*{Problem 2 (Ch. 1.7)}
{\it Show that if $G$ is finite and $H$ and $K$ are subgroups
such that $H \supset K$
then $[G:K] = [G:H] [H:K]$.}
\begin{proof}[Solution]\let\qed\relax
	Using Langrange's theorem (Theorem 1.5) since $G$ is finite:
	\begin{align*}
		|G|
		&= |H|[G:H]\\
		&= |K|[H:K][G:H]
	\end{align*}
	But Lagrange's theorem also says $\frac{|G|}{|K|} = [G:K]$,
	thus
	\[
		[G:K] = [G:H][H:K]
	\]
	as desired.
\end{proof}

\subsection*{Problem 3 (Ch. 1.7)}
{\it Let $H_1$ and $H_2$ be subgroups of $G$.
Show that any right coset relative to $H_1 \cap H_2$
is the intersection of a right coset of $H_1$ with a right coset of $H_2$.
Use this to prove \emph{Poincar\'{e}'s Theorem}
that if $H_1$ and $H_2$ have finite index in $G$ then so has $H_1 \cap H_2$.}
\begin{proof}[Solution]\let\qed\relax
	Let $x \in H_1\cap H_2g$ for an arbitrary $g \in G$.
	Then $x = h_{12}g$ for some $h_{12} \in H_1 \cap H_2$.
	But then $x \in H_1g$ since $h_{12} \in H_1$
	and $x \in H_2g$ since $h_{12} \in H_2$.
	Thus $x \in H_1g \cap H_2g$.
	Thus $H_1\cap H_2g \subset \in H_1g \cap H_2g$.
	Since $g$ was arbitrary, we proved this for an arbitrary right coset
	relative to $H_1 \cap H_2$, so this is true for all of them.

	Now let $x \in H_1g_1 \cap H_2g_2$ for arbitrary $g_1,g_2\in G$.
	If there does not exist such $x$ for the given $g_1,g_2$,
	then our statement is vacuously true.
	So now, assume that our $x$ exists.
	Then $x = h_1g_1 = h_2g_2$ for some $h_1 \in H_1$, $h_2 \in H_2$.
	Want to show that $h_1,h_2 \in H_1 \cap H_2$.
	Note that $h_1 = (h_2g_2)g_1^{-1} = h_2(g_2g_1^{-1})$.
	But then $h_1$ is in a right coset relative to $H_2$.
	Like wise, $h_2 = h_1(g_1g_2^{-1})$, so $h_2$ is in a right coset relative to $H_1$.
\end{proof}

\subsection*{Problem 4 (Ch. 1.7)}
{\it Let $G$ be a finitely generated group,
$H$ a subgroup of finite index.
Show that $H$ is finitely generated.}
\begin{proof}[Solution]\let\qed\relax
	Let $[G : H] = r$.
	Then $G = H \cup Hg_1 \cup \cdots \cup Hg_{r-1}$.
	Note that if $S = \{s_1,s_2,\dots,s_n\}$ is the finite set that generates $G$,
	so $G = \langle S \rangle$,
	then $G = H $...
	something about $H$ being a group so its closed does something?
	Every element in $G$ can be written as a product of $S$,
	but also as $H$ times some other element in $g$ (are we making use of finite index though?).
	ff
\end{proof}


\subsection*{Problem 5 (Ch. 1.7)}
{\it Let $H$ and $K$ be two subgroups of a group $G$.
Show that the set of maps $x \to hxk$, $h \in H$, $k \in K$
is a group of transformatoins of the set $G$.
Show that the orbit of $x$ relative to this group
is the set $HxK = \{hxk \mid h\in H, k\in K\}$.
This is called the \emph{double coset of $x$ relative to the pair $(H,K)$}.
Show that if $G$ is finite then $|HxK| = |H|[K \colon x^{-1}Hx\cap K]
= |K|[H\colon xKx^{-1} \cap H]$.}
\begin{proof}[Solution]\let\qed\relax
	ff
\end{proof}

\subsection*{Problem 3 (Ch. 1.8)}
{\it Let $G$ be the group of pairs of real numbers $(a,b)$ $a\neq0$,
with the product $(a,b)(c,d) = (ac,ad+b)$ (exercise 4, p.36).
Verify that $K = \{(1,b)\mid b\in\R\}$ is a normal subgroup of $G$.
Show that $G/K \cong (\R^*,\cdot,1)$ the multiplicative group of non-zero reals.}
\begin{proof}[Solution]\let\qed\relax
	ff
\end{proof}

\subsection*{Problem 4 (Ch. 1.8)}
{\it Show that any subgroup of index two is normal.
Hence prove that $A_n$ is normal in $S_n$.}
\begin{proof}[Solution]\let\qed\relax
	If a subgroup $H$ of $G$ has index two,
	then there exists $g \in G$, $g \not\in H$
	such that $G = H \cup Hg$.
\end{proof}

\subsection*{Problem 5 (Ch. 1.8)}
{\it Verify that the intersection of any set of normal subgroups
of a group is a normal subgroup.
Show if $H$ and $K$ are normal subgroups,
then $HK$ is a normal subgroup.}
\begin{proof}[Solution]\let\qed\relax
	Let $H$ and $K$ be normal subgroups of $G$.
	If $x \in H \cap K$,
	then $gxg^{-1} \in H$ since $x \in H$ and $H$ is normal,
	and $gxg^{-1} \in K$ since $x \in K$ and $K$ is normal.
	Thus $gxg^{-1} \in H \cap K$, thus $H \cap K$ is normal.

	ff something to do with that silly paragraph at the end of 1.8
\end{proof}
\end{document}
