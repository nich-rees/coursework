\documentclass{article}
\usepackage{amsmath, amsfonts, amsthm, amssymb}
\usepackage{geometry}
\geometry{letterpaper, margin=2.0cm, includefoot, footskip=30pt}

\usepackage{fancyhdr}
\pagestyle{fancy}

\lhead{Math 322}
\chead{Homework 6}
\rhead{Nicholas Rees, 11848363}
\cfoot{Page \thepage}

\newcommand{\N}{{\mathbb N}}
\newcommand{\Z}{{\mathbb Z}}
\newcommand{\Q}{{\mathbb Q}}
\newcommand{\R}{{\mathbb R}}
\newcommand{\C}{{\mathbb C}}
\newcommand{\ep}{{\varepsilon}}

\renewcommand{\theenumi}{(\alph{enumi})}

\begin{document}
\subsection*{Problem 2 (Ch. 1.7)}
{\it Show that if $G$ is finite and $H$ and $K$ are subgroups
such that $H \supset K$
then $[G:K] = [G:H] [H:K]$.}
\begin{proof}[Solution]\let\qed\relax
	Using Langrange's theorem (Theorem 1.5) since $G$ is finite:
	\begin{align*}
		|G|
		&= |H|[G:H]\\
		&= |K|[H:K][G:H]
	\end{align*}
	But Lagrange's theorem also says $\frac{|G|}{|K|} = [G:K]$,
	thus
	\[
		[G:K] = [G:H][H:K]
	\]
	as desired.
\end{proof}

\subsection*{Problem 3 (Ch. 1.7)}
{\it Let $H_1$ and $H_2$ be subgroups of $G$.
Show that any right coset relative to $H_1 \cap H_2$
is the intersection of a right coset of $H_1$ with a right coset of $H_2$.
Use this to prove \emph{Poincar\'{e}'s Theorem}
that if $H_1$ and $H_2$ have finite index in $G$ then so has $H_1 \cap H_2$.}
\begin{proof}[Solution]\let\qed\relax
	Let $x \in (H_1\cap H_2)g$ for an arbitrary $g \in G$.
	Then $x = h_{12}g$ for some $h_{12} \in H_1 \cap H_2$.
	But then $x \in H_1g$ since $h_{12} \in H_1$
	and $x \in H_2g$ since $h_{12} \in H_2$.
	Thus $x \in H_1g \cap H_2g$.
	Thus $H_1\cap H_2g \subseteq H_1g \cap H_2g$ since $x$ was arbitrary.

	Now let $x \in H_1g \cap H_2g$ for the same $g \in G$ as before.
	If these sets are disjoint, then $H_1g \cap H_2g = \emptyset \subseteq (H_1 \cap H_2)g$.
	then our statement is vacuously true.
	So now, assume that our $x$ exists.
	Then $x = h_1g = h_2g$ for some $h_1 \in H_1$, $h_2 \in H_2$.
	Note that $h_1 = h_2gg^{-1} = h_2$.
	Thus $h_1 \in H_2$, so $h_1 \in H_1 \cap H_2$.
	Thus $x \in (H_1 \cap H_2)g$.
	So $H_1g \cap H_2g \subseteq (H_1 \cap H_2)g$ since $x$ was arbitrary.
	Therefore $(H_1 \cap H_2)g = H_1g \cap H_2g$,
	and since the coset relative to $H_1 \cap H_2$ (arbitrary $g$),
	this is true for all right cosets relative to $H_1 \cap H_2$.
\end{proof}

\subsection*{Problem 4 (Ch. 1.7)}
{\it Let $G$ be a finitely generated group,
$H$ a subgroup of finite index.
Show that $H$ is finitely generated.}
\begin{proof}[Solution]\let\qed\relax
	Let $[G : H] = r$.
	Then $G = Hx_1 \sqcup Hx_2 \sqcup \cdots \sqcup Hx_r$ where $x_1 = 1$.
	Note that if $S = \{s_1,s_2,\dots,s_n\}$ is the finite set that generates $G$,
	so $G = \langle S \rangle$.
	Note that without loss of generality,
	we can include $s_i^{-1}$ in $S$ for all $i$
	(since the group generated by it already included it,
	and the set is still finite, since we just double
	the size of our set none of the inverses were in it before).
	Note that for any $i,j$, $x_ig_j = u_{ij}x_{i'}$ for some $u_{ij} \in H$,
	since we must have that $x_ig_j$ is in some coset
	(namely $Hx_{i'}$).

	We claim that $H$ is generated by $\{u_{ij}\}$,
	thus it is finitely generated.
	Let $h = g_{i_1}g_{i_2}\cdots g_{i_l} \in H$, where $g_{ij} \in H$.
	Thus
	\begin{align*}
		h &= (x_1g_{i_1})g_{i_2} \cdots g_{i_l}\\
		  &= (u_{1i_1}x_{1'})g_{i_2}\cdots g_{i_l}\\
		  &= u_{1i_1}(u_{2i_2}x_{2'})\cdots g_{i_l}\\
		  &\vdots\\
		  &=u_{1i_1}u_{2i_2} \cdots u_{li_l}x_{l'} \in H = Hx_1
	\end{align*}
	Thus $x_{l'} = x_1 = 1$, and thus $H$ is generated by $\{u_{ij}\}$.
\end{proof}


\subsection*{Problem 5 (Ch. 1.7)}
{\it Let $H$ and $K$ be two subgroups of a group $G$.
Show that the set of maps $x \to hxk$, $h \in H$, $k \in K$
is a group of transformations of the set $G$.
Show that the orbit of $x$ relative to this group
is the set $HxK = \{hxk \mid h\in H, k\in K\}$.
This is called the \emph{double coset of $x$ relative to the pair $(H,K)$}.
Show that if $G$ is finite then $|HxK| = |H|[K \colon x^{-1}Hx\cap K]
= |K|[H\colon xKx^{-1} \cap H]$.}
\begin{proof}[Solution]\let\qed\relax
	Let $\alpha \colon x \to hxk$, $\beta \colon x \to h'xk'$
	and $\gamma \colon x \to h''xk''$,
	where $h,h',h'' \in H$ and $k,k',k'' \in K$.
	First note that group is closed (under composition),
	since $(\beta\alpha)(x) = h'hxkk'$,
	and $h'h \in H$, $kk' \in K$ since $H$,$K$ are groups,
	so $\beta\alpha$ is in the set of transformations as well.
	Further
	\[
		(\gamma\beta)\alpha(x) = \gamma\beta(hxk) = \gamma(h'hxkk')
		= \gamma (\beta\alpha)(x)
	\]
	so the operation is associative.
	Also, there exists an identity in $H$ and $K$,
	so define $1(x) = 1_Hx1_K$,
	and see $1\alpha(x) = 1_Hhxk1_K = hxk = \alpha(x) = h1_Hx1_Kx = \alpha1(x)$.
	Finally, for any $h,k$, there exist inverses $h^{-1} \in H$, $k^{-1} \in K$,
	so define $\alpha^{-1}(x) = h^{-1}xk^{-1}$
	and see $\alpha^{-1}\alpha(x) = h^{-1}hxkk^{-1} = 1_Hx1_K = 1(x)$
	and $\alpha\alpha^{-1}(x) = hh^{-1}xk^{-1}k = 1_Hx1_K = 1(x)$.
	Thus, since our choice of $h,k$ (and other elements) were abitrary,
	these properties hold for any map of the form $x \to hxk$,
	thus this forms a group of transformations of the set $G$.

	The orbit of $x$ relative to this group is
	$\{\alpha(x) \mid \alpha \in \text{our group of transformations}\}
	= \{hxk \mid h\in H, k \in K\}$.
	But this is the definition of $HxK$,
	thus $HxK$ is the orbit of $x$ relative to this group of transformations.

	Now, let $G$ be finite of order $r$.
	Note that the subgroups of $G$ must also be finite then.
	By Lagrange's theorem, we have
	\[
		\frac{|HxK|}{|H|} = [H : HxK]
	\]
	hmm... I ran out of time
\end{proof}

\subsection*{Problem 3 (Ch. 1.8)}
{\it Let $G$ be the group of pairs of real numbers $(a,b)$ $a\neq0$,
with the product $(a,b)(c,d) = (ac,ad+b)$ (exercise 4, p.36).
Verify that $K = \{(1,b)\mid b\in\R\}$ is a normal subgroup of $G$.
Show that $G/K \cong (\R^*,\cdot,1)$ the multiplicative group of non-zero reals.}
\begin{proof}[Solution]\let\qed\relax
	If $(a,b) = g \in G$, one can verify that $g^{-1} = (1/a, -b/a)$.
	Let $(1,c) = k \in K$.
	See that
	\[
		g^{-1}kg = (1/a, -b/a)(1,c)(a,b) = (\frac{1}{a}, -b/a)(a,b+c)
		= (1,c/a) \in K
	\]
	and since $g \in G$, $k \in K$ were arbitrary,
	this shows that $K$ is normal in $G$.

	Note that $G/K = $ the set of cosets of the form $K, Kg, Kg_2\dots$
	I think just do an explicit bijection.
	I think we map $x \in \R$ to some $g$ such that $kg = (x,*)$ for any $k$.
	Then $g = (x,1)$.

	We provide a map $\phi \colon (\R^*,\cdot,1) \to G/K$
	by $x \mapsto K(x,0)$.
	We show that this is a bijection by providing an explicit inverse,
	namely $\phi^{-1} \colon K(a,b) \mapsto a$.
	Note that this map is well-defined, ie. it maps to the same $a \in \R^*$,
	irrespective of the representative chosen:
	every element in $K(a,b)$ is of the form $k(a,b)$ where $k\in K$.
	Let $k = (1,b')$, then $(1,b')(a,b) = (a,b+b')$,
	which would get mapped to $a$ as well.
	To show that $\phi$ respects the group operation,
	if $x,y \in \R^*$,
	we have $\phi(x)\phi(y) = K(x,0)K(y,0)$;
	but for any $k,k' \in K$, $k(x,0)k'(y,0) = k(x,0)k'(x,0)^{-1}(x,0)(y,0) = kk''(x,0)(y,0)$ (where $k''\in K$), since $K$ is normal in $G$.
	This is true for any $k \in K$, thus $K(x,0)K(y,0) = K(x,0)(y,0)$.
	But then
	\begin{align*}
		\phi(x)\phi(y)
		&= K(x,0)(y,0)\\
		&= K(xy,x)\\
		&= K(xy,0) &\text{since }(xy,x)^{-1} = (1,-x) \in K\\
		&= \phi(xy)
	\end{align*}
	so we have shown they are isomorphic.
\end{proof}

\subsection*{Problem 4 (Ch. 1.8)}
{\it Show that any subgroup of index two is normal.
Hence prove that $A_n$ is normal in $S_n$.}
\begin{proof}[Solution]\let\qed\relax
	If a subgroup $H$ of $G$ has index two,
	then for any $g \in G$, $g \not\in H$
	we have $G = H \sqcup Hg$.
	Recall (from Jacobson) that $Hg = g^{-1}H$.
	Then $g^{-1}Hg = Hgg$.
	Note that $g^2 \in H$.
	For if $g^2 \not\in H$, then $g^2 \in Hg$,
	(since the index is two, so there are no other cosets),
	and so $Hg = Hg^2$.
	But multiplying by $g^{-1}$ on both sides gives
	$H = Hg$, which contradicts the assumption that $H \neq Hg$.
	Thus $g^{-1}Hg = Hg^2 = H$.
	In otherwords, $g^{-1}hg \in H$ for all $h \in H$.

	Now, if $g \in h$, $g^{-1}hg \in H$ since $H$ is a group, and is closed.
	Thus, regardless if $g \in H$ or $g \not\in H$, we have $g^{-1}hg \in H$,
	thus $H$ is normal in $G$.

	Note $|S_n|/|A_n| = [S_n : A_n] = 2$,
	thus $A_n$ is normal in $S_n$.
\end{proof}

\subsection*{Problem 5 (Ch. 1.8)}
{\it Verify that the intersection of any set of normal subgroups
of a group is a normal subgroup.
Show if $H$ and $K$ are normal subgroups,
then $HK$ is a normal subgroup.}
\begin{proof}[Solution]\let\qed\relax
	Let $H$ and $K$ be normal subgroups of $G$.
	If $x \in H \cap K$,
	then $gxg^{-1} \in H$ since $x \in H$ and $H$ is normal,
	and $gxg^{-1} \in K$ since $x \in K$ and $K$ is normal.
	Thus $gxg^{-1} \in H \cap K$, thus $H \cap K$ is normal.

	Now let $x \in HK = \{hk \colon h \in H, k \in K\}$,
	say $x = hk$.
	Then $gxg^{-1} = ghkg^{-1} = ghg^{-1}gkg^{-1} = h'k' \in HK$
	where $h' \in H$, $k' \in K$, since $ghg^{-1} \in H$ by the normality of $H$,
	and $gkg^{-1} \in K$ by the normality of $K$.
	Thus $HK$ is normal.
\end{proof}
\end{document}
