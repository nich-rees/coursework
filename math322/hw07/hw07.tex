\documentclass{article}
\usepackage{amsmath, amsfonts, amsthm, amssymb}
\usepackage{geometry}
\geometry{letterpaper, margin=2.0cm, includefoot, footskip=30pt}

\usepackage{fancyhdr}
\pagestyle{fancy}

\lhead{Math 322}
\chead{Homework 7}
\rhead{Nicholas Rees, 11848363}
\cfoot{Page \thepage}

\newcommand{\N}{{\mathbb N}}
\newcommand{\Z}{{\mathbb Z}}
\newcommand{\Q}{{\mathbb Q}}
\newcommand{\R}{{\mathbb R}}
\newcommand{\C}{{\mathbb C}}
\newcommand{\ep}{{\varepsilon}}

\renewcommand{\theenumi}{(\alph{enumi})}

\begin{document}
\subsection*{Problem 10 (Ch. 1.8)}
{\it Let $G$ be a finite group, $A$ and $B$ non-vacuous subsets of $G$.
Show that $G = AB$ if $|A| + |B| > |G|$.}
\begin{proof}[Solution]\let\qed\relax
	We prove the contrapostive.
	That is, assume that there exists some $g \in G$ such that
	$g \neq ab$ for any $a \in A, b \in B$.
	$A \neq gB$

	If $|A| + |B| > |G|$, we must have $A \cap B \neq \emptyset$,
	since $A$ and $B$ are nonempty subgroups of $G$
	(and so their respective sets are subsets of $G$'s).
	Thus, there exists some $g \in A$ and $g \in B$.
	Also, we must have some $g' \not\in A$ and $g' \not\in B$.

	Note that $A$ must have finite index,
	since $G$ is a finite group and $A$ has positive order,
	so let $[G : A] = r$.
	Lagrange's theorem gives specifically $|G|/|A| = r$.
	We have $G = A \sqcup Ag_1 \sqcup Ag_2 \sqcup \dots \sqcup Ag_r$.
	Similarly, let $l = |G|/|B|$ be the index of $B$ in $G$.
	We have $2|A| + 2|B| > |A|r + |B|l \implies (2-r)|A| + (2-l)|B| > 0$.

	Prove that $|AB| = |G|$?
\end{proof}

\subsection*{Problem 11 (Ch. 1.8)}
{\it Let $G$ be a group of order $2k$ where $k$ is odd.
Show that $G$ contains a subgroup of index $2$.
(\emph{Hint:} Consider the permutation group $G_L$ of
left translations and use exercise 13, p.36.}
\begin{proof}[Solution]\let\qed\relax
	Since $G$ is of even order, by exercise 13,
	there exists some $a \in G$ such that $a \neq 1$ and $a^2 = 1$.
	ff
\end{proof}

\subsection*{Problem 2 (Ch. 1.9)}
{\it ff}
\begin{proof}[Solution]\let\qed\relax
	ff
\end{proof}


\subsection*{Problem 4 (Ch. 1.9)}
{\it ff}
\begin{proof}[Solution]\let\qed\relax
	ff
\end{proof}

\subsection*{Problem 5 (Ch. 1.9)}
{\it ff}
\begin{proof}[Solution]\let\qed\relax
	ff
\end{proof}

\subsection*{Problem 8 (Ch. 1.9)}
{\it ff}
\begin{proof}[Solution]\let\qed\relax
	ff
\end{proof}
\end{document}
