\documentclass{article}
\usepackage{amsmath, amsfonts, amsthm, amssymb}
\usepackage{geometry}
\geometry{letterpaper, margin=2.0cm, includefoot, footskip=30pt}

\usepackage{fancyhdr}
\pagestyle{fancy}

\lhead{Math 322}
\chead{Homework 7}
\rhead{Nicholas Rees, 11848363}
\cfoot{Page \thepage}

\newcommand{\N}{{\mathbb N}}
\newcommand{\Z}{{\mathbb Z}}
\newcommand{\Q}{{\mathbb Q}}
\newcommand{\R}{{\mathbb R}}
\newcommand{\C}{{\mathbb C}}
\newcommand{\ep}{{\varepsilon}}

\renewcommand{\theenumi}{(\alph{enumi})}

\begin{document}
\subsection*{Problem 10 (Ch. 1.8)}
{\it Let $G$ be a finite group, $A$ and $B$ non-vacuous subsets of $G$.
Show that $G = AB$ if $|A| + |B| > |G|$.}
\begin{proof}[Solution]\let\qed\relax
	Since inverses are determined uniquely in groups,
	there is a 1-1 correspondance between $A$ and $A^{-1}$,
	so $|A| = |A^{-1}|$.
	Furthermore, for any $g \in G$, note that $|A^{-1}g| = |A^{-1}| = |A|$,
	since cosets have the same order as the original set.
	But then $|A^{-1}g| + |B| > |G|$,
	and since $A^{-1}g$ and $B$, by definition are subgroups of $G$,
	they have a nonempty intersect
	(otherwise if the intersect were empty,
	there would be $|A^{-1}g| + |B|$ elements in $|G|$,
	but that's a contradiction).
	Thus there exists some $a \in A$ and $b \in B$ so that $a^{-1}g = b$,
	or $g = ab$.
	But this is true for all $g \in G$,
	thus $G \subseteq AB$.
	But $AB \subseteq G$ since $a,b \in G$ for any $a \in A$ and $b \in B$,
	and $ab \in G$ since $G$ is a group so its closed.
	Therefore, $AB = G$.
\end{proof}

\subsection*{Problem 11 (Ch. 1.8)}
{\it Let $G$ be a group of order $2k$ where $k$ is odd.
Show that $G$ contains a subgroup of index $2$.
(\emph{Hint:} Consider the permutation group $G_L$ of
left translations and use exercise 13, p.36.}
\begin{proof}[Solution]\let\qed\relax
	Recall from previous chapters that $G_L$ is a subgroup of $S_{2k}$,
	and $G_L \cong G$.
	Thus, it suffices to show that there is a subgroup
	of index $2$ inside $G_L$.

	First, since $G$ is of even order, by exercise 13,
	there exists some $a \in G$ such that $a \neq 1$ and $a^2 = 1$,
	so it is of order $2$.
	Consider the bijective map $a_L \in G_L$ given by
	$a_L \colon g \mapsto ag$
	for all elements $g \in G$
	(this is clearly a bijective map, since
	there is a well-defined inverse on $G$: $a^{-1}_L \colon g \mapsto a^{-1}g = ag$,
	so $a_L (a^{-1}_L (g)) = (a^{-1}_L \circ a_L) (g) = g$).
	Note $a_L = a^{-1}_L$.
	Thus, $a_L$ is just swapping two elements,
	thus we can represent it as composition of transpositions,
	ie. if we number our elements in $G$ correctly,
	we have $a_L  = (12)(34) \cdots (2k-1\;2k)$.
	Of note is that, since $k$ is odd, $a_L$ is an odd permutation.

	Now, define $H_L = A_{2k}\cap G_L$ where $A_{2k}$ is the
	group of even permutations of $S_{2k}$.
	Note that for any odd permutation $\alpha \in G_L$,
	we have $\alpha a_L \in H_L$,
	so $\alpha \in H_L \alpha_L^{-1} = H_L\alpha_L$.
	Thus, $H_L$ are all the even permutations in $G_L$,
	and since $\alpha$ was arbitrary, $H_La_l$ are all the
	odd permutations of $G_L$,
	thus $G_L = H_L \sqcup H_La_L$.
	But then $H_L$ is a subgroup of $G_L$ with index $2$,
	and since $G_L \cong G$,
	we have that there exists a subgruop of $G$ with index $2$.
\end{proof}

\subsection*{Problem 2 (Ch. 1.9)}
{\it Let $G$ be the set of triples of integers $(k,l,m)$ and define
$(k_1,l_1,m_1)(k_2,l_2,m_2) = (k_1 + k_2 + l_1 m_2, l_1 + l_2, m_1 + m_2)$.
Verify that this defines a group with unit $(0,0,0)$.
Show that $C = \{(k,0,0) \mid k \in \Z\}$ is a normal subgroup
and that $G/C \cong$ the group $\Z^{(2)}=\{(l,m)\mid l,m\in\Z\}$
with the usual addition as composition.}
\begin{proof}[Solution]\let\qed\relax
	Note that $G$ is closed,
	since if $k_1,k_2,l_1,l_2,m_1,m_2 \in \Z$,
	then $k_1 + k_2 + l_1m_2, l_1 + l_2,m_1 + m_2 \in \Z$,
	since $\Z$ is closed under addition and multiplication,
	so $(k_1,l_1,m_1)(k_2,l_2,m_2) \in G$.
	We have associativity:
	\begin{align*}
		((k_1,l_1,m_1)(k_2,l_2,m_2))(k_3,l_3,m_3)
		&= (k_1 + k_2 + l_1m_2, l_1 + l_2, m_1 + m_2)(k_3,l_3,m_3)\\
		&= (k_1 + k_2 + l_1m_2 + k_3 + (l_1 + l_2)m_3, l_1 + l_2 + l_3, m_1 + m_2 + m_3)\\
		&= (k_1 + k_2 + k_3  + l_2m_3 + l_1(m_2 + m_3), l_1 + l_2 + l_3, m_1 + m_2 + m_3)\\
		&= (k_1,l_1,m_1)(k_2 + k_3 + l_2m_3, l_2 + l_3, m_2 + m_3)\\
		&= (k_1, l_1, m_1)((k_2,l_2,m_2)(k_3,l_3,m_3))
	\end{align*}
	Furthermore, $(0,0,0)$ is the identity:
	$(k,l,m)(0,0,0) = (k + 0 + 0, l + 0, m + 0) = (k,l,m)$
	and $(0,0,0)(k,l,m) = (0 + k + 0, 0 + l, 0 + m) = (k,l,m)$.
	Finally, we have inverses:
	we give that the inverse of $(k,l,m) \in G$ is $(-k + lm,-l,-m)$.
	We can verify:
	$(k,l,m)(-k + lm,-l,-m) = (-k + lm,-l,-m)(k,l,m) = (0,0,0)$.

	We show that $C$ is in the kernel of some homomorphism $\phi$,
	and so by the fundamental theorem of homomorphisms,
	$C$ is normal.
	Furthermore, for the sake of efficiency,
	we will construct our homomorphism from $G$ to $\Z^{(2)}$,
	which also by the fundamental theorem,
	$G/C$ is isomorphic to $\Z^{(2)}$.
	We claim that $\phi \colon G \to \Z^{(2)}$ is defined by
	$(k,l,m) \mapsto (l,m)$.
	So what remains to show?
	All we need to show is that $\phi$ is a homomorphism,
	and that $C = \{(k,0,0) \mid k \in \Z\}$ is the kernel of $\phi$.
	$\phi$ is obviously a well-defined map to $\Z^{(2)}$,
	and
	\begin{align*}
		\phi((k_1,l_1,m_1)(k_2,l_2,m_2))
		&= \phi(k_1 + k_2 + l_1m_2, l_1 + l_2, m_1 + m_2)\\
		&= (l_1 + l_2, m_1 + m_2)\\
		&= (l_1,m_1) + (l_2,m_2)\\
		&= \phi(k_1,l_1,m_1) + \phi(k_2,l_2,m_2)
	\end{align*}
	So $\phi$ is a homomorphism.
	Now, $\ker{\phi} = C$.
	The identity of $\Z^{(2)}$ is $(0,0)$,
	and note that $\phi$ will map any element in $G$
	to $(0,0)$ if and only if the last two elements in $G$ are $0$.
	This is exactly $C = \{(k,0,0) \mid k \in \Z\}$,
	thus $C = \ker{\phi}$.
\end{proof}


\subsection*{Problem 4 (Ch. 1.9)}
{\it Determine $\mathrm{Aut}\,G$ for (i) $G$ an infinite cyclic group,
(ii) a cyclic group of order six, (iii) for any finite cyclic group.}
\begin{proof}[Solution]\let\qed\relax
	Consider an infinite cyclic group, $G = \langle g \rangle$.
	Since $G$ is infinite, there is are only two possible generators,
	namely $g$ and $g^{-1}$.
	In order for the image of our homomorphism to be equal to $\langle g\rangle$,
	we need the image to be equal to $\langle g \rangle = \langle g^{-1} \rangle$.
	By Theorem 1.7.
	we need only specify how a homomorphism acts on the generator for $G$
	to specify a homomorphism.
	The only homomorphisms that have a range of $G$
	is $\phi(g) = g$ and $\phi(g) = g^{-1}$.
	Thus, $\mathrm{Aut}\, G =$ the identity map, and the inverse map
	(mapping each element to its inverse).

	Now consider a cyclic group of order six,
	ie. $G = \langle g \rangle$ and $g^6 = 1_G$.
	Note that $\langle g^5 \rangle = G$ as well,
	thus $g^5$ is also a generator for $G$.
	Note that for all other elements $g' \in G\setminus\{g, g^5\}$,
	$\langle g' \rangle \neq G$.
	Thus, any automorphism must map all $g$ into $g$ or $g^5$.
	Thus $\mathrm{Aut}\, G = $ the identity map,
	and $\phi\colon g \to g^5$.

	Finally, consider an arbitrary finite cyclic group.
	Let $G = \langle g \rangle$ where $g$ is of order $n$.
	The potential generators of $G$ are all $a^k$ such that $(n,k) = 1$
	and $k \leq n$
	(as we proved in problem 4 of Jacobson 1.5).
	Thus $\mathrm{Aut}\, G = \{\phi \colon a \mapsto a^k \mid k \leq n, (n,k) = 1\}$.
\end{proof}


\subsection*{Problem 5 (Ch. 1.9)}
{\it Determine $\mathrm{Aut}\, S_3$.}
\begin{proof}[Solution]\let\qed\relax
	Note that $(12)$ and $(123)$ are generators of $S_3$.
	Define $a = (12)$ and $b = (123)$.
	Note that $S_3 = \langle a,b \mid a^2 = b^3 = 1, ab = b^2a\rangle$.
	Also, $((12),(123)),((13),(123)),((23),(123)),((12),(132)),((13),(132))$
	are generators of $S_3$.
	By theorem 1.7, we need only specify mapping the generators of $S_3$.
	But mapping our generators to each of these generators
	is a map from $S_3$ to $S_3$ (since both generate $S_3$).
	Thus any map from $(a,b)$ to one of our generators above is an automorphism,
	and since these are the only generators,
	it must be of this form.
	Thus
	\[
		\mathrm{Aut}\,S_3 = \{\phi(a,b) = (a,b), \phi(a,b) = (a,ba), \phi(a,b) = (a,ba^2), \phi(a,b) = (a^2,b), \phi(a,b) = (a^2,ba), \phi(a,b) = (a^2,ba^2)\}
	\]
\end{proof}

\subsection*{Problem 8 (Ch. 1.9)}
{\it Let $G$ be a group such that $\mathrm{Aut}\, G = 1$.
Show that $G$ is abelian and that every element of $G$ satisfies
the equation $x^2 = 1$.
Show that if $G$ is finite then $|G| = 1$ or $2$
(\emph{Hint:} Use the procedure of finding a base for a vector space
to show that $G$ contains elements $a_1,a_2,\dots,a_r$ such that
every element of $G$ can be written in one and only one way in the form
$a_1^{k_1}a_2^{k_2}\cdots a_r^{k_r}, k_i = 0,1$.
Then show that there exists an automorphism interchanging $a_1$ and $a_2$.)}
\begin{proof}[Solution]\let\qed\relax
	Recall from Jacobson problem 6 from 1.9:
	we have $I_a \in \mathrm{Aut}\, G$ where $I_a \colon x \mapsto axa^{-1}$,
	and if $\mathrm{Inn}\, G = \{I_a \mid a \in G\}$ then $\mathrm{Inn}\, G \cong G/C$
	(where $C$ is the center of $G$).
	So $\mathrm{Inn}\, G \subseteq \mathrm{Aut}\, G$.
	But since $\mathrm{Aut}\,G = 1$, $\mathrm{Inn}\, G$ only has one element,
	namely $I_{1_G}$.
	But since $\mathrm{Inn}\, G \cong G/C$, $|G/C| = 1$ as well.
	Since $G/C$ are the cosets of $C$ in $G$,
	and there is only one possible coset,
	this must mean $C = G$.
	But if the center of $G$ is the entire group,
	we know that $G$ is abelian.

	Further, from Jacobson problem 3 from 1.9,
	which states that $x \to x^{-1}$ is an automorphism of $G$
	if and only if $G$ is abelian,
	we have that the inverse map is an automorphism of $G$.
	But since $\mathrm{Aut}\, G = 1$,
	$x \to x^{-1}$ must also be the identity map,
	thus each element in $G$ is its own inverse.
	Thus $x^2 = 1$ for all $x \in G$.

	Finally, let $G$ be finite.
	Then we have a finite set of generators for $G$,
	$a_1, a_2,\cdots,a_r$.
	Thus, every element $g \in G$ can be written as a finite string
	$g = a_{i_1}^{\alpha_{i_1}}a_{i_2}^{\alpha_{i_2}}\cdots a_{i_j}^{\alpha_{i_j}}$,
	but since the group is abelian, we can rearrange it the $a$ to get
	$g = a_1^{k_1}a_2^{k_2}\cdots a_r^{k_r}$,
	and since $a^2 = 1$ for any $a$,
	we have that $k_i = 0,1$ for all $1 \leq i \leq r$.
	To show that the $k_i$ are unique (namely $k_i = 0$),
	we have $g = a_1^{k'_1}a_2^{k'_2}\cdots a_r^{k'_r}$
	where $k'_i = 0,1$ for all $1 \leq i \leq r$ as well.
	Then $a_1^{k_1}a_2^{k_2}\cdots a_r^{k_r} = a_1^{k'_1}a_2^{k'_2}\cdots a_r^{k'_r}$,
	which implies $a_1^{k_1 - k'_1}a_2^{k_2-k'_2}\cdots a^{k_r - k'_r} = 1$.
	If $k_i - k'_i = 0$, we are done.
	Otherwise, there exists some $a_j^{k_j - k'_j}$
	such that $k_j - k'_j \in \{-1,1\}$.
	Then $1 = a_j^{\pm 1}a_1^{k_1 - k'_1}\cdots $
	so $a_j^{\mp 1} = a_1^{k_1 - k'_1}\cdots$,
	so the right hand side generates the element on the left.
	Repeat this process without $a_j$,
	which we can do since there is a finite set of them.
	Eventually, we reach some set of generators $a_1,\dots,a_k$
	such that there is no $a_j$,
	otherwise there are no generators and $G = 1$ and we are done anyway.
	If there is some set,
	we have that the only representation of $1$ with the generators
	is with $1 = a_1^0\cdots a_q^0$.
	Then $g = a_1^{k_1}\cdots a_q^{k_q} = a_1^{k'_1}\cdots a_q^{k'_q}$
	implies that $k_i = k'_i$ for all $1 \leq i \leq q$,
	thus the representation of $g$ by these generators are unique.

	Consider the map $\phi \colon G \to G$
	that swaps $a_1 \to a_j$ for some $1 \leq j \leq k$.
	This map is well-defined,
	since the representation of an element in $g$ is uniquely determined,
	as we just proved,
	and so $g$ will always get mapped to the same element.
	We claim that this is a automorphism.
	First, we show that it is a homomorphism:
	\begin{align*}
		\phi(a_1^{k_1}\cdots a_q^{k_q})\phi(a_1^{k'_1}\cdots a_q^{k'_q})
		&= (a_j^{k_1}\cdots a_1^{k_j}\cdots a_q^{k_q})(a_j^{k'_1}\cdots a_1^{k'_j}\cdots a_q^{k'_q})\\
		&= a_j^{k_1 + k'_1}\cdots a_1^{k_j + k'_j}\cdots a_q^{k_q + k'_q} \\
		&= \phi(a_1^{k_1 + k'_1}\cdots a_q^{k_q + k'_q})\\
		&= \phi((a_1^{k_1}\cdots a_q^{k_q})(a_1^{k'_1}\cdots a_q^{k'_q}))
	\end{align*}
	Now, we show that it is a bijection to show that it is an isomorphism.
	But we have a well-defined inverse,
	namely itself, since swapping $a_1$ with $a_j$
	and swapping them again is just the identity map
	(and it is well-defined since $\phi$ is well-defined).
	Thus, $\phi$ is an isomorphism.
	Finally, $\phi \in \mathrm{Aut}\, G$,
	since any bijection on a finite set (and $G$ is finite)
	to itself must map each element in $G$ to another element in $G$.
	Now recall that $\mathrm{Aut}\,G = 1$,
	thus swapping $a_1$ with an arbitrary $a_j$
	keeps $g \in G$ the same,
	thus $a_1 = a_2 =\cdots = a_q$.
	But then $G$ is a group with only one generator,
	or $G = \langle a_1 \rangle$.
	But $a_1^2 = 1$,
	thus $G = \{1,a_1\}$, so $|G| = 2$.
	Recall earlier that we could also have $G = 1$ (if there are no generators),
	so we can also have $|G| = 1$, as desired.
\end{proof}
\end{document}
