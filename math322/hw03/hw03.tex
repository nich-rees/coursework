\documentclass{article}
\usepackage{amsmath, amsfonts, amsthm, amssymb}
\usepackage{geometry}
\geometry{letterpaper, margin=2.0cm, includefoot, footskip=30pt}

\usepackage{fancyhdr}
\pagestyle{fancy}

\lhead{Math 322}
\chead{Homework 3}
\rhead{Nicholas Rees, 11848363}
\cfoot{Page \thepage}

\newcommand{\N}{{\mathbb N}}
\newcommand{\Z}{{\mathbb Z}}
\newcommand{\Q}{{\mathbb Q}}
\newcommand{\R}{{\mathbb R}}
\newcommand{\C}{{\mathbb C}}
\newcommand{\ep}{{\varepsilon}}

\renewcommand{\theenumi}{(\alph{enumi})}

\begin{document}
\subsection*{Problem 2 (Ch. 1.3)}
{\it Show that the two groups given in examples 11 and 13 on pages 33 and 34 are isomorphic.
Obtain a subgroup of $S_n$ isomorphic to these groups.}

\begin{proof}[Solution]\let\qed\relax
	We provide the bijection $\phi \colon R_n \to U_n$ below.
	If $\tau = \frac{2j\pi}{n} \in R_n$ where $j \in \Z$, $0 \leq j < n$,
	then
	\[
		\phi(\tau) = e^{i2j\pi/n}
	\]
	If $\tau_1 = \frac{2j_1\pi}{n}$, $\tau_2 = \frac{2j_2\pi}{n}$,
	(so $0 \leq j_1,j_2<n$)
	and $\phi(\tau_1) = \phi(\tau_2)$,
	in other words $e^{i2j_1\pi/n} = e^{i2j_2\pi/n}$,
	then $\tau_1 = \tau_2$ (since
	if $\theta \in [0,2\pi)$, $e^{i\theta}$ is unique so
	we must have $\theta_1 = \theta_2 \implies i2j_1\pi/n =
	i2j_2\pi/n \implies \tau_1 = \tau_2$).
	So $\phi$ is injective.
	Now let $y \in U_n$ where $y = e^{i2j\pi/n}$ for $0 \leq j < n$,
	we have that $\phi(\frac{i2j\pi}{n}) = y$,
	so $\phi$ is surjective,
	therefore $\phi$ is bijective.

	Now we show that $\phi$ respects the group operations.
	Let $\tau_1 = \frac{2j_1\pi}{n}$, $\tau_2 = \frac{2j_2\pi}{n}$
	($0 \leq j_1,j_2<n$).
	We have
	\begin{align*}
		\phi(\tau_1\tau_2)
		&= \phi(\frac{2j_1\pi}{n} + \frac{2j_2\pi}{n})\\
		&= \phi(\frac{2(j_1 + j_2)\pi}{n})\\
		&= e^{i2(j_1 + j_2)\pi/n}\\
		&= e^{i2j_1\pi/n}e^{i2j_2\pi/n}\\
		&= \phi(\tau_1)\phi(\tau_2)
	\end{align*}
	Thus, $R_n \cong U_n$.
\end{proof}

\subsection*{Problem 4 (Ch. 1.3)}
{\it Is the additive group of integers isomorphic to the additive group of rationals
(examples 1 and 2 on p. 32)?}

\begin{proof}[Solution]\let\qed\relax
	For the sake of contradiction,
	assume there exists a bijection $\phi \colon (\Z,+) \to (\Q,+)$
	that preserves the group operations (it's isomorphic).
	Note that there exists an element $n \in (\Z,+)$ such that
	for all $z \in \Z$, there exists $m \in \Z$ where $z = n^{m}$.
	For concreteness, we know that $n = 1$.
	Thus, for all $q \in \Z$, we must have a $m \in \Z$ such that $q = \phi(n)^m$,
	otherwise we have $z \in \Z$ such that $q = \phi(z)$ by surjectivity of $\phi$,
	but $q \neq \phi(\underbrace{n + n + \cdots + n}_{m\text{ times}})
	= \underbrace{\phi(n) + \phi(n) + \cdots + \phi(n)}_{m\text{ times}} = \phi(z) = q$
	which is a contradiction.
	Let $\frac{p}{r} = \phi(n)$ where $p \in \Z\setminus\{0\}, r \in \N\setminus\{0\}$
	and $\gcd(p,r) = 1$
	(we have excluded $\phi(n) \neq 0$ by saying $p \neq 0$,
	because $\underbrace{0 + 0 + \cdots + 0}_{m\text{ times}} = 0$
	and so will never have the desired property
	since $\{0\}\neq\Q$).
	But for all $m \in \N$, $\underbrace{\frac{p}{r} + \frac{p}{r} +
	\cdots \frac{p}{r}}_{m\text{ times}} = \frac{pm}{r} \neq \frac{1}{r+1} \in \Q$,
	since if we did have $m$ where $\frac{pm}{r} = \frac{1}{r+1}$,
	then we would have $pm(r+1) = r$,
	but if $p < 0$ since $m,r+1 \geq 0$ then $pm(r+1) \leq 0 < r$ and so we can't have equality,
	and if $p > 0$ (and $m\neq 0$, otherwise $pm(r+1) = 0 < r$)
	then the left side of the equality is $r+1$ times positive constants,
	so $pm(r+1) \geq r+1 > r$, so we can't have equality again.
	But then we have $q \in \Q$ that is not $\phi(n)^m$,
	which is a contradiction,
	thus $\phi$ is not an isomorphism.
\end{proof}

\subsection*{Problem 5 (Ch. 1.3)}
{\it Is the additive group of rationals isomorphic to the multiplicative group
of non-zero rationals (examples 2 and 5 on p. 32)?}

\begin{proof}[Solution]\let\qed\relax
	For the sake of contradiction, assume that $\phi \colon (\Q,\times) \to (\Q, +)$
	that preserves the group operation (it's isomorphic).
	Then, we have $\phi(4) = \phi(2\times2) = \phi(2) + \phi(2)$.
	We also hae $\phi(4) = \phi(-2\times-2) = \phi(-2) + \phi(-2)$.
	So $\phi(2) + \phi(2) = \phi(-2) + \phi(-2)$.
	But $2 \neq -2$ so $\phi(2) \neq \phi(-2)$ since $\phi$ is injective.
	Now for $q_1, q_2 \in \Q$,
	if $q_1 \neq q_2$ then we know that $2q_1 \neq 2q_2$
	(a property we know of the rationals).
	But then $\phi(4) = \phi(2) + \phi(2) \neq \phi(-2) + \phi(-2) = \phi(4)$,
	but then $\phi$ maps $4$ to more than one value in $\Q$,
	thus $\phi$ is not a function, a contradiction.
\end{proof}

\subsection*{Problem 1 (Ch. 1.5)}
{\it As in section 1.4, let $C(A)$ denote the centralizer of the subset $A$
of a monoid $M$ (or a group $G$).
Note that $C(C(A)) \supset A$ and if $A \subset B$ then $C(A) \supset C(B)$.
Show that these imply that $C(C(C(A))) = C(A)$.
Without using the explicit form of the elements of $\langle A \rangle$
show that $C(A) = C(\langle A \rangle)$.
(\emph{Hint:} Note that if $c \in C(A)$ then $A \subset C(c)$
and hence $\langle A \rangle \subset C(c)$.)
Use the last result to show that if a monoid (or a group) is generated
by a set of elements $A$ which pair-wise commute,
then the monoid (group) is commutative.}

\begin{proof}[Solution]\let\qed\relax
	Let $A$ be an arbitrary subset of a monoid $M$ (or group $G$).
	Since $A \subset C(C(A))$,
	then we know that $C(A) \supset C(C(C(A)))$ by the second given property.
	Now let $B = C(A)$, then plugging $B$ into the first given property,
	we have $C(C(B)) \supset B$.
	But substituting $C(A)$ back in for $B$, we get
	$C(C(C(A))) \supset C(A)$.
	Thus, we have set inclusion in both ways, so $C(C(C(A))) = C(A)$.

	Note that by definition, $\langle A \rangle$ is the smallest submonoid of $M$
	(or subgroup of $G$)
	that contains $A$, thus $\langle A \rangle \supset A$.
	From the hint, we know that $\langle A \rangle \subset C(c)$
	where $c \in C(A)$,
	but then applying the second property we have $C(\langle A \rangle) \supset C(C(c))$.
	But we know that $c \in C(C(c))$, since if $c$ commutes with element
	$m \in M$,
	then $m \in C(c)$ and $c \in C(m)$.
	But this is for all $m \in C(c)$, thus $c \in C(C(c))$.
	And since since $C(\langle A \rangle \supset C(C(c))$,
	we have that $c \in C(\langle A \rangle)$.
	But since $c$ was an arbitrary element in $C(A)$,
	we have that $C(A) \subset C(\langle A \rangle)$.
	Thus $C(\langle A \rangle) = C(A)$.

	Now let $M$ be generated by a set of elements $A$
	which pair-wise commute.
	Then $M = \langle A \rangle$ and $C(A) \supset A$.
	We are seeking to prove that $M = C(M)$.
	Since $M$ and $\langle A \rangle$ are equal,
	we have that their centralizer is the same,
	so $C(M) = C(\langle A \rangle)$.
	But recall we jsut proved $C(\langle A \rangle) = C(A)$,
	thus $C(M) = C(A)$.
	But $C(A)$ is a submonoid that contains $A$,
	and $\langle A \rangle$ is, by definition,
	the smallest submonoid that contains $A$
	and is contained in all submonoids that contain $A$,
	thus $\langle A \rangle \subset C(A) = C(M)$.
	But then $M \subset C(M)$.
	But note that we also have $M \supset C(M)$,
	since $C(M)$ is a submonoid of $M$,
	thus we have proved $M = C(M)$.
\end{proof}

\subsection*{Problem 3 (Ch. 1.5)}
{\it Let $G$ be an abelian group with a finite set of generators which is
\emph{periodic} in the sense that all of its elements have finite order.
Show that $G$ is finite.}

\begin{proof}[Solution]\let\qed\relax
	Let $g_1,g_2,\dots,g_k$ be the generators
	with orders $o_1,o_2,\dots,o_k$ respectively.
	If $g \in G$, then $g = g_1^{e_1}g_2^{e_2}\cdots g_k^{e_k}$
	where $0 \leq e_1 < o_1$, $o \leq e_2 < o_2$ etc.
	Note that we do not care about the permutations of $g_1,g_2,\dots,g_k$,
	because the set is abelian and so changing the order does not change the value.
	Thus, the number of possible elements in $g$ is the different
	combinations of $e_1,e_2,\dots,e_k$, which is a finite value.
	Thus, $G$ contains finite number of elements, so $G$ is finite.
\end{proof}

\subsection*{Problem 4 (Ch. 1.5)}
{\it Show that if $g$ is an element of a group and $o(g) = n$
then $g^k,k\neq0$, has order $[n,k]/k = n/(n,k)$.
Show that the number of generators of $\langle g \rangle$ is the number
of positive integers $<n$ which are relatively prime to $n$.
This number is denoted as $\phi(n)$ and $\phi$ is called the \emph{Euler $\phi$-function}.}

\begin{proof}[Solution]\let\qed\relax
	We assert that $[n,k]/k$ is the order of $g^k$.
	First, we have $(g^k)^{[n,k]/k} = g^{[n,k]} = (g^n)^j = 1^j = 1$
	(where $j \in \N\setminus\{0\}$).
	Now we show that this is the smallest value.
	Assume that $e \in\N\setminus\{0\}$ where $e < [n,k]/k$
	and $(g^k)^e = g^{ke} = 1$.
	But then we must have $ke$ be some multiple of $n$, say $nm$ ($m\in\N\setminus\{0\}$),
	however, $nm = ke < [n,k]$,
	which contradicts that $[n,k]$ is the lowest common multiple of $n$ and $k$.
	Thus, we have $[n,k]/k$ is the order,
	and since $kn = [n,k](n,k)$, thus $n/(n,k) = [n,k]/k$.

	We know $g^k$ is a generator of $\langle g \rangle$ if and only if $o(\langle g^k \rangle) = n$.
	We just showed that $o(g^k) = \frac{n}{(n,k)}$ so we require that $(n,k) = 1$,
	thus they must be coprime.
\end{proof}
\end{document}
