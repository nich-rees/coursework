\documentclass{article}
\usepackage{amsmath, amsfonts, amsthm, amssymb}
\usepackage{geometry}
\geometry{letterpaper, margin=2.0cm, includefoot, footskip=30pt}

\usepackage{fancyhdr}
\pagestyle{fancy}

\lhead{Math 322}
\chead{Homework 2}
\rhead{Nicholas Rees, 11848363}
\cfoot{Page \thepage}

\newcommand{\N}{{\mathbb N}}
\newcommand{\Z}{{\mathbb Z}}
\newcommand{\Q}{{\mathbb Q}}
\newcommand{\R}{{\mathbb R}}
\newcommand{\C}{{\mathbb C}}
\newcommand{\ep}{{\varepsilon}}

\renewcommand{\theenumi}{(\alph{enumi})}

\begin{document}
\subsection*{Problem 1 (Ch. 1.2)}
{\it Determine $\alpha \beta$, $\beta \alpha$, and $\alpha^{-1}$ in $S_5$ if
\[
	\alpha = \begin{pmatrix} 1 & 2 & 3 & 4 & 5 \\ 2 & 3 & 1 & 5 & 4 \end{pmatrix},
	\; \beta = \begin{pmatrix} 1 & 2 & 3 & 4 & 5 \\ 1 & 3 & 4 & 5 & 2 \end{pmatrix}
\]}

\begin{proof}[Solution]\let\qed\relax
	We can compose our two permutations to get:
	\[
		\alpha \beta = 
		\begin{pmatrix} 1 & 2 & 3 & 4 & 5 \\ 2 & 3 & 1 & 5 & 4 \end{pmatrix}
		\begin{pmatrix} 1 & 2 & 3 & 4 & 5 \\ 1 & 3 & 4 & 5 & 2 \end{pmatrix}
		= \begin{pmatrix} 1 & 2 & 3 & 4 & 5 \\ 3 & 4 & 1 & 2 & 5 \end{pmatrix}
	\]
	\[
		\beta\alpha = 
		\begin{pmatrix} 1 & 2 & 3 & 4 & 5 \\ 2 & 3 & 1 & 5 & 4 \end{pmatrix}
		\begin{pmatrix} 1 & 2 & 3 & 4 & 5 \\ 1 & 3 & 4 & 5 & 2 \end{pmatrix}
		= \begin{pmatrix} 1 & 2 & 3 & 4 & 5 \\ 3 & 4 & 1 & 2 & 5 \end{pmatrix}
	\]
	To get our inverse, we will just reverse the permutation of $\alpha$:
	\[
		\alpha^{-1} = 
		\begin{pmatrix} 1 & 2 & 3 & 4 & 5 \\ 3 & 1 & 2 & 5 & 4 \end{pmatrix}
	\]
	We can confirm that this is an inverse by confirming that
	\[
		\alpha \alpha^{-1} =
		\begin{pmatrix} 1 & 2 & 3 & 4 & 5 \\ 2 & 3 & 1 & 5 & 4 \end{pmatrix}
		\begin{pmatrix} 1 & 2 & 3 & 4 & 5 \\ 3 & 1 & 2 & 5 & 4 \end{pmatrix}
		= \begin{pmatrix} 1 & 2 & 3 & 4 & 5 \\ 1 & 2 & 3 & 4 & 5 \end{pmatrix}
		= 1_{S_5} 
	\]
	\[
		\alpha^{-1} \alpha
		\begin{pmatrix} 1 & 2 & 3 & 4 & 5 \\ 3 & 1 & 2 & 5 & 4 \end{pmatrix}
		\begin{pmatrix} 1 & 2 & 3 & 4 & 5 \\ 2 & 3 & 1 & 5 & 4 \end{pmatrix}
		= \begin{pmatrix} 1 & 2 & 3 & 4 & 5 \\ 1 & 2 & 3 & 4 & 5 \end{pmatrix}
		= 1_{S_5} 
	\]
\end{proof}

\subsection*{Problem 4 (Ch. 1.2)}
{\it Let $G$ be the set of pairs of real numbers $(a,b)$ with $a \neq 0$ and define:
$(a,b)(c,d) = (ac, ad + b), 1 = (1,0)$.
Verify that this defines a group.}

\begin{proof}[Solution]\let\qed\relax
	Let $(a,b), (c,d), (e,f) \in G$.
	Note that the group is indeed closed under the multiplication:
	since $G$ is the set of all pairs of real numbers,
	and the reals are closed under addition and multiplication,
	the resulting product is a pair of real numbers as well.

	Now we verify that the multiplication is associative.
	See
	\begin{align*}
		\big((a,b)(c,d)\big)(e,f)
		&= (ac, ad + b)(e,f)\\
		&= (ace, acf + ad + b)\\
		&= (ace, a(cf + d) + b)\\
		&= (a,b)(ce, cf + d)\\
		&= (a,b)\big((c,d)(e,f)\big)
	\end{align*}
	where we use the distributive property of the reals in the third line.
	
	Next, we show that $1 = (1,0)$ is an inverse. See
	\[
		1(a,b) = (1,0)(a,b) = (a,b)
	\]
	\[
		(a,b)1 = (a,b)(1,0) = (a,b)
	\]
	
	Finally, we confirm that every element has an inverse.
	We do this by providing it explicitly for the arbitrary element $(a,b)$,
	namely $(a,b)^{-1} = (\frac{1}{a}, \frac{-b}{a})$. See
	\[
		(a,b)(a,b)^{-1} = (a,b)\left(\frac{1}{a}, \frac{-b}{a}\right) = (1,0)
	\]
	\[
		(a,b)^{-1}(a,b) = \left(\frac{1}{a}, \frac{-b}{a}\right) = (1,0)
	\]
	This is sufficient to show $G$ with the given multiplication is a group.
\end{proof}

\subsection*{Problem 7 (Ch. 1.2)}
{\it Show that if an element $a$ of a monoid has a right inverse $b$,
that is, $ab = 1$;
and a left inverse $c$, that is, $ca = 1$;
then $b = c$, and $a$ is invertible with $a^{-1} = b$.
Show that $a$ is invertible with $b$ as inverse if and only if $aba = a$ and $ab^2a=1$.}

\begin{proof}[Solution]\let\qed\relax
	We have that $(ca)b = c(ab)$ by associativity,
	but then $1\cdot b = c \cdot 1 \implies b = c$.
	Then we have an element such that $ab = ba = 1$
	(we replaced $b$ with $c$ since they are equal),
	which satisifies the condition that $a$ is invertible with inverse $a^{-1} = b$.

	Now let $a$ be invertible with $b$ as an inverse.
	Then $ba = 1$ and so $a(ba) = a\cdot 1 \implies aba = a$.
	Additionally, we have $ba = 1 \implies 1\cdot ba = 1 \cdot 1$
	and since $ab = 1$, we get $(ab)ba = 1 \implies ab^2a = 1$, as desired.
	To show the other direction, assume $aba = a$ and $ab^2a = 1$.
	The second equation gives that $ba$ is a right inverse of $ab$,
	but we just proved that this implies $ba$ is also a left inverse,
	so we have $baab = 1$.
	We can multiply both sides of the equation by $a$ to get $baaba = a$,
	and subbing in with $aba = a$, we have $baa = a$.
	Subbing back into $baab = 1$, we have $ab = 1$.
	So $b$ is a right inverse of $a$.
	Note that $a$ also has a left inverse, namely $ab^2$ (from $ab^2a = 1$),
	and existence is enough to invoke what we proved in the first part of this problem,
	specifically that $a$ is invertible with $b$ as inverse,
	and so we have proven both directions.
\end{proof}

\subsection*{Problem 8 (Ch. 1.2)}
{\it Let $\alpha$ be a rotation about the origin in the plane
and let $\rho$ be the reflection in the $x$-axis.
Show that $\rho \alpha \rho^{-1} = \alpha^{-1}$.}

\begin{proof}[Solution]\let\qed\relax
	We can represent our elements as coordinates in the plane $(x,y)$.
	Then $\alpha$ is $(x,y) \to (x\cos\theta - y\sin\theta, x\sin\theta + y\cos\theta)$
	where $\theta$ is the angle swept by our arbitrary rotation,
	and $\rho$ is $(x,y) \to (x,-y)$.
	Note then that $\rho^{-1} = \rho$, since one can see $\rho\rho = 1$.
	Note that $\alpha^{-1}$ is just a rotation by $\theta$ in the opposite direction,
	or $-\theta$, so $\alpha^{-1} = (x,y) \to (x\cos(-\theta) - y\sin(-\theta),
	x\sin(-\theta) + y\cos(-\theta))$.
	See that 
	\begin{align*}
		\rho \alpha \rho^{-1}(x,y)
		&= \rho(x\cos\theta + y\sin\theta, x\sin\theta - y \cos\theta)\\
		&= (x\cos\theta + y\sin\theta, -x\sin\theta + y \cos\theta)\\
		&= (x\cos(-\theta) - y\sin(-\theta), x\sin(-\theta) + y\cos(-\theta)\\
		&= \alpha^{-1}(x,y)
	\end{align*}
	where we used the evenness of $\cos$ and the oddness of $\sin$ in the third line.
	Thus, since $x,y$ were arbitrary points in the plane,
	we have $\rho\alpha\rho^{-1} = \alpha^{-1}$.
\end{proof}

\subsection*{Problem 11 (Ch. 1.2)}
{\it Show that in a group, the equation $ax = b$ and $ya = b$
are solvable for any $a,b \in G$.
Conversely, show that any semigroup having this property contains a unit and is a group.}

\begin{proof}[Solution]\let\qed\relax
	We want to show that for all $a,b \in G$ fixed,
	there exists $x,y \in G$ such that $ax = b$ and $ya = b$ (ie. its solvable).
	Note that since $G$ is a group, inverses of all elements exist,
	and so if we multiply the first equation on the left by $a^{-1}$,
	we get $a^{-1}ax = a^{-1} b \implies x = a^{-1}b$.
	Since $a^{-1},b \in G$ and a group is closed under its operations,
	$x \in G$ as well.
	We can repeat this with our second equation to get $y = ba^{-1}$,
	so $y \in G$.
	Thus there exists $x,y \in G$ such that our equalities hold,
	namely $x = a^{-1}b$ and $y = ba^{-1}$.

	Conversely, let us assume $S$ is a semigroup such that for all $a,b \in S$,
	there exists $x,y \in S$ such that $ax = b$ and $ya = b$.
	We first prove that $S$ contains a unit.
	Note that our assumptions gaurantees existence of $x,y$
	with the restriction of our equations where $a = b$.
	Then we know there exists $r,l \in S$ such that $ar = la = a$.
	We prove that $sr = s$ for all $s \in S$ now.
	We know that there exists $y \in S$ such that $s = ya$.
	Then $s = y(ar) \implies s = (ya)r \implies s = sr$ as desired.
	We prove the same for $l$,
	that is $ls = s$ for all $s \in S$.
	We know that there exists $x \in S$ such that $s = ax$.
	Then $s = (la)x \implies s = l(ax) \implies s = ls$ as desired.
	Now since we have $lr = r$ and $lr = l$ we have $l = r$.

	Denote our identity $1_S$ (so $r = l = 1_S$).
	Now we use the assumption again with arbitrary $a \in S$ and $b = 1_S$.
	So we have that there exists $x,y \in S$ such that $ax = 1_S$ and $ya = 1_S$.
	But since $S$ is a monoid (a semi-group with a unit),
	we proved in Problem 7 that since $a$ has a left and right inverse,
	$x = y$ and $a$ is invertible with $a^{-1} = x$.
	Thus, $S$ is associative and closed under its operations,
	has a unit,
	and each element has an inverse.
	This is sufficient to show that $S$ is a group.
\end{proof}

\subsection*{Problem 13 (Ch. 1.2)}
{\it Show that any finite group of even order contains an element $a \neq 1$
such that $a^2 = 1$.}

\begin{proof}[Solution]\let\qed\relax
	We prove the contrapositive, that is,
	given a finite group $G$,
	if for all $a \neq 1$, $a^2 \neq 1$,
	then $G$ is not of even order, that is, of odd order.
	Note that given any $a \in G$ where $a \neq 1$,
	we have a corresponding unique $a^{-1}\in G$
	such that $a \neq a^{-1}$ (else $a^2 = 1$).
	Note that $a$ is the inverse of $a^{-1}$ as well,
	and so this pairs up elements together.
	The only element that is not paired up with another distinct element is $1$,
	since $1$ is its own inverse ($1\cdot 1 = 1$).
	Thus we have some number of pairs of elements,
	and the identity, all distinct,
	thus $G$ is of odd order, as desired.
\end{proof}
\end{document}
