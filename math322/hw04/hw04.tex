\documentclass{article}
\usepackage{amsmath, amsfonts, amsthm, amssymb}
\usepackage{geometry}
\geometry{letterpaper, margin=2.0cm, includefoot, footskip=30pt}

\usepackage{fancyhdr}
\pagestyle{fancy}

\lhead{Math 322}
\chead{Homework 4}
\rhead{Nicholas Rees, 11848363}
\cfoot{Page \thepage}

\newcommand{\N}{{\mathbb N}}
\newcommand{\Z}{{\mathbb Z}}
\newcommand{\Q}{{\mathbb Q}}
\newcommand{\R}{{\mathbb R}}
\newcommand{\C}{{\mathbb C}}
\newcommand{\ep}{{\varepsilon}}

\renewcommand{\theenumi}{(\alph{enumi})}

\begin{document}
\subsection*{Problem 2 (Ch. 1.5)}
{\it Let $M$ be a monoid generated by a set $S$ and suppose every element of $S$ is invertible.
Show that $M$ is a group.}
\begin{proof}[Solution]\let\qed\relax
	Since a group is just a monoid where every element is invertible,
	all that is required is to show that for all $m \in M$,
	there exists $m^{-1} \in M$ such that $mm^{-1} = m^{-1}m = 1_M$.
	Since $M$ is generated by $S = \{s_1, s_2, \dots, s_n\}$,
	then any $m \in M$ can be represented as
	\[
		m = s_{j_1}^{e_{j_1}}s_{j_2}^{e_{j_2}}\cdots s_{j_n}^{e_{j_n}}
	\]
	where $j_i \in \{1,2,\dots,n\}$ is some index on the generators of $M$.
	Since every $s_i$ is invertible, $s_i^{-1} = M$,
	and by closure, $(s_i^{-1})^{e_i} = s^{-e_i} \in M$ as well.
	But then
	\[
		m' = s_n^{-e_n}s_{n-1}^{-e_{n-1}}\cdots s_1^{-e_1}\in M
	\]
	as well.
	We have that
	\begin{align*}
		mm'
		&= s_{j_1}^{e_{j_1}}s_{j_2}^{e_{j_2}}\cdots s_{j_{n-1}}^{e_{j_{n-1}}}s_{j_n}^{e_{j_n}}
		s_{j_n}^{-e_{j_n}}s_{j_{n-1}}^{-e_{j_{n-1}}}s_{j_n}\cdots s_{j_1}^{-e_{j_1}}\\
		&= s_{j_1}^{e_{j_1}}s_{j_2}^{e_{j_2}}\cdots s_{j_{n-1}}^{e_{j_{n-1}}}(s_{j_n}^{e_{j_n}}
		s_{j_n}^{-e_{j_n}})s_{j_{n-1}}^{-e_{j_{n-1}}}s_{j_n}\cdots s_{j_1}^{-e_{j_1}}\\
		&= s_{j_1}^{e_{j_1}}s_{j_2}^{e_{j_2}}\cdots s_{j_{n-1}}^{e_{j_{n-1}}}1_M
		s_{j_{n-1}}^{-e_{j_{n-1}}}s_{j_n}\cdots s_{j_1}^{-e_{j_1}}\\
		&= s_{j_1}^{e_{j_1}}s_{j_2}^{e_{j_2}}\cdots s_{j_{n-1}}^{e_{j_{n-1}}}
		s_{j_{n-1}}^{-e_{j_{n-1}}}s_{j_n}\cdots s_{j_1}^{-e_{j_1}}\\
		& \qquad \vdots\\
		&= s_{j_1}^{e_{j_1}}s_{j_1}^{-e_{j_1}}\\
		&= 1_M
	\end{align*}
	where we can introduce the parantheses on line 2 since monoids are associative.
	Likewise
	\begin{align*}
		m'm
		&= s_{j_n}^{-e_{j_n}}s_{j_{n-1}}^{-e_{j_{n-1}}}\cdots s_{j_2}^{-e_{j_2}}s_{j_1}^{-e_{j_1}}
		s_{j_1}^{e_{j_1}}s_{j_2}^{e_{j_2}}\cdots s_{j_n}^{e_{j_n}}\\
		&= s_{j_n}^{-e_{j_n}}s_{j_{n-1}}^{-e_{j_{n-1}}}\cdots s_{j_2}^{-e_{j_2}}(s_{j_1}^{-e_{j_1}}
		s_{j_1}^{e_{j_1}})s_{j_2}^{e_{j_2}}\cdots s_{j_n}^{e_{j_n}}\\
		&= s_{j_n}^{-e_{j_n}}s_{j_{n-1}}^{-e_{j_{n-1}}}\cdots s_{j_2}^{-e_{j_2}}1_M
		s_{j_2}^{e_{j_2}}\cdots s_{j_n}^{e_{j_n}}\\
		&= s_{j_n}^{-e_{j_n}}s_{j_{n-1}}^{-e_{j_{n-1}}}\cdots s_{j_2}^{-e_{j_2}}
		s_{j_2}^{e_{j_2}}\cdots s_{j_n}^{e_{j_n}}\\
		& \qquad \vdots\\
		&= s_{j_n}^{-e_{j_n}}s_{j_n}^{e_{j_n}}\\
		&= 1_M
	\end{align*}
	Thus, $m' = m^{-1}$,
	and since $m$ was arbitrary, every element in $M$ is invertible,
	making $M$ a group.
\end{proof}

\subsection*{Problem 5 (Ch. 1.5)}
{\it Show that any finitely generated subgroup of the additive group
of rationals $(\Q,+,0)$ is cyclic.
Use this to prove that this group is not isomorphic to
the direct product of two copies of it.}
\begin{proof}[Solution]\let\qed\relax
	Let $S$ be an arbitrary finitely generated subgroup of the additive rationals.
	That is, $\langle \frac{p_1}{d_1},\frac{p_2}{d_2},\dots,\frac{p_n}{q_n}\rangle = S$, where the fractions are simplified (ie. $(p_i,d_i) = 1$).
	Let $q = \frac{1}{d_1d_2\cdots d_n}$.
	We claim that $S$ is a subgroup of $\langle q \rangle$.
	We already know that $S$ is a group, since it was generated by a set,
	but we need to show that the collection of elements in $S$ is a subset of $\langle q \rangle$.
	Let $s \in S$.
	Then $s = j_1\frac{p_1}{q_1} + j_2\frac{p_2}{q_2} + \cdots + j_n\frac{p_n}{q_n}$
	(since $(\Q,+,0)$ is abelian, we can ignore when the $\frac{p_i}{d_i}$
	is added after $\frac{p_k}{d_k}$ where $i<k$,
	since we can just move $\frac{p_i}{d_i}$ in front).
	Note
	\[
		s = \frac{(j_1 p_1 d_2d_3\cdots d_n) + (j_2 p_2 d_1d_3\cdots d_n)
		+ \cdots + (j_np_n d_1d_2 \cdots d_{n-1})}{d_1d_2 \cdots d_n}
	\]
	by just multipying each term by the factors not present in
	its denominator.
	But then
	\[
		s = \big((j_1 p_1 d_2d_3\cdots d_n) + (j_2 p_2 d_1d_3\cdots d_n)
	+ \cdots + (j_np_n d_1d_2 \cdots d_{n-1})\big)q \in \langle q \rangle
	\]
	Thus, since $s \in S$ was arbitrary, we can say $S \subseteq \langle q \rangle$.

	But then, by Theorem 1.3 in Jacobson,
	which states that any subgroup of cyclic group is also cyclic,
	we have that $S$ is also cyclic since $\langle q \rangle$ is cyclic.
	Since $S$ was arbitrary, any finitely generated subgroup of $(\Q,+,0)$ is cylic.
	
	We have $S = \langle q \rangle$ where $q$ is defined as above.
	We now prove that $S \not\cong S \times S$.
	Let $\phi \colon S \to S \times S$ be a homomorphism.
	It is sufficient to show that $\phi$ is not surjective.
	For the sake of contradiction, let $\phi$ be surjective.
	Then for any $s \in S$,
	we have that $s = mq$ for some $q \in \Z$.
	Thus $\phi(s) = \phi(mq) = m\phi(q)$.
	Thus, any element in $S \times S$ is of the form $m\phi(q)$ for some $\phi(q) \in S \times S$.
	We can write $\phi(q) = (a,b)$ for $a,b \in S$,
	and since $m(a,b) = (ma,mb)$, all elements in $S \times S$
	can be written in the form $(ma,mb)$,
	But since $b \in S$, we also have $2b \in S$,
	so $(a,2b) \in S \times S$.
	But this contradicts that every element in $S \times S$ can be written as $m(a,b)$.
	Thus, $\phi$ is not surjective,
	and the two groups are not isomorphic.
\end{proof}

\subsection*{Problem 6 (Ch. 6)}
{\it Let $a,b$ be as in Lemma 1.
Show that $\langle a \rangle \cap \langle b \rangle = 1$
and $\langle a,b \rangle = \langle ab \rangle$.}
\begin{proof}[Solution]\let\qed\relax
	Given an abelian group $G$,
	we let $a,b$ be elements with orders $m,n$ respectively,
	such that $(m,n) = 1$.
	
	To prove $\langle a \rangle \cap \langle b \rangle = 1$,
	assume the opposite, specifically
	there exists some $g \in G$ where $g \neq 1_G$,
	and $g \in \langle a \rangle \cap \langle b \rangle$.
	Necessarily, $g = a^{j} = b^{k}$ where $0 < j < m$ and $0 < k < n$.
	We have $g^m = (a^j)^m = (a^m)^j = 1_G$ and $g^n = (b^k)^n = (b^n)^k = 1_G$.
	Note that if $g^m = 1_G$, then $\mathrm{ord}(g) \mid m$.
	To see this, assume the opposite, that is $\mathrm{ord}(g) \nmid m$.
	Then by division algorithm, $m = \mathrm{ord}(g)q + r$ where $0 < r < \mathrm{ord}(g)$.
	But $1_G = g^m = g^{\mathrm{ord}(g)q + r} = (g^{\mathrm{ord}(g)})^qg^r = 1_Gg^r = g^r$,
	and since $r < \mathrm{ord}(g)$, and $\mathrm{ord}(g)$ is the least value
	such that $g$ to the power of it is $1_G$,
	we have that $g^r \neq 1_G$, which is a contradiction.
	Thus, $\mathrm{ord}(g) \mid m$ and $\mathrm{ord}(g) \mid n$.
	But recall that $(m,n) = 1$,
	but the only element that this is true is $1_G$,
	thus, we cannot have $g \neq 1_G$.
	Thus, $\langle a \rangle \cap \langle b \rangle = 1_G$.

	We now prove that $\langle a, b\rangle = \langle ab \rangle$.
	Let $c \in \langle a,b \rangle$.
	Then $c = $ some sequences of $a$'s and $b$'s.
	Note that we can then say $c = a^{j}b^{k}$ where $0 \leq j < m$
	and $0 \leq k < n$,
	since $G$ is abelian
	(we can rearrange to get $c = a^\alpha b^\beta = a^{mq_1 + j}b^{nq_2 + k}$,
	where $j,k$ are as above,
	but then $c = (a^m)^{q_1}a^j(b^n)^{q_1}b^k = 1_G a^j 1_G b^k = a^jb^k$).
	Then $c = (ab)^j b^{k-j}$.
	But note that $k - j$ ... what now?

	Assuming we get to $c \in \langle ab \rangle$ and so $\langle a,b \rangle \subseteq \langle ab \rangle$,
	we now show the other direction, so let $c \in \langle ab \rangle$.
	Trivially $c \in \langle a,b\rangle$, since
	$c = $ to some sequence of $a$'s and $b$'s.
	Thus, we have shown $\langle a,b \rangle = \langle ab \rangle$.
\end{proof}


\subsection*{Problem 7 (Ch. 1.5)}
{\it Show that if $o(a) = n = rs$, where $(r,s) = 1$,
then $\langle a \rangle \cong \langle b \rangle \times \langle c \rangle$,
where $o(b) = r$ and $o(c) = s$.
Hence, prove that any finite cyclic group is isomorphic to a direct product
of cyclic groups of prime power orders.}
\begin{proof}[Solution]\let\qed\relax
	Note that, by Theorem 1.3 in Jacobson,
	there exists only one subgroup of order $r,s$
	in $\langle a \rangle$.
	Note that by Problem 4 in Ch. 1.5
	(which we did on the last homework),
	we know that $a^r$ has order $[n,r]/r = s$,
	and likewise $a^s$ has order $[n,s]/s = r$.
	Thus our unique groups with order $r, s$ are
	$\langle a^s \rangle, \langle a^r \rangle$
	respectively.
	We let $\phi \colon \langle a \rangle \to \langle b \rangle \times \langle c \rangle$ such that
	\[
		\phi(a^j) = (b^m, c^k)
	\]
	where $m\in\N, 0\leq m < r$ such that $j = ms$,
	and $k \in \N, 0\leq k < s$ such that $j = kr$.
	I'm running out of time before the deadline, so I will just sketch what I would have done:
	I would show that this is a bijective map,
	using the fact that since $(r,s) = 1$.
	Then, I would have shown the multiplication is respected by just pushing the symbols.
\end{proof}
\end{document}
