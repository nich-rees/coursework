\documentclass{article}
\usepackage{amsmath, amsfonts, amsthm, amssymb}
\usepackage{geometry}
\geometry{letterpaper, margin=2.0cm, includefoot, footskip=30pt}

\usepackage{fancyhdr}
\pagestyle{fancy}

\lhead{Math 322}
\chead{Homework 11}
\rhead{Nicholas Rees, 11848363}
\cfoot{Page \thepage}

\newcommand{\N}{{\mathbb N}}
\newcommand{\Z}{{\mathbb Z}}
\newcommand{\Q}{{\mathbb Q}}
\newcommand{\R}{{\mathbb R}}
\newcommand{\C}{{\mathbb C}}
\newcommand{\ep}{{\varepsilon}}

\newtheorem{lemma}{Lemma}

\renewcommand{\theenumi}{(\alph{enumi})}

\begin{document}
\subsection*{Problem 4 (page 108)}
{\it Let $G$ be a group and let $T = G \times G$.
\begin{enumerate}
	\item Show that $D = \{(g,g)\in G\times G \mid g \in G\}$ is a group isomorphic to $G$.
	\item Prove that $D$ is normal in $T$ if and only if $G$ is abelian.
\end{enumerate}}
\begin{proof}[Solution]\let\qed\relax
	Let $G,T,D$ be defined as above.
	\begin{enumerate}
		\item We provide the map $\phi \colon D \to G$ by $(g,g) \mapsto g$.
			This map is clearly well-defined.
			It is also clearly surjective,
			since for $g \in G$, $\phi((g,g)) = g$.
			Finally, to show injectivity,
			let $\phi((g_1,g_1)) = \phi((g_2,g_2))$.
			Then $g_1 = g_2 \implies (g_1,g_1) = (g_2,g_2)$.
			Hence $\phi$ is bijective.

			To see that $\phi$ respects the group operation, we have
			\[
				\phi((g_1,g_1)(g_2,g_2)) = \phi(g_1g_2,g_1g_2) = g_1g_2
				= \phi((g_1,g_1))\phi((g_2,g_2))
			\]
			Thus, $\phi$ is an isomorphism between $D$ and $G$.
		\item Let $G$ be abelian.
			If $(g_1,g_2) \in T$ is arbitrary and $(g,g) \in D$,
			we have
			\[
				(g_1,g_2)(g,g)(g_1,g_2)^{-1} = (g_1g,g_2g)(g_1,g_2)^{-1}
				= (gg_1,gg_2)(g_1,g_2)^{-1} = (g,g)(g_1,g_2)(g_1,g_2)^{-1} = (g,g) \in D
			\]
			so $D$ is normal in $T$.

			Now assume that $D$ is normal in $T$.
			Let $g_1,g_2 \in G$ be arbitrary, so $(g_1,g_2) \in T$.
			Note $(g_1,g_1) \in D$,
			so we have $(g_1,g_2)(g_1,g_1)(g_1,g_2)^{-1} = (g_1g_1g_1^{-1},g_2g_1g_2^{-1}) = (g_1,g_2g_1g_2^{-1}) \in D$.
			So $g_1 = g_2g_1g_2^{-1} \implies g_1g_2 = g_2g_1$.
			Since this is true for any $g_1,g_2 \in G$,
			this shows that $G$ is abelian.
	\end{enumerate}
\end{proof}

\subsection*{Problem 5 (page 108)}
{\it Let $G$ be a finite abelian group.
Prove that $G$ is isomorphic to the direct product of its Sylow subgroups.}
\begin{proof}[Solution]\let\qed\relax
	By Theorem 2.13.1 in Herstein,
	it is sufficient to how that $G$ is the internal direct product
	of the Sylow subgroups.
	First, note that all of our Sylow p subgroups ar normal,
	since this is an abelian group and all subgroups are normal.
	Thus there is only one Sylow of each kind, since the Sylow subgroups are conjugate to each other.
	Furthermore, Sylow subgroups are disjoint, since they have coprime order.
	Thus, we can decompose $G$ as the product of Sylow subgroups,
	which satisfy our desired property for an inner directt product.
\end{proof}

\subsection*{Problem 6 (page 108)}
{\it Let $A,B$ be cyclic groups of order $m$ and $n$, respectively.
Prove that $A \times B$ is cyclic if and only if $m$ and $n$ are relatively prime.}
\begin{proof}[Solution]\let\qed\relax
	Let $A \times B$ be cyclic.
	Then there is some $a \in A$ and $b \in B$ such that $\langle (a,b) \rangle = A \times B$.
	There are $|A||B| = mn$ elements in $A \times B$,
	hence $o((a,b)) = mn$ in order to generate the whole group.
	For the sake of contradiction, assume that $\gcd(a,b) \neq 1$.
	Let $\mathrm{lcm}(m,n) = l$.
	Then for any $(a_1,b_1) \in A \times B$, we have
	$(a_1,b_1)^l = (a_1^l,b_1^l) = (1,1)$,
	but since we have that $\gcd \neq 1$, we have $c < mn$.
	Hence, no element in $A \times B$ has order $mn$.
	But then $(a,b)$ doesn't have order $mn$, so a contradiction.

	Then, for any $ 1 \leq r < mn$, $(a,b)^r \neq (1,1)$.
	That is, $(a^r,b^r) \neq (1,1)$.
	Hence, for any $1 \leq r < mn$, $m \nmid r$ or $n \nmid r$,
	otherwise $($.

	Let $(m,n) = 1$.
	For some $(a,b) \in A \times B$, we require that $(a,b)^{\mathrm{lcm}(m,n)} = (1,1)$,
	thus the order of $(a,b)$ must divide $\mathrm{lcm}(m,n)$, call this $k$.
	We must have $a^k = b^k = 1$, so $m \mid k$ and $n \mid k$,
	thus $\mathrm{lcm}(m,n) \mid k$.
	Hence, $k = \mathrm{lcm}(m,n) = mn$ since the $\gcd = 1$.
	Thus, since there are $mn$ elements in $A \times B$,
	and there is an element with order $mn$,
	we have that $A \times B = \langle (a,b) \rangle$, which is cyclic.
\end{proof}

\subsection*{Problem 8 (page 108)}
{\it Give an example of a group $G$ and normal subgroups $N_1, \dots, N_n$
such that $G = N_1N_2\cdots N_n$ and $N_i \cap N_j = \langle e \rangle$
for $i \neq j$ and yet $G$ is \emph{not} the internal direct product of $N_1,\dots,N_n$.}
\begin{proof}[Solution]\let\qed\relax
	We provide the Klein four-group $G$: let $1,a,b \in G$,
	and define $ab = ba \neq 1,a,b$.
	Furthermore, $a^2 = b^2 = 1$ (and clearly then $(ab)^2 = abba = 1$ as well).
	Thus $G$ has four elements.
	Furthermore, it is possible to check that $G$ is abelian as well.
	We provide the subgroups $N_1 = \{1,a\}$, $N_2 = \{1,b\}$, and $N_3 = \{1,ab\}$.
	Since $G$ is abelian, these subgruops are all normal.
	Furthermore, $G = N_1N_2N_3$, since
	\begin{align*}
		1 &= 1\cdot 1 \cdot 1\\
		a &= a \cdot 1 \cdot 1\\
		b &= 1 \cdot b \cdot 1\\
		ab &= 1 \cdot 1 \cdot ab
	\end{align*}
	which are all the elements in $G$.
	Additionally, clearly $N_i \cap N_j = \langle 1 \rangle$ when $i \neq j$
	($1 \leq i,j \leq 3$).
	
	We now show that $G$ is the internal direct product of $N_1,N_2,N_3$.
	Notice that $ab = a \cdot b \cdot 1$ as well as $ab = 1 \cdot 1 \cdot ab$,
	hence, not all $g \in G$ are represented as a unique product
	of elements from $N_1,N_2,N_3$, which violates a requirement
	for $G$ to be the internal direct product of $N_1,N_2,N_3$.
\end{proof}

\subsection*{Problem 11 (page 108)}
{\it Let $G$ be a finite abelian group such that it contains a subgroup
$H_0 \neq \langle e \rangle$ which lies in \emph{every} subgroup $H \neq \langle e \rangle$.
Prove that $G$ must be cyclic.
What can you say about $o(G)$.}
\begin{proof}[Solution]\let\qed\relax
	We prove the contrapositive. That is, assume that $G$ is not cyclic.
	Then, since $G$ is finite,
	there exists a minimal set of generators such that
	$G = \langle a_1,\dots,a_r \rangle$ and $r > 1$.
	We now consider $\langle a_1 \rangle \cap \langle a_2 \rangle$.
	Let $o(a_1) = m, o(a_2) = n$.
	We claim that $\langle a_1 \rangle \cap \langle a_2 \rangle = \{1\}$.
	If $(m,n) = 1$, by problem 6 from section 1.5 of Jacobson
	(we proved in homework 4),
	we have that $\langle a_1 \rangle \cap \langle a_2 \rangle = 1$.
	Now let $(m,n) = d > 1$.
	For the sake of contradiction, assume that $\langle a_1 \rangle \cap \langle a_2 \rangle \neq \{1\}$.
	Then there exists $j,k$ such that $a_1^j = a_2^k$.
	I can't figure this out, but I want to say something like $a_1^j = a_2$,
	and so $\langle a_2 \rangle \subseteq \langle a_1 \rangle$,
	and so $\langle a_1, a_3, \dots ,a_r \rangle$ generates the set.
	But this contradicts our assumption that $a_1,a_2,\dots,a_r$ was minimal.
	Thus, $\langle a_1 \rangle \cap \langle a_2 \rangle = \{1\}$.
	In either case, we have $langle a_1 \rangle \cap \langle a_2 \rangle = \{1\}$,
	but then there is no subgruop $H_0 \neq \langle 1 \rangle$
	that is contained in each subgroup, completing the proof.
\end{proof}
\end{document}
