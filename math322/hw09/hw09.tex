\documentclass{article}
\usepackage{amsmath, amsfonts, amsthm, amssymb}
\usepackage{geometry}
\geometry{letterpaper, margin=2.0cm, includefoot, footskip=30pt}

\usepackage{fancyhdr}
\pagestyle{fancy}

\lhead{Math 322}
\chead{Homework 9}
\rhead{Nicholas Rees, 11848363}
\cfoot{Page \thepage}

\newcommand{\N}{{\mathbb N}}
\newcommand{\Z}{{\mathbb Z}}
\newcommand{\Q}{{\mathbb Q}}
\newcommand{\R}{{\mathbb R}}
\newcommand{\C}{{\mathbb C}}
\newcommand{\ep}{{\varepsilon}}

\newtheorem{lemma}{Lemma}

\renewcommand{\theenumi}{(\alph{enumi})}

\begin{document}
\subsection*{Problem 1 (Ch. 1.13)}
{\it Show that if $P$ is a Sylow subgroup, then $N(N(P)) = N(P)$}
\begin{proof}[Solution]\let\qed\relax
	Let $g \in N(P)$.
	Then for any $x \in N(P)$, $gxg^{-1} \in N(P)$ since $N(P)$ is a group (closure).
	Thus, $g \in N(N(P))$,
	so $N(P) \subseteq N(N(P))$.
	
	Now let $g \in N(N(P))$.
	Then $gxg^{-1} \in N(P)$ for all $x \in N(P)$.
	Note $P \subseteq N(P)$ ($P$ is a group, so if $x \in P$,
	closure implies $pxp^{-1} \in P$ for all $p \in P$).
	Furthermore, $P$ is the only Sylow $p$-subgroup such that $P \subseteq N(P)$,
	by the contrapositive of the lemma from page 81 of Jacobson
	(distinct Sylow $p$-subgroups cannot be subgroups of each other,
	thus since they are of order $p$-power, they cannot be contained in $N(P)$).
	Thus, since if $x \in P$, $gxg^{-1}$ is in a Sylow $p$-subgruop,
	but $gxg^{-1} \in N(P)$ as well, thus $gxg^{-1} \in P$.
	This was true for any $x \in P$, thus $h \in N(P)$.
	Therefore, we have $N(P) = N(N(P))$.
\end{proof}

\subsection*{Problem 2 (Ch. 1.13)}
{\it Show that there are no simple groups of order $148$ or of order $56$.}
\begin{proof}[Solution]\let\qed\relax
	Note the unique prime factor decompositions: $148 = 2^2\cdot 37$
	and $56 = 2^3\cdot 7$.

	I'm first going to prove a Lemma:
	\begin{lemma}
		If $H$ is the only Sylow $p$-subgroup of a group $G$,
		then $H$ is normal in $G$.
	\end{lemma}
	\begin{proof}
		Note that the conjugation action of $G$ on $H$ is an isomorphism:
		the action is a homomorphism,
		and conjugation is the composition of left translation and right translation,
		both of which are bijections,
		this conjugation is also a bijection.
		Thus, conjugation maps $G$ to another group of the same order as it.
		But by assumption, the only such group is $H$.
		Thus for any $a \in G$, $aHa^{-1} = H$,
		hence $H$ is normal in $G$.
	\end{proof}
	
	Let $G$ be a group of order $148$.
	Let $n_{37}$ be the number of Sylow $37$-subgroups of $G$.
	By Sylow's second theorem, we have $n_{37} \equiv 1\, (\mathrm{mod}\, 37)$.
	If $n_{37} = 1$, we are done:
	letting $H$ denote our sole Sylow $37$-subgroup,
	by Lemma 1, we have that $H$ is normal.
	Thus $G$ is not simple, since we have a subgroup
	that is not equal to $G$ or $\{1\}$ (since $|H| = 37 < 148 = |G|$).
	It remains to consider $n_{37} \geq 38$.
	But we must have that $n_{37} \mid 148/37 = 4$ by Sylow's second theorem,
	thus $n_{37} \leq 4$, therefore we cannot have $n_{37} \geq 38$.
	Thus if $|G| = 148$, we must always only have one Sylow $37$-subgroup,
	which we have shown must be normal, thus $G$ is not simple.

	Let $G$ now be a group of order $56$.
	Let $n_7$ be the number of Sylow $7$-subgroups of $G$
	and $n_2$ be the number of Sylow $2$-subgroups of $G$.
	By Sylow's second theorem, we have $n_7 \equiv 1\, (\mathrm{mod}\,37)$
	and $n_2 \equiv 1\, (\mathrm{mod}\,2)$.
	If $n_7 = 1$, we are done, just as the $148$ case ($G$ can't be simple).
	Now consider if $n_7 \geq 8$.
	Note that if $n_7 > 8$, we have that $n_7 \nmid 56/7 = 8$,
	thus we need only look at $n_7 = 8$.
	Note that for each Sylow $7$-subgroup,
	since the group has prime order,
	each element (other than the identity) has order $7$
	(otherwise, they would generate a subgroup with order $m$,
	but no $m \nmid 7$, contradiction by Lagrange's theorem).
	Thus, there are $(7-1)\cdot 8 = 48$ elements of order $7$ in $G$.
	Now consider the Sylow $2$-subgroups of $G$.
	Each of these have $2^3 = 8$ elements in each.
	Note that any element from our $7$-subgroups cannot be in
	a $2$-subgroup,
	since it has order $7$ but $7 \nmid 8$.
	Thus, since there are only $56$ elements in $G$ in total,
	and $48$ of them are assumed to be in our $7$-subgroups,
	we can only have a single Sylow $2$-subgroup.
	But by Lemma 1, this implies that this Sylow $2$-subgroup is normal,
	so $G$ can't be simple.
	This exhausts all possible cases.
\end{proof}

\subsection*{Problem 3 (Ch. 1.13)}
{\it Show that there is no simple group of order $pq$,
$p$, and $q$ primes (cf. exercise 5, p. 77).}
\begin{proof}[Solution]\let\qed\relax
	Let $G$ be a group with order $pq$.
	WLOG, assume $p < q$.
	By Sylow's first theorem,
	there exists a subgroup of order $q$, call it $H$.
	Thus, by Lagrange's theorem, the index of $H$ is $p$.
	But since $p$ is the smallest prime dividing $|G|$,
	we know that $H$ is normal by exercise $5$ on page $78$.
	Thus, since $|\{1\}| = 1 < |H| = p < pq = |G|$, we have that
	$G$ contains a normal subgroup not equal to itself or $\{1\}$,
	thus $G$ cannot be normal.
\end{proof}


\subsection*{Problem 4 (Ch. 1.13)}
{\it Show that every non-abelian group of order $6$ is isomorphic to $S_3$.}
\begin{proof}[Solution]\let\qed\relax
	Assume $G$ is a non-abelian group such that $|G| = 6 = 2\cdot 3$.
	Then by Sylow's second theorem,
	we have that $G$ contains exactly one Sylow $3$-subgroup $H$
	(since if $n_3$ are the number of $3$-subgroups,
	we have $n_3 \equiv 1\,(\mathrm{mod}\,3)$ and $n_3 \mid 6/3 = 2$,
	which only occurs when $n_3 = 1$).
	Furthermore, $G$ can contain one or three Sylow $2$-subgroups
	(for the same reason as before).
	For the sake of contradiction,
	assume that $G$ contains only one Sylow $2$-subgroup, call it $K$.
	Note that $H \cap K = \{1\}$ only,
	since $K = \{1,k\}$ and so if $k \in H$, then $\langle k \rangle$
	is a subgroup of $H$ of order $2$, but this contradicts Langranges theorem.
	Thus, we have a total of $4$ elements in $H$ or in $K$,
	so there are two elements, $x,y \in G$ that are not in either.
	Note that $x^2 \neq 0$ and $x^3 \neq 0$ (otherwise $\langle x \rangle$
	would generate another Sylow subgroup, but we said there were no more)
	thus, since the only remaining divisor of $6$ is $6$,
	we have $o(x) = 6$ (by Lagrange's theorem).
	But then $\langle x \rangle$ is a subgroup of $G$ with the same order as it,
	thus $\langle x \rangle = G$, but then $G$ is a cyclic group,
	so it is abelian, a contradiction.

	Thus, there must be three distinct Sylow $2$-subgroups,
	call them $K_1, K_2, K_3$.
	Note that every group must have the identity,
	thus $K_1 = \{1,a\}$, $K_2 = \{1,b\}$, and $K_3 = \{1,c\}$,
	where $a,b,c \in G$ are distinct and not the identity.
	Furthermore, by closure of the $K_i$'s, we have $a^2 = b^2 = c^2 = 1$.
	Now note that $a,b,c \not\in H$.
	We prove this by showing that $H$ cannot contain any element
	that has order $2$.
	Let $H$ consist of the elements $\{1,m,n\}$ where $m,n \in G$
	are not the identity and $m \neq n$.
	Note that $m^{-1} = n$ and vice versa (since groups must contain inverses).
	If this were not the case, then $m^2 = 1$ and $n^2 = 1$,
	but then we have one of $mn = 1$, $mn = n$, or $mn = m$,
	all three of which cannot happen
	(the first can't because inverses are unique,
	the second can't because then $m = 1$,
	and the third can't because then $n = 1$),
	thus contradiction because $H$ is a group but not closed under the multiplication.
	Thus, $a,b,c$ cannot be in $H$.

	We have that $G$ is a group with elements $\{1,a,b,c,m,n\}$
	where $a^2 = b^2 = c^2 = mn = nm = 1$.
	Note that $ab \neq c$: for the sake of contradiction, assume $ab=c$.
	Then $ba = b^{-1}a^{-1} = (ab)^{-1} = c^{-1} = c$;
	also $cb = a = bc$ and $ac = b = ca$ by an identical argument.
	This defines a group of order $4$, but we know that $G$
	does not contain a subgroup of order $4$ since $4 \nmid 6$,
	so contradiction by Lagrange's theorem.
	Also, $ab \neq a,b$, otherwise $a = 1$ or $b = 1$.
	By an identical argument, we can repeat this for any of our products of $a,b,c$,
	so the product of two of $a,b,c$ must be one of $m$ or $n$.

	We have $ab$ can either equal $m$ or $n$.
	These were both arbitrary elements of $H$, so let $ab = n$;
	then, they must have the same inverse, so we have $ba = m$.
	By an identical argument
	This implies that $bc = n$ too;
	for the sake of contradiction, assume $bc = m$, then $cb = n = ab \implies a = c$,
	a contradiction.
	Thus $cb = m$.
	By a similar argument, we have that $ac = m$ and so $ca = n$.
	We can define the remainder of our products using the relations already derived.

	Thus, $G$ is a group of elements $\{1,a,b,c,m,n\}$ with the following product rules:
	$a^2 = b^2 = c^2 = mn = nm = 1$,
	$ab = n$, $ba = m$, $bc = n$, $cb = m$, $ac = m$, $ca = n$,
	$mc = a$, $cm = b$, $am = c$, $ma = b$, $mb = c$, $bm = a$,
	$nc = b$, $cn = a$, $an = b$, $na = c$, $nb = a$, and $bn = c$.
	This defines every possible product, so there are no more relations on $G$.

	We now prove that this is isomorphic to $S_3$.
	We provide the explicit map $\phi \colon G \to S_3$:
	\begin{align*}
		\phi(1) &\mapsto (1)\\
		\phi(a) &\mapsto (12)\\
		\phi(b) &\mapsto (13)\\
		\phi(c) &\mapsto (23)\\
		\phi(m) &\mapsto (123)\\
		\phi(n) &\mapsto (132)\\
	\end{align*}
	Clearly this is a bijection.
	It remains to show that this is a homomorphism:
	one can consider every product of the form $\phi(x)\phi(y)$
	where $x,y \in G$, and confirm $\phi(x)\phi(y) = \phi(xy)$
	using the product rules we have defined.
	We defined every product,
	so one could check (given a desire to do this very tedious task)
	that our product always matches up,
	so we have an isomorphism.
\end{proof}


\subsection*{Problem 5 (Ch. 1.13)}
{\it Determine the number of non-isomorphic groups of order $15$.}
\begin{proof}[Solution]\let\qed\relax
	We claim that there is only one group (up to isomorphism),
	namely the cyclic group $\Z/15\Z$.
	Let $|G| = 15$.
	Then by Sylow's theorems, there exists $n_3 \equiv 1\,(\mathrm{mod}\,3)$
	subgroups of order $3$, but since $n_3 \mid 5$ only when $n_3 = 1$,
	we have that $n_3 = 1$;
	similarly, there are $n_5 = 1$ subgroups of order $5$.

	Now, there are $8$ elements in $G$ that are not in
	our Sylow $3$-subgroup, or our Sylow $5$-subgroup.
	Pick one of them, call it $x$.
	Consider the subgroup of $G$ generated by $x$, $\langle x \rangle$.
	We have that $|\langle x \rangle | \mid |G|$ by Lagrange's theorem,
	but $|\langle x \rangle | \neq 3,5$ since we assumed
	there was only one group of that order, respectively.
	Thus, $|\langle x \rangle| = 15$.
	But the only subgroup of $G$ of order $15$ is itself,
	thus $\langle x \rangle = G$,
	thus $G$ is a cyclic group of order $15$, or $\Z/15\Z$.
\end{proof}
\clearpage

An element of order $2$ in a group is called an \emph{involution}.
An important insight into the structure of a finite group is obtained
by studying its involutions and their centralizers.
The next five excercises give a program for characterizing $S_5$ in this way.
In all of these exercises, as well as in the rest of this set, $G$ is a finite group.

\subsection*{Problem 6 (Ch. 1.13)}
{\it Let $u$ and $v$ be distinct involutions in $G$.
Show that $\langle u,v \rangle$ is (isomorphic to) a dihedral group.}
\begin{proof}[Solution]\let\qed\relax
	First, note that the dihedral group $D_n$
	is specifically the group
	generated by two elements $\langle x,y \rangle$
	with the only additional relations $x^2 = y^2 = (xy)^n = 1$.
	Thus, it is sufficient to show that $(uv)^n = 1$ for some $n$
	is the only additional structure on $\langle u,v\rangle$.

	Note that all the elements of $\langle u ,v \rangle$ are
	$1, u, v$, $uv, uvuv, uvuvuv, \dots$ and $vu, vuvu, vuvuvu, \dots$.
	Eventually, $(uv)^n = 1$ because this is a finite group,
	and so $(uv)^{-n} = (vu)^n = 1$ as well.
\end{proof}

\subsection*{Problem 7 (Ch. 1.13)}
{\it Let $u$ and $v$ be involutions in $G$.
Show that if $uv$ is of odd order than $u$ and $v$
are conjugate in $G$ ($v = gug^{-1}$).}
\begin{proof}[Solution]\let\qed\relax
	We have that there exists $n$ such that $(uv)^{2n+1} = 1$.
	Then $(uv)^{2n} = (uv)^{-1} = v^{-1}u^{-1} = vu$.
	See
	\[
		v = (uv)^{2n}u = \underbrace{uvu \cdots uv}_{2n\text{ times}}u
		= \underbrace{uv\cdots uv}_{n\text{times}}u\underbrace{vu\cdots vu}_{n\text{times}}
		= (uv)^n u (vu)^n
	\]
	Finally, see that $((uv)^n)^{-1} = ((uv)^{-1})^n = (v^{-1}u^{-1})^n = (vu)^n$,
	thus if $g = (uv)^n$, then $v = gug^{-1}$ as desired.
\end{proof}

\subsection*{Problem 8 (Ch. 1.13)}
{\it Let $u$ and $v$ be involutions in $G$ such that $uv$ has even order $2n$,
so $w = (uv)^n$ is an involution.
Show that $u,v \in C(w)$.}
\begin{proof}[Solution]\let\qed\relax
	Note that $(uv)^n = w = w^{-1} = (uv)^{-n} = (v^{-1}u^{-1})^n = (vu)^n$.
	Thus
	\[
		uw = u(uv)^n = u(vu)^n = (uv)^nu = wu
	\]
	so $u \in C(w)$.
	Similarily,
	\[
		vw = v(uv)^n = (vu)^nv = (uv)^nv = wv
	\]
	so $v \in C(w)$ as well.
\end{proof}


\subsection*{Problem 9 (Ch. 1.13)}
{\it Suppose $G$ contains exactly two conjugacy classes of involutions.
Let $u_1$ and $u_2$ be non-conjugate involutions in $G$.
Let $c_i = |C(u_i)|, i =1,2$.
Let $S_i, i=1,2$, be the set of ordered pairs $(x,y)$ with
$x$ conjugate to $u_1$, $y$ conjugate to $u_2$, and $(xy)^n = u_i$ for some $n$.
Let $s_i = |S_i|$. Prove that $|G| = c_1s_2 + c_2s_1$.
(\emph{Hint}: Count the number of ordered pairs $(x,y)$
with $x$ conjugate to $u_1$ and $y$ conjugate to $u_2$ in two ways.
First, this number is $(|G|/c_1)(|G|/c_2)$.
Since $x$ is no conjugate to $y$, exercises $7$ and $8$ imply that
for $n = o(xy)/2$, $(xy)^n$ is conjugate to either $u_1$ or $u_2$.
This implies that $(|G|/c_1)(|G|/c_2) = (|G|/c_1)s_1 + (|G|/c_2)s_2$.)}
\begin{proof}[Solution]\let\qed\relax
	Let $U_i$ be the conjugacy class of $u_1$.
	Let $G$ act on $U_i$ by conjugation;
	we know this action must be transitive.
	Furthermore, $C(u_i) = \mathrm{Stab}\, u_i$ of this action.
	Then, as mentioned in as a consequence of Theorem 1.10 in Jacobson,
	we get that $|U_i| = [G : C(u_i)] = |G|/|C(u_i)|$.
	Thus, given the ordered pair $(x,y)$ where $x \in U_1$ and $y \in U_2$,
	the number of possible ordered pairs is $(|G|/c_1)(|G|/c_2)$.
	Call this value $\xi$.

	Now, we notice that we can write $\xi$ as a different expression.
	Since $x$ is not conjugate to $y$, $o(xy)$ is even by problem 7,
	and if $n = o(xy)/2$, then $(xy)^n$ is an involution, and so
	$(xy)^n$ is conjugate to either $u_1$ or $u_2$ by problem 8.
	hmm out of time, so let's just say we get:
	$\xi = (|G|/c_1)s_1 + (|G|/c_2)s_2$.
	See
	\[
		(|G|/c_1)(|G|/c_2) = (|G|/c_1)s_1 + (|G|/c_2)s_2
		\implies |G| = c_2s_1 + c_1s_2
	\]
	as desired.
\end{proof}
\end{document}
