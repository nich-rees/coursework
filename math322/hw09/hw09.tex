\documentclass{article}
\usepackage{amsmath, amsfonts, amsthm, amssymb}
\usepackage{geometry}
\geometry{letterpaper, margin=2.0cm, includefoot, footskip=30pt}

\usepackage{fancyhdr}
\pagestyle{fancy}

\lhead{Math 322}
\chead{Homework 9}
\rhead{Nicholas Rees, 11848363}
\cfoot{Page \thepage}

\newcommand{\N}{{\mathbb N}}
\newcommand{\Z}{{\mathbb Z}}
\newcommand{\Q}{{\mathbb Q}}
\newcommand{\R}{{\mathbb R}}
\newcommand{\C}{{\mathbb C}}
\newcommand{\ep}{{\varepsilon}}

\renewcommand{\theenumi}{(\alph{enumi})}

\begin{document}
\subsection*{Problem 1 (Ch. 1.13)}
{\it Show that if $P$ is a Sylow subgruop, then $N(N(P)) = N(P)$}
\begin{proof}[Solution]\let\qed\relax
	I'm pretty sure that $N(P)$ is just the Sylow $p$-subgroups
	where $p^m = |P|$.
	Recall that $N(H) = \{g \in G \mid gHg^{-1} = H\}$,

	Let $G$ be our ambient group of order $p^nm$ where
	$p$ is a prime and $p \nmid m$.
	Then let $P$ be a Sylow $p$-subgroup, that is,
	of order $p^n$.
	Recall that if $P'$ is another subgroup of $G$ with order $p^n$,
	then there exists $a \in G$ such that $P' = aPa^{-1}$
	by Sylow's Theorem.
	ff

	We have $N(P) = \{g \in G \mid gPg^{-1} = P\}$,
	and so $N(N(P)) = \{g_2 \in G \mid g_2N(P)g_2^{-1} = N(P)\}$.
	Set inclusion.

	Let $a \in N(P)$ (so $gPg^{-1} = P$).
	
	Consider the action of the $G$ on the set of Sylow $p$-subgroups by conjugation.
	Recall from equation $(41)$ from Jacobson p. 76 that
	$\mathrm{Stab} \, aPa^{-1} = a(\mathrm{Stab}\,P)a^{-1}$
	where $a \in G$.
	Note that $N(P)$ is the stablizer of $P$ under the conjugancy action,
	thus $N(aPa^{-1}) = aN(P)a^{-1}$ for all $a \in G$.
	ff something about how $G$ acting on Sylow $p$-subgropus by conjugation is transitive.
\end{proof}

\subsection*{Problem 2 (Ch. 1.13)}
{\it Show that there are no simple groups of order $148$ or of order $56$.}
\begin{proof}[Solution]\let\qed\relax
	ff lol Evan Chen problem.
\end{proof}

\subsection*{Problem 3 (Ch. 1.13)}
{\it Show that there is no simple group of order $pq$,
$p$, and $q$ primes (cf. exercise 5, p. 77).}
\begin{proof}[Solution]\let\qed\relax
	ff
\end{proof}


\subsection*{Problem 4 (Ch. 1.13)}
{\it Show that every non-abelian group of order $6$ is isomorphic to $S_3$.}
\begin{proof}[Solution]\let\qed\relax
	ff
\end{proof}


\subsection*{Problem 5 (Ch. 1.13)}
{\it Determine the number of non-isomorphic groups of order $15$.}
\begin{proof}[Solution]\let\qed\relax
	ff
\end{proof}

An element of order $2$ in a group is called an \emph{involution}.
An important insight into the structure of a finite group is obtained
by studying its involutions and their centralizers.
The next five excercises give a program for characterizing $S_5$ in this way.
In all of these exercises, as well as in the rest of this set, $G$ is a finite group.

\subsection*{Problem 6 (Ch. 1.13)}
{\it Let $u$ and $v$ be distinct involutions in $G$.
Show that $\langle u,v \rangle$ is (isomorphic to) a dihedral group.}
\begin{proof}[Solution]\let\qed\relax
	ff
\end{proof}

\subsection*{Problem 7 (Ch. 1.13)}
{\it Let $u$ and $v$ be involutions in $G$.
Show that if $uv$ is of odd order than $u$ and $v$
are conjugate in $G$ ($v = gug^{-1}$).}
\begin{proof}[Solution]\let\qed\relax
	ff
\end{proof}

\subsection*{Problem 8 (Ch. 1.13)}
{\it Let $u$ and $v$ be involutions in $G$ such that $uv$ has even order $2n$,
so $w = (uv)^n$ is an involution.
Show that $u,v \in C(w)$.}
\begin{proof}[Solution]\let\qed\relax
	ff
\end{proof}


\subsection*{Problem 9 (Ch. 1.13)}
{\it Suppose $G$ contains exactly two conjugacy classes of involutions.
Let $u_1$ and $u_2$ be non-conjugate involutions in $G$.
Let $c_i = |C(u_i)|, i =1,2$.
Let $S_i, i=1,2$, be the set of ordered pairs $(x,y)$ with
$x$ conjugate to $u_1$, $y$ conjugate to $u_2$, and $(xy)^n = u_i$ for some $n$.
Let $s_i = |S_i|$. Prove that $|G| = c_1s_2 + c_2s_1$.
(\emph{Hint}: Count the number of ordered pairs $(x,y)$
with $x$ conjugate to $u_1$ and $y$ conjugate to $u_2$ in two ways.
First, this number is $(|G|/c_1)(|G|/c_2)$.
Since $x$ is no conjugate to $y$, exercises $7$ and $8$ imply that
for $n = o(xy)/2$, $(xy)^n$ is conjugate to either $u_1$ or $u_2$.
This implies that $(|G|/c_1)(|G|/c_2) = (|G|/c_1)s_1 + (|G|/c_2)s_2$.)}
\begin{proof}[Solution]\let\qed\relax
	ff
\end{proof}
\end{document}
