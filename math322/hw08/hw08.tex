\documentclass{article}
\usepackage{amsmath, amsfonts, amsthm, amssymb}
\usepackage{geometry}
\geometry{letterpaper, margin=2.0cm, includefoot, footskip=30pt}

\usepackage{fancyhdr}
\pagestyle{fancy}

\lhead{Math 322}
\chead{Homework 8}
\rhead{Nicholas Rees, 11848363}
\cfoot{Page \thepage}

\newcommand{\N}{{\mathbb N}}
\newcommand{\Z}{{\mathbb Z}}
\newcommand{\Q}{{\mathbb Q}}
\newcommand{\R}{{\mathbb R}}
\newcommand{\C}{{\mathbb C}}
\newcommand{\ep}{{\varepsilon}}

\renewcommand{\theenumi}{(\alph{enumi})}

\begin{document}
\subsection*{Problem 2 (Ch. 1.12)}
{\it Determine representatives of the conjugancy classes in $S_5$
and the number of elements in each class.
Use this information to prove that the only normal subgroups
of $S_5$ are $1$, $A_5$, $S_5$.}
\begin{proof}[Solution]\let\qed\relax
	Recall that the conjugancy classes of $S_n$
	are just the possible partitions of $n$ elements.
	ff just silly calculations
\end{proof}

\subsection*{Problem 4 (Ch. 1.12)}
{\it Show that if a finite group $G$ has a subgroup $H$ of index $n$
then $H$ contains a normal subgroup of $G$ of index a divisor of $n!$.
(\emph{Hint}: Consider the action of $G$ on $G/H$ by left translations.}
\begin{proof}[Solution]\let\qed\relax
	Now assume $n > 1$.
	So $G = x_1H \sqcup x_2H \sqcup \cdots \sqcup x_nH$
	where $x_1 = 1_G$.
	Consider the action of $G$ on $G/H$.
	$g$ acts on the coset $xH$ by $gxH$.
	The kernel of this map is the set of all $G$ such that $gxH = xH$
	for all $x \in H$.
	But this is just $g'$ such that $x^{-1}g'x \in H$ for all $x \in H$.
	we claim that $g' \in H$?


	Note that $n \mid |G|$ so $n\leq |G|$,
	but then a map representating the action of $G$ on $G/H$
	is not injective.
	Thus, there is a nontrivial kernel.
	ff
	something something subset $N \in H$ has $g^{-1}Ng \in N$
	for all $g \in G$.

	Theorem 1.10 says action of $G$ on $G/H$ is equivalent to
	action of $G$ on $S$ (transtively) and $H = \mathrm{Stab}\;x$, $x\in S$.
	ff

	Wait, do we not have trivial group?
	$1 \in H$, $1$ is a normal subgroup... ahh index! not order divides $n!$
\end{proof}

\subsection*{Problem 5 (Ch. 1.12)}
{\it Let $p$ be the smallest prime dividing the order of a finite group.
Show that any subgroup of $H$ of $G$ of index $p$ is normal.}
\begin{proof}[Solution]\let\qed\relax
	Apply Problem 4 from 1.12 (above):
	if $H$ is a subgroup of $G$ of index $p$,
	we have that $H$ contains a normal subgroup of $G$
	of index a divisor of $p!$.
	Note that if $p$ is the smallest prime dividing the order of $G$,
	then there is no smaller value (other than $1$)
	that divides the order of $G$,
	since if there were,
	then it must be composite (since $p$ is assumed to be the smallest),
	but composite numbers are products of primes less than it,
	but then those primes are less than $p$, which we can't have.
	ff
\end{proof}


\subsection*{Problem 6 (Ch. 1.12)}
{\it Show that evry group of order $p^2$, $p$ is a prime, is abelian.
Show that up to isomorphism there are only two such groups.}
\begin{proof}[Solution]\let\qed\relax
	ff
\end{proof}


\subsection*{Problem 8 (Ch. 1.12)}
{\it Let $G$ act on $S$, $H$ act on $T$, and assume $S \cap T = \emptyset.$
Let $U = S \cup T$ and define for $g \in G$ ,$h \in H$, $s \in S$ $t \in T$;
$(g,h)s = gs$, $(g,h)t = ht$.
Show that this defines an action of $G \times H$ on $U$.}
\begin{proof}[Solution]\let\qed\relax
	ff
\end{proof}

\subsection*{Problem 9 (Ch. 1.9)}
{\it A group $H$ is said to \emph{act on a group $K$ by automorphisms}
if we have an action of $H$ on $K$
and for every $h \in H$ the map $k \to hk$ of $K$ is an automorphism.
Suppose this is the case and let $H$ be the product set $K \times H$.
Define a binary composition in $K \times H$ by
\[
	(k_1, h_1)(k_2, h_2) = (k_1(h_1k_2), h_1h_2)
\]
and define $1=(1,1)$ -- the units of $K$ and $H$ respectively.
Verify that this defines a group such that $h \to (1,h)$
is a monomorphism of $H$ into $K \times H$
and $k \to (k,1)$ is a monomorphism of $K$ into $K \times H$
whose image is a normal subgroup.
$G$ is called a \emph{semi-direct product of $K$ and $H$}.
Note that if $H$ and $K$ are finite than $|K \times H| = |K||H|$.}
\begin{proof}[Solution]\let\qed\relax
	ff
\end{proof}
\end{document}
