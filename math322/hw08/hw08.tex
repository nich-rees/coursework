\documentclass{article}
\usepackage{amsmath, amsfonts, amsthm, amssymb}
\usepackage{geometry}
\geometry{letterpaper, margin=2.0cm, includefoot, footskip=30pt}

\usepackage{fancyhdr}
\pagestyle{fancy}

\lhead{Math 322}
\chead{Homework 8}
\rhead{Nicholas Rees, 11848363}
\cfoot{Page \thepage}

\newcommand{\N}{{\mathbb N}}
\newcommand{\Z}{{\mathbb Z}}
\newcommand{\Q}{{\mathbb Q}}
\newcommand{\R}{{\mathbb R}}
\newcommand{\C}{{\mathbb C}}
\newcommand{\ep}{{\varepsilon}}

\renewcommand{\theenumi}{(\alph{enumi})}

\begin{document}
\subsection*{Problem 2 (Ch. 1.12)}
{\it Determine representatives of the conjugancy classes in $S_5$
and the number of elements in each class.
Use this information to prove that the only normal subgroups
of $S_5$ are $1$, $A_5$, $S_5$.}
\begin{proof}[Solution]\let\qed\relax
	As is shown in Jacobson, the conjugancy classes of $S_5$
	have a $1$-$1$ correspondance to the partitions of $5$,
	namely positive integers $r \geq s \geq \cdots \geq u$
	such that $r + s + \cdots + u = 5$.
	The partitions of $5$ are $(5)$, $(4,1)$, $(3,2)$, $(3,1,1)$,
	$(2,2,1)$, $(2,1,1,1)$, and $(1,1,1,1,1)$.
	Thus, we present the following representatives
	of the corresponding equivalence classes:
	\begin{align*}
		&(12345) = (15)(14)(13)(12)\\
		&(1234)(5) = (14)(13)(12)\\
		&(123)(45) = (13)(12)(45)\\
		&(123)(4)(5) = (13)(12)\\
		&(12)(34)(5) = (12)(34)\\
		&(12)(3)(4)(5) = (12)\\
		&(1)(2)(3)(4)(5) = (1)\\
	\end{align*}
	If $K$ is a normal subgroup of $S_5$,
	then $\alpha^{-1} K \alpha \in K$ for all $\alpha \in S_5$.
	We know that $K$ must be some union of our conjugancy
	classes (otherwise if it did not contain an entire conjugancy class,
	it fails for $\alpha$ in that conjugancy class),
	and since it is a group, it must contain $(1)$.
	Given this, since conjugancy classes imply $\alpha^{-1}K\alpha \in K$,
	we just need to find the subgroups of $S_5$
	which are unions of conjugancy classes,
	and closed under multiplication.

	If $K \subseteq A_5$,
	then it contains only even conjugancy classes
	(note that conjugancy classes all share the same sign,
	since any element in a conjugancy class given by the representative
	above can be decomposed into transpositions similarily).
	These are $\overline{(12345)}, \overline{(123)}, \overline{(12)(34)}, \overline{(1)}$.
	Note that $(15)(14)\cdot(123) = (12345) \in \overline{(12345)}$,
	$(14)(15)\cdot(12345) = (123) \in \overline{(123)}$
	and $(315)\cdot(12345) = (12)(34)\overline{(12)(34)}$,
	thus the group is only closed when
	all of the even conjugancy classes are included in $K$,
	i.e. $K = A_5$,
	or $K = 1$.

	Now let $K \not\subseteq A_5$.
	Then $K$ must contain at least one of the odd conjugancy classes,
	so $\overline{(1234)}$, $\overline{(123)(45)}$
	and/or $\overline{(12)}$.
	But the product of two odd permutations is even,
	and so one can see that $A_5 \subseteq K$.
	Similar to before, we can construct products
	between the odd conjugancy classes and the even ones
	such that $K$ is only closed when all of the odd conjugancy
	classes are included.
	Thus $K = S_5$.
\end{proof}

\subsection*{Problem 4 (Ch. 1.12)}
{\it Show that if a finite group $G$ has a subgroup $H$ of index $n$
then $H$ contains a normal subgroup of $G$ of index a divisor of $n!$.
(\emph{Hint}: Consider the action of $G$ on $G/H$ by left translations.}
\begin{proof}[Solution]\let\qed\relax
	Consider the action of $G$ on $G/H$ by left translations.
	By definition, this a homomorphism $T \colon G \to \mathrm{Sym}(G/H)$
	(where $\mathrm{Sym}(G/H)$ are the bijective maps on $G/H$ to itself).
	Let $K = \ker(T)$.
	Note that $K$ is a normal subgroup of $G$
	by the fundamental theorem of homomorphisms.
	Furthermore, by the fundamental theorem,
	there is a bijection from $G/K$ to the image $T(G) \subseteq \mathrm{Sym}(G/H)$.
	Since $T$ is a homomorphism, $T(G)$ is a group,
	specifically a subgroup of $\mathrm{Sym}(G/H)$.
	Since the index of $G/H$ is $n$, there are $n$ elements in $G/H$,
	and so there are $n!$ elements in $\mathrm{Sym}(G/H)$
	(the number of permutations of $n$ elements).
	By Lagrange's theorem, the order of a subgroup divides
	the order of the group,
	and so $|G/K| = [G:K] \mid |\mathrm{Sym}(G/H)| = n!$.

	It remains to show that $K$ is a subgroup of $H$.
	The identity of $\mathrm{Sym}(G/H)$ is the identity
	map of the cosets of $G/H$.
	Since $K = \ker(T)$, this means that for any $k \in K$,
	$kxH = xH$ for all $x \in G$.
	Fixing $x = 1_G$, we have $kH = H$.
	But this is true only if $k \in H$ for all $k \in K$,
	thus $H$ contains a normal subgroup of index a divisor of $n!$.
\end{proof}

\subsection*{Problem 5 (Ch. 1.12)}
{\it Let $p$ be the smallest prime dividing the order of a finite group.
Show that any subgroup of $H$ of $G$ of index $p$ is normal.}
\begin{proof}[Solution]\let\qed\relax
	We apply Problem 4 from 1.12 (above):
	if $H$ is a subgroup of $G$ with index $p$,
	then $H$ contains a normal subgroup of $G$, call it $K$,
	such that $[G:K] \mid p!$.
	Now recall from Problem 2 of 1.7 of Jacobson
	that $[G:K] = [G:H][H:K]$ since $K \subset H$,
	thus $p = [G:H] \mid [G:K] \mid p!$.
	Thus $[G:K] = np$ for some $n \in \N$, $1 \leq n \leq p-1$.

	We note that $p$ is the smallest value (other than $1$)
	dividing the order of the group $G$.
	If there was a prime value smaller that divides $|G|$,
	we contradict our assumption that $p$ was the smallest,
	and if there was a composite value smaller,
	than there are primes that divide the compositie value
	which must be smaller than it, which also contradicts our assumption.
	
	By Lagrange's theorem, $|G| = |K|np \implies n \mid |G|$,
	but if $1 < n < p$,
	but since $p$ is the smallest value that divides $|G|$ other than $1$,
	we must have that $n = 1$.
	Thus $[G:K] = p = [G:H]$ as well,
	so $|K| = |H|$ by Lagrange's theorem (ie. $|H|p = |K|p = |G|$),
	and since $K$ is a subgroup of $H$,
	we have that $H = K$.
	Hence, $H$ is normal in $G$ (since $K$ was).
\end{proof}


\subsection*{Problem 6 (Ch. 1.12)}
{\it Show that every group of order $p^2$, $p$ is a prime, is abelian.
Show that up to isomorphism there are only two such groups.}
\begin{proof}[Solution]\let\qed\relax
	Let $G$ be a group with order $p^2$.
	For the sake of contradiction,
	assume that $G$ is not abelian.
	Then $|C(G)| = p^2$,
	and also by Theorem 1.11,
	$C(G) \neq 1$,
	thus since $|C(G)| \mid |G|$, we have that $|C(G)| = p$.
	Let $g \in G$ such that $g \not \in C(G)$.
	Since $[G:C(G)] = p$, $g$ and $C(G)$ generate $G$.
	Thus, if $a \in G$,
	we have that $a = g^ic$ where $c \in C(G)$ and $0 \leq i \leq p-1$.
	So if $b \in G$ such that $b = g^jc'$ with the same assumptions as before,
	we have that $ab = (g^ic)(g^jc') = g^ig^jcc' = (g^jc')(g^ic)$.
	But since $a,b \in G$ were arbitrary,
	this means $G$ is abelian, contradiction.

	Now, note that $G$ can be cyclic.
	If $G$ has an element with order $p^2$, $G$ is cyclic.
	If it is not, then all non-identity elements of $G$ has order $p$
	(to divide the order of the group).
	Any two elements $a,b$ will generate the group
	if $a \neq b^n$ for any $n$, since $p\cdot p = p^2$.
	Thus if two such elements are $a,b$, then $G = \langle a,b \rangle$.
\end{proof}


\subsection*{Problem 8 (Ch. 1.12)}
{\it Let $G$ act on $S$, $H$ act on $T$, and assume $S \cap T = \emptyset.$
Let $U = S \cup T$ and define for $g \in G$, $h \in H$, $s \in S$, $t \in T$;
$(g,h)s = gs$, $(g,h)t = ht$.
Show that this defines an action of $G \times H$ on $U$.}
\begin{proof}[Solution]\let\qed\relax
	Note that this map is well-defined,
	since if $u \in U$, then either $u \in S$ or $u \in T$
	but not both,
	so it always gets mapped to only one element in $U$.

	The identity of $G \times H$ is $(1_G, 1_H)$.
	If $u \in S$, then $(1_G, 1_H)u = 1_Gu = u$.
	If $u \in T$, then $(1_G, 1_H)u = 1_Hu = u$.

	Let $(g_1,h_1),(g_2,h_2) \in G \times H$.
	If $u \in S$, then $((g_1,h_1)(g_2,h_2))u = (g_1g_2,h_1h_2)u
	= g_1g_2u = (g_1,h_1)g_2u = (g_1,h_1)(g_2,h_2)u$.
	If $u \in T$, then $((g_1,h_1)(g_2,h_2))u = (g_1g_2,h_1h_2)u
	= h_1h_2u = (g_1,h_1)h_2u = (g_1,h_1)(g_2,h_2)u$.
	Thus, we have shown that this defines a group action.
\end{proof}

\subsection*{Problem 9 (Ch. 1.9)}
{\it A group $H$ is said to \emph{act on a group $K$ by automorphisms}
if we have an action of $H$ on $K$
and for every $h \in H$ the map $k \to hk$ of $K$ is an automorphism.
Suppose this is the case and let $H$ be the product set $K \times H$.
Define a binary composition in $K \times H$ by
\[
	(k_1, h_1)(k_2, h_2) = (k_1(h_1k_2), h_1h_2)
\]
and define $1=(1,1)$ -- the units of $K$ and $H$ respectively.
Verify that this defines a group such that $h \to (1,h)$
is a monomorphism of $H$ into $K \times H$
and $k \to (k,1)$ is a monomorphism of $K$ into $K \times H$
whose image is a normal subgroup.
$G$ is called a \emph{semi-direct product of $K$ and $H$}.
Note that if $H$ and $K$ are finite than $|K \times H| = |K||H|$.}
\begin{proof}[Solution]\let\qed\relax
	We first show that this defines a group.
	First, the binary composition is closed,
	since $h_1k_2 \in K$,
	since $k \to h_1k$ is an automorphism, so $h_1k \in K$
	for all $k$, which includes $k_2$;
	then the product of $k_1(h_1k_2) \in K$ by closure of $K$.
	Also, $h_1h_2 \in H$ by closure of $H$.
	Second, we show associativity:
	\begin{align*}
		((k_1,h_1)(k_2,h_2))(k_3,h_3)
		&= (k_1(h_1k_2), h_1h_2)(k_3,h_3)\\
		&= (k_1(h_1k_2)(h_1(h_2k_3)), (h_1h_2)h_3)\\
		&= (k_1(h_1(k_2(h_2k_3))), h_1(h_2h_3))\\
		&= (k_1,h_1)(k_2(h_2k_3), h_2h_3)\\
		&= (k_1,h_1)((k_2,h_2)(k_3,h_3))
	\end{align*}
	Where the third line is done by
	$(h_1k_2)(h_1(h_2k_3)) = h_1(k_2(h_2k_3))$
	since $\phi\colon k \to h_1k$ is an isomorphism (since its an automorphism),
	so $\phi(k_2(h_2k_3)) = \phi(k_2) \phi(h_2k_3)$.
	The unit is in $K \times H$, specifically $(1_K,1_H)$.
	Finally, we claim the inverse of $(k,h) \in K \times H$
	is $(h^{-1}k^{-1},h^{-1}) \in K \times H$.
	See:
	\[
		(k,h)(h^{-1}k^{-1},h^{-1}) = (k(h(h^{-1}k^{-1})),hh^{-1}) =
		(k(hh^{-1})k^{-1})),1_H) = (k(1_Hk^{-1}),1_H) = (kk^{-1}, 1_H) = (1_K,1_H)
	\]
	\[
		(h^{-1}(k^{-1},h^{-1})(k,h) = ((h^{-1}k^{-1})(h^{-1}k),h^{-1}h)
		= (k'^{-1}k',1_H) = (1_K,1_H)
	\]
	where we used the fact that $h^{-1}k = (h^{-1}k^{-1})^{-1}$
	since $k \to h^{-1}k$ is an isomorphism (since its an automorphism),
	and so we denoted $h^{-1}k = k' \in K$.
	Thus, ths defines a group.

	To verify $\phi\colon h \to (1,h)$ is a monomorphism,
	we see that it is a homomorphism since $\phi(h_1)\phi(h_2)
	= (1,h_1)(1,h_2) = (1(h_11),h_1h_2) = (1,h_1h_2) = \phi(h_1h_2)$
	where we have made the substitution $h_11 = 1$
	because $x \to h_1x$ is an isomorphism (since its an automorphism)
	and isomorphisms must map the identity to itself;
	to see that the map is injective,
	note that if $\phi(h) = \phi(h')$,
	then $(1,h) = (1,h')$ which is true if and only if $h = h'$.

	To verify $\psi\colon k \to (k,1)$ is a monomorphism,
	we see that it is a homomorphism since $\psi(k_1)\psi(k_2)
	= (k_1,1)(k_2,1) = (k_1(1k_2),1) = (k_1k_2,1) = \psi(k_1k_2)$
	where we have made the substitution $1k_2 = k_2$
	because $k \to 1k$ is the identity automorphism and so
	maps $k_2$ to itself;
	to see that the map is injective,
	note that if $\psi(k) = \psi(k')$,
	then $(k,1) = (k',1)$ which is true if and only if $k = k'$.

	Finally, to see that the image of $\psi$ is normal subgroup of $K \times H$,
	it remains only to show that the image is normal,
	since by the fundamental theorem of homomorphisms,
	since $K$ is a group and $k \to (k,1)$ is a homomorphism (monomorphism),
	then the image is a subgroup of $K \times H$.
	If $(k',1) = \psi(k') \in \psi^{\mathrm{img}}(K)$ for some $k' \in K$,
	consider $(h^{-1}k^{-1},h)(k',1)(k,h)$ for arbitrary $(k,h) \in K \times H$
	(where we have shown previously the expression on the left is our inverse).
	See
	\begin{align*}
		(h^{-1}k^{-1},h^{-1})(k',1)(k,h)
		&= ((h^{-1}k^{-1})(h^{-1}k'), h^{-1})(k,h)\\
		&= (h^{-1}(k^{-1}k'), h^{-1})(k,h)\\
		&= (h^{-1}(k^{-1}k')(h^{-1}k), h^{-1}h)\\
		&= (h^{-1}(k^{-1}k'k), 1)
	\end{align*}
	(where we have been pulling out $h^{-1}$ since
	$k \to h^{-1}k$ is an isomorphism from the fact it is an automorphism,
	and so $(h^{-1}k)(h^{-1}k') = h^{-1}(kk')$).
	But $k^{-1}k'k \in K$ so $h^{-1}(k^{-1}k'k)\in K$
	since $k \to h^{-1}k$ is an automorphism of $K$.
	Thus, $(h^{-1}(k^{-1}k'k), 1) = \psi(h^{-1}(k^{-1}k'k))$,
	thus $(h^{-1}k^{-1},h^{-1})(k',1)(k,h) \in \psi^{\mathrm{img}}(K)$,
	so the subgroup is normal.
\end{proof}
\end{document}
