\documentclass{article}
\usepackage{amsmath, amsfonts, amsthm, amssymb}
\usepackage{geometry}
\geometry{letterpaper, margin=2.0cm, includefoot, footskip=30pt}

\usepackage{fancyhdr}
\pagestyle{fancy}

\lhead{Math 322}
\chead{Homework 5}
\rhead{Nicholas Rees, 11848363}
\cfoot{Page \thepage}

\newcommand{\N}{{\mathbb N}}
\newcommand{\Z}{{\mathbb Z}}
\newcommand{\Q}{{\mathbb Q}}
\newcommand{\R}{{\mathbb R}}
\newcommand{\C}{{\mathbb C}}
\newcommand{\ep}{{\varepsilon}}

\renewcommand{\theenumi}{(\alph{enumi})}

\begin{document}
\subsection*{Problem 1 (Ch. 1.6)}
{\it Write $(456)(567)(671)(123)(234)(345)$ as a product of disjoint cycles.}
\begin{proof}[Solution]\let\qed\relax
	We can explicitly write down where each element
	in the domain of the product of the $3$-cycles,
	specifically $1,2,\dots,7$,
	go according to the $(467)(567)(671)(123)(234)(345)$:
	\begin{align*}
		1 &\mapsto 2\\
		2 &\mapsto 7\\
		3 &\mapsto 3\\
		4 &\mapsto 4\\
		5 &\mapsto 5\\
		6 &\mapsto 6\\
		7 &\mapsto 1
	\end{align*}
	By inspection, we can find that the disjoint cycles in this is
	$(1247)$, $(56)$ and $(3)$, thus
	\[
		(467)(567)(671)(123)(234)(345) = (127)
	\]
\end{proof}

\subsection*{Problem 2 (Ch. 1.6)}
{\it Show that if $n \geq 3$ then $A_n$ is generated by the $3$-cycles $(abc)$.}
\begin{proof}[Solution]\let\qed\relax
	Let $E_n = \{(abc) \colon a,b,c\in \N, a,b,c \leq n\}$ be the set of $3$-cycles in $S_n$.
	We seek to prove that $\langle E_n\rangle = A_n$.
	Let $\eta \in \langle E_n\rangle$.
	Since $(abc) = (ac)(ab)$,
	each $3$-cycle $(abc)$ in $\eta$ can be decomposed into two transpositions,
	so if $\eta$ was the product of $k$ $3$-cycles,
	$\eta$ is equal to the product of $2k$ transpositions,
	which is even, thus $\eta \in A_n$.
	Thus $E_n \subseteq A_n$.
	
	Now let $\alpha \in A_n$.
	Note that by the first part of question 5 of this homework,
	we can decompose $\alpha$ into a product of transpositions of the form $(1\;m)$
	where $1 \leq m \leq n$
	(it is fine to cite this solution;
	my solution to 5 does not rely on this problem,
	and if I really wanted to,
	I could write the proof of 5 above this).
	Note that this will always be an even number of transpositions,
	by the definition of $\alpha$ being even.
	Now, consider pairing up neighbouring elements $(1 \; m_1)(1 \; m_2)$
	from the decomposition of $\alpha$
	(and each will have a pair by evenness of number of transpositions).
	Then we can rewrite this as $(1 \; m_1)(1 \; m_2) = (1\; m_2 \; m_1)$ (one can easily verify this is true).
	But then $\alpha$ can just be written as a product of $3$-cycles,
	so $\alpha \in E_n$.
	Thus $A_n \subseteq E_n$ since $\alpha$ was arbitrary.
	This shows that $A_n = E_n$.
\end{proof}

\subsection*{Problem 3 (Ch. 1.6)}
{\it Determine the sign of the permutation
\[
	\begin{pmatrix}
		1 & 2 & \cdots & n-1 & n\\
		n & n-1 & \cdots & 2 & 1
	\end{pmatrix}
\]}
\begin{proof}[Solution]\let\qed\relax
	Call our permutation $\alpha$.
	Consider when $n$ is even.
	Then we can rewrite $n = 2k$.
	Then one can easily see that we can decompose this permutation
	as the product of disjoint cycles below
	\[
		(1\;n)(2\;[n-1]) \cdots (k\;\left[k+1\right])
	\]
	Then, there are $k = \frac{n}{2}$ transpositions,
	thus the sign of $\alpha$ is the parity of $k$,
	or $\mathrm{sgn}(\alpha) = (-1)^k$.

	If $n$ is odd,
	we can rewrite as $n = 2k + 1$.
	We then decompose the permutation as the following product of disjoint cycles:
	\[
		(1\;n)(2\;[n-1]) \cdots
		(k\;[k+1])
	\]
	Then, there are $k = (n-1)/2$ transpositions,
	thus the sign of the permutation is the parity of $k$,
	or $\mathrm{sgn}(\alpha) = (-1)^k$.

	We can summarize these two statements by noticing that this result is dependent on $n \;(\mathrm{mod}\; 4)$.
	If $n$ is even, and if $n/2$ is even, then our permutation is even;
	if $n/2$ is odd, then our permutation is odd.
	So $n \equiv 0 \; (\mathrm{mod} \; 4) \implies \mathrm{sgn}(\alpha) = 1$
	and $n \equiv 2 \; (\mathrm{mod} \; 4) \implies \mathrm{sgn}(\alpha) = -1$.
	If $n$ is odd, and if $(n-1)/2$ is even, then our permutation is even;
	if $(n-1)/2$ is odd, then our permutation is odd.
	So $n \equiv 1 \; (\mathrm{mod} \; 4) \implies \mathrm{sgn}(\alpha) = 1$
	and $n \equiv 3 \; (\mathrm{mod} \; 4) \implies \mathrm{sgn}(\alpha) = -1$.
	Thus
	\[
		\mathrm{sgn}(\alpha) = 
		\begin{cases}
			1 &\text{if }n \equiv 0,1 \; (\mathrm{mod} \; 4)\\ 
			-1 &\text{if }n \equiv 2,3 \; (\mathrm{mod} \; 4)\\ 
		\end{cases}
	\]
\end{proof}


\subsection*{Problem 4 (Ch. 1.6)}
{\it Show that if $\alpha$ is any permutation then
\[
	\alpha(i_1i_2\cdots i_r)\alpha^{-1}
	= (\alpha(i_1)\alpha(i_2)\cdots\alpha(i_r))
\]}
\begin{proof}[Solution]\let\qed\relax
	First note that $\alpha(i_1i_2\cdots i_r)\alpha^{-1}
	= (\alpha(i_1)\alpha(i_2)\cdots\alpha(i_r))$
	if and only if
	\[
		\alpha(i_1i_2\cdots i_r)
	= (\alpha(i_1)\alpha(i_2)\cdots\alpha(i_r))\alpha
\]
	It is sufficient to show that the permutations on either side
	map each element of the ambient set to the same element.

	Consider $j \not\in \{i_1,\dots,i_r\}$.
	Then our left hand side maps $j$ to $\alpha(j)$
	(since a cycle will just map any element not included in the cycle to itself).
	Furthermore, $\alpha(j) \not \in \{\alpha(i_1)\cdots\alpha(i_r)\}$
	since permutations are injective,
	thus the right hand side maps $j$ to $\alpha(j)$ as well.

	Now let $m \in \N, 1 \leq m \leq r$ be arbitrary.
	Then
	\[
		\big(\alpha\circ(i_1i_2\cdots i_r)\big) (i_m) = \alpha(i_{m+1})
		= (\alpha(i_1) \alpha(i_2) \cdots \alpha(i_m) \alpha(i_{m+1}) \cdots \alpha(i_r)) (\alpha(i_m)) = \big((\alpha(i_1) \cdots \alpha(i_ra))\circ \alpha\big) (i_m) 
	\]
	But since this is true for all elements in $\{i_1,\dots,i_r\}$,
	and we already proved the case for all the elements not in this set,
	we have shown that our two permutations map every element in the ambient set
	to the same element,
	and so permutations on the left and the right are equal,
	thus our original two permutations are equal as well.
\end{proof}

\subsection*{Problem 5 (Ch. 1.6)}
{\it Show that $S_n$ is generated by the $n-1$ transpositions
$(12), (13), \cdots, (1n)$ and also by the $n-1$ transpositions
$(12),(23),\cdots,(n-1n)$.}
\begin{proof}[Solution]\let\qed\relax
	First note that $\langle(12),\dots,(1n)\rangle \subseteq S_n$.
	This follows from the definition of $S_n$.

	Now, let $\alpha \in S_n$ be an arbitrary permutation.
\iffalse
	We can decompose $\alpha$ into a unique composition of disjoint cycles,
	explicitly, if there are $m$ many cycles:
	\[
		\alpha = (i_1^1 i_2^1 \cdots i_{r_1}^1)(i_1^2 i_2^2 \cdots i_{r_2}^2)
		\cdots (i_1^m i_2^m \cdots i_{r_m}^m)
	\]
	Note that the $j$th cycle (where $1 \leq j \leq m$) when $1 \not\in \{i_1^j,i_2^j\dots,i_{r_j}^j\}$ is
	\[
		(i_1^j i_2^j \cdots i_{r_j}^j) = (1 i_1^j)(1 i_{r_j}^j) (1 i_{r_j-1}^j) \cdots(1 i_2^j) (1 i_1^j)
	\]
	One can check this by manually checking how both sides map each element $i \in \{1,2,\dots,n\}$
	to the same element
	(since for any two distinct elements in $S_n$,
	there exists $i\in \{1,2,\dots,n\}$ where they
	map $i$ to a different element in $\{1,2,\dots,n\}$,
	thus they the same if there is no such $i$).
	If $i \not\in \{i_1^{j},i_2^{j},\dots,i_{r_j}^j\} \cup \{1\}$,
	then the permutation on either side maps this to itself.
	Now if $i = 1$,
	the permutation on the left maps $1$ to itself,
	and the permutation on the right first transposes $1$ to $i^j_{r_j}$,
	and then it gets transposed back to $1$,
	since no other transposition maps $i^j_{r_j}$ anywhere;
	thus the permutations affect $1$ the same.
	Finally, if $i \in \{i_1^{j},i_2^{j},\dots,i_{r_j}^j\}$,
	we can write this $i = i_k^j$.
	The permutation on the left will map this to $i_{k+1}^j$ (or $i_1^j$ if $k=r$).
	To consider the permutation on the right,
	note that only $(1i^j_{k+1})(1i^j_k)$ alters the position of $i^j_k$
	(after this, $i^j_k$ is in the $i^j_{k+1}$th position,
	which never gets transposed again,
	since $1$ is not in the cycle).
	This maps $i^j_k$ to $i^j_{k+1}$
	(again treating $i^j_{r+1} = i^j_1$).
	But then the two permutations affect $i_k$ the same,
	thus the two permutations are equal.

	Now, if the $j$th cycle has $1$ in it, namely $1 \in \{i_1^j,i_2^j\dots,i_{r_j}^j\}$,
	then the $j$th cycle is
	\[
		(i_1^j i_2^j \cdots i_{r_j}^j) = (1 i_1^j)(1 i_{r_j}^j) (1 i_{r_j-1}^j) \cdots(1 i_2^j)
	\]
	As before, if $i \not\in \{i_1^j,\dots,i_{r_j}^j\}$,
	then the permutations on either side will map $i$ to itself.
	Now let $i \in \{i_1^j,\dots,i_{r_j}^j\}$.
	We can denote this as $i = i_k^j$.
	The permutation on the left will map this to $i_{k+1}^j$ (or $i_1^j$ if $k=r$).
	To consider the permutation on the right,
	note that only $(1i^j_{k+1})(1i^j_k)$ alters the position of $i^j_k$
	(after this, $i^j_k$ is in the $i^j_{k+1}$th position,
	which never gets transposed again).
	This maps $i^j_k$ to $i^j_{k+1}$
	(again treating $i^j_{r+1} = i^j_1$).
	But then the two permutations affect $i_k$ the same,
	thus the two permutations are equal.

	Regardless, in either case, any arbitrary cycle can be decomposed
	as products of $(12),\dots,(1n)$,
	thus $\alpha \in \langle (12),\dots,(1n)\rangle$,
	so $S_n \subseteq \langle (12),\dots,(1n)\rangle$.
	Therefore,
	we have shown $S_n = \langle (12),\dots,(1n)\rangle$.
\fi
	It was shown in Jacobson that we can decompose any $\alpha$
	into a product of transpositions.
	But for any transposition $(a \; b)$, $a,b \in \{1,\dots,n\}$,
	we can easily see that $(a\; b) = (1\; a)(1 \; b)(1 \; a)$
	(one can easily verify this is true).
	But this applies to every transposition in the decomposition of $\alpha$,
	thus $\alpha$ can be decomposed in terms of a product of elements
	of the form $(1\; a)(1 \; b)(1 \; a)$.
	But then $\alpha \in \langle (12),\dots,(1n)\rangle$,
	so $S_n \subseteq \langle (12),\dots,(1n)\rangle$.
	Therefore,
	we have shown $S_n = \langle (12),\dots,(1n)\rangle$.

	We now seek to prove $S_n = \langle (12),(23),\dots,(n-1 n)\rangle$.
	Note that $\langle(12),(23),\dots,(n-1n)\rangle \subseteq S_n$,
	by definition of $S_n$.
	So it sufficies to prove that $S_n \subseteq \langle (12),(23),\dots,(n-1 n)\rangle$.
	But since we have just shown $S_n = \langle (12),\dots,(1n)\rangle$,
	it suffices to prove that $\langle (12),\dots,(1n)\rangle
	\subseteq \langle (12),(23),\dots,(n-1 n)\rangle$.
	Consider $(1m)$ where $1 \leq m \leq n$.
	Then
	\[
		(1m) = (m-1 \;m)(m-2\; m-1)\cdots (23)(12)(23)\cdots (m-2\; m-1)(m-1\; m)
	\]
	One can check this by manually checking that both sides of the equality map each element $i \in \{1,2,\dots,n\}$
	to the same element
	(since for any two distinct elements in $S_n$,
	there exists $i\in \{1,2,\dots,n\}$ where they
	map $i$ to a different element in $\{1,2,\dots,n\}$,
	thus they are the same if there is no such $i$).
	First, note that if $i \neq 1,m$,
	then $i$ is mapped to itself with the permutation on both sides of the equality.
	If $i = 1,m$, the permutation on the left transposes them.
	The permutation on the right maps $m$ to $1$,
	since $m$ gets mapped to the integer directly below it with each transposition
	until $(12)$,
	and then $1$ is never permuted again, so $m$ stays at $1$;
	and it maps $1$ to $m$,
	since it is not transposed before $(12)$,
	and then after,
	each transposition maps it to the integer directly above it until it reaches $m$.
	Thus the permutations on either side of the equality are equal.
	But since $m$ was arbitrary,
	any element in $\langle (12),\dots,(1n)\rangle$ can be written as a product of
	transpositions from $\{(12),(23),\dots,(n-1\;n)\}$,
	thus $\langle (12),\dots,(1n)\rangle \subseteq \langle(12),(23),\dots,(n-1\;n)\rangle$.
	And so we have $S_n \subseteq \langle(12),(23),\dots,(n-1\;n)\rangle$,
	therefore $S_n = \langle(12),(23),\dots,(n-1\;n)\rangle$.
\end{proof}
\end{document}
