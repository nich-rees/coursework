\documentclass{article}
\usepackage{amsmath, amsfonts, amsthm, amssymb}
\usepackage{geometry}
\geometry{letterpaper, margin=2.0cm, includefoot, footskip=30pt}

\usepackage{fancyhdr}
\pagestyle{fancy}

\lhead{Math 322}
\chead{Homework 5}
\rhead{Nicholas Rees, 11848363}
\cfoot{Page \thepage}

\newcommand{\N}{{\mathbb N}}
\newcommand{\Z}{{\mathbb Z}}
\newcommand{\Q}{{\mathbb Q}}
\newcommand{\R}{{\mathbb R}}
\newcommand{\C}{{\mathbb C}}
\newcommand{\ep}{{\varepsilon}}

\renewcommand{\theenumi}{(\alph{enumi})}

\begin{document}
\subsection*{Problem 1 (Ch. 1.6)}
{\it Write $(456)(567)(671)(123)(234)(345)$ as a product of disjoint cycles.}
\begin{proof}[Solution]\let\qed\relax
	We can explicitly write down where each element
	in the domain of the product of the $3$-cycles,
	specifically $1,2,\dots,7$,
	go according to the $(467)(567)(671)(123)(234)(345)$:
	\begin{align*}
		1 &\mapsto 2\\
		2 &\mapsto 7\\
		3 &\mapsto 3\\
		4 &\mapsto 4\\
		5 &\mapsto 5\\
		6 &\mapsto 6\\
		7 &\mapsto 1
	\end{align*}
	By inspection, we can find that the disjoint cycles in this is
	$(1247)$, $(56)$ and $(3)$, thus
	\[
		(467)(567)(671)(123)(234)(345) = (127)
	\]
\end{proof}

\subsection*{Problem 2 (Ch. 1.6)}
{\it Show that if $n \geq 3$ then $A_n$ is generated by the $3$-cycles $(abc)$.}
\begin{proof}[Solution]\let\qed\relax
	Let $E_n = \{(abc) \colon a,b,c\in \N, a,b,c \leq n\}$ be the set of $3$-cycles in $S_n$.
	We seek to prove that $\langle E_n\rangle = A_n$.
	Let $\eta \in \langle E_n\rangle$.
	Since $(abc) = (ac)(ab)$,
	each $3$-cycle $(abc)$ in $\eta$ can be decomposed into two transpositions,
	so if $\eta$ was the product of $k$ $3$-cycles,
	$\eta$ is equal to the product of $2k$ transpositions,
	which is even, thus $\eta \in A_n$.
	Thus $E_n \subseteq A_n$.
	
	Now let $\alpha \in A_n$.
	Note that for any $\alpha$,
	we can write it as $(i \; \alpha(i))(\alpha(i) \; \alpha^2(i)) \cdots$.
	And then pair up... comeback after question 5.
\end{proof}

\subsection*{Problem 3 (Ch. 1.6)}
{\it Determine the sign of the permutation
\[
	\begin{pmatrix}
		1 & 2 & \cdots & n-1 & n\\
		n & n-1 & \cdots & 2 & 1
	\end{pmatrix}
\]}
\begin{proof}[Solution]\let\qed\relax
	Call our permutation $\alpha$.
	Consider when $n$ is even.
	Then we can rewrite $n = 2k$.
	Then one can easily see that we can decompose this permutation
	as the product of disjoint cycles below
	\[
		(1\;n)(2\;[n-1]) \cdots (k\;\left[k+1\right])
	\]
	Then, there are $k = \frac{n}{2}$ transpositions,
	thus the sign of $\alpha$ is the parity of $k$,
	or $\mathrm{sgn}(\alpha) = (-1)^k$.

	If $n$ is odd,
	we can rewrite as $n = 2k + 1$.
	We then decompose the permutation as the following product of disjoint cycles:
	\[
		(1\;n)(2\;[n-1]) \cdots
		(k\;[k+1])
	\]
	Then, there are $k = (n-1)/2$ transpositions,
	thus the sign of the permutation is the parity of $k$,
	or $\mathrm{sgn}(\alpha) = (-1)^k$.

	We can summarize these two statements by noticing that this result is dependent on $n \;(\mathrm{mod}\; 4)$.
	If $n$ is even, and if $n/2$ is even, then our permutation is even;
	if $n/2$ is odd, then our permutation is odd.
	So $n \equiv 0 \; (\mathrm{mod} \; 4) \implies \mathrm{sgn}(\alpha) = 1$
	and $n \equiv 2 \; (\mathrm{mod} \; 4) \implies \mathrm{sgn}(\alpha) = -1$.
	If $n$ is odd, and if $(n-1)/2$ is even, then our permutation is even;
	if $(n-1)/2$ is odd, then our permutation is odd.
	So $n \equiv 1 \; (\mathrm{mod} \; 4) \implies \mathrm{sgn}(\alpha) = 1$
	and $n \equiv 3 \; (\mathrm{mod} \; 4) \implies \mathrm{sgn}(\alpha) = -1$.
	Thus
	\[
		\mathrm{sgn}(\alpha) = 
		\begin{cases}
			1 &\text{if }n \equiv 0,1 \; (\mathrm{mod} \; 4)\\ 
			-1 &\text{if }n \equiv 2,3 \; (\mathrm{mod} \; 4)\\ 
		\end{cases}
	\]
\end{proof}


\subsection*{Problem 4 (Ch. 1.6)}
{\it Show that if $\alpha$ is any permutation then
\[
	\alpha(i_1i_2\cdots i_r)\alpha^{-1}
	= (\alpha(i_1)\alpha(i_2)\cdots\alpha(i_r))
\]}
\begin{proof}[Solution]\let\qed\relax
	First note that $\alpha(i_1i_2\cdots i_r)\alpha^{-1}
	= (\alpha(i_1)\alpha(i_2)\cdots\alpha(i_r))$
	if and only if
	\[
		\alpha(i_1i_2\cdots i_r)
	= (\alpha(i_1)\alpha(i_2)\cdots\alpha(i_r))\alpha
\]
	It is sufficient to show that the permutations on either side
	map each element of the ambient set to the same element.

	Consider $j \not\in \{i_1,\dots,i_r\}$.
	Then our left hand side maps $j$ to $\alpha(j)$
	(since a cycle will just map any element not included in the cycle to itself).
	Furthermore, $\alpha(j) \not \in \{\alpha(i_1)\cdots\alpha(i_r)\}$
	since permutations are injective,
	thus the right hand side maps $j$ to $\alpha(j)$ as well.

	Now let $m \in \N, 1 \leq m \leq r$ be arbitrary.
	Then
	\[
		\big(\alpha\circ(i_1i_2\cdots i_r)\big) (i_m) = \alpha(i_{m+1})
		= (\alpha(i_1) \alpha(i_2) \cdots \alpha(i_m) \alpha(i_{m+1}) \cdots \alpha(i_r)) (\alpha(i_m)) = \big((\alpha(i_1) \cdots \alpha(i_ra))\circ \alpha\big) (i_m) 
	\]
	But since this is true for all elements in $\{i_1,\dots,i_r\}$,
	and we already proved the case for all the elements not in this set,
	we have shown that our two permutations map every element in the ambient set
	to the same element,
	and so permutations on the left and the right are equal,
	thus our original two permutations are equal as well.
\end{proof}

\subsection*{Problem 5 (Ch. 1.6)}
{\it Show that $S_n$ is generated by the $n-1$ transpositions
$(12), (13), \cdots, (1n)$ and also by the $n-1$ transpositions
$(12),(23),\cdots,(n-1n)$.}
\begin{proof}[Solution]\let\qed\relax
	Let $\alpha \in S_n$ be an arbitrary permutation.

	We can decompose $\alpha$ into a unique composition of disjoint cycles,
	explicitly, if there are $m$ many cycles:
	\[
		\alpha = (i_1^1 i_2^1 \cdots i_{r_1}^1)(i_1^2 i_2^2 \cdots i_{r_2}^2)
		\cdots (i_1^m i_2^m \cdots i_{r_m}^m)
	\]
	Note that the $j$th cycle (where $1 \leq j \leq m$) when $1 \not\in \{i_1^j,i_2^j\dots,i_{r_j}^j\}$ is
	\[
		(i_1^j i_2^j \cdots i_{r_j}^j) = (1 i_1^j)(1 i_{r_j}^j) (1 i_{r_j-1}^j) \cdots(1 i_2^j) (1 i_1^j)
	\]
	One can check this by manually checking how both sides map each element $i \in \{1,2,\dots,n\}$
	to the same element
	(since for any two distinct elements in $S_n$,
	there exists $i\in \{1,2,\dots,n\}$ where they
	map $i$ to a different element in $\{1,2,\dots,n\}$,
	thus they the same if there is no such $i$).
	If $i \not\in \{i_1^{j},i_2^{j},\dots,i_{r_j}^j\} \cup \{1\}$,
	then the permutation on either side maps this to itself.
	Now if $i = 1$, we have that $1$ first gets transposed to $i^j_{r_j}$,
	and then gets transposed back to $1$ 

	ff Don't forget two way set inclusion.
\end{proof}

\end{document}
