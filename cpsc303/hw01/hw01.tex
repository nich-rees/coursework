\documentclass{article}
\usepackage{amsmath, amsfonts, amsthm, amssymb}
\usepackage{geometry}
\geometry{letterpaper, margin=2.0cm, includefoot, footskip=30pt}

\usepackage{fancyhdr}
\pagestyle{fancy}

\lhead{CPSC 303}
\chead{Homework 1}
\rhead{Nicholas Rees, 11848363}
\cfoot{Page \thepage}

\newcommand{\N}{{\mathbb N}}
\newcommand{\Z}{{\mathbb Z}}
\newcommand{\Q}{{\mathbb Q}}
\newcommand{\R}{{\mathbb R}}
\newcommand{\C}{{\mathbb C}}
\newcommand{\ep}{{\varepsilon}}

\newtheorem{lemma}{Lemma}

\renewcommand{\theenumi}{(\alph{enumi})}

\begin{document}
\subsection*{Problem 1}
Who are your group members?
\begin{proof}[Solution]\let\qed\relax
	Nicholas Rees
\end{proof}

\subsection*{Problem 2}
\begin{enumerate}
	\item Show that the linear system
		\[
			\begin{bmatrix}
				1 & 1 & 1\\
				0 & 1 & 2\\
				0 & 1 & 4
			\end{bmatrix}
			\begin{bmatrix} c_0 \\ c_1 \\ c_2 \end{bmatrix}
			= \begin{bmatrix} 0 \\ 1 \\ 0 \end{bmatrix}
		\]
		has the unique solution
		\[
			\begin{bmatrix} c_0 \\ c_1 \\ c_2 \end{bmatrix}
			= \begin{bmatrix} -3/2 \\ 4/2 \\ -1/2 \end{bmatrix}
		\]
	\item Say that $f \colon \R \to \R$ has $f'''(x)$ existing for all $x$.
		Say that $x_0,h \in \R$, and that $\lvert f'''(\xi)\rvert \leq M_3$
		for all $\xi$ between $x_0$ and $x_0 + 2h$.
		Use the fact that
		\begin{align*}
			f(x_0) &= f(x_0)\\
			f(x_0+h) &= f(x_0) + hf'(x_0) + \frac{h^2}{2}f''(x_0) + O(h^3)M_3\\
			f(x_0+2h) &= f(x_0) + 2hf'(x_0) + \frac{(2h)^2}{2}f''(x_0) + O(h^3)M_3
		\end{align*}
		to find a value of $c_0,c_1,c_2$ such that
		\[
			c_0f(x_0) + c_1f(x_0 + h) + c_2f(x_0 + 2h)
			= hf'(x_0) + O(h^3)M_3
		\]
	\item To which formula on page 411 (Section 14.1) of [A\&G]
		are parts (a) and (b) related? Explain.
	\item What is the significance of the solution of
		\[
			\begin{bmatrix}
				1 & 1 & 1 & 1\\
				0 & 1 & 2 & 3\\
				0 & 1 & 4 & 9\\
				0 & 1 & 8 & 27
			\end{bmatrix}
			\begin{bmatrix} c_0 \\ c_1 \\ c_2 \\ c_3 \end{bmatrix}
			= \begin{bmatrix} 0 \\ 1 \\ 0 \\ 0 \end{bmatrix}
		\]
		to approximating $f'(x_0)$?
		[You don't have to solve this system, just state what you can do
		with the solution $c_0,c_1,c_2,c_3$.]
\end{enumerate}
\begin{enumerate}
	\item \begin{proof}[Solution]\let\qed\relax
		Let $A = \begin{bmatrix}
				1 & 1 & 1\\
				0 & 1 & 2\\
				0 & 1 & 4
			\end{bmatrix}$.
		Note that $\det(A) = 1(4-2) - 0 + 0 = 2 \neq 0$ so $A^{-1}$ exists.
		One can compute $A^{-1}$ by finding the adjoint matrix and dividing by $\det(A)$,
		which gives $A^{-1} = \begin{bmatrix}
				1 & -3/2 & 1/2\\
				0 & 2 & -1\\
				0 & -1/2 & 1/2
			\end{bmatrix}$
		(and one can check $AA^{-1} = A^{-1}A = I$).
		Thus, multiplying by $A^{-1}$ on both sides, we get
		\[
			\begin{bmatrix} c_0 \\ c_1 \\ c_2 \end{bmatrix}
			= \begin{bmatrix}
				1 & -3/2 & 1/2\\
				0 & 2 & -1\\
				0 & -1/2 & 1/2
			\end{bmatrix}
			\begin{bmatrix} 0 \\ 1 \\ 0 \end{bmatrix}
			= \begin{bmatrix} -3/2 \\ 2 \\ -1/2 \end{bmatrix}
		\]
		hence $c_0,c_1,c_2$ must be $-3/2,4/2,-1/2$, respectively,
		thus uniquely determining our solution to be the desired one.
	\end{proof}
	\item \begin{proof}[Solution]\let\qed\relax
		We seek to find a linear combination of $f(x_0),f(x_0 + h), f(x_0 + 2h)$
		that adds up to only being $hf'(x_0) + O(h^3)M_3$.
		But this actually corresponds exactly to our linear transformation
		$A$ from part (a),
		where we are changing bases from $f(x_0),f(x_0+h),f(x_0+2h)$
		to $f(x_0),hf'(x_0),\frac{h^2}{2}f''(x_0) + O(h^3)M_3$,
		and our equations that dictate this equivalence is represented in $A$.
		Hence, our coefficients are the solution to part (a),
		namely, $c_0 = -3/2$, $c_1 = 4/2$, $c_2 = -1/2$.
	\end{proof}
	\item \begin{proof}[Solution]\let\qed\relax
		Nicholas Rees
	\end{proof}
	\item \begin{proof}[Solution]\let\qed\relax
		Nicholas Rees
	\end{proof}
\end{enumerate}

\subsection*{Problem 3}
Consider an ODE $y' = f(t,y)$, where as in [A\&G], $y = y(t)$,
and $y'$ refers to $dy/dt$.
Say that $f(t,y)$ is of the special form $f(t,y) = h(t)g(y)$,
where $g$ is a differentiable function and $h$ is continuous.
Then the ODE
\[
	y' = dy/dt = h(t)g(y)
\]
is called a \emph{separable differential equation}, and it can be solved by writing
\[
	\frac{dy}{g(y)} = h(t)dt
\]
and taking indefinite integrals of both sides.
See Section 2.4 (Separable ODE's) of UBC's Calculus 2 Textbook for details,
including Example 2.4.2 there, where they solve the equation $y' = y^2$
(in this textbook, $y'$ refers to $dy/dx$, as is common in math books).
\begin{enumerate}
	\item Solve the ODE $y' = y^3$
		(here $y = y(t)$ and $y'$ refers to $dy/dt$) in the same manner
		as $y' = y^2$ is solved in general form.
	\item Solve $y' = y^3$ for the initial condition $y(1) = 1$.
	\item Solve $y' - y^4$ for the initial condition $y(1) = 1$.
	\item Let $y(t)$ be as in part (b); for $t \geq 1$,
		when does $y(t)$ fail to exist, i.e.,
		for what $T > 1$ does $y(t) \to \infty$ as $t \to T$?
	\item Same question for part (c).
\end{enumerate}
\begin{enumerate}
	\item \begin{proof}[Solution]\let\qed\relax
		Nicholas Rees
	\end{proof}
	\item \begin{proof}[Solution]\let\qed\relax
		Nicholas Rees
	\end{proof}
	\item \begin{proof}[Solution]\let\qed\relax
		Nicholas Rees
	\end{proof}
	\item \begin{proof}[Solution]\let\qed\relax
		Nicholas Rees
	\end{proof}
	\item \begin{proof}[Solution]\let\qed\relax
		Nicholas Rees
	\end{proof}
\end{enumerate}

\subsection*{Problem 4}
Let $y(t) = (3-2t)^{-1/2}, z(t) = (4-3t)^{-1/3}$.
\begin{enumerate}
	\item Examine a plot of $y(t)$ and $z(t)$ for $1 \leq t < 4/3$.
		Is one of these functions larger than the other in the entire interval
		9as far as the plot shows)?
		(Here a simple answer will do.
		You might type \newline \verb|plot (3-2t)^(-1/2) and (4-3t)^(-1/3)|
		into Google, or something like that.)
	\item Show that $y' = y^3$ and $z' = z^4$ and that $y(1) = z(1) = 1$.
	\item Show that $y'(1) = z'(1) = 1$.
	\item Show that $y''(1) = 3$ and $z''(1) = 4$.
		[Hint: differentiate both sides of $y' = y^3$; similarly for $z$.]
	\item Show that for $h > 0$ and $h$ sufficiently small we have $y(1+h) < z(1+h)$.
		[Hint: let $u(t) = z(t) - y(t)$; what are the values of $u(1),u'(1),u''(1)$?]
\end{enumerate}
\end{document}
