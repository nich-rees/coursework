\documentclass{article}
\usepackage{amsmath, amsfonts, amsthm, amssymb}
\usepackage{geometry}
\geometry{letterpaper, margin=2.0cm, includefoot, footskip=30pt}

\usepackage{fancyhdr}
\pagestyle{fancy}

\lhead{CPSC 303}
\chead{Homework 1}
\rhead{Nicholas Rees, 11848363}
\cfoot{Page \thepage}

\newcommand{\N}{{\mathbb N}}
\newcommand{\Z}{{\mathbb Z}}
\newcommand{\Q}{{\mathbb Q}}
\newcommand{\R}{{\mathbb R}}
\newcommand{\C}{{\mathbb C}}
\newcommand{\ep}{{\varepsilon}}

\newtheorem{lemma}{Lemma}

\renewcommand{\theenumi}{(\alph{enumi})}

\begin{document}
\subsection*{Problem 1}
Who are your group members?
\begin{proof}[Solution]\let\qed\relax
	Nicholas Rees
\end{proof}

\subsection*{Problem 2}
\begin{enumerate}
	\item Show that the linear system
		\[
			\begin{bmatrix}
				1 & 1 & 1\\
				0 & 1 & 2\\
				0 & 1 & 4
			\end{bmatrix}
			\begin{bmatrix} c_0 \\ c_1 \\ c_2 \end{bmatrix}
			= \begin{bmatrix} 0 \\ 1 \\ 0 \end{bmatrix}
		\]
		has the unique solution
		\[
			\begin{bmatrix} c_0 \\ c_1 \\ c_2 \end{bmatrix}
			= \begin{bmatrix} -3/2 \\ 4/2 \\ -1/2 \end{bmatrix}
		\]
	\item Say that $f \colon \R \to \R$ has $f'''(x)$ existing for all $x$.
		Say that $x_0,h \in \R$, and that $\lvert f'''(\xi)\rvert \leq M_3$
		for all $\xi$ between $x_0$ and $x_0 + 2h$.
		Use the fact that
		\begin{align*}
			f(x_0) &= f(x_0)\\
			f(x_0+h) &= f(x_0) + hf'(x_0) + \frac{h^2}{2}f''(x_0) + O(h^3)M_3\\
			f(x_0+2h) &= f(x_0) + 2hf'(x_0) + \frac{(2h)^2}{2}f''(x_0) + O(h^3)M_3
		\end{align*}
		to find a value of $c_0,c_1,c_2$ such that
		\[
			c_0f(x_0) + c_1f(x_0 + h) + c_2f(x_0 + 2h)
			= hf'(x_0) + O(h^3)M_3
		\]
	\item To which formula on page 411 (Section 14.1) of [A\&G]
		are parts (a) and (b) related? Explain.
	\item What is the significance of the solution of
		\[
			\begin{bmatrix}
				1 & 1 & 1 & 1\\
				0 & 1 & 2 & 3\\
				0 & 1 & 4 & 9\\
				0 & 1 & 8 & 27
			\end{bmatrix}
			\begin{bmatrix} c_0 \\ c_1 \\ c_2 \\ c_3 \end{bmatrix}
			= \begin{bmatrix} 0 \\ 1 \\ 0 \\ 0 \end{bmatrix}
		\]
		to approximating $f'(x_0)$?
		[You don't have to solve this system, just state what you can do
		with the solution $c_0,c_1,c_2,c_3$.]
\end{enumerate}
\begin{enumerate}
	\item \begin{proof}[Solution]\let\qed\relax
		Let $A = \begin{bmatrix}
				1 & 1 & 1\\
				0 & 1 & 2\\
				0 & 1 & 4
			\end{bmatrix}$.
		Note that $\det(A) = 1(4-2) - 0 + 0 = 2 \neq 0$ so $A^{-1}$ exists.
		One can compute $A^{-1}$ by finding the adjoint matrix and dividing by $\det(A)$,
		which gives $A^{-1} = \begin{bmatrix}
				1 & -3/2 & 1/2\\
				0 & 2 & -1\\
				0 & -1/2 & 1/2
			\end{bmatrix}$
		(and one can check $AA^{-1} = A^{-1}A = I$).
		Thus, multiplying by $A^{-1}$ on both sides, we get
		\[
			\begin{bmatrix} c_0 \\ c_1 \\ c_2 \end{bmatrix}
			= \begin{bmatrix}
				1 & -3/2 & 1/2\\
				0 & 2 & -1\\
				0 & -1/2 & 1/2
			\end{bmatrix}
			\begin{bmatrix} 0 \\ 1 \\ 0 \end{bmatrix}
			= \begin{bmatrix} -3/2 \\ 2 \\ -1/2 \end{bmatrix}
		\]
		hence $c_0,c_1,c_2$ must be $-3/2,4/2,-1/2$, respectively,
		thus uniquely determining our solution to be the desired one.
	\end{proof}
	\item \begin{proof}[Solution]\let\qed\relax
		We seek to find a linear combination of $f(x_0),f(x_0 + h), f(x_0 + 2h)$
		that adds up to only being $hf'(x_0) + O(h^3)M_3$,
		using the given equalities.
		But this actually corresponds exactly to our linear transformation
		$A$ from part (a),
		where we are changing bases from $f(x_0),f(x_0+h),f(x_0+2h)$
		to $f(x_0),hf'(x_0),\frac{h^2}{2}f''(x_0) + O(h^3)M_3$,
		and our equations that dictate this transformation is represented in $A$
		(i.e. the coefficients of $f$, $hf'$, and
		$\frac{h^2}{2}f''$ are the entries of $A$).
		Hence, our coefficients are the solution to part (a),
		namely, $c_0 = -3/2$, $c_1 = 4/2$, $c_2 = -1/2$.
		This gives us
		\[
			-\frac32f(x_0) + \frac{4}{2}f(x_0+h) - \frac{1}{2}f(x_0+2h)
			= hf'(x_0) + O(h^3)M_3
		\]
	\end{proof}
	\item \begin{proof}[Solution]\let\qed\relax
		This corresponds to the three-point, second order, one-sided formula for $f'(x_0)$,
		which was in section 2(b) of 14.1 in [A\&G], written there as
		\[
			f'(x_0) = \frac{1}{2h}\left(-3f(x_0)+4f(x_0+h)-f(x_0+2h)\right)
				+ \frac{h^2}{3}f'''(\xi)
		\]
		where $\xi \in [x_0, x_0 + 2h]$.
		To get the formula we derived from parts (a) and (b),
		we bring in the $\frac12$, multiply everything by $h$,
		and bring the third order term to the other side
		($M_3$ let's ignore the specific $\xi$,
		and $O(h^3)$ let's ignore the specific coefficient of this term).
	\end{proof}
	\item \begin{proof}[Solution]\let\qed\relax
		This gives us a third order approximation of $f'(x_0)$.
		Recall by Taylor's theorem that if $f \colon \R \to \R$ has
		$f^{(4)}(x)$ existing for all $x$,
		and for $x_0,h \in \R$ that $|f^{(4)}(\xi)|\leq M_4$ for all $\xi$
		between $x_0$ and $x_0 + 3h$, we have
		\begin{align*}
			f(x_0) &= f(x_0)\\
			f(x_0 + h) &= f(x_0) + hf'(x_0) + \frac{h^2}{2}f''(x_0)
			\frac{h^3}{6}f'''(x_0) + O(h^4)M_4\\
			f(x_0 + 2h) &= f(x_0) + 2hf'(x_0) + \frac{(2h)^2}{2}f''(x_0)
			\frac{(2h)^3}{6}f'''(x_0) + O(h^4)M_4\\
			f(x_0 + 3h) &= f(x_0) + 3hf'(x_0) + \frac{(3h)^2}{2}f''(x_0)
			\frac{(3h)^3}{6}f'''(x_0) + O(h^4)M_4\\
		\end{align*}
		The coefficients of these equations correspond to the ones in our matrix,
		thus, solving the system for $c_0,c_1,c_2,c_3$ would give us the
		coefficients for the equation
		\[
			c_0f(x_0) + c_1f(x_0 + h) + c_2f(x_0 + 2h) + c_3f(x_0 + 3h)
			= hf'(x_0) + O(h^4)M_4
		\]
		Since the error term is higher order,
		this should decrease our error for small $h$,
		making this a more accurate approximation of $f'(x_0)$
		(dividing everything by $h$ shows it is third order).
	\end{proof}
\end{enumerate}

\subsection*{Problem 3}
Consider an ODE $y' = f(t,y)$, where as in [A\&G], $y = y(t)$,
and $y'$ refers to $dy/dt$.
Say that $f(t,y)$ is of the special form $f(t,y) = h(t)g(y)$,
where $g$ is a differentiable function and $h$ is continuous.
Then the ODE
\[
	y' = dy/dt = h(t)g(y)
\]
is called a \emph{separable differential equation}, and it can be solved by writing
\[
	\frac{dy}{g(y)} = h(t)dt
\]
and taking indefinite integrals of both sides.
See Section 2.4 (Separable ODE's) of UBC's Calculus 2 Textbook for details,
including Example 2.4.2 there, where they solve the equation $y' = y^2$
(in this textbook, $y'$ refers to $dy/dx$, as is common in math books).
\begin{enumerate}
	\item Solve the ODE $y' = y^3$
		(here $y = y(t)$ and $y'$ refers to $dy/dt$) in the same manner
		as $y' = y^2$ is solved in general form.
	\item Solve $y' = y^3$ for the initial condition $y(1) = 1$.
	\item Solve $y' = y^4$ for the initial condition $y(1) = 1$.
	\item Let $y(t)$ be as in part (b); for $t \geq 1$,
		when does $y(t)$ fail to exist, i.e.,
		for what $T > 1$ does $y(t) \to \infty$ as $t \to T$?
	\item Same question for part (c).
\end{enumerate}
\begin{enumerate}
	\item \begin{proof}[Solution]\let\qed\relax
		When $y \neq 0$
		\[
			\frac{dy}{dt} = y^3
			\implies \frac{dy}{y^3} = dt
			\underbrace{\implies}_{\int} \frac{y^{-2}}{-2} = t + C_1
			\implies y^2 = \frac{1}{C - 2t}
		\]
		Presumebly, we are restricting ourselves to real-valued solutions,
		so when $C - 2t > 0 \implies t < C/2$ we have
		$y(t) = \sqrt{(C-2t)^{-1}}$ or $y(t) = -\sqrt{(C-2t)^{-1}}$.

		Now when $y = 0$, we have $y' = 0$ and $y^3 = 0$,
		hence the zero function also satisfies the ODE, giving our solutions
		\[
			y = 0, (C-2t)^{-1/2}, -(C-2t)^{-1/2}
		\]
		where the last two are only defined for $t < C/2$.
	\end{proof}
	\item \begin{proof}[Solution]\let\qed\relax
		Only one of our general forms has positive values,
		namely $y(t) = (C-2t)^{-1/2}$.
		Solving, we get
		\[
			1 = (C-2(1))^{-1/2} \implies 1 = C-2 \implies 3 = C
		\]
		So our solution to the IVP is $y(t) = (3-2t)^{-1/2}$.
	\end{proof}
	\item \begin{proof}[Solution]\let\qed\relax
		We may assume that $y$ is not the zero function, by the initial values.
		Then solving the general form gives
		\[
			\frac{dy}{dt} = y^4 \implies \frac{dy}{y^4} = dt
			\underbrace{\implies}_{\int} \frac{y^{-3}}{-3} = t + C_1
			\implies y^3 = \frac{1}{C-3t} \implies y = (C-3t)^{-1/3}
		\]
		Plugging in the initial values, we get
		\[
			1 = (C - 3(1))^{-1/3} \implies 1 = C - 3 \implies 4 = C
		\]
		So our solution to the IVP is $y(t) = (4 - 3t)^{-1/3}$.
	\end{proof}
	\item \begin{proof}[Solution]\let\qed\relax
		I have already made some coments on this in part (a),
		but I claim that $y(t)$ will fail to exist when $t \geq C/2 = 3/2$.
		So let $T = 3/2$.
		When $t > T$, we have the square root of a negative,
		which is not defined in the reals.
		Now consider $y(t)$ as $t \to T$ (from the left).
		The denominator is going to $0$ and the numerator is a fixed constant.
		Since both the denominator and numerator are positive,
		this gives that $y(t) \to \infty$ as $t \to T$.
		Hence, $y(t)$ from (b) fails to exist for $t \geq 3/2$.
	\end{proof}
	\item \begin{proof}[Solution]\let\qed\relax
		I claim that $y(t)$ will fail to exist when $4 - 3t = 0 \implies t = 4/3$.
		Consider $y(t)$ as $t \to 4/3$.
		From either direction, we have a denominator going to $0$
		and a numerator that is a fixed constant.
		From the left, the denominator is positive so it blows up to $\infty$,
		and from the right, the denominator is negative so it blows up to $-\infty$.
		Either way, the value does not exist.
		Other than that point, the function is well-defined,
		so $y(t)$ from (c) only fails to exist at $t = 4/3$.
	\end{proof}
\end{enumerate}

\subsection*{Problem 4}
Let $y(t) = (3-2t)^{-1/2}, z(t) = (4-3t)^{-1/3}$.
\begin{enumerate}
	\item Examine a plot of $y(t)$ and $z(t)$ for $1 \leq t < 4/3$.
		Is one of these functions larger than the other in the entire interval
		(as far as the plot shows)?
		(Here a simple answer will do.
		You might type \newline \verb|plot (3-2t)^(-1/2) and (4-3t)^(-1/3)|
		into Google, or something like that.)
	\item Show that $y' = y^3$ and $z' = z^4$ and that $y(1) = z(1) = 1$.
	\item Show that $y'(1) = z'(1) = 1$.
	\item Show that $y''(1) = 3$ and $z''(1) = 4$.
		[Hint: differentiate both sides of $y' = y^3$; similarly for $z$.]
	\item Show that for $h > 0$ and $h$ sufficiently small we have $y(1+h) < z(1+h)$.
		[Hint: let $u(t) = z(t) - y(t)$; what are the values of $u(1),u'(1),u''(1)$?]
\end{enumerate}
\begin{enumerate}
	\item \begin{proof}[Solution]\let\qed\relax
		Yes, $z(t) \geq y(t)$ on $1 \leq t < 4/3$ with equality
		only at $t = 1$.
	\end{proof}
	\item \begin{proof}[Solution]\let\qed\relax
		We can compute
		\[
			\frac{dy}{dt} = \left(-\frac{1}{2}\right)(3-2t)^{-3/2}(-2)
			= (3-2t)^{-3/2} = y^3
		\]
		and also
		\[
			\frac{dz}{dt} = \left(-\frac{1}{3}\right)(4-3t)^{-4/3}(-3)
			= (4-3t)^{-4/3} = z^4
		\]
		Furthermore, one can calculate $y(1) = (3-2)^{-1/2} = 1$
		and $z(1) = (4-3)^{-1/3} = 1$.
	\end{proof}
	\item \begin{proof}[Solution]\let\qed\relax
		Using the results shown in part (b), we can compute
		\[
			y'(1) = y(1)^3 = 1^3 = 1
		\]
		\[
			z'(1) = z(1)^4 = 1^4 = 1
		\]
	\end{proof}
	\item \begin{proof}[Solution]\let\qed\relax
		We first compute $y''$. We start by differentiating $y' = y^3$ to get
		\[
			y'' = 3y^2y' = 3y^2y^3 = 3y^5
		\]
		Then
		\[
			y''(1) = 3y(1)^5 = 3\cdot 1^5 = 3
		\]
		Similarly for $z$, we differentiate $z' = z^4$ to get
		\[
			z'' = 4z^3z' = 4z^3z^4 = 4z^7
		\]
		Then
		\[
			z''(1) = 4z(1)^7 = 4\cdot 1^7 = 4
		\]
	\end{proof}
	\item \begin{proof}[Solution]\let\qed\relax
		Let $u(t) = z(t) - y(t)$.
		Note $u'(t) = z'(t) - y'(t)$ and $u''(t) = z''(t) - y''(t)$.
		Then
		\begin{align*}
			u(1) &= z(1) - y(1) = 1-1 = 0\\
			u'(1) &= z'(1) - y'(1) = 1-1 = 0\\
			u''(1) &= z''(1) - y''(1) = 4 - 3 = 1
		\end{align*}
		By Taylor's theorem, if $h \in \R$,
		since $u'''(1+h) = z'''(1+h) - y'''(1+h)$ exists for all
		$h$ sufficiently small,
		there is some $\xi$ between $1$ and $1+h$ such that
		\[
			u(1 + h) = u(1) + hu'(1) + \frac{h^2}{2}u''(1) + \frac{h^3}{6}u'''(\xi)
			= \frac{h^2}{2} + \frac{h^3}{6}u'''(\xi)
		\]
		Hence $z(1+h) - y(1+h) = \frac{h^2}{2}+\frac{h^3}{6}u'''(\xi)$
		or $z(1+h) = y(1+h) + \frac{h^2}{2} + \frac{h^3}{6}u'''(\xi)$.
		When $h$ is sufficiently small, our trailing term is negligible
		compared to our $h^2$ term,
		thus $z(1+h) \approx y(1+h) + \frac{h^2}{2}$.
		Note that $h^2/2 > 0$ for all $h$, and so we have that
		$z(1+h) > y(1+h)$ for $h$ sufficiently small.
	\end{proof}
\end{enumerate}
\end{document}
