\documentclass{article}
\usepackage{amsmath, amsfonts, amsthm, amssymb}
\usepackage{geometry}
\geometry{letterpaper, margin=2.0cm, includefoot, footskip=30pt}

\usepackage{fancyhdr}
\pagestyle{fancy}

\lhead{CPSC 303}
\chead{Homework 4}
\rhead{Nicholas Rees, 11848363}
\cfoot{Page \thepage}

\newcommand{\N}{{\mathbb N}}
\newcommand{\Z}{{\mathbb Z}}
\newcommand{\Q}{{\mathbb Q}}
\newcommand{\R}{{\mathbb R}}
\newcommand{\C}{{\mathbb C}}
\newcommand{\ep}{{\varepsilon}}

\newtheorem{lemma}{Lemma}

\renewcommand{\theenumi}{(\alph{enumi})}

\begin{document}
\subsection*{Problem 1}
Who are your group members?
\begin{proof}[Solution]\let\qed\relax
	Nicholas Rees
\end{proof}


\subsection*{Problem 2}
Recall the (forward) Euler's method (see class notes or [A\&G], page 485)
and the (explicit) trapazoidal method (class notes or [A\&G], page 494),
which solves $y' = f(t,y)$ subject to $y(t_0) = y_0$
(1) choosing a small $h > 0$ and setting $t_i = t_0 + ih$ (as usual),
and (2) takes $y_i$ (an approximation of $y(t_i)$) to produce $y_{i+1}$,
for $i = 0, 1, 2, \dots$ via the two formulas:
\[
	Y_i = y_i + hf(t_i,y_i) \quad y_{i+1} = y_i + h\frac{f(t_i,y_i)+f(t_{i+1},Y_i)}{2}
\]
(hence $Y_i$ is what Euler's method gives for $y_{i+1}$).
\begin{enumerate}
	\item Consider the trapezoidal method for $y' = y$ with $y(t_0) = y_0$.
		Write down a formula for $y_{i+1}$ of the trapezoidal method in terms of $y_i$.
	\item Say that we want to approximate $y(1)$ given $y(0) = 1$
		(i.e. $t_0 = 0$, $y_0 = 1$).
		Choose a natural $N$ (which we think of as large), set $h = 1/N$;
		show that $y_N$ in the trapezoidal method approximation of $y(1)$ equals
		\[
			y_{N,\text{trap}} = \left(1 + \frac{1}{N} + \frac{1}{2N^2}\right)^N.
		\]
	\item Show that Euler's method for the same ODE and intial condition
		gives $y_{N,\text{Eul}} = (1+1/N)^N$.
	\item Recall that actual solution to $y' = y$ with $y(0) = 1$ has $y(1) = e$.
		Show that $\ln(y_{N,\text{trap}})$ is closer to the
		actutal value of $\ln(y(1))$ (namely $\ln(e) = 1$)
		than $\ln(y_{N,\text{Eul}})$, as $N \to \infty$
		(i.e., for all $N$ sufficiently large).
		[It may be helpful to recall that $\ln(1+\ep) = \ep - \ep^2/2 + O(\ep^3)$
		for $|\ep|$ small.]
\end{enumerate}

\begin{enumerate}
	\item \begin{proof}[Solution]\let\qed\relax
		\[
			y_{i+1} = y_i + h\frac{y_i + Y_i}{2} =
			y_i + h\frac{y_i + y_i + hy_i}{2} =
			\frac{2y_i + 2hy_i + h^2y_i}{2}
			= y_i\left(1 + h + \frac{h^2}{2}\right)
		\]
	\end{proof}
	\item \begin{proof}[Solution]\let\qed\relax
		Fix some $N$ large and let our step size be $h = 1/N$.
		We prove by induction that $y_i = y_0(1+\frac{1}{N} + \frac{1}{2N^2})^i$
		for all $i \in \N$.
		The base case when $i = 1$:
		from part (a), we have
		$y_1 = y_0(1 + h + \frac{h^2}{2}) = y_0(1 + \frac{1}{N} + \frac{1}{2N^2})^1$, as desired.
		Now assume that $y_i = y_0(1+\frac{1}{N} + \frac{1}{2N^2})^i$,
		we prove the formula for $y_{i+1}$.
		From part (a), we have
		\[
			y_{i+1} = y_i(1 + h + \frac{h^2}{2}) =
			y_i(1 + \frac{1}{N} + \frac{1}{2N^2}) =
			y_0\left(1 + \frac{1}{N} + \frac{1}{2N^2}\right)^{i+1}
		\]
		as desired.

		Hence, the trapezoidal method at the $N$th step, which is the approximation at $y(Nh) = y(1)$
		gives $y_{N,\text{trap}} = (1 + \frac{1}{N} + \frac{1}{2N^2})^N$.
	\end{proof}
	\item \begin{proof}[Solution]\let\qed\relax
		Recall that $y_{i+1} = y_i(1+h)$ for Euler's method.
		Denote this equation ($*$).
		Fix some $N$ large and let our step size be $h = 1/N$.
		We prove by induction that $y_i = y_0(1+\frac{1}{N})^i$
		for all $i \in \N$.
		The base case when $i = 1$:
		from equation ($*$), we have
		$y_1 = y_0(1 + h) = y_0(1 + \frac{1}{N})^1$, as desired.
		Now assume that $y_i = y_0(1+\frac{1}{N})^i$,
		we prove the formula for $y_{i+1}$.
		From equation ($*$), we have
		\[
			y_{i+1} = y_i(1 + h) =
			y_i(1 + \frac{1}{N}) =
			y_0\left(1 + \frac{1}{N}\right)^{i+1}
		\]
		as desired.

		Hence, Euler's method at the $N$th step, which is the approximation at $y(Nh) = y(1)$
		gives $y_{N,\text{Eul}} = (1 + \frac{1}{N})^N$.
	\end{proof}
	\item \begin{proof}[Solution]\let\qed\relax
		We can compute
		\begin{align*}
			\ln(y_{N,\text{trap}})
			&= \ln\left(\left(1 + \frac{1}{N} + \frac{1}{2N^2}\right)^N\right)\\
			&= N\ln\left(1 + \frac{1}{N} + \frac{1}{2N^2}\right)\\
			&= N\left(\frac{1}{N}+\frac{1}{2N^2} -
				\frac12\left(\frac{1}{N} + \frac{1}{2N^2}\right)^2
			+ O\left(\left(\frac{1}{N}+\frac{1}{2N^2}\right)^3\right)\right)\\
			&= 1 + \frac{1}{2N}
			- \frac{1}{2N} - \frac{1}{2N^2} - \frac{1}{4N^2}
			+ NO\left(\frac{1}{N^3}\right)\\
			&= 1 + O\left(\frac{1}{N^2}\right)
		\end{align*}
		where we replaced $O\left(\left(\frac{1}{N}+\frac{1}{2N^2}\right)^3\right)$
		with $O\left(\frac{1}{N^3}\right)$
		since $(a + \frac{1}{2}a^2)^3 = O(a^3)$.
		We also see
		\begin{align*}
			\ln(y_{N,\text{Eul}})
			&= \ln\left(\left(1 + \frac{1}{N}\right)^N\right)\\
			&= N\ln\left(1 + \frac{1}{N}\right)\\
			&= N\left(\frac{1}{N} - \frac{1}{2N^2} + O\left(\frac{1}{N^3}\right)\right)\\
			&= 1 - \frac{1}{2N} + O\left(\frac{1}{N^2}\right)
		\end{align*}
	\end{proof}
	As $N \to \infty$, $\frac{1}{N^2}$ gets small (much faster than $1/N$),
	and we can neglect it the most,
	so we can see that while $\ln(y_{N,\text{trap}})$ only differs from
	$\ln(y(1))$ on order of $\frac{1}{N^2}$,
	$\ln(y_{N,\text{Eul}})$ differs on order of $\frac{1}{N}$,
	which is a much larger error when $N$ large.
\end{enumerate}


\subsection*{Problem 3}
In this exercise we will solve the ODE $y' - 2y = t^2$.
\begin{enumerate}
	\item Show that there are constants $a,b,c$ such that
		$y(t) = at^2 + bt + c$ is a solution to $y' - 2y = t^2$.
	\item Show that if $y' - 2y = t^2$ and $z' - 2z = 0$,
		then $(y+z)' - 2(y+z) = t^2$.
	\item Show that if $y' - 2y = t^2$ and $(y+z)' - 2(y+z) = t^2$, then $z' - 2z = 0$.
	\item Solve for all $z' - 2z = 0$ (which is called the
		``homogenous form of the equation $y' - 2y = t^2$),
		and use this to write down the general solution to $y' - 2y = t^2$.
	\item Show that for any $t_0,y_0$, your general solution in part (d)
		has a unique $y$ such that $y(t_0) = y_0$.
\end{enumerate}

\begin{enumerate}
	\item \begin{proof}[Solution]\let\qed\relax
		If $a = -1/2$, $b = 1/2$, $c = -1/4$,
		then $y(t) = -\frac12 t^2 + \frac12 t - \frac14$ and
		$y'(t) = - t + \frac{1}{2}$.
		Plugging it into the differential equation, we have
		$y' - 2y = - t + \frac12 + t^2 + t - \frac12 = t^2$,
		hence $y(t)$ is a solution to $y' - 2y = t^2$.
	\end{proof}
	\item \begin{proof}[Solution]\let\qed\relax
		Assume that $y' - 2y = t^2$ and $z' - 2z = 0$.
		Then
		\[
			(y+z)' - 2(y+z)
			= y' + z' - 2y - 2z
			= y' - 2y + z' - 2z = t^2 + 0 = t^2
		\]
		where we have used the linearity of the derivative.
	\end{proof}
	\item \begin{proof}[Solution]\let\qed\relax
		Assume that $y' - 2y = t^2$ and $(y+z)' - 2(y+z) = t^2$.
		Then by the linearity of the derivative, we have
		$y' + z' - 2y - 2z = t^2 \implies y' - 2y + z' - 2z = t^2
		\implies t^2 + z' - 2z = t^2$,
		and subtracting $t^2$ on both sides, we are left with
		$z' = 2z = 0$.
	\end{proof}
	\item \begin{proof}[Solution]\let\qed\relax
		We can rewrite the differential equation as $z' = 2z$,
		and this is now one of our classic solutions (derived on previous homeworks),
		$z(t) = Ae^{2t}$, $A \in \R$ constant;
		one can verify this by plugging it in (and unique from standard theorems).
		
		As we saw from parts (b) and (c),
		any and all solutions to $y' - 2y = t^2$
		is of the form $y_1 + z$ where $y_1$ is a particular solution to the ODE,
		and $z$ is the general solution to the homogenous form of the ODE.
		Hence, using part (a) and the earlier part of (d),
		the general solution is of the form
		\[
			y(t) = -\frac12 t^2 + \frac12 t - \frac14 + Ae^{2t}
		\]
	\end{proof}
	\item \begin{proof}[Solution]\let\qed\relax
		Let $t_0,y_0$ be fixed. Then the solution from (d) gives
		\[
			y_0 = -\frac12 t_0^2 + \frac12 t_0 - \frac14 + Ae^{2t_0}
		\]
		Which implies
		\[
			A = \frac{y_0 + \frac12 t_0^2 - \frac12 t_0 + \frac14}{e^{2t_0}}
		\]
		(and we need not worry about dividing by zero,
		since $e^x \neq 0$ for all $x \in \R$).
		$A$ is determined uniquely be $y_0,t_0$, which gives us the unique solution
		\[
			y(t) = -\frac12 t^2 + \frac12 t - \frac14 +
			\left(\frac{y_0 + \frac12 t_0^2 - \frac12 t_0 + \frac14}{e^{2t_0}}\right)e^{2t}
		\]
	\end{proof}
\end{enumerate}


\subsection*{Problem 4}
In this exericise we will solve the recurrence relation
$x_{n+1} - 2x_n = n^2$ for all $n \in \Z$.
\begin{enumerate}
	\item Show that there are constants $a,b,c$ such that $x_n = an^2 + bn + c$
		is a solution to $x_{n+1} - 2x_n = n^2$.
	\item What is the general solution to $x_{n+1} - 2x_n = 0$?
	\item Use the above two parts to write a general solution to
		the recurrence equation $x_{n+1} - 2x_n = n^2$,
		and explain your reasoning in term of Problem (2)
		above and the appropriate notation of the
		``homogenous form of the recurrence $x_{n+1} - 2x_n = n^2$.
\end{enumerate}

\begin{enumerate}
	\item \begin{proof}[Solution]\let\qed\relax
		If $a = -1$, $b = -2$, $c = -3$,
		then $x_n = -n^2 - 2n - 3$ and
		$x_{n+1} = -1(n+1)^2 + -2(n+1) - 3 = -n^2 -4n - 6$.
		Plugging these together into the recurrence relation, we have
		\[
			x_{n+1} - 2x_{n}
			= -n^2 - 4n - 6 + 2n^2 + 4n + 6 = n^2
		\]
		Hence, $x_n$ is a solution to $x_{n+1} - 2x_n = n^2$.
	\end{proof}
	\item \begin{proof}[Solution]\let\qed\relax
		This is equivalent to $x_{n+1} = 2x_n$.
		As we discussed in class, this is a solution of the form $Ar^n$,
		where $r \in \R$ is a constant, and $A$ is a constant determined by $x_0$.
		For our specific case, $x_n = x_02^n$ works.
	\end{proof}
	\item \begin{proof}[Solution]\let\qed\relax
		We can use the same logic from the previous problem,
		and it would be simple to show as well,
		that any solution to the recurrence relation
		is the sum of a particular solution and the general solution
		to the homogenous form of the recurrence relation.
		Hence, using parts (a) and (b), we have the general solution
		\[
			x_n = -n^2 - 2n - 3 + A2^n
		\]
	\end{proof}
\end{enumerate}


\subsection*{Problem 5}
\begin{enumerate}
	\item At what value of $n$ does MATLAB declare $(1/2)^n$ to be $0$?
		There are a number of ways of doing this,
		but one way is to examine the values of $x$ generated by:
		\begin{verbatim}
		clear
		for n=1:1100, x{n}=(1/2)^n; end

		x
		\end{verbatim}
		(you may need the extra blank line above if you copy and paste).
	\item Find the general solution to the recurrence $x_{n+2} = (3/2)x_{n+1}-(1/2)x_n$.
	\item Find the special case of this recurrence subject to
		the initial conditions $x_1 = 1$ and $x_2 = 1/2$.
	\item Use MATLAB to compute the solution of this recurrence subject to
		$x_1 = 1$ and $x_2 = 1/2$ for $x_1,x_2,\dots,x_{100}$ via:
		\begin{verbatim}
		clear
		x{1} = 1;
		x{2} = 1/2;
		for n=1:98, x{n+2} = (3/2)*x{n+1}-(1/2)*x{n}; end

		x
		\end{verbatim}
		What does MATLAB report for the values of
		\begin{verbatim}
		x{100} * 2^99
		x{100} * 2^99 - 1
		\end{verbatim}
	\item Now run the same code to generate $x_1,x_2,\dots,x_{1200}$ using
		\begin{verbatim}
		clear
		x{1} = 1;
		x{2} = 1/2;
		for n=1:1198, x{n+2} = (3/2)*x{n+1}-(1/2)*x{n}; end

		x
		\end{verbatim}
		Anwser the following:
		\begin{enumerate}
			\item[(i)] Does MATLAB report $x_n = 0$ for some value of $n \leq 1200$?
			\item[(ii)] What is the smallest value of $n_0$ such that
				the values that MATLAB reports for $(1/2)^{n_0-1}$ and $x\{n_0\}$ are not equal?
			\item[(iii)] For the value of $n_0$, by examining the values of
				$(3/2)*x\{n_0-1\}$ and $(1/2)*x\{n_0-2\}$,
				can you see where an error in precision occurs?
			\item[(iv)] What repeating pattern do you see in the values
				of $x\{n\}$ for $n \geq n_0$?
				[Hint: It may be simpler to report this pattern
				in multiples of $(1/2)^{1074}$.]
				[Warning: if you type something like
				\begin{verbatim}
				for n=1070:1100, {n,x{n},x{n}*2^1074}, end
				\end{verbatim}
				then the last cell value will be \verb|Inf|,
				since $2^{1074}$ will yield \verb|Inf|.
				However, if you type:
				\begin{verbatim}
				for n=1070:1100, {n,x{n},(x{n}*2^900)*2^174}, end
				\end{verbatim}
				then you'll get the answers you want.
		\end{enumerate}
\end{enumerate}

\begin{enumerate}
	\item \begin{proof}[Solution]\let\qed\relax
		For any $n \geq 1075$.
	\end{proof}
	\item \begin{proof}[Solution]\let\qed\relax
		Didn't do :)
	\end{proof}
	\item \begin{proof}[Solution]\let\qed\relax
		Didn't do :)
	\end{proof}
	\item \begin{proof}[Solution]\let\qed\relax
		It produces $1$ and $0$, respectively.
	\end{proof}
	\item \begin{proof}[Solution]\let\qed\relax
		\begin{enumerate}
			\item[(i)] didn't do :0
			\item[(ii)] didn't do :0
			\item[(iii)] Didn't do :)
			\item[(iv)] Didn't do :)
		\end{enumerate}
	\end{proof}
\end{enumerate}
\end{document}
