\documentclass{article}
\usepackage{amsmath, amsfonts, amsthm, amssymb}
\usepackage{geometry}
\geometry{letterpaper, margin=2.0cm, includefoot, footskip=30pt}

\usepackage{fancyhdr}
\pagestyle{fancy}

\lhead{CPSC 303}
\chead{Homework 2}
\rhead{Nicholas Rees, 11848363}
\cfoot{Page \thepage}

\newcommand{\N}{{\mathbb N}}
\newcommand{\Z}{{\mathbb Z}}
\newcommand{\Q}{{\mathbb Q}}
\newcommand{\R}{{\mathbb R}}
\newcommand{\C}{{\mathbb C}}
\newcommand{\ep}{{\varepsilon}}

\newtheorem{lemma}{Lemma}

\renewcommand{\theenumi}{(\alph{enumi})}

\begin{document}
\subsection*{Problem 1}
Who are your group members?
\begin{proof}[Solution]\let\qed\relax
	Nicholas Rees
\end{proof}

\subsection*{Problem 2}
Familarize yourself with basic MATLAB syntax,
and make sure you understand what each line in the file
\newline\verb|start_here.txt| is doing,
and what the commands in \verb|exponential_of_a_matrix.txt| are doing.
Answer the following questions with MATLAB (but just write down the answer).
\begin{enumerate}
	\item Let
		\[
			A = \begin{bmatrix} 0 & -1 \\ 1 & 0 \end{bmatrix}
		\]
		What is the largest interger $n$ such that each entry of
		$e^A - \sum_{i=0}^{15} A^i/i!$ is of absolute value at most $10^{-n}$?
	\item Same question for
		\[
			A = \begin{bmatrix} 0 & 1 \\ 1 & 0 \end{bmatrix}
		\]
\end{enumerate}
\begin{enumerate}
	\item \begin{proof}[Solution]\let\qed\relax
		$n = 13$.
	\end{proof}
	\item \begin{proof}[Solution]\let\qed\relax
		$n = 13$.
	\end{proof}
\end{enumerate}

\subsection*{Problem 3}
Create the files \verb|apple.m|, \verb|apple_bad.m|, \verb|apple_worse.m|,
\verb|apple_quiet.m| and see how they are implementing Euler's method
to solve $y' = 2y$ subject to $y(1) = 3$, in order to find $y(2)$.
(The files \verb|apple_bad.m|, \verb|apple_worse.m| will produce error messages;
they are just there as a cautionary note.)
You might also have a look at \verb|chaotic_sqrt.m|.
\begin{enumerate}
	\item Use Euler's method to solve $y' = \lvert y \rvert^{1/2}$
		subject to $y(t_0) = y_0$ to find the value of $y(t_{\mathrm{end}})$,
		where $t_0 = -2$, $y_0 = -1$, and $t_{\mathrm{end}} = 2$.
		Use step size $h = (t_{\mathrm{end}} - t_0)/N$,
		where $N = 1000$ and $N = 100000$.
		What values do you get?
	\item Same question, but with $y_0 = 0$, $t_0 = 0$, and $t_{\mathrm{end}} = 2$.
		Use $N = 10000$.
	\item Same question, but with $y_0 = 10^{-20}$.
	\item Same question, but with $y_0 = 10^{-40}$.
	\item How do you explain the difference between parts (c,d) and part (b)?
\end{enumerate}
\begin{enumerate}
	\item \begin{proof}[Solution]\let\qed\relax
		When $N = 1000$, we get $y(t_{\mathrm{end}}) = 1.0036$.
		When $N = 100000$, we get $y(t_{\mathrm{end}}) = 1.0000$.
	\end{proof}
	\item \begin{proof}[Solution]\let\qed\relax
		We get $y(t_{\mathrm{end}}) = 0$.
	\end{proof}
	\item \begin{proof}[Solution]\let\qed\relax
		We get $y(t_{\mathrm{end}}) = 0.9985$.
	\end{proof}
	\item \begin{proof}[Solution]\let\qed\relax
		We get $y(t_{\mathrm{end}}) = 0.9982$.
	\end{proof}
	\item \begin{proof}[Solution]\let\qed\relax
		From an algorithmic point of view,
		when $y_0 = 0$, Euler's formula, no matter how many iterations,
		will always give $y(t+ih) = 0$,
		since it computes $y_{i+1}(t_0+(i+1)h) = y_{i} + h|y_{i}|^{1/2}$
		so $y_{i} = 0 \implies y_{i+1}$,
		and $y_0 = 0$, so by induction, $y(t_{\mathrm{end}}) = y(t_N) = 0$.
		If, instead, we had a nonzero value,
		$y(t_N)$ will at least sum to a positive value.

		From a theory point of view, it is because $y = 0$ is a solution to the differential equation,
		and the unique solution to the initial value problem $y_0 = t_0 = 0$ is $y(t) = 0$.
		On the other hand, we get a nontrivial solution when $y_0 > 0$,
		so it approaches a similar value to what we had before in part (a).
	\end{proof}
\end{enumerate}

\subsection*{Problem 4}
If $y \colon \R \to \R$ is a function and $T \in \R$,
then the \emph{translation of $y$ by $T$},
denoted $\mathrm{Trans}_T(y)$, refers to the function $z$ given by
\[
	\forall t \in \R, \quad z(t) = y(t-T)
\]
(or, equivalently, $z(t+T) = y(t)$)
(hence $z(T) = y(0)$, $z(T+1) = y(1)$, etc.).
Similarly, the \emph{time reversal of $y$ at time $T$},
denoted $\mathrm{Reverse}_T(y)$ refers to the function $z$ given by
\[
	\forall t \in \R, \quad z(t) = y(2T - t)
\]
(hence $z(T) = y(T)$, and $z(T+a) = y(T-a)$).
\begin{enumerate}
	\item If $T_1,T_2 \in \R$ and $y$ is any function, what is
		\[
			\mathrm{Trans}_{T_1}(\mathrm{Trans}_{T_2}(y))
		\]
		in simpler terms?
	\item If $T \in \R$ and $y$ is any function, what is
		\[
			\mathrm{Reverse}_T(\mathrm{Reverse}_T(y))
		\]
		in simpler terms?
	\item If $T_1,T_2 \in \R$ and $y$ is any function, what is
		\[
			\mathrm{Reverse}_{T_1}(\mathrm{Reverse}_{T_2}(y))
		\]
		in simpler terms?
	\item Show that if for some function $f \colon \R \to \R$,
		$y$ satisfies the ODE $y' = f(y)$ globally
		(meaning $y'(t) = f(y(t))$ for all $t \in \R$),
		then $z = \mathrm{Trans}_T(y)$ satisfies the same ODE,
		i.e., $z' = f(z)$ (globally).
	\item Show directly that if $y(t) = e^{At}$ for some $A \in \R$,
		and if $z$ satisfies the ODE $z' = Az$ with $z(t) > 0$ for some $t \in \R$,
		then $z$ is a translation of $y$, \textbf{PROVIDED THAT} $A \neq 0$.
	\item If similarly $y' = f(y)$ globally,
		then $z = \mathrm{Reverse}_T(y)$ satisfies the ODE $z' = -f(z)$.
	\item If similarly $y'' = f(y)$ globally,
		then $z = \mathrm{Reverse}_T(y)$ satisfies the ODE $z'' = f(z)$.
\end{enumerate}
\begin{enumerate}
	\item \begin{proof}[Solution]\let\qed\relax
		We can compute:
		\begin{align*}
			\mathrm{Trans}_{T_1}(\mathrm{Trans}_{T_2}(y))
			&= \mathrm{Trans}_{T_1}(y(t-T_2))\\
			&= y((t - T_1) - T_2)\\
			&= y(t - T_1 - T_2)
		\end{align*}
	\end{proof}
	\item \begin{proof}[Solution]\let\qed\relax
		We can compute:
		\begin{align*}
			\mathrm{Reverse}_{T}(\mathrm{Reverse}_{T}(y))
			&= \mathrm{Reverse}_{T}(y(2T-t))\\
			&= y(2T - (2T - t))\\
			&= y(t)
		\end{align*}
	\end{proof}
	\item \begin{proof}[Solution]\let\qed\relax
		We can compute:
		\begin{align*}
			\mathrm{Reverse}_{T_1}(\mathrm{Reverse}_{T_2}(y))
			&= \mathrm{Reverse}_{T_1}(y(2T_2-t))\\
			&= y(2T_2 - (2T_1 - t))\\
			&= y(t + 2T_2 - 2T_1)
		\end{align*}
	\end{proof}
	\item \begin{proof}[Solution]\let\qed\relax
		Assume that $y'(t) = f(y(t))$ for all $t \in \R$.
		Note $z(t) = \mathrm{Trans}_T(y) = y(t-T)$.
		Hence, $z'(t) = y'(t-T)\cdot1 = f(y(t-T)) = f(z(t))$
		for all $t \in \R$.
		So we have shown $z' = f(z)$ globally.
	\end{proof}
	\item \begin{proof}[Solution]\let\qed\relax
		Let $A \in \R$, and $z' = Az$.
		Since $A \neq 0$ and $z(t) > 0$ for some $t \in \R$ so $z$ is not the zero function,
		we can divide by $Az$ to get
		\begin{align*}
			\frac{dz}{dt} = Az
			&\implies \frac{1}{Az}dz = dt\\
			&\implies \int \frac{1}{Az}dz = \int dt\\
			&\implies \frac{1}{A}\ln{z} = t + C\\
			&\implies z = e^{A(t+C)}
		\end{align*}
		Hence, $z = y(t+C)$ and so $z$ is a translation of $y$ by $-C$,
		i.e. $z = \mathrm{Trans}_{-C}(y)$.
	\end{proof}
	\item \begin{proof}[Solution]\let\qed\relax
		Assume that $y'(t) = f(y(t))$ for all $t \in \R$.
		Note $z(t) = \mathrm{Reverse}_T(y) = y(2T-t)$.
		Hence $z'(t) = y'(2T-t)\cdot(-1) = -y'(2T-t) = -f(y(2T-t) = -f(z(t))$
		for all $t \in \R$.
		So we have shown $z' = -f(z)$.
	\end{proof}
	\item \begin{proof}[Solution]\let\qed\relax
		Assume that $y''(t) = f(y(t))$ for all $t \in \R$.
		Note $z(t) = \mathrm{Reverse}_T(y) = y(2T-t)$.
		Hence $z'(t) = y'(2T-t)\cdot(-1) = -y'(2T-t)$
		for all $t \in \R$.
		We take another derivative to get
		$z''(t) = -y''(2T-t)\cdot(-1) = y''(2T-t) = f(y(2T-t)) = f(z(t))$
		for all $t \in \R$.
		So we have shown $z'' = f(z)$.
	\end{proof}
\end{enumerate}
\end{document}
