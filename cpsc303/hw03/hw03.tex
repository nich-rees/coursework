\documentclass{article}
\usepackage{amsmath, amsfonts, amsthm, amssymb}
\usepackage{geometry}
\geometry{letterpaper, margin=2.0cm, includefoot, footskip=30pt}

\usepackage{fancyhdr}
\pagestyle{fancy}

\lhead{CPSC 303}
\chead{Homework 3}
\rhead{Nicholas Rees, 11848363}
\cfoot{Page \thepage}

\newcommand{\N}{{\mathbb N}}
\newcommand{\Z}{{\mathbb Z}}
\newcommand{\Q}{{\mathbb Q}}
\newcommand{\R}{{\mathbb R}}
\newcommand{\C}{{\mathbb C}}
\newcommand{\ep}{{\varepsilon}}

\newtheorem{lemma}{Lemma}

\renewcommand{\theenumi}{(\alph{enumi})}

\begin{document}
\subsection*{Problem 1}
Who are your group members?
\begin{proof}[Solution]\let\qed\relax
	Nicholas Rees
\end{proof}


\subsection*{Problem 2}
Using the expansion $e^{Ah} = I + hA + (1/2)(hA)^2 + O(h^3)$ for $|h|$ small
(i.e., near $0$) (where the constant in $O(h^3)$ depends on $A$),
show that for any fixed square matrices $A,B$ of the same dimensions,
\[
	e^{(A+B)h} - e^{Ah}e^{Bh} = (1/2)(BA - AB)h^2 + O(h^3)
\]
for $|h|$ small.
\begin{proof}[Solution]\let\qed\relax
	We just algebraically verify:
	\begin{align*}
		e^{Ah}e^{Bh}
		&= (I + hA + (1/2)(hA)^2 + O(h^3))(I + hB + (1/2)(hB)^2 + O(h^3))\\
		&= I + hB + (h^2/2)B^2 + hA + h^2AB + (h^3/2)AB^2 +
		(h^2/2)A^2 + (h^3/2)A^2B + (h^4/4)A^2B^2 + O(h^3)\\
		&= I + h(A + B) + (h^2/2)B^2 + h^2AB + (h^2/2)A^2 + O(h^3)\\
		&= I + h(A+B) + (h^2/2)(B^2 + 2AB + A^2) + O(h^3)
	\end{align*}
	Then
	\begin{align*}
		e^{(A+B)h} &- e^{Ah}e^{Bh}\\
		&= I + h(A+B) + (1/2)(h(A+B))^2 + O(h^3)
		- \left(I + h(A+B) + (h^2/2)(B^2 + 2AB + A^2) + O(h^3)\right)\\
		&= \frac{1}{2}(A^2 + AB + BA + B^2 - B^2 - 2AB - A^2)h^2 + O(h^3)\\
		&= (1/2)(BA - AB)h^2 + O(h^3)
	\end{align*}
	as desired.
\end{proof}


\subsection*{Problem 3}
Recall that Euler's method for solving $y' = f(y)$ subject to $y(t_0) = y_0$
involves (page 485 of [A\&G], where $a,c$ there are $t_0,y_0$ here)
(1) choosing an $h > 0$,
(2) setting $t_i = t_0 + ih$, and
(3) approximating $y(t_i)$ as $y_i$ using the formula
\[
	y_{i+1} = y_i + hf(y_i)
\]
Now consider the ODE where $f(y) = |y|^{1/2}$.
\begin{enumerate}
	\item Say that for some $i \geq 1$, $y_{i+1} = 0$.
		Show that $y_i$ is either $0$ or $-h^2$.
	\item Say that for some $i \geq 1$, $y_{i+1} < 0$. Show that:
		\begin{enumerate}
			\item[(i).] $y_i < 0$; and
			\item[(ii).] using part (i), setting $x = \sqrt{-y_i}$, i.e.,
				$x$ is the positive square root of $-y_i$, show that
				\[
					x = \frac{h+\sqrt{h^2 - 4y_{i+1}}}{2},
				\]
				(where $\sqrt{h^2 - 4y_{i+1}}$ refrs to the positive square root)
				and hence
				\[
					y_i = -\left(\frac{h+\sqrt{h^2 - 4y_{i+1}}}{2}\right)^2
				\]
		\end{enumerate}
	\item Show that if for some $i \geq 1$, $y_{i+1} < 0$ and
		$y_{i+1} = -uh^2$ for a real $u > 0$, then
		\[
			y_i = -h^2g(u) \quad \text{where} \quad
			g(u) = \frac{1+2u+\sqrt{1+4u}}{2}
		\]
\end{enumerate}
\begin{enumerate}
	\item \begin{proof}[Solution]\let\qed\relax
		Plugging in our values for variables, we have $0 = y_i + h|y_i|^{1/2}$,
		and rearranging gives $-y_i = h|y_i|^{1/2}$.
		First, note that $y_i = 0$ satisfies our equation.
		Now, assume that $y_i \neq 0$.
		Squaring both sides gives $y_i^2 = h^2|y_i| \implies |y_i| = h^2$,
		where we can divide by $|y_i|$ because it is nonzero.
		If $y_i = h^2$, we have $0 = h^2 + h|h^2|^{1/2} = h^2 + h^2$.
		But squares are nonnegative, and the sum of two nonnegative numbers
		is also nonnegative;
		but also we don't have equality becuase $y_i \neq 0 \implies h^2 \neq 0$,
		hence $0 = h^2 + h^2 > 0$, so we cannot have $y_i = h^2$.
		Now if $y_i = -h^2$, we have $0 = -h^2 + h|-h^2|^{1/2} = -h^2 + h^2 = 0$,
		so $y_i = -h^2$ works.

		Hence, we must either have $y_i = 0$ or $y_i = -h^2$.
	\end{proof}
	\item \begin{proof}[Solution]\let\qed\relax
		\begin{enumerate}
			\item[(i).] We prove the contrapositive.
				That is, for some $i \geq 1$, assume that $y_i \geq 0$.
				Note $h|y_i|^{1/2} \geq 0$ since $h > 0$ and $|x|^{1/2} \geq 0$
				for all $x \in \R$.
				Hence, $y_i + h|y_i|^{1/2} = y_{i+1} \geq 0$.
				Therefore, we have shown $y_{i+1} < 0 \implies y_i < 0$.
			\item[(ii).] Rearranging the Euler's method formula, we get
				\[
					y_{i+1} = y_i + h|y_i|^{1/2}
				\]
				Plugging in $x = \sqrt{-y_i}$, note that
				$y_i = -x^2$ and since $-y_i = |y_i|$ since $y_i < 0$ by part (i),
				we have $|y_i|^{1/2} = \sqrt{-y_i} = x$.
				Then subbing back in, we get
				\[
					y_{i+1} = -x^2 + hx \implies x^2 - hx + y_{i+1} = 0
				\]
				The quadratic equation gives us
				\[
					x = \frac{h \pm \sqrt{h^2 - 4y_{i+1}}}{2}
				\]
				But $y_{i+1} < 0 \implies -4y_{i+1} > 0 \implies
				h^2 - 4y_{i+1} > h^2 \implies h = \sqrt{h^2} < \sqrt{h^2 - 4y_{i+1}}$
				hence $\frac{h-\sqrt{h^2 - 4y_{i+1}}}{2} < 0$,
				but we said $x$ was positive.
				Thus
				\[
					x = \frac{h + \sqrt{h^2 - 4y_{i+1}}}{2}
				\]
				(which is obviously positive, since we are adding two
				positive values in the numerator).

				Now, we substitute $x = \sqrt{-y_i}$ to recover
				\[
					\sqrt{-y_i} = \frac{h + \sqrt{h^2 - 4y_{i+1}}}{2}
					\implies y_i = -\left(\frac{h + \sqrt{h^2 - 4y_{i+1}}}{2}\right)^{2}
				\]
				as desired.
		\end{enumerate}
	\end{proof}
	\item \begin{proof}[Solution]\let\qed\relax
		Using the formula from the previous part, we plug in $y_{i+1} = -uh^2$ to get
		\begin{align*}
			y_i &= -\left(\frac{h + \sqrt{h^2+4uh^2}}{2}\right)^2\\
				&= -\left(\frac{h + h\sqrt{1+4u}}{2}\right)^2\\
				&= -\left(\frac{h(1 + \sqrt{1+4u})}{2}\right)^2\\
				&= -h^2\left(\frac{1 + \sqrt{1+4u}}{2}\right)^2\\
				&= -h^2\left(\frac{1 + 2\sqrt{1+4u} + 1 + 4u}{4}\right)\\
				&= -h^2\left(\frac{1 + 2u + \sqrt{1+4u}}{4}\right)\\
				&= -h^2g(u)
		\end{align*}
		as desired.
	\end{proof}
\end{enumerate}

\subsection*{Problem 4}
Consider the ODE $y' = |y|^{1/2}$, with subject to $y(0) = y_0$ and arbitrary $h > 0$.
\begin{enumerate}
	\item Using the results of the previous exercise,
		find a value of $y_0 < 0$ (as a function of $h$)
		such that (in exact arithmetic) $y_1 = 0$,
		and therefore $y_2 = 0$, $y_3 = 0$, etc.
	\item Using the results of the previous exercise,
		find a value of $y_0 < 0$ (as a function of $h$) such that
		(in exact arithmetic) $y_1 < 0$ and $y_2 = y_3 = \cdots = 0$.
	\item Use the code from Homework 2 called \verb|chaotic_sqrt.m|,
		or implement your own Euler's method solver for $y' = |y|^{1/2}$.
		What does MATLAB report for $y(2)$ on the initial condition
		$y(0) = y_0$ where $h = 2/N$ for the values:
		$N = 2,4,16,64,65,63,99,100,101$ and $y_0 = -(2/N)^2 = -h^2$?
		Make a table of computed values of $y(2)$.
		Which values of $N$ above yields something likely due to roundoff
		or truncation error in finite precision?
		(Therefore you might want to type something like
		\verb|chaotic_sqrt(0,2,63,-4/(63^2));| into MATLAB for $N=63$.)
	\item Use the code from Homework 2 called \verb|chaotic_sqrt.m|,
		or implement your own Euler's method solver for $y' = |y|^{1/2}$.
		What does MATLAB report for $y(2)$ on the initial condition $y(0) = y_0$
		where $h = 2/N$ for the values: $N = 7,8,9,10,100,101,10^6,10^6+1$ and for
		\[
			y_0 = -h^2\left(\frac{3+\sqrt{5}}{2}\right)?
		\]
		Type this same expression into MATLAB as written above
		(e.g., don't write down an approximation for $\sqrt{5}$).
		(Therefore you might want to type something like
		\verb|chaotic_sqrt(0,2,100,-(4/(100^2))*(3+sqrt(5))/2);|
		into MATLAB for $N = 100$.)
		Which values of $N$ above yields something likely due to roundoff
		or truncation error in finite precision?
\end{enumerate}
\begin{enumerate}
	\item \begin{proof}[Solution]\let\qed\relax
		Since $y_1 = 0$, and our ODE is $y' = |y|^{1/2}$,
		we can invoke question 3(a) to get
		$y_0 = 0, -h^2$.
		Since we ask for $y_0 \neq 0$, we have $y_0 = -h^2$,
		which is less than $0$ since $h^2 > 0$.
	\end{proof}
	\item \begin{proof}[Solution]\let\qed\relax
		As with the previous part, we can use our results from question 3.
		Now, since $y_2 = 0$, we can invoke question 3(a) to get
		$y_1 = 0, -h^2$. But since we say $y_1 \neq 0$,
		this gives us $y_1 = -h^2$.
		Now, from question 3(c) (since $y_1 = y_{0+1} < 0$), we have $y_{0 + 1} = -uh^2$
		where $u = 1$, and so we get $y_0 = -h^2g(1)$.
		But $g(1) = \frac{1+2(1) + \sqrt{1 + 4(1)}}{2} = \frac{3 + \sqrt{5}}{2}$.
		Hence $y_0 = -h^2\frac{3+\sqrt{5}}{2}$,
		which is less than $0$ since $h^2 > 0$.
	\end{proof}
	\item \begin{proof}[Solution]\let\qed\relax
		Here is my table of computed values:
			\begin{center}
				\begin{tabular}{ | c | c | }
					\hline
					$N$ & $y(2)$ \\
					\hline
					$2$ & $0$ \\ 
					$4$ & $0$ \\  
					$16$ & $0$ \\
					$64$ & $0$ \\
					$65$ & $0$ \\
					$63$ & $0.8042$ \\
					$99$ & $0$ \\
					$100$ & $0$ \\
					$101$ & $0.8703$ \\
					\hline
				\end{tabular}
			\end{center}
		By part (a) of this problem, since $y_0 = -h^2$ regardless of $N$,
		$y_1 = y_2 = \cdots = 0$.
		Hence, theory says $y(2) = 0$.
		Out of the above values, $N = 63, 101$ are instances
		of something due to roundoff or truncation error in finite precision,
		since they are not equal to $0$.
	\end{proof}
	\item \begin{proof}[Solution]\let\qed\relax
		Here is my table of computed values:
			\begin{center}
				\begin{tabular}{ | c | c | }
					\hline
					$N$ & $y(2)$ \\
					\hline
					$7$ & $0$ \\ 
					$8$ & $0$ \\  
					$9$ & $0$ \\
					$10$ & $0$ \\
					$100$ & $0.8507$ \\
					$101$ & $0$ \\
					$10^6$ & $0$ \\
					$10^6+1$ & $1.000$ \\
					\hline
				\end{tabular}
			\end{center}
		By part (b) of this problem, since $y_0 = -h^2\left(\frac{3+\sqrt{5}}{2}\right)$
		regardless of $N$, $y_2 = y_3 = \cdots = 0$.
		Hence, theory says $y(2) = 0$ when $N \geq 2$.
		Out of the above values, $N = 100, 10^6+1$ are instances
		of something due to roundoff or truncation error in finite precision,
		since they are not equal to $0$.
	\end{proof}
\end{enumerate}
\end{document}
