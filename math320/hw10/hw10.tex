\documentclass{article}
\usepackage{amsmath, amsfonts, amsthm, amssymb}
\usepackage{geometry}
\geometry{letterpaper, margin=2.0cm, includefoot, footskip=30pt}

\usepackage{fancyhdr}
\pagestyle{fancy}

\lhead{Math 320}
\chead{Homework 10}
\rhead{Nicholas Rees, 11848363}
\cfoot{Page \thepage}

\newtheorem*{problem}{Problem}

\newcommand{\N}{{\mathbb N}}
\newcommand{\Z}{{\mathbb Z}}
\newcommand{\Q}{{\mathbb Q}}
\newcommand{\R}{{\mathbb R}}
\newcommand{\C}{{\mathbb C}}
\newcommand{\ep}{{\varepsilon}}
\newcommand{\SR}{{\mathcal R}}

\renewcommand{\theenumi}{(\alph{enumi})}

\begin{document}
\subsection*{Problem 1}
{\it Prove the following theorem (terminology is given below):
\begin{center}
	Suppose $X$ is compact and $f \colon X \to \R$ is lower semicontinuous.
	Then $F$ is bounded below on $X$,
	and there exists a point $z \in X$ satisfying $f(z) \leq f(x)$ for all $x \in X$.
\end{center}
Recall that in a HTS $(X, \mathcal{T})$, a function $f \colon X \to \R$
is called \emph{lower semicontinuous} if the following set
is closed for every $p \in \R$:
\[
	f^{-1}((-\infty,p]) = \{x \in X \colon f(x) \leq p \}.
\]
(One approach uses the family of closed sets
$f^{-1}((-\infty,p])$ satisfyin $p > \inf f(x)$.)}

\begin{proof}[Solution]\let\qed\relax
	ff
\end{proof}
\clearpage
~\clearpage

\subsection*{Problem 2}
{\it Let $(X, d)$ be a metric space, with $K \subseteq X$ a compact set.
Prove that whenever $\mathcal{G}$ is an open cover for $K$,
there exists $r < 0$ with this property:
for every pair of points $x,y \in K$ obeying $d(x,y) < r$,
some open set $G \in \mathcal{G}$ contains both $x$ and $y$.}

\begin{proof}[Solution]\let\qed\relax
	Let $G_1, G_2, \dots, G_N$ be the finite subcover of $K$
	such that $G_i \in \mathcal{G}$ for all $i \in 1,2,\dots,N$
	and $K \subseteq \bigcup_{1\leq i \leq N}G_i$,
	which we know exists from the compactness of $K$.
	Define $I_{i,j} := G_i \cap G_j$ where $1 \leq i < j \leq N$.
	Note then that $I_{i,j}$ is open as well.
	For each $I_{i,j}$, pick some $x_{i,j} \in I_{i,j}$ if $I_{i,j} \neq \emptyset$.
	Then, since $X$ is a metric space and $I_{i,j}$ is an open set,
	we must have that there exists some $r_{i,j} > 0$
	such that $\mathbb{B}[x_{i,j};r_{i,j}) \subseteq I_{i,j}$
	(if $I_{i,j} = \emptyset$, just let $r_{i,j} = 1$).
	Let $r = \min_{1\leq i<j\leq N}\{r_{i,j}\}$.
	Since there are only finitely many $r_{i,j}$, all of them greater than $0$,
	we must have $r > 0$ as well.
	
	Now consider any $x,y \in K$ such that $d(x,y) < r$.
	Consider $\mathbb{B}[x;r)$.
	By our distance condition, we have $y \in \mathbb[x;r)$.
	It is sufficient to show now that there exists some $G \in \mathcal{G}$
	such that $\mathbb{B}[x;r) \subseteq G$.
	Since $\{G_i\}_{1\leq i\leq N}$ is a covering of all of $K$,
	there exists some $G_x \in \{G_i\}_{1\leq i\leq N}$ such that $x \in G_x$.
	If $\mathbb{B}[x;r) \subseteq G_x$, we are done.
	For the sake of contradiction, assume then that $\mathbb{B}[x;r) \not\subseteq G_x$.
	hmm it's not this, it's either in $G_x$, or in $G'_x$...
	consider the $G_i$ for which $x$ is in?

	I don't even have that $\mathbb{B}[x;r) \subseteq K$.
	Hmm, so we don't want to lose $y$, because we'll always have
	$x \in \partial K$ (since $K$ is closed) violate our condition with the ball.
	ff

	So let $x \in G_x$ and $y \in G_y$.
	We claim that either $x \in G_y$ or $y \in G_x$
	(this is trivial if $G_x = G_y$, so we now assume that $G_x \neq G_y$).
	For the sake of contradiction, assume this is false:
	$x, y \not\in G_x \cap G_y$.
	Then by construction, for some $x_{x,y} \in G_x \cap G_y$,
	$x,y \not\in \mathbb{B}[x_{x,y},r)$.
	Then $d(x_{x,y},x) > r$ and $d(x_{x,y},y) > r$.
	We hav $d(x,y) \leq d(x_{x,y},x) + d(x_{x,y},y)$.
	Hmm... the thing on the right is bounded below by $2r$,
	and the thing on the left is bounded above by $r$.
	We want to use this inequality in a different way I think,
	like $d(x_{x,y},x) < d(x,y)$ and $d(x_{x,y},y) < d(x,y)$ somehow.
	Maybe something to do with them being in the same $G_i$?

	So we have $d(x_{x,y}, x) < d(x,y) + d(x_{x,y},y)$.
	But $d(x_{x,y},y) < d(x,y) + d(x_{x,y},x)$, thus
	$d(x_{x,y}, x) < 2d(x,y) + d(x_{x,y},x) \implies 0 < 2d(x,y)$ trivially.
	ff
	So the claim I want to make is that the intersection between
	two open sets in a metric space is closer to each other than
	they are to each other.


	Hmm, we might need to define $r$ in terms of points that are
	not in the same $G_i$, but are close to each other.
	Think of a square donut that his a tiny slit that doesn't touch,
	and an open cover of two sets that intersect on the non-slit side.

	So it seems the problem is when I fix the slit,
	I get a problem with the intersection.
	When I fix the intersection, I get a problem with the slit.
	I will think about this later.
\end{proof}
\clearpage
~\clearpage

\subsection*{Problem 3}
{\it Define the set-valued ``projection" mapping
$p_1 \colon \mathcal{P}(\R^2) \to \mathcal{P}(\R)$ by
\[
	p_1(S) = \{x_1 \in \R \colon (x_1,x_2) \in S \text{ for some }x_2\},
	\qquad S \subseteq \R^2
\]
\begin{enumerate}
	\item If $S$ is bounded, must $p_1(S)$ be bounded? (Why or why not?)
	\item If $S$ is closed, must $p_1(S)$ be closed? (Why or why not?)
	\item If $S$ is compact, must $p_1(S)$ be compact? (Why or why not?)
\end{enumerate}}

\begin{enumerate}
	\item \begin{proof}[Solution]\let\qed\relax
		ff
	\end{proof}
	\item \begin{proof}[Solution]\let\qed\relax
		ff
	\end{proof}
	\item \begin{proof}[Solution]\let\qed\relax
		ff
	\end{proof}
\end{enumerate}
\clearpage
~\clearpage

\subsection*{Problem 4}
{\it Recall the set $\ell^2$ from HW07 Q3,
and the standard ``unit vectors" $\hat{\mathbb{e}}_p = (0,0,\dots,0,1,0,\dots)$,
where the only nonzero entry in $\hat{\mathbb{e}}_p$ occurs in component $p$.
For any $x$ in $\ell^2$ and subset $V \subseteq \ell^2$, write
\[
	\Omega(x;V) = \{y \in \ell^2 \colon -1 < (v,y-x)<1, \forall v \in V\}.
\]
Then define a collection $\mathcal{T}$ of subsets of $\ell^2$ by saying
$G \in \mathcal{T}$ if and only if every point $x \in G$ has the property
that $x \in \Omega(x;V) \subseteq G$ for some \emph{finite set} $V \subseteq \ell^2$.
\begin{enumerate}
	\item Prove that $\Omega(x;V) \in \mathcal{T}$ of every finite set
		$V \subseteq \ell^2$ and point $x \in \ell^2$.
	\item Prove that $(\ell^2, \mathcal{T})$ is a Hausdorff Topological Space.
	\item Let $S = \{\hat{\mathbb{e}}_p \colon p \in \N\}$.
		Prove that $0 \in S'$.
		(Here $0$ denotes $(0,0,\dots)$, the ``origin in $\ell^2$.)
		\emph{Note}: This fact proves that $\mathcal{T}$ is
		different from the metric topology on $\ell^2$.
	\item Prove that every $G$ in $\mathcal{T}$ has the property:
		for every $x$ in $G$, there exists $r>0$ such that
		\[
			G \supseteq \mathbb{B}[x;r) = \{y \in \ell^2 \colon
			\lVert y - x \rVert < r\}.
		\]
		This fact proves that every set considered ``open" in $\mathcal{T}$
		is also open in the metric topology on $\ell^2$.
		This explains why $\mathcal{T}$ gets called ``the weak topology"
		and the metric topology is also called ``the strong topology."
	\item Prove that the following set is closed in the weak topology of $\ell^2$:
		$\mathbb{B}[0;1] = \{y \in \ell^2 \colon \lVert y \rVert \leq 1\}$.
\end{enumerate}}

\begin{enumerate}
	\item \begin{proof}[Solution]\let\qed\relax
		ff
	\end{proof}
	\item \begin{proof}[Solution]\let\qed\relax
		ff
	\end{proof}
	\item \begin{proof}[Solution]\let\qed\relax
		ff
	\end{proof}
	\item \begin{proof}[Solution]\let\qed\relax
		ff
	\end{proof}
	\item \begin{proof}[Solution]\let\qed\relax
		ff
	\end{proof}
\end{enumerate}
\clearpage
\end{document}
