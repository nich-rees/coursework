\documentclass{article}
\usepackage{amsmath, amsfonts, amsthm, amssymb}
\usepackage{geometry}
\geometry{letterpaper, margin=2.0cm, includefoot, footskip=30pt}

\usepackage{fancyhdr}
\pagestyle{fancy}

\lhead{Math 320}
\chead{Homework 10}
\rhead{Nicholas Rees, 11848363}
\cfoot{Page \thepage}

\newtheorem*{problem}{Problem}

\newcommand{\N}{{\mathbb N}}
\newcommand{\Z}{{\mathbb Z}}
\newcommand{\Q}{{\mathbb Q}}
\newcommand{\R}{{\mathbb R}}
\newcommand{\C}{{\mathbb C}}
\newcommand{\ep}{{\varepsilon}}
\newcommand{\SR}{{\mathcal R}}

\renewcommand{\theenumi}{(\alph{enumi})}

\begin{document}
\subsection*{Problem 1}
{\it Prove the following theorem (terminology is given below):
\begin{center}
	Suppose $X$ is compact and $f \colon X \to \R$ is lower semicontinuous.
	Then $f$ is bounded below on $X$,
	and there exists a point $z \in X$ satisfying $f(z) \leq f(x)$ for all $x \in X$.
\end{center}
Recall that in a HTS $(X, \mathcal{T})$, a function $f \colon X \to \R$
is called \emph{lower semicontinuous} if the following set
is closed for every $p \in \R$:
\[
	f^{-1}((-\infty,p]) = \{x \in X \colon f(x) \leq p \}.
\]
(One approach uses the family of closed sets
$f^{-1}((-\infty,p])$ satisfying $p > \inf f(x)$.)}

\begin{proof}[Solution]\let\qed\relax
	We first prove that $f$ is bounded below on $X$,
	that is, $\inf{f(x)} > -\infty$.
	For the sake of contradiction, assume the opposite, that $\inf{f(x)} = -\infty$.
	Consider $f^{-1}((-\infty, p])$ for some $p \in \R$.
	ff

	Supposedly, we just have $f^{-1}((-\infty,p])^c = f^{-1}((-\infty,p]^c) = f^{-1}((p,\infty))$
	which is open.
	And maybe a union?
	And open covers over all of $X$, and can get a finite subcover.ff

	Consider the family of closed sets of $f^{-1}((-\infty,p])$
	satisfying $p > \inf{f(x)}$, call it $\mathcal{F}$.
	First, remark that each element in $\mathcal{F}$ is nonempty,
	otherwise $f^{-1}(-\infty,p])$ is nonempty,
	thus there is no $x_0 \in X$ where $f(x_0) \in (-\infty,p]$
	and so $p \leq \inf{f(x)}$, which we assumed not true.
	Secondly, by the assumption that $f$ is lower semicontinuous,
	each element in $\mathcal{F}$ is also closed.
	Finally, note that $\mathcal{F}$ has the finite intersection property:
	let $N \in \N$ and $F_1, \dots, F_N$ are sets in $\mathcal{F}$,
	which we can write explicitly as $F_i = f^{-1}((-\infty, p_i])$
	where $p_i > \inf{f(x)}$;
	denote $p_0 = \min_i\{p_i\}$.
	Then $F_0 = f^{-1}((-\infty, p_0]) \subseteq F_i$ for all $1 \leq i \leq N$,
	and since we're just minimizing over a finite number of sets,
	$F_0 \in \{F_1,\dots,F_n\} \subseteq \mathcal{F}$, thus
	\[
		\bigcap_{i=1}^N F_i = f^{-1}((-\infty,p_o]) = F_0 \neq \emptyset
	\]
	so we have the finite intersection property.

	Now, since we're in a a HTS and $X$ is compact,
	any collection of elements of $\mathcal{F}$ has nonempty intersection,
	by the theorem proved in class
	(every element is a subset of $X$ and are closed,
	and any finite collection has the finite intersection property).
	Notably, $\bigcap \mathcal{F} \neq \emptyset$.
	This means that there exists some $z \in X$ where $z \in \bigcap\mathcal{F}$.
	Then, for all $p > \inf{f(x)}$, we have $z \in f^{-1}((-\infty,p])$.
	ff
\end{proof}
\clearpage
~\clearpage

\subsection*{Problem 2}
{\it Let $(X, d)$ be a metric space, with $K \subseteq X$ a compact set.
Prove that whenever $\mathcal{G}$ is an open cover for $K$,
there exists $r < 0$ with this property:
for every pair of points $x,y \in K$ obeying $d(x,y) < r$,
some open set $G \in \mathcal{G}$ contains both $x$ and $y$.}

\begin{proof}[Solution]\let\qed\relax
	Let $G_1, G_2, \dots, G_N$ be the finite subcover of $K$
	such that $G_i \in \mathcal{G}$ for all $i \in 1,2,\dots,N$
	and $K \subseteq \bigcup_{1\leq i \leq N}G_i$,
	which we know exists from the compactness of $K$.
	Define $I_{i,j} := G_i \cap G_j$ where $1 \leq i < j \leq N$.
	Note then that $I_{i,j}$ is open as well.
	For each $I_{i,j}$, pick some $x_{i,j} \in I_{i,j}$ if $I_{i,j} \neq \emptyset$.
	Then, since $X$ is a metric space and $I_{i,j}$ is an open set,
	we must have that there exists some $r_{i,j} > 0$
	such that $\mathbb{B}[x_{i,j};r_{i,j}) \subseteq I_{i,j}$
	(if $I_{i,j} = \emptyset$, just let $r_{i,j} = 1$).
	Let $r = \min_{1\leq i<j\leq N}\{r_{i,j}\}$.
	Since there are only finitely many $r_{i,j}$, all of them greater than $0$,
	we must have $r > 0$ as well.
	
	Now consider any $x,y \in K$ such that $d(x,y) < r$.
	Consider $\mathbb{B}[x;r)$.
	By our distance condition, we have $y \in \mathbb[x;r)$.
	It is sufficient to show now that there exists some $G \in \mathcal{G}$
	such that $\mathbb{B}[x;r) \subseteq G$.
	Since $\{G_i\}_{1\leq i\leq N}$ is a covering of all of $K$,
	there exists some $G_x \in \{G_i\}_{1\leq i\leq N}$ such that $x \in G_x$.
	If $\mathbb{B}[x;r) \subseteq G_x$, we are done.
	For the sake of contradiction, assume then that $\mathbb{B}[x;r) \not\subseteq G_x$.
	hmm it's not this, it's either in $G_x$, or in $G'_x$...
	consider the $G_i$ for which $x$ is in?

	I don't even have that $\mathbb{B}[x;r) \subseteq K$.
	Hmm, so we don't want to lose $y$, because we'll always have
	$x \in \partial K$ (since $K$ is closed) violate our condition with the ball.
	ff

	So let $x \in G_x$ and $y \in G_y$.
	We claim that either $x \in G_y$ or $y \in G_x$
	(this is trivial if $G_x = G_y$, so we now assume that $G_x \neq G_y$).
	For the sake of contradiction, assume this is false:
	$x, y \not\in G_x \cap G_y$.
	Then by construction, for some $x_{x,y} \in G_x \cap G_y$,
	$x,y \not\in \mathbb{B}[x_{x,y},r)$.
	Then $d(x_{x,y},x) > r$ and $d(x_{x,y},y) > r$.
	We hav $d(x,y) \leq d(x_{x,y},x) + d(x_{x,y},y)$.
	Hmm... the thing on the right is bounded below by $2r$,
	and the thing on the left is bounded above by $r$.
	We want to use this inequality in a different way I think,
	like $d(x_{x,y},x) < d(x,y)$ and $d(x_{x,y},y) < d(x,y)$ somehow.
	Maybe something to do with them being in the same $G_i$?

	So we have $d(x_{x,y}, x) < d(x,y) + d(x_{x,y},y)$.
	But $d(x_{x,y},y) < d(x,y) + d(x_{x,y},x)$, thus
	$d(x_{x,y}, x) < 2d(x,y) + d(x_{x,y},x) \implies 0 < 2d(x,y)$ trivially.
	ff
	So the claim I want to make is that the intersection between
	two open sets in a metric space is closer to each other than
	they are to each other.


	Hmm, we might need to define $r$ in terms of points that are
	not in the same $G_i$, but are close to each other.
	Think of a square donut that his a tiny slit that doesn't touch,
	and an open cover of two sets that intersect on the non-slit side.

	So it seems the problem is when I fix the slit,
	I get a problem with the intersection.
	When I fix the intersection, I get a problem with the slit.
	I will think about this later.


	So Tighe said to do a proof by contradiction.
\end{proof}
\clearpage
~\clearpage

\subsection*{Problem 3}
{\it Define the set-valued ``projection" mapping
$p_1 \colon \mathcal{P}(\R^2) \to \mathcal{P}(\R)$ by
\[
	p_1(S) = \{x_1 \in \R \colon (x_1,x_2) \in S \text{ for some }x_2\},
	\qquad S \subseteq \R^2
\]
\begin{enumerate}
	\item If $S$ is bounded, must $p_1(S)$ be bounded? (Why or why not?)
	\item If $S$ is closed, must $p_1(S)$ be closed? (Why or why not?)
	\item If $S$ is compact, must $p_1(S)$ be compact? (Why or why not?)
\end{enumerate}}

\begin{enumerate}
	\item \begin{proof}[Solution]\let\qed\relax
		It must. If $S$ is bounded, then by definition,
		there exists $x \in S$ and $R>0$ such that
		$S \subseteq \mathbb{B}[x;R)$.
		Using the the standard metric on $\R^2$
		(namely $d(x,y) = \sqrt{(y_1 - x_1)^2 + (y_2 - x_2)^2}$),
		this means for any $y \in S$, we have
		$d(x,y) < r$, or $\sqrt{(y_1 - x_1)^2 + (y_2 - x_2)^2} < R$.
		Consider $x_1 = p_1(x)$.
		Then for any $y_1 \in p_1(S)$
		(using the standard metric on $\R$, $d(x,y) = |y-x|$), we have
		\[
			d(x_1,y_1) = |y_1 - x_1| = \sqrt{(y_1-x_1)^2}
			\leq \sqrt{(y_1-x_1)^2 + (y' - x_2)^2} < R
		\]
		where $y' \in p^{-1}(y_1)$,
		and so the last inequality follows from the boundedness of $S$.
		Thus, $p_1(S) \subseteq \mathbb{B}[x_1;R)$,
		so $p_1(S)$ is bounded.
	\end{proof}
	\item \begin{proof}[Solution]\let\qed\relax
		It must. Consider some $x_1 \in p_1(S)^c$,
		and some open set $U \in \mathcal{N}(x_1)$.
		We can write this explicitly as $U = \mathbb{B}[x_1;r)$
		for some $r>0$,
		or $U = \{y_1 \in \R \colon |y_1 - x_1| < r\}$.
		We have $p_1^{-1}(U) = $
		We want to show that $U \subseteq p_1(S)^c$.
		ff
	\end{proof}
	\item \begin{proof}[Solution]\let\qed\relax
		It must.
		By the Heine-Borel Theorem,
		since $S$ is compact in $\R^2$, we have that $S$ is closed and bounded.
		Then by parts (a) and (b) of this problem,
		$p_1(S)$ must also be closed and bounded.
		Hence, by Heine-Borel again,
		we have that $p_1(S)$ is compact.
	\end{proof}
\end{enumerate}
\clearpage
~\clearpage

\subsection*{Problem 4}
{\it Recall the set $\ell^2$ from HW07 Q3,
and the standard ``unit vectors" $\hat{\mathbf{e}}_p = (0,0,\dots,0,1,0,\dots)$,
where the only nonzero entry in $\hat{\mathbf{e}}_p$ occurs in component $p$.
For any $x$ in $\ell^2$ and subset $V \subseteq \ell^2$, write
\[
	\Omega(x;V) = \{y \in \ell^2 \colon -1 < \langle v,y-x\rangle<1, \forall v \in V\}.
\]
Then define a collection $\mathcal{T}$ of subsets of $\ell^2$ by saying
$G \in \mathcal{T}$ if and only if every point $x \in G$ has the property
that $x \in \Omega(x;V) \subseteq G$ for some \emph{finite set} $V \subseteq \ell^2$.
\begin{enumerate}
	\item Prove that $\Omega(x;V) \in \mathcal{T}$ for every finite set
		$V \subseteq \ell^2$ and point $x \in \ell^2$.
	\item Prove that $(\ell^2, \mathcal{T})$ is a Hausdorff Topological Space.
	\item Let $S = \{\hat{\mathbf{e}}_p \colon p \in \N\}$.
		Prove that $0 \in S'$.
		(Here $0$ denotes $(0,0,\dots)$, the ``origin in $\ell^2$.)
		\emph{Note}: This fact proves that $\mathcal{T}$ is
		different from the metric topology on $\ell^2$.
	\item Prove that every $G$ in $\mathcal{T}$ has the property:
		for every $x$ in $G$, there exists $r>0$ such that
		\[
			G \supseteq \mathbb{B}[x;r) = \{y \in \ell^2 \colon
			\lVert y - x \rVert < r\}.
		\]
		This fact proves that every set considered ``open" in $\mathcal{T}$
		is also open in the metric topology on $\ell^2$.
		This explains why $\mathcal{T}$ gets called ``the weak topology"
		and the metric topology is also called ``the strong topology."
	\item Prove that the following set is closed in the weak topology of $\ell^2$:
		$\mathbb{B}[0;1] = \{y \in \ell^2 \colon \lVert y \rVert \leq 1\}$.
\end{enumerate}}

\begin{enumerate}
	\item \begin{proof}[Solution]\let\qed\relax
		Let $x' \in \Omega(x;V)$.
		Want to show there exists a finite set $V' \subseteq \ell^2$
		such that $\Omega(x';V') \subseteq \Omega(x,V)$.
		Then $-1 < \langle v, x' - x \rangle < 1$ for all $v \in V$.
		This is equivalent to
		\[
			-1 < \sum_{n = 1}^\infty v_n(x'_n - x_n) < 1
		\]
		for all $v \in V$.
		ff
	\end{proof}
	\item \begin{proof}[Solution]\let\qed\relax
		We have that $\emptyset \in \mathcal{T}$,
		since there does not exist $x \in \emptyset$ so it
		satisfies our condition to be in $\mathcal{T}$ vacuously.
		We also have $\ell^2 \in \mathcal{T}$,
		since $\Omega(x;V)$ is composed of elements of $\ell^2$,
		and so for any $x \in \ell^2$, $\Omega(x;V) \subseteq \ell^2$.
		
		Now consider $\mathcal{G} \subseteq \mathcal{T}$.
		Consider an arbitrary element $x \in \bigcup \mathcal{G}$.
		Then for some $G \in \mathcal{G}$, we have $x \in G$.
		Then $\Omega(x;V) \subseteq G \subseteq \bigcup\mathcal{G}$
		for some finite set $V \subseteq \ell^2$ since $G \in \mathcal{T}$,
		so $\bigcup \mathcal{G} \in \mathcal{T}$ as well.

		Now consider $U_1,\dots,U_N \in \mathcal{T}$ where $N \in \N$.
		Consider an arbitrary element $x \in \bigcap_i^N U_i$.
		Then for all $1 \leq i \leq N$, $x \in U_i$.
		Then by definition of each $U_i$ being in $\mathcal{T}$,
		we have that there exists a finite set $V_i \subseteq \ell^2$
		such that $\Omega(x;V_i) \subseteq U_i$.
		ff

		Finally, let $x, y \in \ell^2$ such that $x \neq y$.
		ff this one doesn't look bad, we use part (a)
	\end{proof}
	\item \begin{proof}[Solution]\let\qed\relax
			Let $U \in \mathcal{N}(0)$ be an arbitrary open set,
			ie. $U \in \mathcal{T}$ such that $0 \in U$.
			We want to show that $(U\setminus\{0\})\cap S \neq \emptyset$;
			since $0 \not\in S$ anyway, we just need to show
			$U \cap S \neq \emptyset$.

			Since $0 \in U$, there exists a finite set $V\subseteq \ell^2$
			such that $\Omega(0;V) \subseteq U$.
			If $V = \emptyset$, then $\Omega(0;V) = \ell^3$
			since $-1 < \langle v, y-x \rangle < 1$ is now
			vacuously true for all $y \in \ell^2$;
			then $\Omega(0;V) \cap S$ since $\hat{\mathbf{e}}_1 \in \ell^2 \cap S$,
			and since $\Omega(0;V) \subseteq U$, $U \cap S \neq \emptyset$.
			So now assume $V$ is not empty.
			Denote the elements of $V$ as $v^{i}$ where $1 \leq i \leq k$.
			Then since $v^i \in \ell^2$,
			we must have that $\lim_n (v^i_n)^2 = 0$ (crude divergence test).
			Then there exists some $N_i$ where $(v^i_{N_i})^2 < 1$ by the definition of convergence.
			Let $N = \min_i\{N_i\}$.
			Then $-1 < v^i_N < 1$ as well.
			See
			\[
				\langle v^i, \hat{\mathbf{e}}_N \rangle =
				\sum_{n=1}^\infty v^i_n (\hat{\mathbf{e}}_N)_n
				= v^i_N
			\]
			Thus $\hat{\mathbf{e}}_N \in \Omega(0,V)$ since
			$-1 < \langle v, \hat{\mathbf{e}}_N - 0 \rangle = v_N < 1$
			for all $v \in V$.
			Thus, $\hat{\mathbf{e}}_N \in \Omega(0,V) \subseteq U$.
			Since $\hat{\mathbf{e}}_N \in S$,
			thus shows that $S \cap U \neq \emptyset$,
			so we are done since $U$ was arbitrary
			(this works for any open $U \in \mathcal{N}(0)$).
	\end{proof}
	\item \begin{proof}[Solution]\let\qed\relax
		ff
	\end{proof}
	\item \begin{proof}[Solution]\let\qed\relax
		ff

		Recap of what still needs to be done:
		\begin{itemize}
			\item Wrapping up Q1: bounded case and end of attaining minimum
			\item All of Q2 (with contradiction)
			\item 3(b) (closed implies closed)
			\item 4(a), 4(b) with the second two conditions for HTS,
				4(d), 4(e)
		\end{itemize}
	\end{proof}
\end{enumerate}
\clearpage
\end{document}
