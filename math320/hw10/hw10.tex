\documentclass{article}
\usepackage{amsmath, amsfonts, amsthm, amssymb}
\usepackage{geometry}
\geometry{letterpaper, margin=2.0cm, includefoot, footskip=30pt}

\usepackage{fancyhdr}
\pagestyle{fancy}

\lhead{Math 320}
\chead{Homework 10}
\rhead{Nicholas Rees, 11848363}
\cfoot{Page \thepage}

\newtheorem*{problem}{Problem}

\newcommand{\N}{{\mathbb N}}
\newcommand{\Z}{{\mathbb Z}}
\newcommand{\Q}{{\mathbb Q}}
\newcommand{\R}{{\mathbb R}}
\newcommand{\C}{{\mathbb C}}
\newcommand{\ep}{{\varepsilon}}
\newcommand{\SR}{{\mathcal R}}

\renewcommand{\theenumi}{(\alph{enumi})}

\begin{document}
\subsection*{Problem 1}
{\it Prove the following theorem (terminology is given below):
\begin{center}
	Suppose $X$ is compact and $f \colon X \to \R$ is lower semicontinuous.
	Then $f$ is bounded below on $X$,
	and there exists a point $z \in X$ satisfying $f(z) \leq f(x)$ for all $x \in X$.
\end{center}
Recall that in a HTS $(X, \mathcal{T})$, a function $f \colon X \to \R$
is called \emph{lower semicontinuous} if the following set
is closed for every $p \in \R$:
\[
	f^{-1}((-\infty,p]) = \{x \in X \colon f(x) \leq p \}.
\]
(One approach uses the family of closed sets
$f^{-1}((-\infty,p])$ satisfying $p > \inf f(x)$.)}

\begin{proof}[Solution]\let\qed\relax
	Consider the family of closed sets of $f^{-1}((-\infty,p])$
	satisfying $p > \inf{f(x)}$, call it $\mathcal{F}$.
	First, remark that each element in $\mathcal{F}$ is nonempty,
	otherwise $f^{-1}((-\infty,p])$ is empty,
	thus there is no $x_0 \in X$ where $f(x_0) \in (-\infty,p]$
	and so $p \leq \inf{f(x)}$, which we assumed not true.
	Secondly, by the assumption that $f$ is lower semicontinuous,
	each element in $\mathcal{F}$ is also closed.
	Finally, note that $\mathcal{F}$ has the finite intersection property:
	let $N \in \N$ and $F_1, \dots, F_N$ are sets in $\mathcal{F}$,
	which we can write explicitly as $F_i = f^{-1}((-\infty, p_i])$
	where $p_i > \inf{f(x)}$;
	denote $p_0 = \min_i\{p_i\}$.
	Then $F_0 = f^{-1}((-\infty, p_0]) \subseteq F_i$ for all $1 \leq i \leq N$,
	and since we're just minimizing over a finite number of sets,
	$F_0 \in \{F_1,\dots,F_n\} \subseteq \mathcal{F}$, thus
	\[
		\bigcap_{i=1}^N F_i = f^{-1}((-\infty,p_o]) = F_0 \neq \emptyset
	\]
	so we have the finite intersection property.

	Now, since we're in a a HTS and $X$ is compact,
	any collection of elements of $\mathcal{F}$ has nonempty intersection,
	by the theorem proven in class
	(every element is a subset of $X$ and are closed,
	and any finite collection has the finite intersection property).
	Notably, $\bigcap \mathcal{F} \neq \emptyset$.
	This means that there exists some $z \in X$ where $z \in \bigcap\mathcal{F}$.
	Then, for all $p > \inf{f(x)}$, we have $z \in f^{-1}((-\infty,p])$.
	If $x \in X$, then $z \in f^{-1}((-\infty,f(x)])$, thus $f(z) \leq f(x)$.
	Therefore, $f$ is bounded below on $X$, specifically
	by $f(z)$ where $z \in X$,
	since $f(z) \leq f(x)$ for all $x \in X$.
\end{proof}
\clearpage
~\clearpage

\subsection*{Problem 2}
{\it Let $(X, d)$ be a metric space, with $K \subseteq X$ a compact set.
Prove that whenever $\mathcal{G}$ is an open cover for $K$,
there exists $r < 0$ with this property:
for every pair of points $x,y \in K$ obeying $d(x,y) < r$,
some open set $G \in \mathcal{G}$ contains both $x$ and $y$.}

\begin{proof}[Solution]\let\qed\relax
	For the sake of contradiction,
	assume that for all $r>0$, there are some $x_r,y_r \in K$ such that $d(x_r,y_r) <r$
	but for any $G \in \mathcal{G}$, $x_r,y_r$ are not both in $G$.
	Note that 
	Since $\mathcal{G}$ covers $K$,
	we have that both $x_r,y_r$ are in some $G \in \mathcal{G}$,
	so have $x \in G_x$ and $y \in G_y$ but $G_x \neq G_y$.

	Hmm, have a two sequences go towards each other.

	TFAE: $K$ is compact and every sequence $(x_n)$ in $K$ has
	a convergent subsequence whose limit lies in $K$.

	TFAE: $A$ is an open set and for every $x \in A$ and every sequence
	$(x_n)$ obeying $x_n \to x$, one has $x_n \in A$ for all $n$ sufficiently large.

	Evan has a solution ff
\end{proof}
\clearpage
~\clearpage

\subsection*{Problem 3}
{\it Define the set-valued ``projection" mapping
$p_1 \colon \mathcal{P}(\R^2) \to \mathcal{P}(\R)$ by
\[
	p_1(S) = \{x_1 \in \R \colon (x_1,x_2) \in S \text{ for some }x_2\},
	\qquad S \subseteq \R^2
\]
\begin{enumerate}
	\item If $S$ is bounded, must $p_1(S)$ be bounded? (Why or why not?)
	\item If $S$ is closed, must $p_1(S)$ be closed? (Why or why not?)
	\item If $S$ is compact, must $p_1(S)$ be compact? (Why or why not?)
\end{enumerate}}

\begin{enumerate}
	\item \begin{proof}[Solution]\let\qed\relax
		It must. If $S$ is bounded, then by definition,
		there exists $x \in S$ and $R>0$ such that
		$S \subseteq \mathbb{B}[x;R)$.
		Using the the standard metric on $\R^2$
		(namely $d(x,y) = \sqrt{(y_1 - x_1)^2 + (y_2 - x_2)^2}$),
		this means for any $y \in S$, we have
		$d(x,y) < r$, or $\sqrt{(y_1 - x_1)^2 + (y_2 - x_2)^2} < R$.
		Consider $x_1 = p_1(x)$.
		Then for any $y_1 \in p_1(S)$
		(using the standard metric on $\R$, $d(x,y) = |y-x|$), we have
		\[
			d(x_1,y_1) = |y_1 - x_1| = \sqrt{(y_1-x_1)^2}
			\leq \sqrt{(y_1-x_1)^2 + (y' - x_2)^2} < R
		\]
		where $y' \in p^{-1}(y_1)$,
		and so the last inequality follows from the boundedness of $S$.
		Thus, $p_1(S) \subseteq \mathbb{B}[x_1;R)$,
		so $p_1(S)$ is bounded.
	\end{proof}
	\item \begin{proof}[Solution]\let\qed\relax
		This is not true. We provide the counter-example
		$S = \{(2^{-n}, 2^n) \in \R^2 \colon n \in \N\}$.

		We first prove that $S$ is closed.
		Note that $S' = \emptyset$.
		To see this, for the sake of contradiction, let $s \in S'$.
		Then for some sequence $s_n$ of distinct elements of $S$,
		we have $\lim_{n\to\infty} s_n = s$ (by the proposition proven in class).
		Unraveling the definition of the limit, this means that for
		any $\ep > 0$, there exists some $N \in \N$ where $\forall n \geq N$,
		we have $d(s,s_n) < \ep$.
		For the sake of contradiction, assume that this is true;
		then let $\ep = \frac12$, which gives us some $N$
		where $d(s,s_n) < \frac12$ when $n \geq N$.
		But note that for any $s_n, s_{n+1} \in S$, since $s_n \neq s_{n+1}$,
		we have that $d(s_n, s_{n+1}) > 2$
		(by construction, since $2 \leq 2^{n+1} - 2^n =
		y_{s_{n+1}} - y_{s_{n+1}} = \sqrt{(y_{s_{n+1}} - y_{s_{n+1}})^2}
		\leq \sqrt{(y_{s_{n+1}} - y_{s_{n+1}})^2 + (x_{s_{n+1}} - x_{s_{n+1}})^2} = d(s_n,s_n+1)$).
		Thus $2 \leq d(s_{n+1},s_n) \leq d(s,s_n) + d(s,s_{n+1}) \leq \frac12 + d(s,s_{n+1})
		\implies \frac32  < d(s,s_{n+1})$.
		But this contradicts our assumption, since $n+1 > n \geq N$,
		but $d(s,s_{n+1} > \frac32 > \frac12 = \ep$.
		Thus, there are no limit points of $S$, so $S' = \emptyset$.
		
		Now recall the theorem proven in class, $\overline{S} = S \cup S'$.
		Since $S' = \emptyset$, this leaves us $\overline{S} = S$.
		But recall that this is true only if $S$ is closed.

		We now prove that $p_1(S)$ is not closed.
		Note that $p_1(S) = \{2^{-n} \colon n \in \N\}$.
		See that $0 \in p_1(S)'$ but $0 \not\in p_1(S)$.
		The second of these is obvious, $0 < 2^{-n}$ for all $n \in \N$.
		To see that $0$ is a limit point,
		we have $\lim_{n\to\infty} 2^{-n} = 0$ (obviously, we are in $\R$),
		and $2^{-n} \in p_1(S)$ are distinct points,
		thus $0 \in p_1(S)'$ (by our proposition in metric spaces).
		Thus, $p_1(S) \neq p_1(S) \cup p_1(S)' = \overline{p_1(S)}$.
		But this is true only if $p_1(S)$ is not closed.
		Hence, $S$ is closed but $p_1(S)$ is not closed.
	\end{proof}
	\item \begin{proof}[Solution]\let\qed\relax
		Now, assume that $S$ is compact.
		Consider an open cover $\mathcal{G}$ of $p_1(S)$.
		If $G \in \mathcal{G}$, extract the ``open column" corresponding
		to $G$, namely $G^2 = \{(x,y) \in \R^2 \colon x \in G, y \in \R\}$.
		Notice two things:
		each $G^2$ is open, and
		$\mathcal{G}^2$, the collection of all $G^2$ such that $G \in \mathcal{G}$,
		covers $S$.

		To see the first, let $g \in G^2$ and denote $g = (x_0, y_0)$.
		Since $x_0 \in G$, we have that there exists $r$
		such that $\mathbb{B}_1[x_0,r) = \{x \in \R \colon |x_0-x|<r\} \subseteq G$
		by the fact that $G$ is open.
		We claim that $\mathbb{B}_2[g,r) \subset G^2$ as well;
		for the sake of contradiction, assume there is
		$(x_1,y_1) \in \mathbb{B}_2[g,r)$ but $(x_1,y_1) \not\in G^2$.
		Then $x_1 \not\in G$ so $x_1 \not\in \mathbb{B}_1[x_0,r)$,
		which give us $|x_0 - x_1| \geq r$.
		But then $\sqrt{(x_1-x_0)^2 + (y_1 - y_0)^2} \geq \sqrt{r^2 (y_1 - y_0)^2}
		\geq \sqrt{r^2} = r$,
		which contradicts the fact that $(x_1,y_1) \in \mathbb{B}_2[g,r)$.
		Thus, $\mathbb{B}_2[g,r) \subset G^2$,
		and since $g \in G$ was arbitrary, this proves that $G^2$ is open.
		
		To see the second claim, that $\mathcal{G}^2$ covers $S$,
		let $s = (x_0,y_0) \in S$.
		Then since $\mathcal{G}$ covers $p_1(S)$,
		we have $p_1(s) = x_0 \in G$ for some $G \in \mathcal{G}$.
		Then $(x_0,y_0) \in G^2$ by definition,
		and $G^2 \in \mathcal{G}^2 \subseteq \bigcup_{G^2 \in \mathcal{G}^2} G^2$.
		Thus $s \in \bigcup_{G^2 \in \mathcal{G}^2} G^2$,
		so $\mathcal{G}^2$ covers $S$ since $s \in S$ was arbitrary.

		Hence, $\mathcal{G}^2$ is an open cover of $S$.
		By the compactness of $S$, there exists a finite subcover of $\mathcal{G}^2$,
		which we can denote as $G_1^2, G_2^2, \dots, G_N^2$.
		So $S \subseteq \bigcup_i G_i^2$.
		But recall that each $G_i^2 \in \mathcal{G}^2$ had some corresponding
		$G \in \mathcal{G}$ by definition
		(recall that the $x$-values of each $G^2$ were determined by some $G$),
		so we have a finite collection of open sets $G_1,G_2,\dots,G_N \in \mathcal{G}$.
		We prove that this is a finite subcover of $p_1(S)$.
		Let $x_0 \in p_1(s)$.
		Then $\exists y_0$ such that $(x_0,y_0) \in S$.
		Then $(x_0,y_0) \in G^2_j$ for some $1 \leq j \leq N$.
		But then by definition of $G^2_j$, $x_0 \in G_j$.
		Therefore, since $x_0 \in p_1(S)$ was arbitrary,
		$p_1(S) \subseteq \bigcup_i G_i$,
		hence $G_1,\dots,G_N \in \mathcal{G}$ is a finite subcover,
		thus $p_1(S)$ is compact.
	\end{proof}
\end{enumerate}
\clearpage

\subsection*{Problem 4}
{\it Recall the set $\ell^2$ from HW07 Q3,
and the standard ``unit vectors" $\hat{\mathbf{e}}_p = (0,0,\dots,0,1,0,\dots)$,
where the only nonzero entry in $\hat{\mathbf{e}}_p$ occurs in component $p$.
For any $x$ in $\ell^2$ and subset $V \subseteq \ell^2$, write
\[
	\Omega(x;V) = \{y \in \ell^2 \colon -1 < \langle v,y-x\rangle<1, \forall v \in V\}.
\]
Then define a collection $\mathcal{T}$ of subsets of $\ell^2$ by saying
$G \in \mathcal{T}$ if and only if every point $x \in G$ has the property
that $x \in \Omega(x;V) \subseteq G$ for some \emph{finite set} $V \subseteq \ell^2$.
\begin{enumerate}
	\item Prove that $\Omega(x;V) \in \mathcal{T}$ for every finite set
		$V \subseteq \ell^2$ and point $x \in \ell^2$.
	\item Prove that $(\ell^2, \mathcal{T})$ is a Hausdorff Topological Space.
	\item Let $S = \{\hat{\mathbf{e}}_p \colon p \in \N\}$.
		Prove that $0 \in S'$.
		(Here $0$ denotes $(0,0,\dots)$, the ``origin in $\ell^2$.)
		\emph{Note}: This fact proves that $\mathcal{T}$ is
		different from the metric topology on $\ell^2$.
	\item Prove that every $G$ in $\mathcal{T}$ has the property:
		for every $x$ in $G$, there exists $r>0$ such that
		\[
			G \supseteq \mathbb{B}[x;r) = \{y \in \ell^2 \colon
			\lVert y - x \rVert < r\}.
		\]
		This fact proves that every set considered ``open" in $\mathcal{T}$
		is also open in the metric topology on $\ell^2$.
		This explains why $\mathcal{T}$ gets called ``the weak topology"
		and the metric topology is also called ``the strong topology."
	\item Prove that the following set is closed in the weak topology of $\ell^2$:
		$\mathbb{B}[0;1] = \{y \in \ell^2 \colon \lVert y \rVert \leq 1\}$.
\end{enumerate}}

\begin{enumerate}
	\item \begin{proof}[Solution]\let\qed\relax
		Let $x' \in \Omega(x;V)$.
		Want to show there exists a finite set $V' \subseteq \ell^2$
		such that $\Omega(x';V') \subseteq \Omega(x,V)$.
		Then $-1 < \langle v, x' - x \rangle < 1$ for all $v \in V$.
		This is equivalent to
		\[
			-1 < \sum_{n = 1}^\infty v_n(x'_n - x_n) < 1
		\]
		for all $v \in V$.
		Let $v'$ be the sequence $v_n(y_n - x_n) = v'_n(y_n - x'_n)$,
		$v_ny_n - v_nx_n = v_n'y_n - v_n'x'_n$
		if $v'_n = v_nx_n/x'_n$ we get $v_ny_n = v_ny_nx_n/x'_n$.

		Let $v'$ be the sequence defined by $v'_n = v_nx_n/x'_n$.

		Let $y \in \Omega(x';V')$.
		Then
		\[
			-1 < \sum_{n=1}^\infty v'_n(y_n - x'_n) < 1
		\]
		for all $v' \in V'$.
		Then
		\[
			-1 < \sum_{n=1}^\infty v_n(y_nx_n/x'_n - x_n) < 1
		\]
		ff

		Actually, let $v'_n = v_n(x'_n - x_n)$.
		It is clear that $v' \in \ell^2$,
		since ff

		Or can use Cauchy-Schwartz: $|\langle v, y-x \rangle|
		\leq \lVert v \rVert \lVert y-x \rVert$
		ff
	\end{proof}
	\item \begin{proof}[Solution]\let\qed\relax
		We have that $\emptyset \in \mathcal{T}$,
		since there does not exist $x \in \emptyset$ so it
		satisfies our condition to be in $\mathcal{T}$ vacuously.
		We also have $\ell^2 \in \mathcal{T}$,
		since $\Omega(x;V)$ is composed of elements of $\ell^2$,
		and so for any $x \in \ell^2$, $\Omega(x;V) \subseteq \ell^2$.
		
		Now consider $\mathcal{G} \subseteq \mathcal{T}$.
		Consider an arbitrary element $x \in \bigcup \mathcal{G}$.
		Then for some $G \in \mathcal{G}$, we have $x \in G$.
		Then $\Omega(x;V) \subseteq G \subseteq \bigcup\mathcal{G}$
		for some finite set $V \subseteq \ell^2$ since $G \in \mathcal{T}$,
		so $\bigcup \mathcal{G} \in \mathcal{T}$ as well.

		Now consider $U_1,\dots,U_N \in \mathcal{T}$ where $N \in \N$.
		Consider an arbitrary element $x \in \bigcap_i^N U_i$.
		Then for all $1 \leq i \leq N$, $x \in U_i$.
		Then by definition of each $U_i$ being in $\mathcal{T}$,
		we have that there exists a finite set $V_i \subseteq \ell^2$
		such that $\Omega(x;V_i) \subseteq U_i$.
		Note that by definition, if $y \in \Omega(x; V_i \cup V_j)$,
		then $y \in \Omega(x; V_i)$ and $y \in \Omega(x; V_j)$.
		So let $V = \cup_i V_i$. This is still a finite set,
		since we are just unioning a finite number of finite sets.
		Then if $y \in \Omega(x;V)$, we have that $y \in \Omega(x;V_i)$
		for all $1 \leq i \leq N$,
		thus $y \in U_i$.
		Since $y \in \Omega(x;V)$ was arbitrary, we have that
		$\Omega(x;V) \subseteq U_i$ for all $1 \leq i \leq N$,
		thus $\Omega(x;V) \subseteq \bigcap_i^N U_i$,
		hence $\bigcap_i^N U_i \in \mathcal{T}$ as well.

		Finally, let $x, y \in \ell^2$ such that $x \neq y$.
		We have that $y_N \neq x_N$ for some $N \in \N$.
		Let $v \in \ell^2$ be defined by
		$v_N = 2(y_N - x_N)^{-1}$ and $v_n = 0$ for all other $n$.
		Define $V = \{v\}$.
		We claim that $x\in\Omega(x;V)$ and $y\in\Omega(y;V)$ are disjoint
		(and they are open sets by part (a) of this problem).
		Let $x' \in \Omega(x;V)$.
		It is sufficient to show $x' \not\in \Omega(y;V)$.
		We have $-1 < \langle v, x' - x \rangle =
		\sum_n v_n(x'_n - x_n) = 2(x'_N - x_N)/(y_N - x_N) < 1$,
		or $0 \leq |2(x'_N - x_N)/(y_N - x_N)| < 1$,
		so $0 \leq |x'_N - x_N| < \frac12|y_N - x_N|$.
		Note $|y_N - x_N| \leq |x'_N - y_N| + |x'_N - x_N| < |x'_N - y_N| + \frac12|y_N - x_N|$
		thus $\frac12|y_N - x_N| < |x'_N - y_N|$.
		But then $1 < |2(x'_N - y_N)/(y_N - x_N)|$
		and $\sum_n v_n(x'_n - y_n) = 2(x'_N - y_N)/(y_N-x_N)$.
		Thus $\sum_n v_n(x'_n - y_n) < -1$ or $\sum_n v_n(x'_n - y_n) > 1$,
		in either case, $x' \not\in \Omega(y;V)$.

		This satisfies all the conditions for a HTS,
		thus $(\ell^2,\mathcal{T})$ is a HTS.
	\end{proof}
	\item \begin{proof}[Solution]\let\qed\relax
		Let $U \in \mathcal{N}(0)$ be an arbitrary open set,
		ie. $U \in \mathcal{T}$ such that $0 \in U$.
		We want to show that $(U\setminus\{0\})\cap S \neq \emptyset$;
		since $0 \not\in S$ anyway, we just need to show
		$U \cap S \neq \emptyset$.

		Since $0 \in U$, there exists a finite set $V\subseteq \ell^2$
		such that $\Omega(0;V) \subseteq U$.
		If $V = \emptyset$, then $\Omega(0;V) = \ell^3$
		since $-1 < \langle v, y-x \rangle < 1$ is now
		vacuously true for all $y \in \ell^2$;
		then $\Omega(0;V) \cap S$ since $\hat{\mathbf{e}}_1 \in \ell^2 \cap S$,
		and since $\Omega(0;V) \subseteq U$, $U \cap S \neq \emptyset$.
		So now assume $V$ is not empty.
		Denote the elements of $V$ as $v^{i}$ where $1 \leq i \leq k$.
		Then since $v^i \in \ell^2$,
		we must have that $\lim_n (v^i_n)^2 = 0$ (crude divergence test).
		Then there exists some $N_i$ where $(v^i_{N_i})^2 < 1$ by the definition of convergence.
		Let $N = \min_i\{N_i\}$.
		Then $-1 < v^i_N < 1$ as well.
		See
		\[
			\langle v^i, \hat{\mathbf{e}}_N \rangle =
			\sum_{n=1}^\infty v^i_n (\hat{\mathbf{e}}_N)_n
			= v^i_N
		\]
		Thus $\hat{\mathbf{e}}_N \in \Omega(0,V)$ since
		$-1 < \langle v, \hat{\mathbf{e}}_N - 0 \rangle = v_N < 1$
		for all $v \in V$.
		Thus, $\hat{\mathbf{e}}_N \in \Omega(0,V) \subseteq U$.
		Since $\hat{\mathbf{e}}_N \in S$,
		thus shows that $S \cap U \neq \emptyset$,
		so we are done since $U$ was arbitrary
		(this works for any open $U \in \mathcal{N}(0)$).
	\end{proof}
	\item \begin{proof}[Solution]\let\qed\relax
		If $x \in G$, there exists a finite set $V \subseteq \ell^2$
		where $\Omega(x;V) \subseteq G$.
		If there exists some $v_0 \in V$ such that $\lVert v_0 \rVert =0$,
		then $v_0$ must be the zero sequence
		(otherwise $\lVert v_0 \rVert = \sum_n v_0^2 > 0$),
		but then regardless of $y \in \ell^2$,
		$\langle v_0, y-x\rangle = \sum_n 0(y_n-x_n) = 1$,
		so $G = \ell^2$, and so $\mathbb{B}[x;1) \subseteq \ell^2 = G$
		and we are done (note we just made $r=1>0$ here).

		Now consider the remaining case when $0 < \lVert v_i \rVert$
		where $v_i \in \{v_1, \dots, v_N\} = V$.
		Let $r = \min_i\{\lVert v_i \rVert^{-1}\}$.
		Note that since is just the minimum of a finite number of values,
		all greater than zero, we have $r > 0$ as well.
		Let $y \in \mathbb{B}[x;r)$.
		Then $\lVert y - x \rVert < r \leq \lVert v \rVert^{-1}$ for all $v \in V$.
		Thus $\lVert v \rVert\lVert y - x \rVert < 1$.
		Now using Cauchy-Schwartz (which we proved for this norm in homework 7),
		we have $|\langle v, y-x \rangle \leq \lVert v \rVert\lVert y - x \rVert < 1$,
		but this is equivalent to $-1 < \langle v, y-x \rangle < 1$,
		so $y \in \Omega(x;V)$ (since this was true for any $v \in V$),
		thus $y \in G$.
		Since $y \in \mathbb{B}[x;r)$ was arbitrary,
		this means $\mathbb{B}[x;r) \subseteq G$,
		as desired.
	\end{proof}
	\item \begin{proof}[Solution]\let\qed\relax
		It is sufficient to show that $\mathbb{B}[0;1]^c
		= \{y \in \ell^2 \colon \lVert y \rVert > 1\}$ is open.
		Let $x \in \mathbb{B}[0;1]^c$.
		Consider $y \in \Omega(x;V)$ (ff don't forget to choose $V$).
		Then $|\langle v, y-x \rangle| < 1$ for all $v \in V$.
		ff

		we want to show that $\lVert y \rVert > 1$.
		It would be sufficient to show that $|\langle v, y-x \rangle|/\lVert v \rVert > 1$.
		So we want
		\[
			\left\lvert\sum_{n=1}^\infty v_n(y_n-x_n) \right\rvert
			 > \sqrt{\sum_{n=1}^\infty v^2_n} > 0
		\]
		\[
			\left\lvert\sum_{n=1}^\infty v_n(y_n-x_n) \right\rvert^2
			 > \sum_{n=1}^\infty v^2_n > 0
		\]


		Problem 8 of last homework
		(it was a bonus problem, but solutions were still posted):
		if $A,B$ are closed and one of them are bounded, then $A + B$ is closed.
		$\mathbb{B}[0,1] = \{y \in \ell^2 \colon \lVert y \rVert \leq 1\}$.
		oh but this was for $A,B \subseteq \R$.

		Boundary point??
		We claim $\partial \mathbb{B}[0;1]$ are the $y \in \ell^2$
		such that $\lVert y \rVert = 1$.
		This means $\sum_n y_n^2 = 1$.
		Let $x \in \Omega$... neighbourhoods?
		Show that $\Omega(y,V)$ has some sequences $z$ $\lVert z \rVert > 1$
		and $\lVert z \rVert \leq 1$ (but this is trivial from $y$).
		ff
	\end{proof}
\end{enumerate}
Recap of what still needs to be done:
	\begin{itemize}
		\item Evan Chen Q2
		\item 4(a), 4(e)
	\end{itemize}
\clearpage
\end{document}
