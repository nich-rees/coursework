\documentclass{article}
\usepackage{amsmath, amsfonts, amsthm, amssymb}
\usepackage{geometry}
\geometry{letterpaper, margin=2.0cm, includefoot, footskip=30pt}

\usepackage{fancyhdr}
\pagestyle{fancy}

\lhead{Math 320}
\chead{Homework 9}
\rhead{Nicholas Rees, 11848363}
\cfoot{Page \thepage}

\newtheorem*{problem}{Problem}

\newcommand{\N}{{\mathbb N}}
\newcommand{\Z}{{\mathbb Z}}
\newcommand{\Q}{{\mathbb Q}}
\newcommand{\R}{{\mathbb R}}
\newcommand{\C}{{\mathbb C}}
\newcommand{\ep}{{\varepsilon}}
\newcommand{\SR}{{\mathcal R}}

\renewcommand{\theenumi}{(\alph{enumi})}

\begin{document}
\subsection*{Problem 1}
{\it Solve in either order:
\begin{enumerate}
	\item Construct, with justification, a subset $A$ of $\R$ such that
	every point of $A$ is isolated and $A'\neq\emptyset$.
	\item Rudin Chapter 2, problem 5, page 43:
		Construct a bounded set of real numbers with exactly three limit points.
\end{enumerate}}

\begin{enumerate}
	\item \begin{proof}[Solution]\let\qed\relax
		We provide the subset $A = \{\frac{1}{n} \colon n \in \N\}$.
		First, if $a \in A$, denote this as $a = \frac{1}{n'}$.
		We show $a$ is an isolated point:
		let $m = \mathrm{min}\left\{\left|\frac{1}{n'} - \frac{1}{n'+1}\right|,
		\left|\frac{1}{n'} - \frac{1}{n'-1}\right|\right\}$.
		Note that since for any $\overline{n} \geq n'$,
		we have that $\left|\frac{1}{n'+1} - \frac{1}{n'}\right| \leq
		\left|\frac{1}{\overline{n}} - \frac{1}{n'}\right|$,
		thus any ball around $\frac{1}{n'}$ that contains $\frac{1}{\overline{n}}$
		must also contain $\frac{1}{n'+1}$.
		Now for any $\underline{n} \leq n'$,
		we have that $\left|\frac{1}{n'-1} - \frac{1}{n'}\right| \leq
		\left|\frac{1}{\underline{n}} - \frac{1}{n'}\right|$,
		thus any ball around $\frac{1}{n'}$ that contains $\frac{1}{\underline{n}}$
		must also contain $\frac{1}{n'-1}$.

		Consider the open set of $R$
		$U = \mathbb{B}[\frac{1}{n'}; \frac12m)$.
		Note that $\frac{1}{n'-1} \not\in U$
		and $\frac{1}{n'+1} \not\in U$.
		Thus by the contrapositive of the claims we just said,
		for any $n \in \N$ where $n \neq n'$,
		we have that $\frac{1}{n} \not\in U$.
		Thus, $\frac{1}{n'}$ is isolated.
		Since this is true for arbitrary $n' \in \N$,
		every point in $A$ is isolated.

		Now, note that $0 \in A'$ so $A' \neq \emptyset$.
		If $U$ is be an arbitrary open set in the neighbourhood of $0$.
		Note that we will always have an element of $A$ in $U$.
		Assume otherwise, that there exists an open set $U$ such that
		$U \cap A = \emptyset$.
		Conside a ball in $U$, specifically $\mathbb{B}[0,r) \subseteq U$.
		Note by the Archimedean property of the reals,
		there exists $n \in \N$ such that $n\cdot1 > r^{-1} > 0$,
		thus $0 < \frac{1}{n} < r$.
		But then $\frac{1}{n} \in \mathbb{B}[0,r)$,
		thus a contradiction.
		Thus, since $U \in \mathcal{N}(0)$ was arbitrary,
		we have that every open set in the neighbourhood of $0$
		has a non empty intersection with $A$,
		thus $0$ is a limit piont of $A$.
		Thus, $A' \neq \emptyset$.
	\end{proof}
	\item \begin{proof}[Solution]\let\qed\relax
		We give the set $A = \{\frac{1}{n} \colon n \in \N\}
		\cup \{\frac{1}{n} + 10 \colon n \in \N\}
		\cup \{\frac{1}{n} + 20 \colon n \in \N\}$.
		From part (a) of this problem,
		we note that no element in $A$ is a limit point of $A$,
		since they are all isolated (and thus cannot be limit points);
		the argument is the same, since the additional term
		just makes it so that we have three subsets of $A$
		that do not intersect.
		Furthermore, we can make identical arguments as from part (a)
		to show that $10$ and $20$ are in $A'$,
		as well as $0$.
		Thus, $A$ has exactly three limit points.
	\end{proof}
\end{enumerate}
\clearpage
~\clearpage

\subsection*{Problem 2}
{\it \begin{enumerate}
	\item Give an example of two sets $A$ and $B$ in some HTS satisfying
		\[
			\mathrm{int}(A\cup B) \neq \mathrm{int}(A)\cup\mathrm{int}(B)
		\]
	\item Give an example of two sets $A$ and $B$ in some HTS satisfying
		\[
			\overline{A\cap B} \neq \overline{A} \cap \overline{B}
		\]
	\item Working $\R^k$ with the usual topology, express the open ball
		$\mathbb{B}[0;1)$ as a union of closed sets.
		Can $\mathbb{B}[0;1)$ be expressed as an intersection of closed sets?
\end{enumerate}}

\begin{enumerate}
	\item \begin{proof}[Solution]\let\qed\relax
		Let our HTS be $\R$,
		and let $A = [0,1]$ and $B = [1,2]$.
		We have $\mathrm{int}(A\cup B) = \mathrm{int}([0,2]) = (0,2)$
		and $\mathrm{int}(A) \cup \mathrm{int}(B) = (0,1) \cup (1,2) = (0,2)\setminus \{1\}$.
		Hence we have shown $\mathrm{int}(A)\cup\mathrm{int}(B) = (0,2)\setminus\{1\}
		\neq (0,2) = \mathrm{int}(A\cup B)$,
		so we are done.
	\end{proof}
	\item \begin{proof}[Solution]\let\qed\relax
		Let our HTS be $\R$,
		and let $A = (0,1)$ and $B = (1,2)$.
		We have $\overline{A \cap B} = \overline{\emptyset} = (\R^o)^c$,
		but since $A$ is open if and only if $A^0$ is open (from notes)
		and $\R$ must be open, we have $\overline{A\cap B} = \R^c = \emptyset$.
		Now, see that $\overline{A} \cap \overline{B} = [0,1] \cap [1,2] = \{1\}$,
		where we have used the fact that in $\R$, $\overline{(a,b)} = (((a,b)^c)^o)^c
		= (((-\infty,a]\cup [b,\infty))^o)^c$ but taking the largest
		open subset we get $((-\infty,a)\cup (b,\infty))^c = [a,b]$.
		Hence, $\overline{A \cap B} = \emptyset \neq \{1\} = \overline{A} \cap \overline{B}$.
	\end{proof}
	\item \begin{proof}[Solution]\let\qed\relax
		We have
		\[
			\mathbb{B}[0;1) = \bigcup_{n\in\N} \mathbb{B}\left[0;1-\frac{1}{n}\right]
		\]
		To verify this equality, if $x \in \mathbb{B}[0;1)$ and $x\neq0$
		(the $x = 0$ case is trivial, $x$ is definitely in the RHS)
		note that $x_k < 1$, so
		then by Archimedean, there exists some $n \in \N$ such that
		$0 < \frac{1-x_k} < n\cdot 1 \implies 0 < x_k < 1 - \frac{1}{n}$,
		thus $x \in \mathbb[0;1-\frac{1}{n}]$, so $x$ is in the RHS.
		Now if $x \in \bigcup_{n\in\N}\mathbb{B}[0;1-\frac{1}{n}]$,
		there exists some $n$ such that $0 \leq x_k \leq 1-\frac{1}{n}$.
		But then $0 \leq x_k < 1$ for all $k$,
		so $x \in \mathbb{B}[0;1)$, as desired.

		We cannot write $\mathbb{B}[0;1)$ as an intersection of closed sets,
		since we know that for any HTS< the arbitrary intersection of closed sets is also closed,
		and $\mathbb{B}[0;1)$ is open by definition.
	\end{proof}
\end{enumerate}
\clearpage
~\clearpage

\subsection*{Problem 3}
{\it Define a family $\mathcal{T}$ of subsets of $\R$ as follows:
	\begin{center}
		A set $G \subseteq \R$ belongs to $\mathcal{T}$ if and only if
		for every $x$ in $G$, there exists $r>0$ such that $[x,x+r)\subseteq G$.
	\end{center}
	\begin{enumerate}
		\item Prove that $(\R,\mathcal{T})$ is a HTS. (It is called the \emph{Sorgenfrey line}.)
	\end{enumerate}
	All our terminology -- open set, closed set, boundary point, limit point, convergence
	-- depends on what topology we use. Use the Sorgenfrey topology in parts (b)-(d):
	\begin{enumerate}
		\item[(b)] Show that the interval $[0,1)$ is open.
		\item[(c)] Find all boundary points of the interval $(0,1)$.
		\item[(d)] Let $s_n = -1/n$ and $t_n=1/n$.
			Prove that one of these sequences converges to $0$, and the other does not.
			Use the definition given in class, i.e. $x_n \to \hat{x}$
			means that for every open set $U$ containing $\hat{x}$,
			there exists $N \in \N$ such that for all $n > N$, $x_n \in U$.
\end{enumerate}}

\begin{enumerate}
	\item \begin{proof}[Solution]\let\qed\relax
		First, note that $\R$ and $\emptyset$ are both open.
		If $x \in \R$, then $[x,x+1) \in \R$ as well, so $\R \in \mathcal{T}$.
		Additionally, $\emptyset$ vacuously satisfies our criteria,
		and so $\emptyset \in \mathcal{T}$ as well.

		Now consider a collection $\mathcal{G}$ of open sets in $\mathcal{T}$.
		Let $V$ denote the set $\bigcup \mathcal{G}$.
		If $x \in V$, then $x \in G$ for some $G \in \mathcal{G}$,
		and by assumption, since $G \in \mathcal{T}$,
		there exists some $r>0$ such that $[x,x+r) \in G$,
		which means $[x,x+r) \in V$ as well (by definition of union).
		Thus, $V \in \mathcal{T}$ as well.

		Now let $U_1, U_2, \cdots , U_N \in \mathcal{T}$ (for some $N \in \N$),
		and let $V$ denote the set $U_1 \cap U_2 \cap \cdots \cap U_N$.
		If $x \in V$, then we must have that $x \in U_i$ for all $i \in \{1,\dots,N\}$.
		And since $U_i \in \mathcal{T}$, there is some $r_i>0$ such that
		$[x, x+r_i) \in U_i$.
		Let $r = \min_i\{r_i\} > 0$.
		Then surely $[x, x+r) \subseteq [x, x+r_i)$ for all $i \in \{1,\dots,N\}$,
		thus $[x, x+r) \in U_i$ for all $i$,
		so $[x, x+r) \in V$.
		Thus, since $r>0$, $V \in \mathcal{T}$.

		Let $x,y \in \R$ such that $x \neq y$.
		WLOG assume $x < y$.
		Clearly $x \in [x,y)$ and $y \in [y,y+1)$.
		These are clearly distinct subsets of $\R$.
		We now show that both are open.
		If $z \in [x,y)$,
		let $y - z = \delta > 0$.
		We have that $[z, z + \delta) = [z,y) \in [x,z)$.
		Thus, $[x,y)$ is an open set.
		If $w \in [y,y+1)$,
		let $y +1 - w = \delta > 0$.
		We have that $[w, w + \delta) = [w,y+1) \in [y,y+1)$.
		Thus, $[y,y+1)$ is an open set.
		Thus, we have shown $(\R,\mathcal{T})$ is a HTS, as desired.
	\end{proof}
	\item \begin{proof}[Solution]\let\qed\relax
		Let $x \in [0,1)$.
		Then let $1 - x = \delta > 0$.
		We have that $[x, x + \delta) = [x,1) \in [0,1)$.
		Thus, $[0,1) \in \mathcal{T}$ and so is an open set.
	\end{proof}
	\item \begin{proof}[Solution]\let\qed\relax
		We claim the only boundary point of $(0,1)$ is $0$.
		If $U$ is an open set containing $0$,
		it is of the form $[a,b)\in\mathcal{T}$ such that $0 \in [a,b)$
		($b = a + r$, $r > 0 \implies b > a$).
		In order to have $0$ fall in it, we must have $a \leq 0$ and $b > 0$.
		Since $0\in U$ and $0 \not\in (0,1)$, then $U \cap (0,1)^c \neq \emptyset$.
		Now, either $b \geq 1$ or $ b < 1$.
		If it is the former, then we have that $(0,1) \subset [a,b)$
		and so $U \cap (0,1) \neq \emptyset$.
		If it is the latter, then we have $b/2 \in [a,b)$ and $b/2 \in (0,1)$,
		thus $U \cap (0,1) \neq \emptyset$ again.
		This is sufficient to show that $0$ is a boundary point.
		
		We now show that there are no more boundary points of the set.
		Let $x \in \R$ and $x \neq 0$.
		If $x < 0$, the open set $[x,0)$ contains $x$ but does not
		intersect $(0,1)$, so cannot be a boundary point for it.
		If $x \in (0,1)$, then $[x,1)$ contains $x$ but does not intersect $(0,1)^c$,
		so cannot be a boundary point for $(0,1)$.
		And if $x \geq 1$, then $[x,x+1)$ contains $x$ but does not
		intersect $(0,1)$ so cannot be a boundary point for it.
		Thus, since $x$ was arbitrary, $x \in \R^*$ cannot be a boundary point for $(0,1)$.
	\end{proof}
	\item \begin{proof}[Solution]\let\qed\relax
		We claim that $s_n$ does not converge to $0$.
		If $U = [0, 1)$,
		then we have that $U$ is open from part (b) of this problem
		and it contains $0$,
		however, for all $n \in \N$, we have that $-1/n < 0$ and so
		for no $n \in \N$ do we have $s_n \in U$.
		Thus, $s_n \not\in U$ for all $n$, thus $s_n$ does not converge to $0$.

		We claim that $t_n$ does converge to $0$.
		Consider an open set $U$ that contains $0$.
		That is, $U$ is of the form $[a,b)$ where $b > a$
		(since $b = a+r$ and $r>0$ thus $b>a$)
		such that $0 \in U$.
		In order for this to happen, it is necessary that $a \leq 0$ and $b > 0$.
		Thus, it is sufficient to show that $0 < t_n < b$ for large enough $n$
		to show $t_n \in U$.
		Clearly, $t_n > 0$ for all $n \in \N$.
		Now, by Archimedean, there exists $N \in \N$ such that $N\cdot 1 > b^{-1}>0$,
		thus $0 < \frac{1}{N} = t_N < b$.
		Furthermore, $\frac{1}{n} \leq \frac{1}{N}$ for $n \geq N$
		(this is obvious, but if really desired, this can be shown with induction),
		thus for $n \geq N$, we have $0 < t_n < b$, thus $t_n \in U$.
		Since $U$ was an arbitrary open set that contained $0$,
		this is sufficient to show convergence of $t_n \to 0$.
	\end{proof}
\end{enumerate}
\clearpage
~\clearpage

\subsection*{Problem 4}
{\it Let $A$ be a subset of a HTS $(X,\mathcal{T})$.
The \textbf{boundary of} $A$ is a set denoted $\partial A$:
we say $z \in \partial A$ if and only if every open $U$ containing $z$
satisfies both $U \cap A \neq \emptyset$ and $U \cap A^c \neq \emptyset$. Prove:}
\begin{enumerate}
	\item $\partial A = \overline{A} \cap \overline{A^c}$.
	\item $A$ is closed if and only if $\partial A \subseteq A$.
	\item $A$ is open if and only if $A \cap \partial A = \emptyset$.
\end{enumerate}

\begin{enumerate}
	\item \begin{proof}[Solution]\let\qed\relax
		Let $x \in \overline{A} \cap \overline{A^c}$.
		Let $G \in \mathcal{N}(x)$.
		We first prove that $A \cap G \neq \emptyset$.
		If $x \in A$, we are done, since $\{x\} \in A \cap G$.
		Now, assume that $x \not\in A$.
		For the sake of contradiction, assume that $A \cap G = \emptyset$.
		This means that $G \subseteq A^c$
		Thus, $A$ is closed (by the lemma from class).
		Then $A = \overline{A}$.
		But recall that $x \in \overline{A} = A$, thus a contradiction.
		We now prove that $A^c \cap G \neq \emptyset$.
		If $x \in A^c$, we are done, since $\{x\} \in A^c \cap G$.
		Now, assume that $x \not\in A^c$.
		For the sake of contradiction, assume that $A^c \cap G = \emptyset$.
		This means that $G \subseteq A$
		Thus, $A^c$ is closed (by the lemma from class).
		Then $A^c = \overline{A^c}$.
		But recall that $x \in \overline{A^c} = A^c$, thus a contradiction.
		Thus, since $G$ was arbitrary, we have both
		$A \cap G \neq \emptyset$ and $A^c \cap G \neq 0$
		for all $G \in \mathcal{N}(x)$,
		which by definition, means that $x \in \partial A$.
		Thus, $\overline{A}\cap\overline{A^c} \subseteq \partial A$.

		Now let $x \in \partial A$.
		We now prove that $x \in \overline{A}$.
		If $x \in A$, we are done, since $A \subset \overline{A}$.
		If $x \not\in A$, since $\overline{A}$ is closed,
		we have that there exists some neighbourhood $U \in \mathcal{N}(x)$
		such that $U \subseteq A^c$,
		and since $A^c \cap A = \emptyset$,
		this implies that $U \cap A = \emptyset$.
		But recall that since $x \in \partial A$, we have that $A \cap U \neq \emptyset$,
		thus, a contradiction.
		Hence, $x \in \overline{A}$.
		Now we prove that $x \in \overline{A^c}$.
		If $x \in A^c$, we are done, since $A^c \subset \overline{A^c}$.
		If $x \not\in A^c$, since $\overline{A^c}$ is closed,
		we have that there exists some neighbourhood $U \in \mathcal{N}(x)$
		such that $U \subseteq (A^c)^c = A$,
		and since $A^c \cap A = \emptyset$,
		this implies that $U \cap A^c = \emptyset$.
		But recall that since $x \in \partial A$, we have that $A^c \cap U \neq \emptyset$,
		thus, a contradiction.
		Hence, $x \in \overline{A^c}$.
		Thus, $x \in \overline{A} \cap \overline{A^c}$,
		so $\overline{A} \cap \overline{A^c} \subseteq \partial A$.
		
		We have shown set inclusion in both directions,
		so $\partial A = \overline{A} \cap \overline{A^c}$.
	\end{proof}
	\item \begin{proof}[Solution]\let\qed\relax
		Assume that $A$ is closed.
		Let $x \in \partial A$.
		From part (a) of this problem,
		we have $\partial A = \overline{A} \cap \overline{A^c}$,
		thus $z \in \overline{A} \cap\overline{A^c} \implies z \in \overline{A}$.
		But since $A$ is closed, $A = \overline{A}$,
		thus $z \in A$.
		Since $z$ was arbitrary, this implies that $\partial A \subseteq A$.

		Now assume that $\partial A \subseteq A$.
		From part (a) of this problem, we have
		$\partial A = \overline{A} \cap \overline{A^c} =
		\overline{A}\cap(((A^c)^c)^o)^c = \overline{A} \cap (A^o)^c
		= \overline{A} \setminus A^o$,
		thus $\partial A \cup A^o = \overline{A}$.
		Thus, since $\partial{A} \subseteq A$, by assumption,
		and $A^o \subseteq A$,
		we have $\overline{A} \subseteq A$.
		But $A \subseteq \overline{A}$ by definition,
		so $A = \overline{A}$.
		But this is true only $A$ is closed.
	\end{proof}
	\item \begin{proof}[Solution]\let\qed\relax
		Assume that $A$ is open.
		Then $A^c$ is closed, and thus $A^c = \overline{A^c}$.
		Invoking part (a) of this problem, we have
		\begin{align*}
			A \cap \partial A
			&= A \cap (\overline{A} \cap \overline{A^c})\\
			&= (A \cap A^c) \cap \overline{A}\\
			&= \emptyset \cap \overline{A}\\
			&= \emptyset
		\end{align*}

		Now assume that $A \cap \partial A = \emptyset$.
		But then $A \cap \overline{A} \cap \overline{A^c} = \emptyset$.
		Since $A \subseteq \overline{A}$, we have $A \cap \overline{A} = A$,
		thus we have $A \cap \overline{A^c} = \emptyset$.
		Writing out the full definition for the closure of $A^c$, we have
		$\overline{A^c} = (((A^c)^c)^o)^c = (A^o)^c$,
		thus we have $A \cap (A^o)^c = \emptyset$.
		For the sake of contradiction, assume that $A$ is not open.
		Then, there exists $x \in A$ such that $x \not \in A^o$
		(since $A^o \subset A$ and $A^o \neq A$).
		Then, $x \in (A^o)^c$.
		But then $x \in A \cap (A^o)^c) \implies A \cap (A^o)^c \neq \emptyset$,
		thus a contradiction.
		Hence $A$ must be open.
	\end{proof}
\end{enumerate}
\clearpage
~\clearpage

\subsection*{Problem 5}
{\it Prove: For every set $A$ in a HTS $(X, \mathcal{T})$, $A'$ is closed.}

\begin{proof}[Solution]\let\qed\relax
	It is sufficient to show that $\partial A' \subseteq A'$,
	by problem 4(b).
	Let $x \in \partial A'$.
	By definition, this means that for all open sets $U \in \mathcal{N}(x)$,
	we have $U \cap A' \neq \emptyset$.
	Thus, let $y \in U \cap A'$.
	Since $U$ is an open set and $y \in U$,
	there exists $U' \in \mathcal{N}(y)$ such that $U' \subseteq U$.
	Now define $V = U'\setminus{x}$.
	Note that $y \in V$ still,
	and $V$ is open, since $V = U' \setminus{x} = U' \cap \{x\}^c$
	which is a finite intersection of open sets
	($\{x\}^c$ is open since $\{x\}$ is closed), so is also open,
	thus $V \in \mathcal{N}(y)$ still.
	Now, since $y \in A'$, then for all open $W \in \mathcal{N}(y)$,
	we have $W\setminus\{y\} \cap A\neq \emptyset$,
	so $V\setminus\{y\} \cap A \neq \emptyset$.
	Thus, there exists $z \in V \setminus\{y\} \cap A \implies
	z \in U$ since $V\setminus\{y\} \subseteq U$.
	Thus, we se that there exists $z \in U \cap A$
	where $z \neq x$, or $U \setminus \{x\} \cap A \neq \emptyset \implies x \in A'$.
	Hence, we have that $\partial A' \subseteq A'$,
	which shows that $A'$ is closed.
\end{proof}
\clearpage
~\clearpage

\subsection*{Problem 6}
{\it Recall the sequence space $\ell^2$ from HW07 Q3.
Given a specific $M = (M_1,M_2,\dots)$ in $\ell^2$, let
\[
	S = \{x \in \ell^2 \colon \forall n \in \N, |x_n| \leq M_n\}.
\]
Prove: every sequence $\left(x^{(n)}\right)$ in $S$ has a convergent subsequence,
whose limit lies in $S$.}

\begin{proof}[Solution]\let\qed\relax
	Write out the sequence of the first components of our sequence, namely
	\[
		\left(x_1^{(n)}\right) = x_1^{(1)}, x_1^{(2)}, x_1^{(3)}, \dots
	\]
	Each term of $\left(x_1^{(n)}\right)$ is bounded by $M_1$,
	thus, by Bolzano-Weierstrass
	(which we can use since this is just a real-valued sequence),
	there exists a subsequence $(n_{k_1})$ such that
	$\left(x_1^{(n_{k_1})}\right)$ converges.
	This is a convergent sequence,
	so denote $\lim_{k_1} x_1^{(n_{k_1})} = s_1$;
	since $|x_1^{(i)}| \leq M_1$ for all $i \in n_k$,
	the limit is at most $M_1$, thus $|s_1| \leq M_1$ as well.
	We will now use the notation $k_1(i)$ to be the $i$th
	integer of $n_{k_1}$.
	Thus our new subsequence of elements in $S$ is
	$\left(x^{(n_{k_1})}\right) = x^{(k_1(1))}, x^{(k_1(2))}, x^{(k_1(3))}, \dots$.
	
	Now consider the second components of our subsequence $\left(x^{(n_{k_1})}\right)$:
	\[
		\left(x_2^{(n_{k_1})}\right) = x_2^{(k_1(1))}, x_2^{(k_1(2))}, x_2^{(k_1(3))}, \dots
	\]
	Again, by Bolzano Weierstrass from boundedness by $M_2$,
	there exists a subsequence of $n_{k_1}$, call it $n_{k_2}$ such that
	$\left(x_2^{(n_{k_2})}\right)$ converges.
	Again, $k_2(i)$ is the $i$th integer of $n_{k_2}$.
	
	Now, for any $j$, we can iteratively acquire a subsequence $n_{k_j}$.
	We write it them all out below:
	\begin{align*}
		\left(x^{(n_{k_1})}\right) &= x^{(k_1(1))}, x^{(k_1(2))}, x^{(k_1(3))}, \dots\\
		\left(x^{(n_{k_2})}\right) &= x^{(k_2(1))}, x^{(k_2(2))}, x^{(k_2(3))}, \dots\\
						&\vdots\\
		\left(x^{(n_{k_j})}\right) &= x^{(k_j(1))}, x^{(k_j(2))}, x^{(k_j(3))}, \dots
	\end{align*}
	Some remarks to note:
	\begin{enumerate}
		\item $\left(x^{(n_{k_j})}\right)$ is a subsequence of
			$\left(x^{(n_{k_{j-1}})}\right)$ for all $j = 2,3,4,\dots$.
		\item $\left(x_j^{(n_{k_j})}\right)$ converges as $k_j \to \infty$
		(since each step, Bolzano Weierstrass lets us pick a convergent subsequence),
		whose value we denote $s_j$ and $|s_j| \leq M_j$
		(by the argument we provided when $j=1$).
		\item By the definition of a subsequence,
			if $x^{(m)}$ comes before $x^{(m')}$ in $\left(x^{(n_{k_j})}\right)$,
			we must have that $x^{(m)}$ comes before $x^{(m')}$ in
			$\left(x^{(n_{k_i})}\right)$ for all $1 \leq i \leq j$
			by the definition of a subsequence.
	\end{enumerate}
	We now pick entries from our diagonal to form a new subsequence of
	$\left(x^{(n)}\right)$, which we will call $\left(x^{(n_k)}\right)$:
	\[
		\left(x^{(n_k)}\right) = x^{(k_1(1))}, x^{(k_2(2))}, x^{(k_3(3))}, \dots
	\]
	Define $\hat{s} = s_{k_1(1)}, s_{k_2(2)}, s_{k_3(3)}, \dots$.
	Recall that $0 < |s_n| \leq M_n \implies 0 < |s_n|^2 \leq M_n^2$.
	Note since $\sum_n M_n^2$ converges by definition of being in $\ell^2$,
	we also have that $\sum_n |s_n|^2$ converges,
	thus $\hat{s} \in \ell^2$ as well.
	Hence, $\hat{s} \in S$.
	Therefore, it is sufficient to show now that
	$\left(x^{(n_k)}\right) \to \hat{s}$ as $k \to \infty$.

	From our previous homework, our notion of distance is
	\[
		d(x,y) = \lVert x - y\rVert = \sqrt{\sum_{n=1}^\infty (x_n-y_n)^2}
	\]
	Thus to say that $\left(x^{(n_k)}\right) \to \hat{s}$ as $k \to \infty$,
	we would need to show that for all $\ep > 0$, there exists $K\in\N$,
	such that for all $k > K$, we have
	$\sqrt{\sum_{i=1}^\infty (x^{(n_k)}_i-\hat{s}_i)^2} < \ep$.

	Thus let $\ep > 0$ be arbitrary.
	Since $(M_n) \in \ell^2$, $\sum_n M_n^2$ converges;
	denote the value it converges to by $L$.
	Note that $(\sum_n M_{n_k})$ is a subsequence,
	and subsequences converge to the same limit.
	Then there exists $K-1 \in \N$ such that for all $k \geq K-1$, we have
	\[
		\left\lvert L - \sum_{i=1}^k M_i^2\right\rvert < \frac{\ep^2}{4}
	\]
	This applies when $k = K-1$, thus
	\[
		\left\lvert  L - \sum_{i=1}^{K-1} M_i^2\right\rvert
		< \frac{\ep^2}{4} \implies \sum_{i=K}^\infty M_i^2 < \frac{\ep^2}{4}
	\]
	where the implication is because $L = \sum_{i=1}^{K-1} M_i^2 + \sum_{i=K}^\infty M_i^2$,
	and we remove the absolute value, because $M_i^2 > 0$ always,
	so the sum is $>0$ always.

	Now, note that for any $k$,
	$0 < |x^{(n_k)}_i - \hat{s}_i| \leq |x^{(n_k)}_i| + |\hat{s}_i| \leq 2M_i$
	for all $i$.
	Since $0 < |x^{(n_k)}_i - \hat{s}_i|^2 \leq 4M_i^2$,
	and we know that $\sum_n 4M_i^2$ converges,
	we know that $\sum_n|x^{(n_k)}_i - \hat{s}_i|^2$ converges too.
	Thus, writing an infinite series means something,
	and so for any $p \in P$,
	we have $\sum_{i=p}^\infty|x^{(n_k)}_i - \hat{s}_i|^2 \leq
	\sum_{i=p}^\infty M_i^2$.
	Thus, if we let $P_K = \sum_{i=1}^{K-1}|x^{(n_k)}_i - \hat{s}_i|^2$,
	we have
	\[
		\sum_{i=1}^\infty |x^{(n_k)}_i - \hat{s}_i|^2
		= P_K + \sum_{i=K}^\infty |x^{(n_k)}_i - \hat{s}_i|^2
		\leq P_K + 4\sum_{i=K}^\infty M_i^2
	\]
	Taking the $\limsup$ of both sides, we get
	\[
		\limsup_{k\to\infty} \sum_{i=1}^\infty |x^{(n_k)}_i - \hat{s}_i|^2
		\leq \limsup_{k\to\infty}\left(P_K + 4\sum_{i=K}^\infty M_i^2\right)
		\leq \limsup_{k\to\infty P_K} + 4\limsup_{k\to\infty}\sum_{i=K}^\infty M_i^2
	\]
	On the right hand side, note that $\limsup P_K = 0$,
	since $x^{(n_k)}_i \to \hat{s}_i$ as $k \to \infty$ by construction
	(note (b) from above);
	additionally $\sum_{i=K}^\infty M_i^2$ does not vary with $k$.
	Thus, we have
	\[
		\limsup_{k\to\infty} \sum_{i=1}^\infty |x^{(n_k)}_i - \hat{s}_i|^2
		\leq 4\sum_{i=K}^\infty M_i^2 < \ep^2
	\]
	By the definition of $\limsup$, this gives us
	\[
		\sum_{i=1}^\infty |x^{(n_k)}_i - \hat{s}_i|^2 < \ep^2
	\]
	hence
	\[
		\sqrt{\sum_{i=1}^\infty |x^{(n_k)}_i - \hat{s}_i|^2} < \ep
	\]
	for all $k \geq K$, as desired.
\end{proof}
\clearpage
\end{document}
