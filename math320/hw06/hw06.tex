\documentclass{article}
\usepackage{amsmath, amsfonts, amsthm, amssymb}
\usepackage{geometry}
\geometry{letterpaper, margin=2.0cm, includefoot, footskip=30pt}

\usepackage{fancyhdr}
\pagestyle{fancy}

\lhead{Math 320}
\chead{Homework 6}
\rhead{Nicholas Rees, 11848363}
\cfoot{Page \thepage}

\newtheorem*{problem}{Problem}

\newcommand{\N}{{\mathbb N}}
\newcommand{\Z}{{\mathbb Z}}
\newcommand{\Q}{{\mathbb Q}}
\newcommand{\R}{{\mathbb R}}
\newcommand{\C}{{\mathbb C}}
\newcommand{\ep}{{\varepsilon}}
\newcommand{\SR}{{\mathcal R}}

\renewcommand{\theenumi}{(\alph{enumi})}

\begin{document}
\subsection*{Problem 1}
{\it Let $0 < a_1 < b_1$ and define
	$a_{n+1} = \sqrt{a_nb_n}$, $b_{n+1} = \frac{a_n + b_n}{2}$, $n \in \N$.
\begin{enumerate}
	\item Prove that the sequences $(a_n)$ and $(b_n)$ both converge.
		(Suggestion: Use induction to prove $0 < a_n < a_{n+1} < b_{n+1} < b_n$.)
	\item Prove that the sequences $(a_n)$ and $(b_n)$ have the same limit.
\end{enumerate}}

\begin{enumerate}
	\item \begin{proof}[Solution]\let\qed\relax
		First, we use induction to prove $0 < a_n < a_{n+1} < b_{n+1} < b_n$.
		We do this in steps:
		first we prove that $0 < a_n$ and $0< b_n$ for all $n \in \N$.
		When $n = 1$, by assumption, $0 < a_1$ and $0 < b_1$.
		Now let $n = j$, and $0 < a_j,b_j$.
		Then $0 < \sqrt{a_jb_j} = a_{j+1}$
		and $0 < \frac{a_j+b_j}{2} = b_{j+1}$
		as desired.
		Thus $0 < a_n,b_n, \forall n \in \N$.

		Now we prove that $a_n < b_n$ for all $n \in \N$.
		When $n = 1$, by assumption, $a_1 < b_1$.
		Now let $n = j$, and $a_j < b_j$.
		Then $a_{j+1} = \sqrt{a_jb_j}$
		and $b_{j+1} = \frac{a_j + b_j}{2}$.
		We have
		\begin{align*}
			b_{j+1}^2
			&= (a_j^2 + 2a_jb_j + b_j^2)/4\\
			&> (2a_jb_j + 2a_jb_j)/4\\
			&= a_jb_j = a_{j+1}^2
		\end{align*}
		where the second line is because $a_j - b_j > 0 \implies (a_j - b_j)^2 > 0$
		so $a_j^2 - 2a_jb_j + b_j^2 > 0 \implies a_j^2 + b_j^2 > 2a_jb_j$.
		But since we know that both $a_{j+1}, b_{j+1} > 0$,
		then $b^2_{j+1} > a^2_{j+1} \implies |b_{j+1}| > |a_{j+1}|
		\implies b_{j+1} > a_{j+1}$, which closes the induction.

		Now we prove that $a_n < a_{n+1}$ and $b_n > \frac{a_n+b_n}{2}$.
		We have $a_{n+1} = \sqrt{a_nb_n} > \sqrt{a_n^2} = a_n$
		(since $x^2 > y^2 \implies x > y$ when $x,y > 0$).
		Furthermore, $b_{n+1} = \frac{a_n + b_n}{2} > \frac{2b_n}{2} = b_n$.
		Both of these are true for all $n \in \N$,
		so we have fully shown
		\[
			0 < a_n < a_{n+1} < b_{n+1} < b_n
		\]

		Note then that $(a_n)$ is monotonically increasing,
		and bounded by $a_1$ and $b_1$,
		and $(b_n)$ is monotonically decreasing,
		and bounded by $a_1$ and $b_1$ as well.
		Thus by the Monotone Convergence Theorem,
		and the fact $\R$ is complete,
		we have that $(a_n)$ and $(b_n)$ both converge.
	\end{proof}
	\item \begin{proof}[Solution]\let\qed\relax
		We claim that $b_n - a_n < (b_1 - a_1)/2^{n-1}$
		for all $n \geq 2$.
		This is true in the base case ($n = 2$): $b_2 - a_2 = (b_1 + a_1)/2 - \sqrt{a_1b_1}
		< (2b_1)/2 + (a_1 - b_1)/2 + \sqrt{a_1^2}
		= b_1 - a_1 - (b_1 - a-1)/2 = (b_1 - a_1)/2$.
		Now assume that $b_n - a_n < (b_1 - a_1)/2^{n-1}$.
		Then
		\begin{align*}
			b_{n+1} - a_{n+1}
			&= \frac{b_n+a_n}{2} - \sqrt{a_nb_n}\\
			&< \frac{2b_n}{2} + \frac{a_n - b_n}{2} - \sqrt{a_n^2}\\
			&= b_n - a_n - \frac{b_n - a_n}{2}\\
			&= \frac{b_n - a_n}{2}\\
			&< \frac{b_1 - a_1}{2^n}
		\end{align*}
		Thus we have closed the induction.

		Now let $\alpha \in \R$ be the value that $(a_n)$ converges to
		(as proven will exist with part (a)).
		Let $\ep > 0$.
		Let $\ep' = \ep/2$.
		We know when there exists some $N_1 \in \N$ such that for all $n \geq N$,
		we have $|\alpha - a_n| < \ep'$.
		Additionally, by Archimedes, there exists some $N_2 \in \N$ such that
		$0 < \frac{b_1-a_1}{\ep'} < N_2$,
		and $N_2 \leq 2^{N_2}$,
		thus $\ep' > (b_1 - a_1)/2^{N_2-1}$.
		Since $(b_1 - a_1)/2^{n-1}$ is a monotonically decreasing function,
		this $\ep' > (b_1 - a_1)/2^{n-1}$ for all $n \geq N_2$.
		So $0 < b_n - a_n < \ep'$ for all $n \geq N_2$.
		Thus $|b_n - \alpha| < |\ep' + a_n - \alpha| \leq |\ep'| + |a_n - \alpha|
		< \ep' + \ep' = \ep$,
		thus $b_n \to \alpha$ as well
		(where the first substitution is valid, since $b_n, a_n, \ep' > 0$).
	\end{proof}
\end{enumerate}
\clearpage

\subsection*{Problem 2}
{\it
\begin{enumerate}
	\item Suppose $(z_n)_{n\in\N}$ is a bounded sequence with integer values.
	\begin{enumerate}
		\item Prove that both numbers below are integers:
			\[
				\lambda = \liminf_{n\to\infty} z_n \qquad
				\mu = \limsup_{n\to\infty} z_n
			\]
		\item Prove that there are infinitely many integers
			$n$ for which $z_n = \lambda$.
	\end{enumerate}
	\item Let $d_n = p_{n+1} - p_n$ ($n \in \N$)
		denote the sequence of prime differences,
		built from the sequence of primes
		\[
			p_1 = 2, p_2 = 3, p_3 = 5, p_4 = 7, \dots
		\]
	\begin{enumerate}
		\item Prove that $\limsup_{d\to\infty} = +\infty$.
		\item Some mathematicians believe that
			\begin{equation}\label{twin}
				\delta := \liminf_{n \to\infty} d_n = 2
			\end{equation}
			However, the best estimate of $\delta$ known to date
			is $2 \leq \delta \leq 246$.
			Identify by name a famous unsolved problem in mathematics
			that is equivalent to proving or disproving line (\ref{twin}).
			(After giving the name, clearly explain the required relationship.)
	\end{enumerate}
\end{enumerate}}

\begin{enumerate}
	\item
	\begin{enumerate}
		\item \begin{proof}[Solution]\let\qed\relax
			Let $E$ be the set of number $x$ such that
			there exists a convergent subsequence $\{n_k\}$ where
			$z_{n_k}$ converges to $x$
			(which we know is nonempty by Bolzano-Weierstrass,
			because $(z_n)$ is bounded).
			We claim that each $x \in E$ is an integer.
			Since every convergent sequence is Cauchy,
			there must exist some $N$ where for all $k,k' \geq N$,
			$|z_{n_k} - z_{n_k'}| < \frac{1}{2}$,
			but since $z_{n_k}, z_{n_k'}$ are integers,
			this is only true when $z_{n_k} = z_{n_k'}$.
			Thus, for all $k \geq N$ $z_{n_k}$ is the same integer, call it $j$.
			$x$ must equal $j$;
			otherwise say $x = j + \delta$ for some fixed $\delta\in\R$,
			then if $\ep = \delta$, for all $k \geq N$,
			$|z_{n_k} - x| = |j - j \pm \delta| = |\delta| \not< \ep$,
			which contradicts that $s_{n_k}$ converges to $x$.
			Thus $x = j$, which is an integer.
			Therefore, every element in $E$ is an integer.

			Now, recall definition 3.16 in Rudin:
			$\lambda = \liminf_{n \to \infty}z_n = \inf{E}$ and
			$\mu = \limsup_{n\to\infty}z_n = \sup{E}$.
			Furthermore, theorem 3.17 in Rudin tells us that
			$\lambda \in E$ and $\mu \in E$.
			But every element in $E$ is an integer, thus
			$\lambda,\mu \in \Z$.
		\end{proof}
		\item \begin{proof}[Solution]\let\qed\relax
			Recall that from the previous part of the question
			that $\lambda \in E$ where $E$
			are all such $x$ where there is a subsequence
			$z_{n_k}$ that converges to $x$.
			So there is a subsequence $z_{n_l}$
			that converges to $\lambda$.
			In order for $z_{n_l} \to \lambda$,
			we must have that for all $\ep$,
			there is some $N\in\N$ such that for all $l \geq N$,
			we have $|z_{n_l} - \lambda| < \ep$.
			If $\ep = \frac{1}{2}$,
			since $z_{n_l}, \lambda$ are integers,
			we must have that $z_{n_l} = \lambda$.
			This is true for all $l \geq N$,
			which there are infinitely many of in,
			so there are infinitely many $z_{n_l} = \lambda$,
			thus there are infitly many $n$ where $z_{n_l} = \lambda$
			($n = n_l$ when $l \geq N$).
		\end{proof}
	\end{enumerate}
	\item
	\begin{enumerate}
		\item \begin{proof}[Solution]\let\qed\relax
			Let $k$ be given.
			We want to show that 
		\end{proof}
		\item \begin{proof}[Solution]\let\qed\relax
			ff
		\end{proof}
	\end{enumerate}
\end{enumerate}
\clearpage

\subsection*{Problem 3}
{\it \begin{enumerate}
	\item Prove: For any sequences $(a_n)$ and $(b_n)$ of nonnegative real numbers,
		\[
			\limsup_{n\to\infty}(a_nb_n) \leq \left(\limsup_{n\to\infty}a_n\right)
			\left(\limsup_{n\to\infty} b_n\right),
		\]
		provided the right side does not involve the product of $0$ and $\infty$.
	\item Give an example in which the result of part (a) holds with a strict inequality.
\end{enumerate}}

\begin{enumerate}
	\item \begin{proof}[Solution]\let\qed\relax
		Let $\alpha = \limsup_{n\to\infty} a_n$ and $\beta = \limsup_{n\to\infty}b_n$.

		First let us consider when $\alpha = +\infty$.
		Regardless of $\beta$ (since we are assuming $\beta > 0$),
		our right hand side becomes $+\infty$,
		and regardless of the left-hand side,
		in the extended reals, our inequality holds.
		WLOG, this is also true when $\beta = +\infty$.

		Now we can assume $\alpha > 0$, $\beta > 0$.
		Let $A > \alpha$.
		Then by definition, $A > \inf_{n \in \N}\left(\sup_{k\geq n} a_k\right)$,
		thus there exists some $N_1 \in \N$ such that $A > \sup_{k \geq N_1} a_k$,
		so $A > a_k$ when $k \geq N_1$.
		Similarly, if $B > \beta$,
		there exists some $N_2 \in \N$ such that $B > b_k$ when $k \geq N_2$.
		Let $N = \max\{N_1,N_2\}$.
		Then for all $k \geq N$,
		$a_k < A$ and $b_k < B$,
		thus $a_kb_k < AB$
		since both sides are nonnegative, and $A>\alpha\geq0$ and $B>\beta\geq0$.
		Thus $\sup_{k \geq N} (a_kb_k) \leq AB$.
		Any lower bound for a set of values indexed by $n$
		must be less than or equal to each of them
		(e.g. one where $n = N$).
		Thus the previous inequality implies
		\[
			\limsup_{n\to\infty}(a_nb_n) = \inf_{n\in\N}\sup_{k\geq n}(a_nb_n)
			\leq AB
		\]
		Since this holds for arbitrary $A > \alpha$ and $B > \beta$,
		the real number on the left cannot exceed $\alpha \beta$.
		Thus, we must have
		\[
			\limsup_{n\to\infty}(a_nb_n) \leq \alpha\beta = \limsup_{n\to\infty} a_n + \limsup_{n\to\infty} b_n
		\]
	\end{proof}
	\item \begin{proof}[Solution]\let\qed\relax
		We give the sequences $a_n = (-1)^n$ and $b_n = (-1)^{n+1}$.
		Then $a_nb_n = -1$,
		so $\limsup_{n\to\infty}(a_nb_n) = -1$.
		Additionally, $\limsup_{n\to\infty} a_n = 1$,
		since for any $n \in \N$,
		there always exists some $a_k = 1$ where $k \geq n$,
		and so $\limsup_{n\to\infty} = \inf_{n\in\N} \sup_{k \geq n} a_n
		= \inf_{n\in\N} 1 = 1$);
		and for the same reason, $\limsup_{n\to\infty} b_n = 1$.
		Thus
		\[
			\limsup_{n\to\infty} (a_nb_n) = -1 < 1 =
			\left(\limsup_{n\to\infty} a_n\right)\left(\limsup_{n\to\infty}b_n\right)
		\]
	\end{proof}
\end{enumerate}
\clearpage
~\clearpage

\subsection*{Problem 4}
{\it \begin{enumerate}
	\item Show that for any $r \geq 1$, one has
		\[
			n(r-1) \leq r^n - 1 \leq nr^{n-1}(r-1) \quad \forall n \in \N
		\]
	\item Prove that for each real $a \geq 1$, the following sequences converges:
		\[
			x_n = n\left(a^{1/n} - 1\right), \quad n \in \N
		\]
	\item Prove that the sequence in (b) also converges for each real $a \in (0,1)$.
	\item Let $L(x) = \lim_{n\to\infty} n\left(x^{1/n}-1\right)$ for $x > 0$.
		Prove that $L(ab) = L(a) + L(b)$ for all $a > 0, b > 0$.
\end{enumerate}
\emph{Note:} Present solutions that use only methods discussed in MATH 320.
No calculus, please!}

\begin{enumerate}
	\item \begin{proof}[Solution]\let\qed\relax
		Recall the identity $b^n - a^n =
		(b-a)(b^{n-1} + b^{n-2}a + \cdots + a^{n-1})$
		for any $n \in \N$.
		Since $r \geq 1$,
		applying the identity when $b = r$ and $a = 1$,
		we get
		\[
			r^n - 1 = (r-1)(r^{n-1} + r^{n-2} + \cdots + 1) \leq (r-1)nr^{n-1}
		\]
		since each $r^{n-j} \leq r^{n-1}$ ($1 \leq j \leq n$),
		and there are $n$ many $r^{n-j}$ in the factor. 

		To prove the second inequality,
		we use the identity again,
		however
		\[
			r^n - 1 = (r-1)(r^{n-1} + r^{n-2} + \cdots + 1) \geq n(r-1)
		\]
		since each $r^{n-j} \geq 1$ ($1 \leq j \leq n$),
		and there are $n$ many $r^{n-j}$ in the factor.
		Thus we have shown $n(r-1) \leq r^n - 1 \leq nr^{n-1}(r-1)$.
	\end{proof}
	\item \begin{proof}[Solution]\let\qed\relax
		We have that for all $n \in \N$,
		by part (a) of this problem,
		$x_n = n\left(a^{1/n}-1\right) \leq a - 1$,
		thus $x_n$ is bounded above.
		Additionally, $a \geq 1 = 1^n \implies a^{1/n} \geq 1$,
		thus $n(a^{1/n}-1) \geq 0$, and so $x_n$ is bounded below.
		Furthermore, we claim that $(x_n)$ is monotonically decreasing:
		$x_{n+1} = (n+1)(a^{1/(n+1)} - 1) = n(a^{1/(n+1)} - 1) + a^{1/(n+1)} - 1$. 
		That is $x_n  - x_{n+1} \geq 0$.
		We see
		\begin{align*}
			x_n - x_{n+1}
			&= n\left(a^{1/n} - 1\right) - (n+1)\left(a^{1/(n+1)}-1\right)\\
			&\geq n^2\left(a^{1/n^2} - 1\right) - (n+1)\left(a^{1/(n+1)}-1\right)\\
			&= 
		\end{align*}
		I can't figure this one out, but let's just assume it's monotonically decreasing.
		Since it is bounded and monotonically decreasing,
		it converges.
	\end{proof}
	\item \begin{proof}[Solution]\let\qed\relax
		The sequence is bounded above by $0$,
		since $0 < a < 1 = 1^n \implies a^{1/n} < 1$,
		thus $n(a^{1/n} - 1) < 0$.
		Furthermore, it is bounded below, and monotonically decreasing
		(same as above).
		Thus the series converges.
	\end{proof}
	\item \begin{proof}[Solution]\let\qed\relax
			We have
			\begin{align*}
				L(ab)
				&= \lim_{n\to\infty} n((ab)^{1/n} - 1)\\
				&= \lim_{n\to\infty} n(a^{1/n}b^{1/n} - 1)\\
				&= \lim_{n\to\infty} n/2(-(a^{1/n} - b^{1/n})^2 - 2 + a^{2/n} + b^{2/n})\\
				&=
			\end{align*}
			.
	\end{proof}
\end{enumerate}
\clearpage

\end{document}
