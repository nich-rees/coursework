\documentclass{article}
\usepackage{amsmath, amsfonts, amsthm, amssymb}
\usepackage{geometry}
\geometry{letterpaper, margin=2.0cm, includefoot, footskip=30pt}

\usepackage{fancyhdr}
\pagestyle{fancy}

\lhead{Math 320}
\chead{Homework 12}
\rhead{Nicholas Rees, 11848363}
\cfoot{Page \thepage}

\newtheorem*{problem}{Problem}

\newcommand{\N}{{\mathbb N}}
\newcommand{\Z}{{\mathbb Z}}
\newcommand{\Q}{{\mathbb Q}}
\newcommand{\R}{{\mathbb R}}
\newcommand{\C}{{\mathbb C}}
\newcommand{\ep}{{\varepsilon}}
\newcommand{\SR}{{\mathcal R}}

\renewcommand{\theenumi}{(\alph{enumi})}

\begin{document}
\subsection*{Problem 1}
{\it If $f \colon X \to Y$ is a continuous mapping between
Hausdorff topological spaces $X$ and $Y$,
prove that
\[
	f(\overline{E}) \subseteq \overline{f(E)}
\]
for every set $E \subseteq X$.
Show, by an example, that $f(\overline{E})$ can be a proper subset of $\overline{f(E)}$.}

\begin{proof}[Solution]\let\qed\relax
	Let $y \in f(\overline{E})$,
	then there is some $x \in \overline{E}$ such that $f(x) = y$.
	If $x \in E$, we have that $y \in f(E)$ hence $y \in \overline{f(E)}$.
	If $x \not\in E$, we must have that $x \in E'$ (since $\overline{E} = E \cup E'$).
	
	So now assume that $x \in E'$.
	If $y \in f(E)$, we have $y \in \overline{f(E)}$ as well.
	So now also assume that $y \not\in f(E)$.
	Consider some arbitrary open set $W$ containing $y$.
	Since $f$ is continuous, $U = f^{-1}(W)$ is open,
	and furthermore, $x \in U$.
	Since $x \in E'$, we have that $U \setminus\{x\} \cap E \neq \emptyset$.
	So there is some $x_1 \in U\setminus\{x\}$ and $x_1 \in E$.
	So $f(x_1) \in f(E)$ and
	$f(x_1) \in f(U \setminus \{x\}) \subseteq f(U) = W$.
	Hence, we have that $f(x_1) \in f(E) \cap W \neq \emptyset$,
	and since $y \not\in f(E)$, we have $f(x_1) \in f(E) \cap (W \setminus \{y\})$.
	Hence, $f(E) \cap (W \setminus \{y\}) \neq \emptyset$.
	But since $W$ was an arbitrary open set containing $y$,
	this gives us that $y \in f(E)'$.
	Thus, $y \in \overline{f(E)}$.

	In each case, whether $x \in E$,
	$x \not\in E$ and $y \in f(E)$,
	or $x \not\in E$ and $y \not \in f(E)$,
	we have shown that $y \in \overline{f(E)}$.
	Since $y$ was an arbitrary point in $f(\overline{E})$,
	this shows that $f(\overline{E}) \subseteq \overline{f(E)}$ as deired.

	To show that $f(\overline{E})$ can be a proper subset of $\overline{f(E)}$,
	we give the following example:
	let $X$ be $\R$ with the discrete topology,
	and $Y$ be $\R$ with the usual topology (from the metric).
	As we have mentioned in class, these are both Hausdorff.
	Recall that, by definition, every set in $X$ is open,
	which implies that for any $U \subseteq X$, $U^c$ is open as well,
	so every set in $X$ is closed as well.
	Hence, if we let $E \subseteq X$ be the interval $(a,b)$,
	since $E$ is closed, we have $E = \overline{E} = (a,b)$.
	Now let $f$ be the identity map.
	This is obviously continuous, since the pre-image of every
	open set in $Y$ is a set in $X$, which is open.
	So we have $f(\overline{E}) = f((a,b)) = (a,b)$.
	Note though that $\overline{f(E)} = \overline{(a,b)} = [a,b]$.
	So $f(\overline{E}) = (a,b) \subsetneq [a,b] = \overline{f(E)}$.
	Hence, we have some continuous $f$ that maps
	between two Hausdorff spaces,
	and there is some $E \subseteq X$ such that $f(\overline{E}) \subsetneq \overline{f(E)}$.
\end{proof}
\clearpage
~\clearpage

\subsection*{Problem 2}
{\it
\begin{enumerate}
	\item Let $X$ and $Y$ be metric spaces. Prove that for $f \colon X \to Y$, TFAE:
	\begin{enumerate}
		\item $f$ is uniformly continuous on $X$;
		\item for any sequences $(x_n)$ and $(x'_n)$ in $X$
			satisfying $d_X(x_n,x'_n) \to 0$,
			one has $d_Y(y_n,y'_n) \to 0$, where $y_n = f(x_n), y'_n=f(x'_n)$.
	\end{enumerate}
	\item Identify, with proof, all real numbers $p$ for which the function
		$f(x) = x^p$ is uniformly continuous on $X = (0,+\infty)$.
		[It's OK to use a little calculus to support your findings.]
\end{enumerate}}


\begin{enumerate}
	\item \begin{proof}[Solution]\let\qed\relax
		We first prove $(a) \implies (b)$.
		So assume that $f$ is uniformly continuous on $X$.
		Let $(x_n)$, $(x'_n)$ be arbitrary sequences in $X$
		such that $d_X(x_n,x'_n) \to 0$.
		Let $\ep > 0$ be arbitrary.
		Since $f$ is uniformly continuous on $X$,
		we get some $\delta > 0$ such that
		$\forall t \in \mathbb{B}_X[x_n;\delta)$,
		$d_Y(f(x_n),f(t)) < \ep$.
		Now using the definition of $d_X(x_n,x'_n) \to 0$,
		we have that for some $N$, $d(x_n,x'_n) < \delta$
		for all $n \geq N$.
		Hence, $x'_n \in \mathbb{B}_X[x_n;\delta)$,
		thus $d_Y(f(x_n),f(x'_n)) < \ep$ for all $n \geq N$.
		But then if $y_n = f(x_n)$, $y'_n = f(x'_n)$,
		since $\ep$ was arbitrary,
		this is the definition for $d_Y(y_n,y'_n) \to 0$.

		Now, to prove $(b) \implies (a)$,
		we use contraposition,
		so assume that $f$ is not uniformly continuous on $X$.
		Then there is some $\ep' > 0$ such that for any $\delta > 0$,
		there is some $s \in X$ where for some $t \in \mathbb{B}_X[s;\delta)$,
		we have $d(f(s),f(t)) \geq \ep'$.
		Consider $\delta_n = \frac{1}{n}$.
		For each $\delta_n$, from above, we can get $s,t$ where
		$t \in \mathbb{B}_X[s;\delta_n)$, but $d(f(s),f(t))\geq \ep'$;
		so for each $\delta_n$, we let $x_n = s$ and $x'_n = t$.
		See that $d_X(x_n,x'_n) \to 0$,
		since for any $\ep > 0$, using the Archimedean property,
		we can find $N > \frac{1}{\ep} > 0$ so $\frac{1}{N} = \delta_N < \ep$
		and $\forall n \geq N$, $\delta_n \leq \delta_N < \ep$,
		so for all $n \geq N$, $x'_n \in \mathbb{B}_X[x_n;\delta_n)
		\implies d_X(x_n,x'_n) < \delta_n < \ep$.
		However, notice that if $y_n = f(x_n)$ and $y'_n = f(x'_n)$,
		we have $d_Y(y_n,y'_n) \not\to 0$,
		since regardless of $n$, we have $d_Y(y_n,y'_n) \geq \ep'$,
		hence $d_Y(y_n,y'_n)$ cannot converge to $0$.
		This shows that $(a) \iff (b)$, as desired.
	\end{proof}
	\item \begin{proof}[Solution]\let\qed\relax
		We note that $|x|^p$ is not uniformly continuous on $X$
		when $p < 0$ and $p >1$:
		via calculus, the slopes are increasing at some point,
		and so for any $\ep$, we can always find an two points $x_1,x_2$
		such that $|x_1-x_2| < \delta$ but $|f(x_1) - f(x_2)| \geq \ep$.

		In the case when $p = 0$, this is uniformly continuous trivially:
		we just pick $\delta = \ep/2$.

		In the $p \in (0,1]$ case, we have uniform continiuity.
		I ran out of time, but the essential idea
		is that since the slope doesn't increase,
		we can find $\delta$ small enough that points are going to be
		close enough together that $|f(x_1) - f(x_2)| < \ep$...
		might use a little MVT.
	\end{proof}
\end{enumerate}
\clearpage
~\clearpage

\subsection*{Problem 3}
{\it A metric space $(X,d)$ is called an \emph{ultrametric space} if $d$ satisfies the condition
\[
	\forall x,y,z \in X, \quad d(x,z) \leq \max\{d(x,y),d(y,z)\}.
\]
(This makes $d$ itself ``an ultrametric".) Show that in any ultrametric space $(X,d)$,...
\begin{enumerate}
	\item every open ball $\mathbb{B}[x;r)$ is a closed set;
	\item one has $y \in \mathbb{B}[x;r)$ if and only if $\mathbb{B}[y;r) = \mathbb{B}[x;r)$; and
	\item if $\mathbb{B}[x;r_1) \cap \mathbb{B}[y;r_2) \neq \emptyset$,
		then one of these balls must contain the other, i.e.,
		\[
			\mathbb{B}[x;r_1) \subseteq \mathbb{B}[y;r_2) \neq \emptyset
			\; \text{ or }\;
			\mathbb{B}[x;r_1) \supseteq \mathbb{B}[y;r_2) \neq \emptyset
		\]
\end{enumerate}
[The ``$p$-adic numbers" form an ultrametric space of interest in number theory.]}

\begin{enumerate}
	\item \begin{proof}[Solution]\let\qed\relax
		So normally, triangle inequality gives us $d(x,z) \leq d(x,y) + d(y,z)$.
		
		By a corollary from class, it is sufficient to show that
		$\mathbb{B}[x;r)' \subseteq \mathbb{B}[x;r)$.
		If $\mathbb{B}[x;r)' = \emptyset$, we are done.
		So assume that there exists some $y \in \mathbb{B}[x;r)'$.
		So for any $r \geq r_1 > 0$, we have
		$\mathbb{B}[x;r) \cap (\mathbb{B}[y;r_1)\setminus\{y\}) \neq \emptyset$.
		So there is some $z \in \mathbb{B}[x;r)$ and $z \in \mathbb{B}[y;r_1) \setminus \{y\}$.
		Hence $d(x,z) < r$ and $0 < d(y,z) < r_1$.
		Using the inequality, we then get that $d(x,y) < \max\{r,r_1\}$.
		Since $r \geq r_1$, we have $d(x,y) < r \implies y \in \mathbb{B}[x,r)$.
		Since $y \in \mathbb{B}[x;r)'$ was arbitrary,
		this shows that $\mathbb{B}[x;r)' \subseteq \mathbb{B}[x;r)$,
		so $\mathbb{B}[x;r)$ is closed.
	\end{proof}
	\item \begin{proof}[Solution]\let\qed\relax
			Assume that $y \in \mathbb{B}[x;r)$.
			So $d(x,y) < r$.
			Let $z \in \mathbb{B}[y;r)$,
			so $d(y,z) < r$.
			Thus, $d(x,z) \leq \max\{d(x,y),d(y,z)\} < r$.
			Hence, $z \in \mathbb{B}[x;r)$,
			and since $z$ was arbitrary, we get $\mathbb{B}[y;r) \subseteq \mathbb{B}[x;r)$.
			Now let $z \in \mathbb{B}[x;r)$,
			so $d(x,z) < r$.
			Thus, $d(y,z) \leq \max\{d(x,y),d(x,z)\} < r$.
			Hence, $z \in \mathbb{B}[y;r)$,
			and since $z$ was arbitrary, we get $\mathbb{B}[y;r) \subseteq \mathbb{B}[x;r)$.
			Therefore, we get that $\mathbb{B}[x;r) = \mathbb{B}[y;r)$.

			Now assume that $\mathbb{B}[y;r) = \mathbb{B}[x;r)$.
			Since $y \in \mathbb{B}[y;r)$, we must then have that
			$y \in \mathbb{B}[x;r)$ by equality.
			Thus, we have $y \in \mathbb{B}[x;r)$ if and
			only if $\mathbb{B}[x;r) = \mathbb{B}[y;r)$.
	\end{proof}
	\item \begin{proof}[Solution]\let\qed\relax
			If $\mathbb{B}[x;r_1) \cap \mathbb{B}[y;r_2) \neq \emptyset$,
			then $\mathbb{B}[x;r_1) \neq \emptyset \neq \mathbb{B}[y;r_2)$,
			and there exists some $z \in \mathbb{B}[x;r_1)$
			and $z \in \mathbb{B}[y;r_2)$.
			So $d(x,z) < r_1$ and $d(y,z) < r_2$.
			Assume that $\mathbb{B}[x;r_1) \not\supseteq \mathbb{B}[y;r_2)$.
			Then there is some point $y_1 \in \mathbb{B}[y;r_2)$ such that
			$y_1 \not\in \mathbb{B}[x;r_1)$.
			So $d(y,y_1) < r_2$ but $d(x,y_1) \geq r_1$.

			We now show that this gives us $r_2 \geq r_1$.
			Note that $d(y_1,z) \leq \max\{d(y_1,y),d(z,y)\} \leq r_2$.
			Now we have that $\max\{d(y_1,z),d(x,z)\} \geq d(y_1,x) \geq r_1$.
			However, $d(x,z) < r_1$, so to have our left hand side $\geq$ than the right,
			we must have that $d(y_1,z) \geq r_1 > d(x,z)$.
			Thus, $r_1 \leq d(y_1,z) \leq r_2$ so by transitivity of order in the reals,
			we get that $r_1 \leq r_2$ as well.

			Now take any $x_1 \in \mathbb{B}[x;r_1) \neq \emptyset$,
			We know that $d(x_1,z) \leq \max\{d(x_1,x),d(x,z)\} < r_1$.
			Furthermore, we know that $d(y,z) < r_2$,
			so $d(x_1,y) \leq \max\{d(x_1,z),d(z,y)\} < \max\{r_1,r_2\} = r_2$.
			So $x_1 \in \mathbb{B}[y;r_2)$.
			But since $x_1 \in \mathbb{B}[x_1;r_1)$ was arbitrary,
			this gives us that $\mathbb{B}[x;r_1) \subseteq \mathbb{B}[y;r_2)$.
			Therefore we must have $\mathbb{B}[y;r_2) \subseteq \mathbb{B}[x;r_1)$,
			or if this is not true, we have proven that we must then have
			$\mathbb{B}[x;r_1) \subseteq \mathbb{B}[y;r_2)$.
			Furthermore, we stated our balls are not $\emptyset$ at the start,
			and so we have proven what is desired.
	\end{proof}
\end{enumerate}
\clearpage
~\clearpage

\subsection*{Problem 4}
{\it Given Hausdorff Topological Spaces $(X,\mathcal{T}_X)$ and $(Y,\mathcal{T}_Y)$,
and continuous functions $f,g \colon X \to Y$, consider the \emph{equalizer}:
\[
	E = \{x \in X \colon f(x) = g(x)\}.
\]
Prove that $E$ is closed in $X$.}

\begin{proof}[Solution]\let\qed\relax
	For the sake of contradiction, assume that $E$ is not closed in $X$.
	Then $E \subsetneq \overline{E}$.
	So there is some $x_1 \in \overline{E}$ but $x_1 \not\in E$.
	Recall the following theorem from class:
	for continuous function $f_1,f_2$ and $Q \subseteq X$,
	if $f_1(q)=f_2(q)$ for all $q \in Q$, then $f_1(x) = f_2(x)$ for all $x \in \overline{Q}$.
	Using the theorem with the definition of $E$ then, we must have that $f(x)=g(x)$
	for all $x \in \overline{E}$, which includes $x_1$,
	so $f(x_1) = g(x_1)$.
	But then $x_1 \in E$, a contradiction, since we assumed that $x_1 \not\in E$.
	Hence, we must have that $E = \overline{E}$ and so $E$ is closed.
\end{proof}
\clearpage
~\clearpage

\subsection*{Problem 5}
{\it Three continuous functions $f,g,h \colon \R \to \R$ are related by the identity
\[
	f(x+y) = g(x) + h(y)
\]
\begin{enumerate}
	\item In the special case where $f=g=h$, show that there must
	be a real number $m$ such that $f(t) = mt$ for all real $t$.
	\item Drop the hypothesis that $f,g,h$ are identical.
	Describe the most general trio of continuous functions
	compatible with the given identity.
\end{enumerate}}

\begin{enumerate}
	\item \begin{proof}[Solution]\let\qed\relax
		Since $f(x+y) = f(x) + f(y)$, we get that $f(2y) = 2f(y)$.
		Now, we can prove that $f(ky) = kf(y)$ by induction,
		since if $f(ky) = kf(y)$, we have $f((k+1)y) = f(ky)+f(y)=(k+1)f(y)$,
		and our base case is done above.
		Hence $f(ky) = kf(y)$ for all $k \in \N$ and $y \in \R$.

		Now, when $x = y = 0$, we have that $f(0) = 2f(0)$,
		so $f(0) = 0$ is the only value that satisfies this.
		So $0 = f(0) = f(ky +(-ky)) = f(ky) + f(-ky)$
		so $f(-ky) = -kf(y) \implies f(ky) = kf(y)$ is true for any $k \in \Z$ as well.

		Now since $f(y) = qf(y/q)$, we get that $\frac{1}{q}f(y)=f(y/q)$
		when $q \in \Z \setminus \{0\}$.
		Soo $f(\frac{p}{q}y) = \frac{p}{q}f(y)$ for any $a \in \Z$ and $b \in \Z\setminus \{0\}$.
		So $f(ky) = kf(y)$ when $k \in \Q$ as well.

		Now let $y \in \R$ be fixed.
		We have that two functions, $f(ky)$ and $kf(y)$
		are fixed for all $k \in \Q$.
		Since $f$ is continuous, and $\Q$ is dense in $\R$,
		by the theorem in class,
		this gives us that $f(ky) = kf(y)$ when $k \in \R$ as well.
		Since $y \in \R$ is arbitrary, this means that $f(ky) = kf(y)$
		for all $k,y \in \R$.

		It is clear then that for $t \in \R$,
		we have $f(t) = tf(1)$.
		So if $f(1) = m \in \R$, we recover $f(t) = mt$, as desired.
	\end{proof}
	\item \begin{proof}[Solution]\let\qed\relax
		Sorry ran out of time...
	\end{proof}
\end{enumerate}
\clearpage
~\clearpage

\subsection*{Problem 6}
{\it Here's a key fact every math student should know:
\begin{center}
	Every nonempty open set in $\R$ can be expressed
	as a finite or countable union of disjoint open intervals
\end{center}
Prove this, referring to a given open set $U \neq \emptyset$, by following these steps:
\begin{enumerate}
	\item For each $x \in U$, let $I(x) = (\alpha(x),\beta(x))$, where
	\[
		\alpha(x) = \inf\{a \colon \text{ one has }x\in(a,b) \text{ for some }(a,b)\subseteq U\}\\
		\beta(x) = \sup\{a \colon \text{ one has }x\in(a,b) \text{ for some }(a,b)\subseteq U\}
	\]
	Prove that $x \in I(x)$ and $I(x) \subseteq U$,
	while $\alpha(x) \not\in U$ and $\beta(x) \not\in U$.
	[Argue carefully, since both $\alpha(x) = -\infty$ and $\beta(x) = +\infty$ are possible.]
	\item Let $\mathcal{G} = \{I(x) \colon x \in U\}$.
		Show that any two intervals in $\mathcal{G}$ must be either disjoint or identical.
	\item Explain why the key fact stated above must hold.
\end{enumerate}}

\begin{enumerate}
	\item \begin{proof}[Solution]\let\qed\relax
		Sorry ran out of time...
	\end{proof}
	\item \begin{proof}[Solution]\let\qed\relax
		Sorry ran out of time...
	\end{proof}
	\item \begin{proof}[Solution]\let\qed\relax
		The basic argument would go somethng like,
		our $I(x)$ will cover any open set in $\R$ by (a),
		and there are at most countable many of them by,
		and so by (b), they are disjoint.
	\end{proof}
\end{enumerate}
\clearpage
~\clearpage
\end{document}
