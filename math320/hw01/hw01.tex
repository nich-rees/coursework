\documentclass{article}
\usepackage{amsmath, amsfonts, amsthm, amssymb}
\usepackage{geometry}
\geometry{letterpaper, margin=2.0cm, includefoot, footskip=30pt}

\usepackage{fancyhdr}
\pagestyle{fancy}

\lhead{Math 320}
\chead{Homework 1}
\rhead{Nicholas Rees, 11848363}
\cfoot{Page \thepage}

\newtheorem*{problem}{Problem}

\newcommand{\N}{{\mathbb N}}
\newcommand{\Z}{{\mathbb Z}}
\newcommand{\Q}{{\mathbb Q}}
\newcommand{\R}{{\mathbb R}}
\newcommand{\C}{{\mathbb C}}
\newcommand{\ep}{{\varepsilon}}

\renewcommand{\theenumi}{(\alph{enumi})}

\begin{document}
\subsection*{Problem 1}
\textit{Prove or disprove: For each $n\in\N, n^2-n+41$ is prime.}
\begin{proof}[Solution]\let\qed\relax
This claim is not true.
We provide a counterexample: $n = 41$.
Then $n^2 - n + 41 = 41^2$
which is obviously not prime since it has the factor $41$.
\end{proof}
\clearpage

\subsection*{Problem 2}
{\it If $A$, $B$, and $C$ are sets, prove that
\begin{enumerate}
	\item $A \cap (B \cup C) = (A \cap B) \cup (A \cap C)$,
	\item $C \setminus (A \cup B) = (C \setminus A) \cap (C \setminus B)$,
	\item $C \setminus (A \cap B) = (C \setminus A) \cup (C \setminus B)$,
\end{enumerate}}
\begin{enumerate}
\item \begin{proof}[Solution]\let\qed\relax
	Let $x \in A \cap (B \cup C)$.
	Then $x$ is in both $A$ and $B \cup C$.
	But then $x$ is in at least one of $B$ or $C$.
	So $x$ is either in both $A$ and $B$,
	or both $A$ and $C$.
	Thus, either $x\in A \cap B$ or $x\in A\cap C$.
	Therefore, $x \in (A \cap B) \cup (A \cap C)$.
	Thus, every element in the set on the LHS is in the RHS,
	so $A \cap (B \cup C) \subset (A \cap B) \cup (A \cap C)$.

	We now prove the other direction.
	Let $y \in (A \cap B) \cup (A \cap C)$.
	Then $y$ is in either in $(A \cap B)$ or $(A \cap C)$.
	If $y \in (A \cap B)$, then $y$ is in both $A$ and $B$.
	Thus $y$ is in both $A$ and $B \cup C$.
	Now if $y \in (A \cap C)$, then $y$ is in both $A$ and $C$.
	Thus $y$ is in both $A$ and $B \cup C$ as well.
	So $y \in A \cap (B \cup C)$ in either case.
	So $A \cap (B \cup C) \supset (A \cap B) \cup (A \cap C)$,
	and so $A \cap (B \cup C) = (A \cap B) \cup (A \cap C)$.
\end{proof}
\item \begin{proof}[Solution]\let\qed\relax
	Let $x \in C \setminus (A \cup B)$.
	Then $x$ is in $C$ but is not in $A \cup B$.
	This means $x$ is not in $A$ nor $B$.
	This means $x$ is in both $C \setminus A$ and $C \setminus B$.
	But then $x \in (C \setminus A) \cap (C \setminus B)$.
	So $C \setminus (A \cup B) \subset (C \setminus A) \cap (C \setminus B)$.

	We now prove the other direction.
	Let $y \in (C \setminus A) \cap (C \setminus B)$.
	Then $y$ is in both $C \setminus A$ and in $C \setminus B$.
	So $y \in C$ and $y \not\in A$ and $y\not\in B$.
	Thus $y \not\in A \cup B$.
	Thus $y \in C \setminus (A\cup B)$.
	So $C \setminus (A \cup B) \supset (C \setminus A) \cap (C \setminus B)$.
	Therefore $C \setminus (A \cup B) = (C \setminus A) \cap (C \setminus B)$.
\end{proof}
\item \begin{proof}[Solution]\let\qed\relax
	Let $x \in C \setminus (A \cap B)$.
	Then $x$ is in $C$ but is not in $A \cap B$.
	This means $x$ is not in $A$ or is not in $B$.
	This means $x$ is in at least one of $C \setminus A$ and $C \setminus B$.
	But then $x \in (C \setminus A) \cup (C \setminus B)$.
	So $C \setminus (A \cap B) \subset (C \setminus A) \cup (C \setminus B)$.

	We now prove the other direction.
	Let $y \in (C \setminus A) \cup (C \setminus B)$.
	Then $y$ is in at least one of $C \setminus A$ or $C \setminus B$.
	So $y \in C$, and $y \not\in A$ or $y\not\in B$.
	Thus $y \not\in A \cup B$.
	Thus $y \in C \setminus (A\cup B)$.
	So $C \setminus (A \cup B) \supset (C \setminus A) \cap (C \setminus B)$.
	Therefore $C \setminus (A \cup B) = (C \setminus A) \cap (C \setminus B)$.
\end{proof}

\end{enumerate}
\clearpage

\subsection*{Problem 3}
{\it Let $f \colon A \to B$. Let $C$, $C_1$, and $C_2$ be subsets of $A$,
and let $D$ be a subset of $B$. Prove:
\begin{enumerate}
	\item If $f$ is one-to-one, then $f(C_1 \cap C_2) = f(C_1)\cap f(C_2)$.
	\item If $f$ is $1$-$1$, then $f^{-1}(f(C)) = C$.
	\item If $f$ is onto, then $f(f^{-1}(D)) = D$.
\end{enumerate}
In each part, find an inclusion relation
(either ``$\subseteq$" or ``$\supseteq$")
that can be used to replace the symbol ``$=$" and produce
a true statement even without the given hypothesis.
[Recall that $f^{-1}(y) = \{x\in A \colon f(x) = y\}$ is,
in general, a set-valued operation.
It is not safe to infer that $f$ is invertible
just because the symbol $f^{-1}$ appears.]}
\begin{enumerate}
\item \begin{proof}[Solution]\let\qed\relax
Let $y \in f(C_1 \cap C_2)$.
Let $y \in f(C_1 \cap C_2)$.
Then there is some $x \in C_1 \cap C_2$ such that $f(x) = y$.
We know $x \in C_1$ and $x \in C_2$ as well.
Thus, $f(x) = y \in f(C_1)$ and $f(x) = y \in f(C_2)$,
so $y \in f(C_1) \cap f(C_2)$.
This means $f(C_1 \cap C_2) \subset f(C_1) \cap f(C_2)$.

We now prove the other direction.
Let $y \in f(C_1) \cap f(C_2)$ now.
Then there is some $x_1\in C_1,x_2\in C_2$
such that $f(x_1) = f(x_2) = y$.
But since $f$ is $1$-$1$, we have that $x_1 = x_2$ since $f(x_1) = f(x_2)$; we let $x = x_1 = x_2$.
But since $x = x_1 \in C_1$ and $x = x_2 \in C_2$, we have $x \in C_1 \cap C_2$.
Thus $f(x) = y \in f(C_1 \cap C_2)$.
Therefore, $f(C_1) \cap f(C_2) \subset f(C_1 \cap C_2)$,
and so $f(C_1) \cap f(C_2) = f(C_1 \cap C_2)$.

Note that we only used $f$ being $1$-$1$ in proving $f(C_1) \cap f(C_2) \subset f(C_1 \cap C_2)$ and not the other direction,
and so we can still say $f(C_1 \cap C_2) \subseteq f(C_1)\cap f(C_2)$.
\end{proof}
\item \begin{proof}[Solution]\let\qed\relax
Let $x \in f^{-1}(f(C))$.
Then, there is some $c \in C$ such that $f(x) = f(c)$.
But since $f$ is $1$-$1$, this implies that $x = c$.
So $x \in C$, implying that $f^{-1}(f(C)) \subset C$.

Proving the other direction, note that trivially,
$C$ maps to $f(C)$, and so $C$ is at least a subset of the set that maps to $f(C)$.
This means that $C \subset f^{-1}(f(C))$,
therefore $f^{-1}(f(C)) = C$.

Note that we only used $f$ being $1$-$1$ in proving $f^{-1}(f(C)) \subset C$ and not the other direction,
and so we can still say $f^{-1}(f(C)) \supseteq C$.
\end{proof}
\item \begin{proof}[Solution]\let\qed\relax
Consider the inverse image of $D$, $f^{-1}(D)$.
If $f^{-1}(D)$ has no elements,
then $f(f^{-1}(D)) = \emptyset \subset D$ and we are done this direction.
If $f^{-1}(D) \neq \emptyset$, then we can pick an element $x \in f^{-1}(D)$.
Then, by definition, $f(x) \in D$.
Since this is true for all $x \in f^{-1}(D)$ since $x$ was arbitrary,
we have $f(f^{-1}(D)) \subset D$. 

Proving the other direction,
since $f$ is onto, we have $D \subset f(A)$,
or more specifically $D \subset f(A_1)$,
where $A_1 \subset A$
is exactly the set of elements in $A$ that map to $D$.
But this is just the definition of $f^{-1}(D)$, so $A_1 = f^{-1}(D)$,
and thus $D \subset f(f^{-1}(D))$.
Therefore, $f(f^{-1}(D)) = D$.

Note that we only used $f$ being onto in proving $D \subset f(f^{-1}(D))$
and not the other direction,
and so we can still say $f(f^{-1}(D)) \subseteq D$.
\end{proof}
\end{enumerate}
\clearpage

\subsection*{Problem 4}
{\it For Question 3(a), construct a specific example
in which the indicated equation fails.
(Of course the given hypothesis will have to be false too.)
Repeat for parts 3(b) and 3(c).}
\begin{enumerate}
\item \begin{proof}[Solution]\let\qed\relax
Consider $x^2$ (and so $A,B = \R$).
Let $C_1 = [-1,2]$ and $C_2 = [-2,1]$.
Note that $x^2$ is not $1$-$1$, since it maps multiple elements in the domain
to the same element in the codomain (eg. $f(-1)=f(1)$ but $-1\neq 1$),
so the hypothesis fails.
Note that $f(C_1 \cap C_2) = f([-1,1]) = [0,1]$,
however $f(C_1) \cap f(C_2) = [0,4] \cap [0,4] = [0,4]$.
Thus $f(C_1 \cap C_2) \neq f(C_1) \cap f(C_2)$,
specifically $f(C_1 \cap C_2) \subset f(C_1) \cap f(C_2)$ as mentioned previously.
\end{proof}
\item \begin{proof}[Solution]\let\qed\relax
Consider $x^2$ (and so $A,B = \R$).
Let $C = [0,1]$.
Like before, $x^2$ is not $1$-$1$ and so the hypothesis fails.
Note that $f^{-1}(f(C)) = f^{-1}([0,1]) = [-1,1]$,
however $C = [0,1]$.
Thus $f^{-1}(f(C)) \neq C$, specifically $f^{-1}(f(C))\supset C$ as mentioned previously.
\end{proof}
\item \begin{proof}[Solution]\let\qed\relax
Consider $x^2$ (and so $A,B = \R$).
Let $D = [-1,1]$.
Note that $x^2$ is not onto $B = \R$ and so the hypothesis fails.
See that $f(f^{-1}(D)) = f(f^{-1}([-1,1])) = f([0,1]) = [0,1] \neq [-1,1] = D$,
where $f^{-1}([-1,1]) = [0,1]$, since the only elements in $A = \R$ that map to an element in $[-1,1]$
are in $[0,1]$ (specifically only mapping to $[0,1]$).
Thus $f(f^{-1}(D)) \neq D$, specifically $f(f^{-1}(D))\subset D$ as mentioned previously.
\end{proof}
\end{enumerate}
\clearpage

\subsection*{Problem 5}
{\it Prove that there is no $(a,b)$ in $\Z \times \Z$ for which $a^2 = 4b + 3$.
(Hint: Every integer $a$ must be either even or odd.)}
\begin{proof}[Solution]
For the sake of contradiction, assume there do exist such $(a,b)$.
Then either $a$ is even or $a$ is odd, since it is an integer.
First consider the case when $a$ is even.
Then $a^2$ is even (since an even number squared is also even:
an even number can be written as $2k$, $k \in \Z$,
and $(2k)^2 = 4k^2 = 2(2k^2)$,
which is also of the form of an even number,
since $2k^2 \in Z$).
Note that $4b$ is also always even,
so $4b + 3$ is always odd (the form of an odd number is $2k-1$,
$k \in \Z$, and $4b + 3 = 2(2b+2) - 1$).
But a number can't both be odd and even simultaneously,
and so $a^2 \neq 4b + 3$, a contradiction,
thus $a$ cannot be even.

Now consider when $a$ is odd.
Then we can write $a = 2k+1$, $k \in \Z$.
We can see $a^2 = (2k+1)^2 = 4k^2+4k+1 = 2(2k^2+2k) + 1$, so
\begin{align*}
	4b &= a^2 - 3\\
	&= 2(2k^2+2k) + 1 - 3\\
	&= 2(2k^2 + 2k - 1)\\
	2b &= 2(k^2 + k) - 1
\end{align*}
But the LHS of the equation is of the form of an even number,
and the RHS is of the form of an odd number (since $k^2+k\in\Z$),
and we cannot have an even number be equal to an odd number,
so we get a contradiction, thus $a$ cannot be odd.
This has exhausted all possiblities of $a$,
thus there cannot exist any such $a \in \Z$ that satisfies the equation,
and so there is no $(a,b) \in \Z \times \Z$ either.
\end{proof}
\clearpage

\subsection*{Problem 6}
{\it Let $f \colon A \to B$ and $g \colon B \to C$ be given functions.
Use the symbol $g \circ f$ to denote the function from $A$ to $C$
defined by $(g \circ f)(x) = g\left(f(x)\right)$ for all $x \in A$. Prove:
\begin{enumerate}
	\item If $f$ and $g$ are one-to-one, then $g \circ f$ is one-to-one.
	\item If $g \circ f$ is one-to-one, then $f$ is one-to-one.
	\item If $f$ is onto and $g \circ f$ is one-to-one, then $g$ is one-to-one.
	\item It can happen that $g \circ f$ is one-to-one, but $g$ is not.
	(To ``prove" this, simply provide a specific example with the indicated properties.)
\end{enumerate}}
\begin{enumerate}
\item \begin{proof}[Solution]\let\qed\relax
Let $x_1,x_2 \in A$ and $(g\circ f)(x_1) = (g \circ f)(x_2)$.
Recall the property that for a $1$-$1$ function $f$, if $f(x) = f(x')$, then $x=x'$.
Since $g$ is $1$-$1$, this means $f(x_1) = f(x_2)$.
Then since $f$ is $1$-$1$, this means $x_1 = x_2$,
which is enough to show that $(g \circ f)(x)$ is $1$-$1$.
\end{proof}
\item \begin{proof}[Solution]\let\qed\relax
Let $x_1,x_2 \in A$ and $f(x_1) = f(x_2)$.
Recall the property that for a $1$-$1$ function $f$, if $f(x) = f(x')$, then $x=x'$.
Trivially, $g(f(x_1)) = g(f(x_2))$ because $g$ is a function.
Since $g \circ f$ is $1$-$1$, this means $x_1 = x_2$,
which is enough to show that $f$ is $1$-$1$.
\end{proof}
\item \begin{proof}[Solution]\let\qed\relax
Let $y_1,y_2 \in B$ and $g(y_1) = g(y_2)$.
Since $f$ is onto, we know that there exists $x_1,x_2\in A$
such that $f(x_1) = y_1$ and $f(x_2) = y_2$.
So we have $g(f(x_1)) = g(f(x_2))$,
but $g \circ f$ is $1$-$1$,
so we know that $x_1 = x_2$.
But if $x_1 = x_2$, then $f(x_1) = f(x_2)$ since $f$ is a function,
and so $y_1 = f(x_1) = f(x_2) = y_2$,
which is enough to show that $g$ is $1$-$1$.
\end{proof}
\item \begin{proof}[Solution]\let\qed\relax
Let $A,B,C = \R$, and $f(x) = e^x$ and $g(y) = y^2$.
Obviously $g$ is not $1$-$1$,
since it maps multiple elements in the domain
to the same element in the codomain (eg. $f(-1)=f(1)$ but $-1\neq 1$).
But see that $g \circ f$ is $1$-$1$:
since the range of $e^x$ is only $(0,\infty)$,
and $x^2 \colon (0,\infty) \to (0,\infty)$ is one-to-one,
then by part (a) of this problem,
$g \circ f$ is $1$-$1$
(apply the statement of 6(a) with $f=e^x$, $g=y^2$, $A=\R$, $B=(0,\infty)$, and $C=(0,\infty)$;
our definition for $B,C$ might be different than the rest of the problem,
but this is fine, since in the context of $g \circ f$,
$g$ is only taking $(0,\infty)$ as an input).
Thus $g \circ f$ is $1$-$1$, but $g$ is not.
\end{proof}
\end{enumerate}
\clearpage

\subsection*{Problem 7}
{\it \begin{enumerate}
	\item Suppose $f \colon X \to X$ is a function,
	and define $g = f \circ f$.
	Prove: If $g(x) = x$ for all $x \in X$, then $f$ is one-to-one and onto.
	\item Extend the result in (a) to the function $g = f\circ f \circ \cdots \circ f$
	defined by composing $f$ with itself $n$ times.
	Show that the result is valid for each $n \in \N$.
\end{enumerate}}
\begin{enumerate}
\item \begin{proof}[Solution]\let\qed\relax
We start by showing $f$ is $1$-to-$1$.
Let $x_1,x_2\in X$ and $f(x_1) = f(x_2)$.
Since $f$ is a function, we have then $f(f(x_1)) = f(f(x_2))$.
But we know $(f \circ f)(x) = x$, so
$x_1 = x_2$, which shows $f$ is $1$-$1$.
Now we show $f$ is onto.
Let $x \in X$.
Then $f(f(x)) = x$.
But then $f(x) \in X$ maps to $x$ under $f$,
so there is always an element in $X$ (namely $f(x)$) that maps to any $x\in X$,
which shows that $f$ is onto.
\end{proof}
\item \begin{proof}[Solution]\let\qed\relax
Let $g = \underbrace{f \circ f \circ \cdots \circ f}_{n \text{ times}} = \underbrace{f \circ \cdots \circ f}_{n-1 \text{ times}}\circ f$
where $n \in \N$ is arbitrary, so that $g(x) = x$.
Note that $g(x)$ is $1$-$1$:
Let $x_1, x_2 \in X$ and $g(x_1) = g(x_2)$.
But then we have $g(x_1) = x_1 = x_2 = g(x_2)$ as well;
thus $g$ is $1$-$1$.
But problem 6(b) implies then that $f$ is $1$-$1$ as well,
since $g = \underbrace{f \circ \cdots \circ f}_{n-1 \text{ times}}\circ f$ is $1$-$1$.
We now show that $f$ is onto.
Let $x \in X$.
Then $\underbrace{(f \circ \cdots \circ f)}_{n \text{ times}}(x) = x$.
But then $\underbrace{(f \circ \cdots \circ f)}_{n \text{ times}}(x)$
maps to $x$ under $f$,
so there is always an element in $X$ (namely $\underbrace{(f \circ \cdots \circ f)}_{n \text{ times}}(x)$)
that maps to any $x \in X$, which shows that $f$ is onto.
Note that in all of this, $n$ was an arbitrary element in $\N$,
and so $f$ is $1$-$1$ and onto for all $n \in \N$.
\end{proof}
\end{enumerate}
\clearpage

\subsection*{Problem 8}
{\it Let $f \colon A \to B$ and let $C \subseteq A$.
\begin{enumerate}
	\item Proof or counterexample: $f(A\setminus C) \subseteq f(A) \setminus f(C)$.
	\item Proof or counterexample: $f(A\setminus C) \supseteq f(A) \setminus f(C)$.
	\item What condition on $f$ will gaurantee $f(A\setminus C) = f(A) \setminus f(C)$?
	(Choose between ``$f$ is $1$-$1$" and ``$f$ is onto":
	prove that your answer is correct.)
\end{enumerate}}
\begin{enumerate}
\item \begin{proof}[Solution]\let\qed\relax
We provide a counterexample.
Consider $f = x^2$, $A, B = \R$, and $C = [0,1]$.
Then, $f(A \setminus C) = f((-\infty,0) \cup (1,\infty))
= (0, \infty)$.
On the other hand, $f(A) \setminus f(C) = f(\R) \setminus f([0,1])
= [0,\infty) \setminus [0,1] = (1,\infty)$.
But then some elements in $f(A\setminus C)$ are not in $f(A) \setminus f(C)$
(consider $0.5$) and so $f(A\setminus C) \not\subseteq f(A) \setminus f(C)$.
\end{proof}
\item \begin{proof}[Solution]\let\qed\relax
This is true, we provide a proof.
Let $y \in f(A) \setminus f(C)$.
Then $y \in f(A)$, but $y \not\in f(C)$.
So there exists some element, call it $x$, such that $x \in A$
but $x\not\in C$,
and $f(x) = y$.
We have then that $x \in A \setminus C$.
Thus $f(x) = y \in f(A\setminus C)$
(since the $\in$ operation is preserved when applying functions).
Since $y$ was an arbitrary element in $f(A) \setminus f(C)$,
we have $f(A \setminus C) \supseteq f(A) \setminus f(C)$.
\end{proof}
\item \begin{proof}[Solution]\let\qed\relax
We give the condition that $f$ is $1$-$1$.
Note that we already have $f(A \setminus C) \supseteq f(A) \setminus f(C)$ from 8(b),
so we need only to show that $f(A \setminus C) \subseteq f(A) \setminus f(C)$.
Let $y \in f(A \setminus C)$.
Then there exists some unique $x \in A \setminus C$ such that $f(x) = y$.
$x\in A$ but $x \not\in C$.
So $f(x) = y \in A$.
Since $f$ is $1$-$1$, there is no $x' \in C$ such that $f(x) = f(x')$, since otherwise $x = x'$ but $x \not\in C$.
Thus $f(x) = y \not\in C$.
But then $y \in f(A) \setminus f(C)$.
Since $y$ was an arbitrary element in $f(A\setminus C)$,
we have that $f(A \setminus C) \subseteq f(A) \setminus f(C)$.
Therefore, we get $f(A \setminus C) = f(A) \setminus f(C)$ when $f$ is $1$-$1$.
\end{proof}
\end{enumerate}
\end{document}
