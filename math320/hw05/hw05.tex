\documentclass{article}
\usepackage{amsmath, amsfonts, amsthm, amssymb}
\usepackage{geometry}
\geometry{letterpaper, margin=2.0cm, includefoot, footskip=30pt}

\usepackage{fancyhdr}
\pagestyle{fancy}

\lhead{Math 320}
\chead{Homework 5}
\rhead{Nicholas Rees, 11848363}
\cfoot{Page \thepage}

\newtheorem*{problem}{Problem}

\newcommand{\N}{{\mathbb N}}
\newcommand{\Z}{{\mathbb Z}}
\newcommand{\Q}{{\mathbb Q}}
\newcommand{\R}{{\mathbb R}}
\newcommand{\C}{{\mathbb C}}
\newcommand{\ep}{{\varepsilon}}

\renewcommand{\theenumi}{(\alph{enumi})}

\begin{document}
\subsection*{Problem 1}
{\it Consider a real-valued sequence $(x_n)$ and a real number $\hat{x}$.
	Prove that the following are equivalent:
\begin{enumerate}
	\item $x_n \to \hat{x}$,
	\item $\forall \ep > 0, \exists N \in \N \colon
		\forall n \geq 23N, |x_n - \hat{x}| < 20\ep$.
\end{enumerate}}

\begin{proof}[Solution]\let\qed\relax
	Trivial, basically just set $\ep = \ep'/20$ or something.
	Will probably be easier to prove the two directions seperately.
	ff
\end{proof}
\clearpage
~\clearpage

\subsection*{Problem 2}
{\it Extend our collection of equivalent formulations of
the completeness property for $\R$ by proving
that the following are equivalent (TFAE).
Proceed directly, without relying on the completeness property in one of its other forms.
(So, for example, do not assume existence of suprema and infima.)
\begin{enumerate}
	\item For any sequence of nonempty closed real intervals
		$I_1 [a_1,b_1], I_2=[a_2,b_2], \dots$,
		such that $I_1 \supseteq I_2 \supseteq I_3 \supseteq \cdots$
		(such intervals are called ``nested"), one has
		\[
			\bigcap_{n\in\N} I_n \neq \emptyset
		\]
	\item Every bounded monotonic sequence in $\R$ converges.
		(Recall Rudin's Definition 3.13.)
\end{enumerate}
(\emph{Note}: The inverval notation $[a,b] = \{t \in \R\colon a \leq t \leq b\}$
is reserved for the case where both $a$ and $b$ are real numbers.
To encode $\{t \in \R \colon t \geq 0\}$, for example,
we would write $[0,+\infty)$, not $[0,+\infty]$.)}

\begin{proof}[Solution]\let\qed\relax
	Backwards direction:
	$a_n$ and $b_n$ are bounded monotone sequences
	($a_n$ and $b_n$ are reals,
	so bounded above/below,
	and bounded below/above by each other).
	Either $a_n,b_n$ converge to the same number,
	then nonempty,
	or some distance apart, so also nonempty.
	ff
\end{proof}
\clearpage
~\clearpage

\noindent{\it Note: Questions 3-6 contribute to the major project of constructing $\R$ from $\Q$.
Therefore they must be completed entirely in the context of the ratoinal numbers.
Present solutions that make no reference at all to the completeness property of $\R$,
in any of its equivalent forms.}

\subsection*{Problem 3}
{\it Introduce the following notation:\newline
$CS(\Q)\colon$ the set of all Cauchy sequences with rational elements.\newline
$x,y,z\colon$ typical symbols for elements of $CS(\Q)$.
Thus, e.g., $x = (x_1,x_2,\dots)$.\newline
$R[x]\colon$ the subset of $CS(\Q)$ associated with a given $x \in CS(\Q)$ as follows:
\[
	R[x] = \left\{x' \in CS(\Q) \colon \lim_{n\to\infty}\lvert x'_n - x_n \rvert = 0\right\}.
\]
$\Phi \colon$ the function that takes each rational number $q$ into
the subset of $CS(\Q)$ containing the corresponding constant sequence, i.e.,
\[
	\Phi(q) = R[(q,q,\dots)] \quad \forall q \in \Q
\]
\begin{enumerate}
	\item Prove: $R[x] \neq \emptyset$ for every $x \in CS(\Q)$.
	\item Prove: For any $x,y \in CS(\Q)$, $R[x] = R[y] \iff R[x]\cap R[y] \neq \emptyset$
\end{enumerate}}

\begin{enumerate}
	\item \begin{proof}[Solution]\let\qed\relax
		Define $x'$ where $x'_n = \begin{cases}
			x_n+1 &\text{if }n=1\\
			x_n &\text{if }n>1
		\end{cases}$.
		Then for any $n > 1$, $|x'_n - x_n| = 0$,
		so $\lim_{n\to\infty}|x'_n - x_n| = 0$,
		thus $x' \in R[x]$.
		And so $R[x] \neq \emptyset$.
	\end{proof}
	\item \begin{proof}[Solution]\let\qed\relax
		Something about converging to the same value.ff
	\end{proof}
\end{enumerate}
\clearpage

\subsection*{Problem 4}
{\it Continue with the notation from Question 3.
	We would like to define a relatoin denoted ``$<$" on $\Q^*$ as follows:
	\[
		R[x] < R[y] \iff \exists r > 0 (r \in \Q), \exists N \in \N
		\colon \forall n > N, y_n - x_n > r
	\]
	This relation looks like one that is familiar for rational numbers,
	but here it compares two \emph{sets}.
	Each of the properties we take for granted when manipulating inequalities
	relating numbeers requires careful thinking in this new context.
	Prove the following.
\begin{enumerate}
	\item Whenever $R[x'] = R[x]$ and $R[y'] = R[y]$
		for some given $x,x',y,y' \in CS(\Q)$,
		the definition proposed above gaurantees that
		\[
			R[x'] < R[y'] \iff R[x] < R[y].
		\]
		(that is, the proposed definition is unambiguous.
		Or, more conventionally,
		``the relation $<$ is well-defined".)
	\item If $x,y,z \in CS(\Q)$ obey $R[x] < R[y]$ and $R[y] < R[z]$,
		then $R[x] < R[z]$.
	\item The inequality $R[x] < R[x]$ never happens, for any $x \in CS(\Q)$.
	\item For any $p,q \in \Q$, we have $p<q \iff \Phi(p) < \Phi(q)$.
\end{enumerate}}

\begin{enumerate}
	\item \begin{proof}[Solution]\let\qed\relax
		ff
	\end{proof}
	\item \begin{proof}[Solution]\let\qed\relax
		ff
	\end{proof}
	\item \begin{proof}[Solution]\let\qed\relax
		ff
	\end{proof}
	\item \begin{proof}[Solution]\let\qed\relax
		ff
	\end{proof}
\end{enumerate}
\clearpage

\subsection*{Problem 5}
{\it Continue with the notation from Questions 3 and 4. Prove the following:
\begin{enumerate}
	\item For any $x \in CS(\Q)$, exactly one of the following holds:
		\[
			R[x] < \Phi(0), \qquad R[x] = \Phi(0), \qquad \Phi(0) < R[x]
		\]
	\item For each $x$ in $CS(\Q)$, there exist $q,r\in\Q$
		such that $\Phi(q) < R[x] < \Phi(r)$.
	\item For any $x,y \in CS(\Q)$ with $R[x] < R[y]$,
		there exists $q \in \Q$ such that $R[x] < \Phi(q) < R[y]$.
\end{enumerate}}

\begin{enumerate}
	\item \begin{proof}[Solution]\let\qed\relax
		ff
	\end{proof}
	\item \begin{proof}[Solution]\let\qed\relax
		ff
	\end{proof}
	\item \begin{proof}[Solution]\let\qed\relax
		ff
	\end{proof} 
\end{enumerate}
\clearpage

\subsection*{Problem 6}
{\it Continue with the notation from Questions 3 and 4. Prove the following:
\begin{center}
	If $x \in CS(\Q)$ has $R[x] \neq \Phi(0)$,
	then there exists $z \in CS(\Q)$ for which $R[x\cdot z] = \Phi(1)$.
\end{center}
Here $x \cdot z$ denotes the sequence whose $n$th term is $x_nz_n$.
(Recall from Assignment 4, Question 6,
that $x\cdot z \in CS(\Q)$ whenever $x,z \in CS(\Q)$.)}

\begin{proof}[Solution]\let\qed\relax
	ff
\end{proof}
\clearpage
~\clearpage

\subsection*{Problem 7}
{\it [Rudin problem 3.5]
For any two real sequences $(a_n)$ and $(b_n)$, prove that the inequality
\[
	\limsup_{n\to\infty}(a_n + b_n) \leq \limsup_{n\to\infty}a_n + \limsup_{n\to\infty}b_n
\]
holds whenever the right side is not of the form $(+\infty) + (+\infty)$.
Give a specific example to show that inequality may hold.}

\begin{proof}[Solution]\let\qed\relax
	ff
\end{proof}
\clearpage
~\clearpage

\subsection*{Problem 8}
{\it Let $X$ denote the collection of all functions $f \colon [0,1] \to \R$
for which the set of real numbers $f([0,1]) = \{f(x) \colon x \in [0,1]\}$ is bounded.
For each $f \in X$, define
\[
	\lVert f \rVert = \sup\{|f(x)| \colon x \in [0,1]\}
\]
Prove that for all rael $c$ and all $f,g,h \in X$,
\begin{enumerate}
	\item $\lVert cf \rVert = |c|\lVert f \rVert$,
	\item $\lVert f + g \rVert \leq \lVert f \rVert + \lVert g \rVert$,
	\item $\lVert f - h \rVert - \lVert g - h\rVert \leq \lVert f - g\rVert$.
\end{enumerate}
Give an example where (b) holds as a strict inequality (``$<$").}

\begin{enumerate}
	\item \begin{proof}[Solution]\let\qed\relax
		ff
	\end{proof}
	\item \begin{proof}[Solution]\let\qed\relax
		ff
	\end{proof}
	\item \begin{proof}[Solution]\let\qed\relax
		ff
	\end{proof} 
\end{enumerate}

\end{document}
