\documentclass{article}
\usepackage{amsmath, amsfonts, amsthm, amssymb}
\usepackage{geometry}
\geometry{letterpaper, margin=2.0cm, includefoot, footskip=30pt}

\usepackage{fancyhdr}
\pagestyle{fancy}

\lhead{Math 320}
\chead{Homework 5}
\rhead{Nicholas Rees, 11848363}
\cfoot{Page \thepage}

\newtheorem*{problem}{Problem}

\newcommand{\N}{{\mathbb N}}
\newcommand{\Z}{{\mathbb Z}}
\newcommand{\Q}{{\mathbb Q}}
\newcommand{\R}{{\mathbb R}}
\newcommand{\C}{{\mathbb C}}
\newcommand{\ep}{{\varepsilon}}
\newcommand{\SR}{{\mathcal R}}

\renewcommand{\theenumi}{(\alph{enumi})}

\begin{document}
\subsection*{Problem 1}
{\it Consider a real-valued sequence $(x_n)$ and a real number $\hat{x}$.
	Prove that the following are equivalent:
\begin{enumerate}
	\item $x_n \to \hat{x}$,
	\item $\forall \ep > 0, \exists N \in \N \colon
		\forall n \geq 23N, |x_n - \hat{x}| < 20\ep$.
\end{enumerate}}

\begin{proof}[Solution]\let\qed\relax
	Assume (a).
	For any $\ep > 0$,
	since $x_n$ converges to $\hat{x}$ (and $20\ep>0$),
	we have that there exists some $N \in \N$
	such that for all $n > N$,
	$|x_n - \hat{x}| < 20 \ep$.
	But since if $n \geq 23N$, it is certainly true that $n > N$,
	thus $|x_n - \hat{x}| < 20\ep$ for all $n \geq 23N$,
	which is (b).

	Now assume (b).
	For any $\ep > 0$,
	let $\ep' = \frac{\ep}{20}$.
	We know that there exists $N' \in \N$
	such that for all $n \geq 23N'$,
	we have that $|x_n - \hat{x}| < 20\ep' = \ep$.
	But since $24N' \in \N$,
	let us fix $N = 24N'$,
	then for all $n \geq N > 23N'$, we have
	$|x_n - \hat{x}| < \ep$ again,
	which, by definition, means $x_n \to \hat{x}$.
\end{proof}
\clearpage
~\clearpage

\subsection*{Problem 2}
{\it Extend our collection of equivalent formulations of
the completeness property for $\R$ by proving
that the following are equivalent (TFAE).
Proceed directly, without relying on the completeness property in one of its other forms.
(So, for example, do not assume existence of suprema and infima.)
\begin{enumerate}
	\item For any sequence of nonempty closed real intervals
		$I_1 [a_1,b_1], I_2=[a_2,b_2], \dots$,
		such that $I_1 \supseteq I_2 \supseteq I_3 \supseteq \cdots$
		(such intervals are called ``nested"), one has
		\[
			\bigcap_{n\in\N} I_n \neq \emptyset
		\]
	\item Every bounded monotonic sequence in $\R$ converges.
		(Recall Rudin's Definition 3.13.)
\end{enumerate}
(\emph{Note}: The inverval notation $[a,b] = \{t \in \R\colon a \leq t \leq b\}$
is reserved for the case where both $a$ and $b$ are real numbers.
To encode $\{t \in \R \colon t \geq 0\}$, for example,
we would write $[0,+\infty)$, not $[0,+\infty]$.)}

\begin{proof}[Solution]\let\qed\relax
	Backwards direction:
	$a_n$ and $b_n$ are bounded monotone sequences
	($a_n$ and $b_n$ are reals,
	so bounded above/below,
	and bounded below/above by each other).
	Either $a_n,b_n$ converge to the same number,
	then nonempty,
	or some distance apart, so also nonempty.

	Now assume (b).
	Consider the sequence of nonempty closed real intervals $I_n$ from the problem statement.
	Note that $a_n$ and $b_n$ form monotonic sequences.
	Since in order for $I_n \supseteq I_{n+1}$ to be true,
	we must have that $a_{n} \geq a_{n+1}$ and $b_{n} \leq b_{n+1}$.
	Thus, $a_n$ is monotonically decreasing and $b_n$ is monotonically decreasing.
	Note that the sequence $(a_n)$ must be bounded.
	We know that any $a_n \leq a_1$ since its monotone decreasing,
	so its bounded above,
	and $a_n \geq b_1$,
	otherwise $b_1 > a_n \geq b_n$,
	which contradicts our assumption that $b_n$ monotonically increases,
	so $(a_n)$ is bounded below.
	Likewise, $(b_n)$ is bounded above and below with the same argument,
	(just swap the $a$'s and $b$'s).

	By (b), we then know that $a_n$ and $b_n$ both converge to some value,
	call them $\hat{a}$ and $\hat{b}$.
	Note that $\hat{a} \leq a_n$ for all $n \in \N$,
	since $(a_n)$ is monotonically decreasing:
	if there existed some $N$ where $\hat{a} > a_N$,
	then for all $n \geq N$, $\hat{a} > a_N \geq a_n$
	since the sequence is monotonically decreasing,
	which violates convergence to $\hat{a}$
	(let $\ep = |\hat{a} - a_N|$).
	Furthermore, $\hat{a} \geq b_n$.
	Otherwise, if $b_{N_1} > \hat{a}$,
	using the definition of convergence, setting $\ep = b_{N_1} - \hat{a} > 0$
	gaurantees that there exists $a_{N_2}$ such that $|\hat{a} - a_{N_2}| < b_{N_1}$.
	Then since $(b_n)$ is monotonically increasing and $(a_n)$ is monotonically decreasing,
	we have $N = \max\{N_1,N_2\}$ so
	\[
		\hat{a} - a_N \leq \hat{a} - a_{N_2} < b_{N_1} \leq b_N - \hat{a}
	\]
	(we can get rid of the absolute value bars,
	since $a_n < \hat{a}$ for all $n$).


	I'm gonna come back to this. I don't know what my goal is right now.
	Something to do with existence of $\hat{a}, \hat{b}$,
	because $\Q$ fails because these sequences could converge
	on something that is not $\Q$ (ie. $\sqrt{2}$).
	Right now, I don't know what it means to converge...
	like what I'm going for now seems like it should still work for $\Q$.

	Wait if I just get $a_n \geq \hat{a} \geq b_n$, I'm done,
	because $\hat{a}$ is in every set,
	so intersection is nonempty.
\end{proof}
\clearpage
~\clearpage

\noindent{\it Note: Questions 3-6 contribute to the major project of constructing $\R$ from $\Q$.
Therefore they must be completed entirely in the context of the ratoinal numbers.
Present solutions that make no reference at all to the completeness property of $\R$,
in any of its equivalent forms.}

\subsection*{Problem 3}
{\it Introduce the following notation:\newline
$CS(\Q)\colon$ the set of all Cauchy sequences with rational elements.\newline
$x,y,z\colon$ typical symbols for elements of $CS(\Q)$.
Thus, e.g., $x = (x_1,x_2,\dots)$.\newline
$R[x]\colon$ the subset of $CS(\Q)$ associated with a given $x \in CS(\Q)$ as follows:
\[
	R[x] = \left\{x' \in CS(\Q) \colon \lim_{n\to\infty}\lvert x'_n - x_n \rvert = 0\right\}.
\]
$\Phi \colon$ the function that takes each rational number $q$ into
the subset of $CS(\Q)$ containing the corresponding constant sequence, i.e.,
\[
	\Phi(q) = R[(q,q,\dots)] \quad \forall q \in \Q
\]
\begin{enumerate}
	\item Prove: $R[x] \neq \emptyset$ for every $x \in CS(\Q)$.
	\item Prove: For any $x,y \in CS(\Q)$, $R[x] = R[y] \iff R[x]\cap R[y] \neq \emptyset$
\end{enumerate}}

\begin{enumerate}
	\item \begin{proof}[Solution]\let\qed\relax
		Define $x'$ where $x'_n = x_n$.
		Then for any $n \in \N$, $|x'_n - x_n| = 0$,
		so $\lim_{n\to\infty}|x'_n - x_n| = 0$,
		thus $x' \in R[x]$.
		And so $R[x] \neq \emptyset$.
	\end{proof}
	\item \begin{proof}[Solution]\let\qed\relax
		Assume $\SR[x] = \SR[y]$.
		By part (a) of this problem, these are nonempty sets.
		Thus, there exists an element in $\SR[x]$ which must also be in
		$\SR[y]$ by equality.
		But then $\SR[x] \cap \SR[y] \neq \emptyset$.

		Now assume $\SR[x] \cap \SR[y] \neq \emptyset$.
		Thus, there exists $z \in CS(\Q)$ such that
		$z \in \SR[x]$ and $z \in \SR[y]$.
		Since both $\SR[x]$ and $\SR[y]$ are nonempty by part (a),
		let $x' \in \SR[x]$ and $y'\in\SR[y]$ be arbitrary elements in
		their respective sets.
		We first prove that $\SR[x] \subseteq \SR[y]$
		by showing $x' \in \SR[y]$ since it is arbitrary.
		We will show first that for all $\ep > 0$,
		there exists $N \in \N$ such that for all $n \geq N$, $|z_n - x'_n| < \ep$.
		Let $\ep > 0$ be arbitrary.
		By definition of $z,x' \in \SR[x]$,
		we have for $\ep_1 = \frac{\ep}{2}$,
		there exists $N_1 \in \N$ such that for all
		$n \geq N_1$, $|x'_n - x_n| < \ep_1$,
		and for $\ep_2 = \frac{\ep}{2}$,
		there exists $N_2$ such that for all
		$n \geq N_2$, $|z_n - x_n| < \ep_2$.
		Let $N = \max\{N_1,N_2\} \in \N$.
		Then we know for all $n \geq N$
		\[
			|x_n' - z_n| \leq |x'_n - x_n| + |x_n - z_n| \leq \ep_1 + \ep_2 = \ep
		\]
		(where we have used the triangle inequality).
		Now, we will show first that for all $\ep > 0$,
		there exists $N \in \N$ such that for all $n \geq N$, $|x'_n - y_n| < \ep$,
		thus $x'_n \in \SR[y] \implies \SR[x] \subseteq \SR[y]$
		(since $x'_n$ was an arbitrary element of $\SR[x]$).
		Let $\ep > 0$ be arbitrary.
		By definition of $z \in \SR[y]$,
		we have for $\ep_1 = \frac{\ep}{2}$,
		there exists $N_1 \in \N$ such that for all
		$n \geq N_1$, $|z_n - y_n| < \ep_1$,
		and using the fact we just proved,
		for $\ep_2 = \frac{\ep}{2}$,
		there exists $N_2$ such that for all
		$n \geq N_2$, $|z_n - x'_n| < \ep_2$.
		Let $N = \max\{N_1,N_2\} \in \N$.
		Then we know for all $n \geq N$
		\[
			|x_n' - y_n| \leq |x'_n - z_n| + |z_n - y_n| \leq \ep_1 + \ep_2 = \ep
		\]
		Thus $\SR[x] \subseteq \SR[y]$.

		Finally, note that the argument to show that
		$y'_n \in \SR[x]$ is identical,
		and we one would just need to swap the $x$'s and $y$'s,
		thus $\SR[y] \subseteq \SR[x]$ as well
		(since $y'_n$ was an arbitrary element of $\SR[y]$).
		Therefore, $\SR[x] = \SR[y]$.
		This shows both directions of the implication.
	\end{proof}
\end{enumerate}
\clearpage
~\clearpage

\subsection*{Problem 4}
{\it Continue with the notation from Question 3.
	We would like to define a relatoin denoted ``$<$" on $\Q^*$ as follows:
	\[
		R[x] < R[y] \iff \exists r > 0 (r \in \Q), \exists N \in \N
		\colon \forall n > N, y_n - x_n > r
	\]
	This relation looks like one that is familiar for rational numbers,
	but here it compares two \emph{sets}.
	Each of the properties we take for granted when manipulating inequalities
	relating numbers requires careful thinking in this new context.
	Prove the following.
\begin{enumerate}
	\item Whenever $R[x'] = R[x]$ and $R[y'] = R[y]$
		for some given $x,x',y,y' \in CS(\Q)$,
		the definition proposed above gaurantees that
		\[
			R[x'] < R[y'] \iff R[x] < R[y].
		\]
		(that is, the proposed definition is unambiguous.
		Or, more conventionally,
		``the relation $<$ is well-defined".)
	\item If $x,y,z \in CS(\Q)$ obey $R[x] < R[y]$ and $R[y] < R[z]$,
		then $R[x] < R[z]$.
	\item The inequality $R[x] < R[x]$ never happens, for any $x \in CS(\Q)$.
	\item For any $p,q \in \Q$, we have $p<q \iff \Phi(p) < \Phi(q)$.
\end{enumerate}}

\begin{enumerate}
	\item \begin{proof}[Solution]\let\qed\relax
		Assume $\SR[x'] < \SR[y']$.
		Let $r$ be given such there exists $N_1 \in \N$ where for
		all $n \geq N_1$, $y'_n - x'_n > r$.
		Since $\SR[x'] = \SR[x]$,
		if we let $\ep = \frac{r}{2}$, we know there exists $N_2 \in \N$
		such that for all $n \geq N_2$, $-\frac{r}{2} < x'_n - x_n < \frac{r}{2}
		\implies x_n - \frac{r}{2} < x'_n$.
		Thus, if $N' = \max\{N_1,N_2\}\in\N$, for all $n \geq N'$,
		\[
			y'_n - x_n + \frac{r}{2} > y'_n - x'_n > r \implies
			y'_n - x_n > \frac{r}{2}
		\]
		.
		Since $\SR[y'] = \SR[y]$,
		if we let $\ep = \frac{r}{4}$, we know there exists $N_3 \in \N$
		such that for all $n \geq N_3$, $-\frac{r}{4} < y'_n - y_n < \frac{r}{4}
		\implies y'_n < y_n + \frac{r}{4}$.
		Thus, if $N = \max\{N_3,N'\}\in\N$, for all $n \geq N$,
		\[
			y_n + \frac{r}{4} - x_n > y'_n - x_n > \frac{r}{2} \implies
		y_n - x_n > \frac{r}{4}
		\]
		But if $0 < r \in \Q$, certainly $0 < \frac{r}{4} \in \Q$,
		thus $\SR[x] < \SR[y]$.

		Now assume $\SR[x] < \SR[y]$.
		Note that we can do an identical proof as before,
		just swapping the $x$'s and $x'$'s and the $y$'s with $y'$'s,
		to show that this implies $\SR[x'] < \SR[y']$.
		Thus, we have shown both directions of the implication.
	\end{proof}
	\item \begin{proof}[Solution]\let\qed\relax
		Since $\SR[x] < \SR[y]$,
		there exists $0 < r_1 \in \Q$ where there exists $N_1 \in \N$
		such that for all $n\geq N_1$,
		$y_n - x_n > r_1$.
		Secondly, since $\SR[y] < \SR[z]$,
		there exists $0 < r_2 \in \Q$ where there exists $N_2 \in \N$
		such that for all $n\geq N_2$,
		$z_n - y_n > r_2 \implies z_n - r_2 > y_n$.
		Now let $N = \max\{N_1,N_2\}$,
		then for all $n \geq N$,
		\[
			r_1 < y_n - x_n < z_n - r_2 - x_n \implies
			r_1 + r_2 < z_n - x_n
		\]
		But if $0 < r_1,r_2 \in \Q$, certainly $0 < r_1 + r_2 \in \Q$,
		thus $\SR[x] < \SR[z]$.
	\end{proof}
	\item \begin{proof}[Solution]\let\qed\relax
		Note that for all $N \in \N$,
		we have that if $n \geq N$,
		$x_n - x_n = 0$.
		Thus, for any $r > 0$, $x_n - x_n < r$.
		Thus, we cannot have $\SR[x] < \SR[x]$.
	\end{proof}
	\item \begin{proof}[Solution]\let\qed\relax
		Define $p_n = p$ and $q_n = q$ for all $n \in \N$.
		Thus $\Phi(p) = \SR[(p_n)]$, $\Phi(q) = \SR[(q_n)]$ with this notation.
		Let $p < q$.
		This is true if and only if $0 < q - p = 2r$
		(and note $2r \in \Q$ since the rationals are closed under addition).
		And this is true if and only if
		$q_n - p_n = 2r > 0$ for all $n \in \N$ (when the sequences are as defined above).
		Since $0 < 2r \in \Q \implies 0< r \in \Q$,
		our inequality is true if and only if
		$q_n - p_n > r$ for all $n \in \N$.
		But then, this is our definition of $\SR[(p_n)] < \SR[(q_n)]$
		(where $N = 1$),
		or equivalently,
		$\Phi(p) < \Phi(q)$.
		All of our statements imply both directions,
		so we have proven both directions of the implication.
	\end{proof}
\end{enumerate}
\clearpage
~\clearpage

\subsection*{Problem 5}
{\it Continue with the notation from Questions 3 and 4. Prove the following:
\begin{enumerate}
	\item For any $x \in CS(\Q)$, exactly one of the following holds:
		\[
			R[x] < \Phi(0), \qquad R[x] = \Phi(0), \qquad \Phi(0) < R[x]
		\]
	\item For each $x$ in $CS(\Q)$, there exist $q,r\in\Q$
		such that $\Phi(q) < R[x] < \Phi(r)$.
	\item For any $x,y \in CS(\Q)$ with $R[x] < R[y]$,
		there exists $q \in \Q$ such that $R[x] < \Phi(q) < R[y]$.
\end{enumerate}}

\begin{enumerate}
	\item \begin{proof}[Solution]\let\qed\relax
		We first show that at most,
		only one of the three statements can be true.
		First, if $\SR[x] = \Phi(0)$,
		then by part (a) and part (c) of Problem 4,
		we have that $\SR[x] < \Phi(0)$ and $\Phi(0) < \SR[x]$
		do not occur.
		Thus, we can only have $\SR[x] = \Phi(0)$.
		Now let $\SR[x] < \Phi(0)$.
		This means there exists some $r_1>0$ such that for some $N_1 \in \N$,
		$-x_n > r_1$ for all $n \geq N_1$.
		For the sake of contradiction, assume we have $\SR[x] > \Phi(0)$
		as well.
		Then there exists some $r_2>0$ such that for some $N_2 \in \N$,
		$x_n > r_2$ for all $n \geq N_2$.
		Taking $r = \min\{r_1,r_2\}$ and $N = \max\{N_1,N_2\}$,
		we have $-x_n > r$ and $x_n > r$ for all $n \geq N$,
		but these imply $x_n > r > -r > x_n$ which is a contradiction,
		by the trichotomy of order for rational numbers.
		Thus at most one of these is true.

		Now we show that they cannot all be false, i.e. at least one is true.
		First, assume $\SR[x] \not< \Phi(0)$ and $\SR[x] \not> \Phi(0)$.
		Let $\ep > 0$ be arbitrary.
		Then, for all $r > 0 (r \in \Q)$ and for all $N \in \N$,
		there exists $N_1 \geq N$ such that $x_{N_1} \leq r + \frac{\ep}{3}$,
		and $N_2 \geq N$ such that $-x_{N_2} \leq r + \frac{\ep}{3}$.
		Let $r = \frac{\ep}{3}$ and $N = 1$ then.
		Recall that $x$ is Cauchy,
		so for all $m,n \geq N_1$, $|x_m-x_n| < r$.
		So $x_m < r + x_{N_1} \leq 2r + \frac{\ep}{3} = \ep$.
		Furthermore, for all $m,n \geq N_2$, $|x_n - x_m| < r \implies
		-x_m < r - x_n \leq 2r + \frac{\ep}{3} = \ep \implies x_m > -\ep$.
		Thus if $M = \max\{N_1,N_2\}$, then for all $m \geq M$,
		$-\ep < x_m < \ep \implies |x_m| < \ep$,
		which is sufficent to show that $\lim_{m\to\infty}|x_m - 0| = 0$,
		thus $\SR[x] = \Phi(0)$.

		Now assume that $\SR[x] \not< \Phi(0)$ and $\SR[x] \neq \Phi(0)$.
		Then there exists $\ep' > 0$ such that for all $N \in \N$,
		there exists $n \geq N$ where $|x_n| > \ep$.
		Furthermore, for all $r \geq 0$ and all $N \in \N$,
		there exists $n \geq N$ such that $x_n \leq r$.
		
		Note that we could repeat an identical argument to show $\SR[x] < \Phi(0)$
		must be true, assuming $\SR[x] \neq \Phi(0)$ and $\SR[x] \not> \Phi(0)$
		by simply swapping all the signs on our $x_n$'s.
		Therefore, we have proven that at least one of our statements
		must be true, and at most one is true,
		thus exactly one of them hold.
	\end{proof}
	\item \begin{proof}[Solution]\let\qed\relax
		ff something to do with all Cauchy sequences are bounded
	\end{proof}
	\item \begin{proof}[Solution]\let\qed\relax
		From the definition of $\SR[x] < \SR[y]$,
		there exists $0 < r \in \Q$ and $N \in \N$
		such that for all $n \geq N$,
		$y_n - x_n > r$.
		ff


		Tao said to prove for any $\SR[x] > 0$, there exists $N \in \N$ where
		$\SR[x] > \frac{1}{N} > 0$,
		then argue by contradiction.
	\end{proof} 
\end{enumerate}
\clearpage

\subsection*{Problem 6}
{\it Continue with the notation from Questions 3 and 4. Prove the following:
\begin{center}
	If $x \in CS(\Q)$ has $R[x] \neq \Phi(0)$,
	then there exists $z \in CS(\Q)$ for which $R[x\cdot z] = \Phi(1)$.
\end{center}
Here $x \cdot z$ denotes the sequence whose $n$th term is $x_nz_n$.
(Recall from Assignment 4, Question 6,
that $x\cdot z \in CS(\Q)$ whenever $x,z \in CS(\Q)$.)}

\begin{proof}[Solution]\let\qed\relax
	If $\SR[x] \neq \Phi(0)$,
	then part (b) of Problem 3 says that $\SR[x] \cap \Phi(0) = \emptyset$.
	Then $(0,0,\dots) \in \Phi(0)$ so $(0,0,\dots) \not\in \SR[x]$.
	Then there exists some $\ep' > 0$ such that for all
	$N_1 \in \N$, if $n \geq N_1$ then $|x_n| > \ep'$.
	That is to say that, for all $n\geq N_1$, $x_n \neq 0$.
	Now define $z_n = x_n^{-1}$ for $n \geq N_1$ and $z_n = 2023$
	when $1 \leq n < N_1$.

	We claim that $z$ is Cauchy.
	Let $\ep > 0$.
	Since $x$ is Cauchy,
	we know that there exists $N_2$ such that for all $m,n \geq N_2$,
	$|x_m - x_n| \leq \ep\ep'^2$.
	Let $N = \max\{N_1,N_2\}$.
	Then for all $m,n > N$:
	\begin{align*}
		|z_m - z_n|
		&= \left\lvert\frac{1}{x_m} - \frac{1}{x_n}\right\rvert\\
		&= \left\lvert \frac{x_n - x_m}{x_nx_m} \right\rvert\\
		&= \frac{|x_n-x_m|}{|x_n||x_m|}\\
		&< \frac{\ep\ep'^2}{|x_n||x_m|}\\
		&< \frac{\ep\ep'^2}{\ep'^2}\\
		&= \ep
	\end{align*}
	thus $z$ is Cauchy as well.
	Therefore, we have that $x \cdot z$ is Cauchy from Assignment 4, Question 6 (b).

	We now claim that $(1,1,\dots) \in \SR[x\cdot z]$.
	Let $0 < \ep \in \Q$ be given.
	Then if $n \geq N$,
	we have that
	\[
		|1 - x_n\cdot z_n| = |1 - x_n\cdot x_n^{-1}| = |1-1| = 0 < \ep
	\]
	which satisifies the definition that $(1,1,\dots) \in \SR[x\cdot z]$.
	But since $(1,1,\dots) \in \Phi(1)$,
	$\Phi(1) \cap \SR[x \cdot z] \neq \emptyset$,
	and from part (b) of Problem 3, we have that $\SR[x\cdot z] = \Phi(1)$.
\end{proof}
\clearpage
~\clearpage

\subsection*{Problem 7}
{\it [Rudin problem 3.5]
For any two real sequences $(a_n)$ and $(b_n)$, prove that the inequality
\[
	\limsup_{n\to\infty}(a_n + b_n) \leq \limsup_{n\to\infty}a_n + \limsup_{n\to\infty}b_n
\]
holds whenever the right side is not of the form $(+\infty) + (-\infty)$.
Give a specific example to show that inequality may hold.}

\begin{proof}[Solution]\let\qed\relax
	I'm going to use the Rudin definition of $\limsup$
	(which we proved was equivalent to ours in class),
	which says that
	\[
		\limsup_{n \to \infty} s_n = 
	\]
	
	Rudin Theorem 3.6 (b): Every bounded sequence in $\R^k$ contains a convergent subsequence.

	Case work to get rid of divergent examples.

	Example: $a_n$ alternates $1,-1$, and $b_n$ is the other way (use inf sup definition)
\end{proof}
\clearpage
~\clearpage

\subsection*{Problem 8}
{\it Let $X$ denote the collection of all functions $f \colon [0,1] \to \R$
for which the set of real numbers $f([0,1]) = \{f(x) \colon x \in [0,1]\}$ is bounded.
For each $f \in X$, define
\[
	\lVert f \rVert = \sup\{|f(x)| \colon x \in [0,1]\}
\]
Prove that for all real $c$ and all $f,g,h \in X$,
\begin{enumerate}
	\item $\lVert cf \rVert = |c|\lVert f \rVert$,
	\item $\lVert f + g \rVert \leq \lVert f \rVert + \lVert g \rVert$,
	\item $\lVert f - h \rVert - \lVert g - h\rVert \leq \lVert f - g\rVert$.
\end{enumerate}
Give an example where (b) holds as a strict inequality (``$<$").}

\begin{enumerate}
	\item \begin{proof}[Solution]\let\qed\relax
		We first deal with the case when $c = 0$.
		Then
		\begin{align*}
			\lVert cf \rVert
			&= \sup\{|cf(x)| \colon x \in [0,1]\}\\
			&= \sup\{0 \colon x \in [0,1]\}\\
			&= 0\\
		\end{align*}
		And note that since $f([0,1])$ is bounded,
		there is an upper bound, so $f([0,1])$ has a supremum $\beta \in \R$.
		But since $0\cdot \beta = 0$,
		we have shown $\lVert cf \rVert = |c|\lVert f\rVert$ when $c=0$.

		Now let $c \neq 0$.
		We have
		\begin{align*}
			\lVert cf \rVert
			&= \sup\{|cf(x)| \colon x \in [0,1]\}\\
			&= \sup\{|c||f(x)| \colon x \in [0,1]\}\\
		\end{align*}
		We now prove that $|c|\sup\{|f(x)| \colon x \in [0,1]\}$
		is also a supremum for $|c||f([0,1])|$.
		Note that since $|f(x')| \leq \sup\{|f(x)| \colon x \in [0,1]\}$
		for all $x' \in [0,1]$ (by definition of a supremum),
		we have $|c||f(x')| \leq \sup\{|f(x)| \colon x \in [0,1]\}$
		for all $x' \in [0,1]$ (since $|c| > 0$).
		Thus, $|c|\sup\{|f(x)| \colon x \in [0,1]\}$ is an upper bound for $|c||f([0,1])|$.
		Let $\ep > 0$.
		By the definition of supremum, we know that there exists $x'\in [0,1]$ such that
		$\sup\{|f(x)| \colon x \in [0,1]\} - \frac{\ep}{|c|} < |f(x')|$
		since $|c| > 0$.
		But then $|c|\sup\{|f(x)| \colon x \in [0,1]\} - \ep < |c||f(x')|$.
		Thus, $|c|\sup\{|f(x)| \colon x \in [0,1]\}$
		satisifies both our conditions to be the supremum of $|c||f([0,1])|$.
		But note that a supremum is unique (both are upper bounds and
		the only way for them to be simultaneously less than or equal to each other
		is if they are equal).
		Thus
		\[
			|c|\sup\{|f(x)| \colon x \in [0,1]\} = \sup\{|c||f(x)|\colon x \in [0,1]\}
		\]
		\[
			\implies |c|\lVert f \rVert = \lVert cf\rVert
		\]
		Thus we have shown this is true for all real $c$.
	\end{proof}
	\item \begin{proof}[Solution]\let\qed\relax
		Recall that since $|f(x') + g(x')| \leq |f(x')| + |g(x')|$
		for every $x' \in [0,1]$ (triangle inequality),
		we have $|f(x') + g(x')| \leq \sup_{x\in[0,1]}\{|f(x)|\} + |g(x')|
		\leq \sup_{x\in[0,1]}\{|f(x)|\} + \sup_{x\in[0,1]}\{|g(x)|\}$,
		thus
		\[
			\sup_{x \in [0,1]}\{|f(x) + g(x)|\}
			\leq \sup_{x\in[0,1]}\{|f(x)|\} + \sup_{x\in[0,1]}\{|g(x)|\}
		\]
		\[
			\implies \lVert f + g \rVert \leq \lVert f \rVert + \lVert g \rVert
		\]

		We now give an example when this is a strict inequality:
		Let $f = 1$ and $g = -1$ for all $x \in [0,1]$.
		Then
		\[
			\lVert f + g \rVert = \lVert 0 \rVert = 0 <
			2 = 1 + 1 = \lVert f \rVert + \lVert g \rVert
		\]
	\end{proof}
	\item \begin{proof}[Solution]\let\qed\relax
		ff
	\end{proof} 
\end{enumerate}

\end{document}
