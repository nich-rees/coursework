\documentclass{article}
\usepackage{amsmath, amsfonts, amsthm, amssymb}
\usepackage{geometry}
\geometry{letterpaper, margin=2.0cm, includefoot, footskip=30pt}

\usepackage{fancyhdr}
\pagestyle{fancy}

\lhead{Math 320}
\chead{Homework 8}
\rhead{Nicholas Rees, 11848363}
\cfoot{Page \thepage}

\newtheorem*{problem}{Problem}

\newcommand{\N}{{\mathbb N}}
\newcommand{\Z}{{\mathbb Z}}
\newcommand{\Q}{{\mathbb Q}}
\newcommand{\R}{{\mathbb R}}
\newcommand{\C}{{\mathbb C}}
\newcommand{\ep}{{\varepsilon}}
\newcommand{\SR}{{\mathcal R}}

\renewcommand{\theenumi}{(\alph{enumi})}

\begin{document}
\subsection*{Problem 1}
{\it Prove: If $\sum a_n$ converges and $\sum b_n$ converges absolutely,
then $\sum a_nb_n$ converges.
Is this statement still true if the word ``absolutely" is removed?}

\begin{proof}[Solution]\let\qed\relax
	It is sufficient to show that $\sum_n a_nb_n$ is absolutely convergent.
	Consider the series $\sum_n|a_nb_n| = \sum_n|a_n||b_n|$.
	Since $\lim_{n\to\infty} a_n = 0$
	(by the contrapositive of the ``crude" divergence test since $\sum_n a_n$ converges),
	we have that $a_n$ is bounded,
	and so $|a_n|$ is bounded as well
	(the upper bound is just the max of the lower and upper bound of $a_n$,
	and it is bounded below by $0$).
	Let $|a_n| \leq M$ for all $n \in \N$.
	Then $|a_n||b_n| < M|b_n|$.
	We have that $\sum_n M|b_n|$ converges,
	since if $s_N = \sum_{n=1}^N |b_n|$, then
	\[
		\sum_n M|b_n| = \lim_{n\to\infty} M|b_0| + M|b_1| + \cdots + M|b_n|
		= \lim_{n\to\infty} M(|b_0| + |b_1| + \cdots + |b_n|)
		= \lim_{n\to\infty} Ms_n
	\]
	and since $(s_n)$ converges (by the absolute convergence of $b_n$),
	by our constant multiplication limit law, $(Ms_n) = \sum_nM|b_n|$
	converges as well.

	Now since $0 \leq |a_nb_n| \leq M|b_n|$,
	by the comparison test, $\sum_n |a_nb_n|$ converges,
	thus $\sum_n a_nb_n$ is absolutely convergent,
	which implies that $\sum_n a_nb_n$ converges.
\end{proof}
\clearpage
~\clearpage

\subsection*{Problem 2}
{\it For each series below, find the set of $x \in \R$ where the series converges.
\begin{enumerate}
	\item $\sum_{n=1}^\infty c^{n^2}(x-1)^n$ ($c > 0$ const.)
	\item $\sum_{n=1}^\infty \frac{x^n(1-x^n)}{n}$
	\item $\sum_{n=1}^\infty \frac{1}{\sqrt{n}}\left[\frac{x+1}{2x+1}\right]^n$
	\item $\sum_{n=1}^\infty \left[\frac{(2n)!}{n(n!)^2}\right](x-e)^n$
\end{enumerate}}

\begin{enumerate}
	\item \begin{proof}[Solution]\let\qed\relax
		Fix some arbitrary $x \in \R$.
		Let $a_n = c^{n^2}(x-1)^n$
		and $\alpha = \limsup_n|a_n|^{1/n}$.
		We can compute
		\[
			\alpha = \limsup_n \left\lvert c^{n^2}(x-1)^n\right\rvert^{1/n}
			= \limsup_n \left\lvert c^n(x-1)\right\rvert
			= \left\lvert x-1\right\rvert\limsup_n c^n
		\]
		where we've brought the exponent $n$ out in the first step,
		since $|a^nb^n| = |ab|^n$.

		If $x = 1$, then $\alpha = 0$,
		so the series converges by the root test regardless of $c$.
		Now let $x \in \R\setminus\{1\}$.
		We know that $\lim_{n\to\infty} c^n \to +\infty$ if $c > 1$,
		so $\limsup_n c^n = +\infty$, thus the series diverges for all $x$.
		Additionally, $\lim_{n\to\infty}c^n = 0$ if $1 > c > 0$,
		so $\limsup_n c^n = 0$, thus the series converges for all $x$.
		Finally, if $c = 1$, $a_n = (x-1)^n$ which is a geometric series:
		it will converge when $|x-1| < 1 \implies 0 < x < 2$ and will diverge otherwise.

		In summary:
		\begin{itemize}
			\item If $c > 1$, $x \in \{1\}$ makes the series converge
			\item If $c = 1$, $x \in (0,2)$ makes the series converge
			\item If $0 < c < 1$, $x \in \R$ makes the series converge
		\end{itemize}
	\end{proof}
	\item \begin{proof}[Solution]\let\qed\relax
		Let $a_n = \frac{x^n(1-x^n)}{n}$.
		Let $x \in \{0,1\}$.
		Then $a_n = 0$ for all $n$, thus the series converges.
		Let $x = -1$.
		Then our series is $\sum_n a_n = \sum_{n\text{ odd}}\frac{2}{n}$.
		We can rewrite our sum to be $\sum_{n}\frac{1}{2\lfloor (n-1)/2\rfloor+1}$
		(since if $n = 2k$, $2\lfloor (n-1)/2\rfloor+1 =n - 1$: the odd
		number directly below it,
		and if $n = 2k+1$, $2\lfloor (n-1)/2\rfloor+1 = n$: itself)
		and then since $0 <2\lfloor (n-1)/2\rfloor+1<n$
		so $0 < \left\lvert\frac{1}{n}\right\rvert
		\leq \frac{1}{2\lfloor (n-1)/2\rfloor+1}$,
		comparison test says this series diverges.
		
		Now consider when $|x| > 1$.
		Then we claim there exists an $N\in \N$ such that $x^n(1-x^n) < -1$
		for all $n \geq N$.
		We prove this by considering when $n$ is positive and negative.
		Let $x > 1$.
		Note that there exists an $N$ such that $x^n > 2$
		for all $n \geq N$:
		using the inequality from Problem 4(a) of Homework 6 since $x > 1$,
		we have that $x^n > x^n - 1 \geq n(x-1)$
		and then invoke Archimedean property to find $N$ such that $N(x-1) > 2$,
		it's trivial to see that $n \geq N$ also implies $x^n > 2$.
		Now if $n \geq N$, we have $1-x^n < -1$ and since
		$x^n > 1$, we have $x^n(1-x^n) < 1-x^n < -1$.
		Now let $x < -1$.
		If $n$ is even, $x^n(1-x^n) = |x|^n(1-|x|^n)$,
		and we have the same $N$ from when $x > 1$
		to have $x^n(1-x^n) < -1$.
		If $n$ is odd, $x^n(1-x^n) = (-1)|x|^n(1-(-1)|x|^n) = -|x|^n(|x|^n + 1)$,
		and using the $N$ from before, we have $-|x|^n (|x|^n+1) < -2(x^n+1) < -2 < -1$.
		This proves our claim.
		But then for all $n \geq N$,
		we have that
		\[
			\frac{x^n(1-x^n)}{n} < \frac{-1}{n} < 0 \implies
			0 < \frac{1}{n} = \left\lvert \frac{1}{n} \right\rvert < -\frac{x^n(1-x^n)}{n}
		\]
		And so by the comparison test,
		$\sum_n -a_n$ diverges.
		But this is true only if $\sum_n a_n$ diverges,
		since if $s_N = \sum_{n=1}^N a_n$ and $s'_N = \sum_{n=1}^N -a_n$,
		we have that $s'_N -\sum_{n=1}^N a_n = -s_N$,
		and if $s_N$ converged as $N \to \infty$,
		constant multiplication limit law would tell us that $s'_N$
		would converge as well.
		Thus, if $|x| > 1$, we have that the series diverges.

		Now consider when $0 < x < 1$.
		Consider $\sum_k 2^k\frac{x^{2^k}(1-x^{2^k})}{2^k} = \sum_k x^{2^k}(1-x^{2^k})$.
		We have $0 < x^{2^k}(1-x^{2^k}) < x^{2^k} < x^k$
		(where the inequality is due to the fact that
		$x^a$ is monotonically decreasing when $0 < x < 1$,
		and $2^k > k$),
		and $\sum_k x^k$ converges since it is geometric series
		with ratio $x < 1$.
		Thus, by the comparison test,
		we have that $\sum_k 2^k\frac{x^{2^k}(1-x^{2^k})}{2^k}$
		converges as well.
		Finally, $\frac{x^n(1-x^n)}{n}$ is monotonically
		decreasing and bounded below by $0$:
		all the terms are positive, so $a_n > 0$ for all $n$;
		now see
		\[
			\frac{a_{n+1}}{a_n}
			= \frac{x^{n+1}(1-x^{n+1})n}{x^n(1-x^n)(n+1)}
			< x\frac{1-x^{n+1}}{1-x^n}{}
		\]
		and $x\frac{1-x^{n+1}}{1-x^n} \to x$ as $n \to \infty$
		(limit laws),
		thus for sufficiently large $n$,
		we have that $\frac{a_{n+1}}{a_n} < x + \ep$
		where setting $\ep = 1 - x > 0$ gives $\frac{a_{n+1}}{a_n} < 1$,
		hence the series is monotonically decreasing past that point.
		Thus, by Cauchy condensation, the series converges
		when $0 < x < 1$
		(technically, Cauchy Condensation only tells us that
		the series converges starting from our $n$
		where the series begins to be monotonically decreasing,
		but then we have the sum of a convergent series
		and a finite sum, which itself converges).

		It remains to consider the case when $-1 < x < 0$.
		Now if $n$ is odd, we have $a_n = \frac{(-1)|x|^n(1+|x|^n)}{n}
		< \frac{-2|x|^n}{n}$.
		Since $\lim_{n\to\infty}\frac{2n}{(1/|x|)^n} = 0$ by Rudin Theorem 3.20 (d),
		we have some $N$ such that $\left(\frac{1}{|x|}\right)^n > 2n > 0$
		for all $n \geq N$,
		thus $0 < 2|x|^n < \frac{1}{n}$,
		thus $|a_n| < \frac{2|x|^n}{n} < \frac{1}{n^2}$.
		Furthermore, when $n$ is even, we have $a_n = \frac{|x|^n(1-|x|^n)}{n} < \frac{|x|^n(1 + |x|^n)}{n} < \frac{1}{n^2}$ as well.
		Thus $|a_n| < \frac{1}{n^2}$ for all $n \geq N$.
		And since $\sum_n \frac{1}{n^2}$ is a convergent $p$-series ($p > 1$),
		the comparison test tells us that $\sum_n a_n$ converges as wel.

		In summary: the series converges when $x \in (-1,1]$,
		and diverges otherwise.
	\end{proof}
	\item \begin{proof}[Solution]\let\qed\relax
		Fix some arbitrary $x \in \R$.
		Let $a_n = \frac{1}{\sqrt{n}}\left[\frac{x+1}{2x+1}\right]^n$
		and $\alpha = \limsup_n|a_n|^{1/n}$.
		We can compute
		\[
			\alpha =
			\limsup_n \left\lvert\frac{1}{\sqrt{n}}
			\left[\frac{x+1}{2x+1}\right]^n\right\rvert^{1/n}
			= \limsup_n (n^{1/(2n)})^{-1}\left\lvert\frac{x+1}{2x+1}\right\rvert
			= \left\lvert\frac{x+1}{2x+1}\right\rvert\limsup_n(n^{1/(2n)})^{-1}
			=\left\lvert\frac{x+1}{2x+1}\right\rvert
		\]
		where our final equality is due to $\lim_{n\to\infty} n^{1/(2n)} = 1$,
		and so $\liminf_n n^{1/{2n}} = 1$ ($\liminf$ agrees with convergent limits),
		and by Problem 8(c) from homework 4,
		$\limsup_n (n^{1/(2n)})^{-1} = 1^{-1} = 1$
		(we've also used the fact $|a^n| = |a|^n$ for our first equality).
		Now, ratio test gives convergence when $|x+1| < |2x+1|$.
	\end{proof}
	\item \begin{proof}[Solution]\let\qed\relax
		ff
	\end{proof}
\end{enumerate}
\clearpage

\subsection*{Problem 3}
{\it Discuss the series whose $n$th terms are shown below:
\begin{align*}
	&a_n = (-1)^n\frac{n^n}{(n+1)^{n+1}},
	&b_n = \frac{n^n}{(n+1)^{n+1}},\\
	&c_n = (-1)^n\frac{(n+1)^n}{n^n},
	&d_n = \frac{(n+1)^n}{n^n}.\\
\end{align*}

\begin{proof}[Solution]\let\qed\relax
	ff
\end{proof}
\clearpage
~\clearpage

\subsection*{Problem 4}
{\it Suppose $x_1 \geq x_2 \geq x_3 \geq \cdots$ and $\lim_{n \to \infty}x_n = 0$.
Show that the following series converges:
\[
	x_1 - \frac{1}{2}(x_1 + x_2) + \frac{1}{3}(x_1 + x_2 + x_3)
	- \frac{1}{4}(x_1 + x_2 + x_3 + x_4) \pm \cdots.
\]}
\begin{proof}[Solution]\let\qed\relax
	Note that the sequence is bounded below by $0$.
	We have a sum of geometric series.
\end{proof}
\clearpage
~\clearpage

\subsection*{Problem 5}
{\it \begin{enumerate}
	\item Prove: if $a_n \geq a_{n+1} \geq 0$ for all $n$,
	and $\sum a_n$ converges, then $\lim_{n\to\infty}na_n = 0$.
	\item Prove: If $\sum(b_n^2/n)$ converges,
	$\frac{1}{N}\sum_{j=1}^N b_j \to 0$ as $N \to \infty$.
\end{enumerate}
[Hint: In part (a), it's enough to prove that $\frac12 na_n \to 0$.]

\begin{enumerate}
	\item \begin{proof}[Solution]\let\qed\relax
		It would be sufficient to show that $\sum_n \frac{1}{2}n a_n$ converges.
		This converges if and only if $\sum_n 2^n a_n$
		by the Cauchy condensation... this is not Cauchy condensation
		because of $n$... but what about $n < 2^k$.

		Since $a_n \geq a_{n+1} \geq 0$ for all $n$
		and $\sum a_n$ converges,
		the Cauchy Condensation Test gives $\sum_k 2^k a_{2^k}$ converges as well.
		Since our sequence monotonically decreases and is always positive,
		we have $2^k a_n \leq 2^ka_{2^k}$ for $n \in \N$ such that $2^{k} \leq n < 2^{k+1}$.
		Note that for any $k$, $2^{k} \leq n < 2^{k+1}$ implies $\frac{n}{2} < 2^k$.
		Thus, $0 \leq |\frac{n}{2}a_n| = \frac{n}{2}a_n < 2^ka_{n} \leq 2^k a_{2^k}$
		for $2^{k} \leq n < 2^{k+1}$.
		ff
	\end{proof}
	\item \begin{proof}[Solution]\let\qed\relax
		In Problem 8(a) from homework 3,
		we proved that if $a_n \to 0$ as $n \to \infty$,
		then $(a_1+a_2+\cdots a_n)/n \to 0$ as well.
		Thus, it is sufficent to show that $b_n \to 0$
		as $n \to \infty$.
		
		Since $\sum (b_n^2/n)$ converges,
		by the crude divergence test, we have that $b_n^2/n \to 0$ as $n\to\infty$.
		ff
	\end{proof}
\end{enumerate} 
\clearpage
~\clearpage

\subsection*{Problem 6}
{\it Define $f(\theta) = \sum_{k=1}^\infty \frac{1}{2k-1}\sin((2k-1)\theta)$.
Determine the domain of $f$, namely,
the set of all real $\theta$ where the series converges,
by completing the steps below.
\begin{enumerate}
	\item Obtain the following identities, valid for each $n \in \N$
		at all points where $\sin\theta \neq 0$:
		\[
			C_n(\theta) = \cos(\theta) + \cos(3\theta) + \cos(5\theta)
			+ \cdots + \cos((2n-1)\theta) = \frac{\sin(2n\theta)}{2\sin\theta},
		\]
		\[
			S_n(\theta) = \sin(\theta) + \sin(3\theta) + \sin(5\theta)
			+ \cdots + \sin((2n-1)\theta) = \frac{1-\cos(2n\theta)}{2\sin\theta},
		\]
		[Suggestion: Use geometric sums of complex numbers,
		with $e^{it} = \cos(t) + i\sin(t)$.]
	\item Prove that the domain of $f$ is the interval $(-\infty,+\infty)$.
	\item Find a sequence $(\theta_n)$ such that $\theta_n \to 0$
		and $S_n(\theta_n) \to +\infty$ as $n \to \infty$.
		Explain why your solution in part (b) is correct in spite
		of the evident unboundedness of the sequence $(S_n(\theta_n))$.
\end{enumerate}
}

\begin{enumerate}
	\item \begin{proof}[Solution]\let\qed\relax
		ff
	\end{proof}
	\item \begin{proof}[Solution]\let\qed\relax
		ff
	\end{proof}
	\item \begin{proof}[Solution]\let\qed\relax
		ff
	\end{proof}
\end{enumerate}
\clearpage
\end{document}
