\documentclass{article}
\usepackage{amsmath, amsfonts, amsthm, amssymb}
\usepackage{geometry}
\geometry{letterpaper, margin=2.0cm, includefoot, footskip=30pt}

\usepackage{fancyhdr}
\pagestyle{fancy}

\lhead{Math 320}
\chead{Homework 8}
\rhead{Nicholas Rees, 11848363}
\cfoot{Page \thepage}

\newtheorem*{problem}{Problem}

\newcommand{\N}{{\mathbb N}}
\newcommand{\Z}{{\mathbb Z}}
\newcommand{\Q}{{\mathbb Q}}
\newcommand{\R}{{\mathbb R}}
\newcommand{\C}{{\mathbb C}}
\newcommand{\ep}{{\varepsilon}}
\newcommand{\SR}{{\mathcal R}}

\renewcommand{\theenumi}{(\alph{enumi})}

\begin{document}
\subsection*{Problem 1}
{\it Prove: If $\sum a_n$ converges and $\sum b_n$ converges absolutely,
then $\sum a_nb_n$ converges.
Is this statement still true if the word ``absolutely" is removed?}

\begin{proof}[Solution]\let\qed\relax
	It is sufficient to show that $\sum_n a_nb_n$ is absolutely convergent.
	Consider the series $\sum_n|a_nb_n| = \sum_n|a_n||b_n|$.
	Since $\lim_{n\to\infty} a_n = 0$
	(by the contrapositive of the ``crude" divergence test since $\sum_n a_n$ converges),
	we have that $a_n$ is bounded,
	and so $|a_n|$ is bounded as well
	(the upper bound is just the max of the lower and upper bound of $a_n$,
	and it is bounded below by $0$).
	Let $|a_n| \leq M$ for all $n \in \N$.
	Then $|a_n||b_n| < M|b_n|$.
	We have that $\sum_n M|b_n|$ converges,
	since if $s_N = \sum_{n=1}^N |b_n|$, then
	\[
		\sum_n M|b_n| = \lim_{n\to\infty} M|b_0| + M|b_1| + \cdots + M|b_n|
		= \lim_{n\to\infty} M(|b_0| + |b_1| + \cdots + |b_n|)
		= \lim_{n\to\infty} Ms_n
	\]
	and since $(s_n)$ converges (by the absolute convergence of $b_n$),
	by our constant multiplication limit law, $(Ms_n) = \sum_nM|b_n|$
	converges as well.

	Now since $0 \leq |a_nb_n| \leq M|b_n|$,
	by the comparison test, $\sum_n |a_nb_n|$ converges,
	thus $\sum_n a_nb_n$ is absolutely convergent,
	which implies that $\sum_n a_nb_n$ converges.
\end{proof}
\clearpage
~\clearpage

\subsection*{Problem 2}
{\it For each series below, find the set of $x \in \R$ where the series converges.
\begin{enumerate}
	\item $\sum_{n=1}^\infty c^{n^2}(x-1)^n$ ($c > 0$ const.)
	\item $\sum_{n=1}^\infty \frac{x^n(1-x^n)}{n}$
	\item $\sum_{n=1}^\infty \frac{1}{\sqrt{n}}\left[\frac{x+1}{2x+1}\right]^n$
	\item $\sum_{n=1}^\infty \left[\frac{(2n)!}{n(n!)^2}\right](x-e)^n$
\end{enumerate}}

\begin{enumerate}
	\item \begin{proof}[Solution]\let\qed\relax
		ff
	\end{proof}
	\item \begin{proof}[Solution]\let\qed\relax
		ff
	\end{proof}
	\item \begin{proof}[Solution]\let\qed\relax
		ff
	\end{proof}
	\item \begin{proof}[Solution]\let\qed\relax
		ff
	\end{proof}
\end{enumerate}
\clearpage

\subsection*{Problem 3}
{\it Discuss the series whose $n$th terms are shown below:
\begin{align*}
	&a_n = (-1)^n\frac{n^n}{(n+1)^{n+1}},
	&b_n = \frac{n^n}{(n+1)^{n+1}},\\
	&c_n = (-1)^n\frac{(n+1)^n}{n^n},
	&d_n = \frac{(n+1)^n}{n^n}.\\
\end{align*}

\begin{proof}[Solution]\let\qed\relax
	ff
\end{proof}
\clearpage
~\clearpage

\subsection*{Problem 4}
{\it Suppose $x_1 \geq x_2 \geq x_3 \geq \cdots$ and $\lim_{n \to \infty}x_n = 0$.
Show that the following series converges:
\[
	x_1 - \frac{1}{2}(x_1 + x_2) + \frac{1}{3}(x_1 + x_2 + x_3)
	- \frac{1}{4}(x_1 + x_2 + x_3 + x_4) \pm \cdots.
\]}
\begin{proof}[Solution]\let\qed\relax
	Note that the sequence is bounded below by $0$.
	We have a sum of geometric series.
\end{proof}
\clearpage
~\clearpage

\subsection*{Problem 5}
{\it \begin{enumerate}
	\item Prove: if $a_n \geq a_{n+1} \geq 0$ for all $n$,
	and $\sum a_n$ converges, then $\lim_{n\to\infty}na_n = 0$.
	\item Prove: If $\sum(b_n^2/n)$ converges,
	$\frac{1}{N}\sum_{j=1}^N b_j \to 0$ as $N \to \infty$.
\end{enumerate}
[Hint: In part (a), it's enough to prove that $\frac12 na_n \to 0$.]

\begin{enumerate}
	\item \begin{proof}[Solution]\let\qed\relax
		It would be sufficient to show that $\sum_n \frac{1}{2}n a_n$ converges.
		This converges if and only if $\sum_n 2^n a_n$
		by the Cauchy condensation... this is not Cauchy condensation
		because of $n$... but what about $n < 2^k$.

		Since $a_n \geq a_{n+1} \geq 0$ for all $n$
		and $\sum a_n$ converges,
		the Cauchy Condensation Test gives $\sum_k 2^k a_{2^k}$ converges as well.
		Since our sequence monotonically decreases and is always positive,
		we have $2^k a_n \leq 2^ka_{2^k}$ for $n \in \N$ such that $2^{k} \leq n < 2^{k+1}$.
		Note that for any $k$, $2^{k} \leq n < 2^{k+1}$ implies $\frac{n}{2} < 2^k$.
		Thus, $0 \leq |\frac{n}{2}a_n| = \frac{n}{2}a_n < 2^ka_{n} \leq 2^k a_{2^k}$
		for $2^{k} \leq n < 2^{k+1}$.
		ff
	\end{proof}
	\item \begin{proof}[Solution]\let\qed\relax
		In Problem 8(a) from homework 3,
		we proved that if $a_n \to 0$ as $n \to \infty$,
		then $(a_1+a_2+\cdots a_n)/n \to 0$ as well.
		Thus, it is sufficent to show that $b_n \to 0$
		as $n \to \infty$.
		
		Since $\sum (b_n^2/n)$ converges,
		by the crude divergence test, we have that $b_n^2/n \to 0$ as $n\to\infty$.
		ff
	\end{proof}
\end{enumerate} 
\clearpage
~\clearpage

\subsection*{Problem 6}
{\it Define $f(\theta) = \sum_{k=1}^\infty \frac{1}{2k-1}\sin((2k-1)\theta)$.
Determine the domain of $f$, namely,
the set of all real $\theta$ where the series converges,
by completing the steps below.
\begin{enumerate}
	\item Obtain the following identities, valid for each $n \in \N$
		at all points where $\sin\theta \neq 0$:
		\[
			C_n(\theta) = \cos(\theta) + \cos(3\theta) + \cos(5\theta)
			+ \cdots + \cos((2n-1)\theta) = \frac{\sin(2n\theta)}{2\sin\theta},
		\]
		\[
			S_n(\theta) = \sin(\theta) + \sin(3\theta) + \sin(5\theta)
			+ \cdots + \sin((2n-1)\theta) = \frac{1-\cos(2n\theta)}{2\sin\theta},
		\]
		[Suggestion: Use geometric sums of complex numbers,
		with $e^{it} = \cos(t) + i\sin(t)$.]
	\item Prove that the domain of $f$ is the interval $(-\infty,+\infty)$.
	\item Find a sequence $(\theta_n)$ such that $\theta_n \to 0$
		and $S_n(\theta_n) \to +\infty$ as $n \to \infty$.
		Explain why your solution in part (b) is correct in spite
		of the evident unboundedness of the sequence $(S_n(\theta_n))$.
\end{enumerate}
}

\begin{enumerate}
	\item \begin{proof}[Solution]\let\qed\relax
		ff
	\end{proof}
	\item \begin{proof}[Solution]\let\qed\relax
		ff
	\end{proof}
	\item \begin{proof}[Solution]\let\qed\relax
		ff
	\end{proof}
\end{enumerate}
\clearpage
\end{document}
