\documentclass{article}
\usepackage{amsmath, amsfonts, amsthm, amssymb}
\usepackage{geometry}
\geometry{legalpaper, margin=1.7cm, includefoot, footskip=30pt}

\usepackage{fancyhdr}
\pagestyle{fancy}

\lhead{Math 320}
\chead{Homework 2}
\rhead{Nicholas Rees, 11848363}
\cfoot{Page \thepage}

\newtheorem*{problem}{Problem}

\newcommand{\N}{{\mathbb N}}
\newcommand{\Z}{{\mathbb Z}}
\newcommand{\Q}{{\mathbb Q}}
\newcommand{\R}{{\mathbb R}}
\newcommand{\C}{{\mathbb C}}
\newcommand{\ep}{{\varepsilon}}

\renewcommand{\theenumi}{(\alph{enumi})}

\begin{document}
\subsection*{Problem 1}
{\it  The Fibonacci sequence is defined recursively by saying $F(0) = 1, F(1) = 1$, and
\[F(n) = F(n-1) + F(n-2) \text{ for each }n=2,3,\dots\]
Prove that if $\phi=\frac{1+\sqrt{5}}{2}$ denotes the ``Golden Ratio"
(observe $\phi^2 = \phi + 1$), we have
\[F(n) = \frac{\phi^{n+1}-(-\phi)^{-n-1}}{\sqrt{5}}, \; n=1,2,3,\dots\]}

\begin{proof}[Solution]\let\qed\relax
	(Apologies for the gross formatting of the algebra,
	we were told to keep the solution contained to a page;
	read equations left to right, then down.)
	We prove using induction on $n$.
	We start with the base cases, $n = 1,2$.
	First, we look to prove our expression is equivalent to $F(1) = 1$.
	\begin{align*}
		\frac{\phi^2-(-\phi)^{-2}}{\sqrt{5}}
		&= \frac{\phi + 1 - 1/(\phi+1)}{\sqrt{5}}
		= \frac{((\phi+1)^2-1)/(\phi+1)}{\sqrt{5}}\\
		&= \frac{\phi(\phi+2)}{\sqrt{5}(\phi+1)}
		= \frac{\phi+1 + 2\phi}{\sqrt{5}(\phi+1)}\\
		&= \frac{\phi + 2 + \sqrt{5}}{(\sqrt{5}+5+2\sqrt{5})/2}
		= \frac{\sqrt{5} + 5 + 2\sqrt{5}}{\sqrt{5} + 5 + 2\sqrt{5}}\\
		&= 1\\ &= F(1)
	\end{align*}
	And now we want to show our expression is equivalent to $F(2) = 2$.
	We have
\begin{align*}
	\frac{\phi^3 - (-\phi)^{-3}}{\sqrt{5}}
	&= \frac{\phi(\phi+1) + 1/(\phi(\phi+1))}{\sqrt{5}}\\
	&= \frac{(\phi^2(\phi+1)^2 + 1)/(\phi(\phi+1))}{\sqrt{5}}
	= \frac{(\phi+1)^3 + 1}{\sqrt{5}\phi(\phi+1)}\\
	&= \frac{\phi^3 + 3\phi^2 + 3\phi + 1 + 1}{\sqrt{5}\phi(\phi+1)}
	= \frac{\phi(\phi+1) + 6\phi + 5}{\sqrt{5}\phi(\phi+1)}\\
	&= \frac{2\phi + 1 + 6\phi + 5}{\sqrt{5}(2\phi + 1)}
	= \frac{8\phi + 6}{\sqrt{5}(2\phi + 1)}\\
	&= 2\frac{4\phi + 3}{\sqrt{5}(2\phi+1)}
	= 2\frac{2 + 2\sqrt{5} + 3}{\sqrt{5}(1+\sqrt{5}+1)}\\
	&= 2\frac{5 + 2\sqrt{5}}{5 + 2\sqrt{5}}\\
	&= 2\\ &= F(2)
\end{align*}
thus the base cases are true.

Before we prove the inductive step, note that
\[
	1 - \phi^{-1} = \phi^{-2}
\]
We can use our given identity,
specifically
\[
	\phi^{-2} = (\phi + 1)^{-1} = 1/\left(\frac{1+\sqrt{5}}{2}+1\right)
	= \frac{2}{3+\sqrt{5}} = \frac{2(3 - \sqrt{5})}{4}
	= \frac{5 - 2\sqrt{5} + 1}{5 - 1} = \frac{\sqrt{5} - 1}{\sqrt{5} + 1}
	= 1 - \frac{2}{1 + \sqrt{5}} = 1 - \phi^{-1}
\]
We will use this in the steps below,
specifically invoking $(-\phi)^{-2} = 1 - \phi^{-1}$, since $(-\phi)^{-2} = \phi^{-2}$.
Now, let us assume that $F(j) = \frac{\phi^{j+1} - (-\phi)^{-j-1}}{\sqrt{5}}$
and $F(j-1) = \frac{\phi^{j} - (-\phi)^{-j}}{\sqrt{5}}$
for an arbitrary $2 \leq j \in \N$. Then for proving $F(j+1)$, see that
\begin{align*}
	\frac{\phi^{j+2} - (-\phi)^{-j-2}}{\sqrt{5}}
	&= \frac{\phi^{j}\phi^2 - (-\phi)^{-j}(-\phi)^{-2}}{\sqrt{5}}
	= \frac{\phi^{j}(\phi+1) - (-\phi)^{-j}(1 - \phi^{-1})}{\sqrt{5}}\\
	&= \frac{\phi^{j+1} + \phi^j - (-\phi)^{-j} - (-\phi)^{-j-1}}{\sqrt{5}}\\
	&=  \frac{\phi^j - (-\phi)^{-j}}{\sqrt{5}} +
	\frac{\phi^{j+1} - (-\phi)^{-j-1}}{\sqrt{5}}\\
	&= F(j-1) + F(j)\\ &= F(j+1)
\end{align*}
Thus the induction is closed, and so our formula for $F(n)$ is valid
for all $n = 1,2,3\dots$.
\end{proof}
\clearpage

\subsection*{Problem 2}
{\it Construct a countable family of increasing sequences with entries from $\N$:
\[s^{(1)} = \left(s_1^{(1)}, s_2^{(1)}, s_3^{(1)}, \dots \right),
s^{(2)} = \left(s_1^{(2)}, s_2^{(2)}, s_3^{(2)}, \dots \right),
s^{(3)} = \left(s_1^{(3)}, s_2^{(3)}, s_3^{(3)}, \dots \right), \dots\]
Arrange your construction so that every positive integer appears
in one and only one of the sequences,
and the inequality $s_n^{(i)} < s_{n+1}^{(i)}$ holds for all $i,n\in\N$.
Illustrate your construction by writing down explicitly
the first four entries of sequences $s^{(1)}$, $s^{(2)}$, $s^{(3)}$ and $s^{(4)}$.
}
\begin{proof}[Solution]\let\qed\relax
	First, note that the prime numbers are countable.
	This is because they are both infinite (see many of the standard proofs of this fact),
	but they are a subset of the naturals,
	and so they are at most countable (as we proved in class, week 2).
	Then, we can enumerate the prime numbers, $p_1,p_2,\dots$
	(say from least to greatest, since the naturals are well-ordered).

	Let $1$ reside in $s^{(1)}$.
	Now recall that every $1 < s \in \N$ has a unique prime factorization
	by the Fundamental Theorem of Arithmetic.
	This means every $s$ is determined uniquely by a product
	of primes raised to some powers,
	$s = p_1^{\alpha_1}p_2^{\alpha_2}\cdots$ with $0 \geq \alpha_i \in \N$.
	Let $s$ be in the sequence $s^{(i)}$ if and only if
	$p_i$ is smallest prime such that $p_i\mid s$
	(and we know we can always find a smallest prime,
	since the primes are subsets of the naturals and the naturals are well-ordered).
	Now we let these $s \in s^{(i)}$ be ordered by the ordering from $\N$,
	namely $s_n^{(i)} < s_{n+1}^{(i)}$ so our sequences are always increasing.
	
	Note that by construction, every $s \in \N$ will only appear in a sequence,
	and furthermore, $s$ will appear only once in all the sequences,
	since $s \neq 1$ has a unique representation of primes,
	and so will only have one prime $p_i$ that can be considered its smallest prime,
	and so $s$ will only be in $s^{(i)}$,
	(and $1$ is only in $s^{(1)}$).

	Explicitly:
	\begin{align*}
		s^{(1)} &= (1, 2, 4, 6, \dots)\\
		s^{(2)} &= (3, 9, 15, 21, \dots)\\
		s^{(3)} &= (5, 25, 35, 55, \dots)\\
		s^{(4)} &= (7, 49, 77, 91, \dots)
	\end{align*}
\end{proof}
\clearpage

\subsection*{Problem 3}
{\it Let $\mathcal{S}$ denote the collection (set) of $\N$-valued
sequences that are increasing.
That is, each object $s$ in $\mathcal{S}$ has the form
\[s = (s_1, s_2, s_3, \dots), \text{ where } \forall k \in \N, s_k < s_{k+1}.\]
Denote if the set $\mathcal{S}$ is countable or uncountable.
Prove your answer.
}
\begin{proof}[Solution]\let\qed\relax
	We claim that the set is uncountable.
	For the sake of contradiction, assume it is countable.
	That is to say, there exists some bijection $\phi \colon \N \to \mathcal{S}$.
	Then $\phi$ is injective,
	mapping each $n \to \N$ to a unique $s^{(n)} \in \mathcal{S}$.
	Note that we can construct a sequence $s'$ whose $i$th-term is given by
	\[s'_i = \sum_{j = 1}^i s_{j}^{(j)} + 1\]
	One can easily see that $s'$ is increasing,
	since each term is always adding a positive number to the previous one,
	so $s' \in \mathcal{S}$.
	However, $s'$ differs from every element $s^{(n)}$ in the image of $\phi$,
	since $s'$ and $s^{(n)}$ will always differ at the $n$th-term.
	This is because we are adding a positive value to $s_n^{(n)}$,
	and so $s'_n > s_n^{(n)}$ and so they are not equal.
	But then $\phi$ is not onto.
	But then the bijection $\phi$ cannot exist.
\end{proof}
\clearpage

\subsection*{Problem 4}
{\it Prove that the real intervals $I = [0,1]$ and $J=(0,1)$
have the same cardinal number constructing
an explicit bijection $\phi \colon I \to J$.}
\begin{proof}[Solution]\let\qed\relax
	We provide the bijection $\phi \colon I \to J$ as the following:
	\[\phi(x) = \begin{cases}
		\frac{1+x}{3} & x=0,1\\
		\frac{x}{3} & x\in\{\frac{1}{3^n} \colon n\in\N\}
		\cup \{\frac{2}{3^n} \colon n \in \N\}\\
		x & \text{ otherwise}
	\end{cases}\]
\end{proof}
We require to show that $\phi$ is both injective and surjective.
To show injectivity, let $x_1,x_2 \in [0,1]$ such that $\phi(x_1) = \phi(x_2)$.
Since the ranges of $\frac{1+x}{3}$, $\frac{x}{3}$, and $x$
are disjoint for their given domain
($\{\frac{1}{3},\frac{2}{3}\}$,
$\{\frac{1}{3^n} \colon 1<n\in\N\}
\cup \{\frac{1}{3^n} \colon 1<n\in\N\}$,
and the complement of the union of the other two ranges, respectively)
we know that $x_1$, $x_2$ must be in the image of the same function inside $\phi$.
If it had been $\frac{1+x}{3}$, then $\frac{1+x_1}{3} = \frac{1+x_2}{3} \Rightarrow x_1=x_2$.
If it had been $\frac{x}{3}$, then $\frac{x_1}{3} = \frac{x_2}{3} \Rightarrow x_1 = x_2$.
And if it had been $x$, then $x_1 = x_2$.
Thus, $\phi$ is injective.

We now show surjectivity.
Let $y \in (0,1)$.
$y$ is either of the form $y = \frac{k}{3^n}, k \in \{1,2\}, n\in\N$ or it is not.
If it is, then it is in the range of the first two functions,
and if it is not, it is in the range of the third.
Thus $\phi$ is onto $(0,1)$,
therefore $\phi$ is a bijection.
\clearpage

\subsection*{Problem 5}
{\it Taking as given an enumeration of the rationals as
\[\Q = \{q_1, q_2, q_3, \dots \},\]
construct an explicit bijection $f$ from $\R$ to $\R \setminus \Q$.
To confirm that your bijection is explicit enough,
return decimal approximations (correct to $6$ significant digits)
for these four numbers: $f(\pi)$, $f(\sqrt{3})$, $f(q_2)$, and $f(q_3)$.
(Hint: It is well known that $\sqrt{2} \in \R\setminus\Q$. You don't need to prove this.)
}
\begin{proof}[Solution]\let\qed\relax
First, note that the prime numbers are countable.
This is because they are both infinite (see many of the standard proofs of this fact),
but they are a subset of the naturals,
and so they are at most countable (as we proved in class, week 2).
Then, we can enumerate the prime numbers, $p_1,p_2,\dots$
(say from least to greatest, since the naturals are well-ordered).

We provide the following bijection $\phi \colon \R \to \R\setminus\Q$:
\[\phi(x) = \begin{cases}
	p_i + \sqrt{2} & x = q_i \in \Q\\
	p_i^{\alpha + 1} + \sqrt{2} & x = p_i^{\alpha}+\sqrt{2}, 1 \leq \alpha \in \N\\
	x & \text{ otherwise}
\end{cases}
\]
First, note that the range of $\phi$ is entirely contained in $\R\setminus \Q$,
and so satisfies partially what we were seeking to construct.
One can see this because either an irrational is being mapped to itself
(it's prime),
or a number is being mapped to a value of the form $y = n + \sqrt{2}$
where $n\in\N$ (since primes and their powers are all in $\N$),
but since $\Q$ is closed under addition, $-n \in \Q$, $\sqrt{2}\not\in\Q$,
and $y - n = \sqrt{2}$ implies that we must have that $y \not\in \Q$.

Now we seek to prove that $\phi$ is a bijection.
To see that $\phi$ is injective,
let $x_1,x_2 \in \R$ such that $\phi(x_1) = \phi(x_2)$.
Either $\phi(x_1) = \phi(x_2)$ is of the form $p^\alpha + \sqrt{2}$ or it is not
(where $p$ is a prime and $1\leq\alpha\in\N$).
If it is not, then it was mapped to by the third function in $\phi$,
and so $x_1 = x_2$ and we are done.
If it is, then if $\alpha \neq 1$,
it was mapped to by the second function,
and so $x_1 = p^{\alpha-1}+\sqrt{2}$ and $x_2 = p^{\alpha-1}+\sqrt{2}$,
but then $x_1 = x_2$.
Finally, if $\alpha = 1$, $x_1$ and $x_2$ were mapped by the first function,
and so if $p$ was the $i$th smallest prime
(ie. $\phi(x_1) = \phi(x_2) = p_i + \sqrt{2}$), we have $x_1 = q_i$ and $x_2 = q_i$,
and so $x_1 = x_2$.
This exhausts all cases, so $\phi$ is injective.

We now prove that $\phi$ is surjective.
Let $y \in \R \setminus \Q$.
Either $y$ is of the form $p^{\alpha} + \sqrt{2}$ or it is not
($p$ is prime and $1 \leq \alpha \in \N$).
If it is of the form, then $y = p_i^{\alpha} + \sqrt{2}$ is mapped to from
$p_i^{\alpha-1} + \sqrt{2}$ if $\alpha \neq 1$,
and is mapped to from $q_i$ if $\alpha = 1$.
If it is not of the form, then $y$ was mapped to from itself.
In either case, we can find some $x \in \R$ such that $\phi(x) = y$,
and so $\phi$ is surjective.
Thus, $\phi$ is bijective.

Computing explicitly:
\begin{align*}
	f(\pi) &= \pi = 3.14159\\
	f(\sqrt{3}) &= \sqrt{3} = 1.73205\\
	f(q_2) &= 3 + \sqrt{2} = 4.41421\\
	f(q_3) &= 5 + \sqrt{2} = 6.41421\\
\end{align*}
\end{proof}
\clearpage

\subsection*{Problem 6}
{\it \begin{enumerate}
	\item Show that the set of polynomials with integer coefficients is countable.
	\item Show that the set of real numbers $x$ arising
	as zeros of nonzero polynomials with integer coefficients is countable.
	(Such a real number is called algebraic.)
	\item A real number that is not algebraic is called transcendental.
	Prove that the set of transcendental numbers is not empty.
	\item Is the set of transcendental numbers finite, countable, or uncountable? Why?
\end{enumerate}
}
\begin{enumerate}
\item \begin{proof}[Solution]\let\qed\relax

	First, note that the prime numbers are countable.
	This is because they are both infinite (which we can assume by Piazza @80)
	but they are a subset of the naturals,
	and so they are at most countable (as we proved in class, week 2).
	Then, we can enumerate the prime numbers, $p_0,p_1,\dots$
	(say from least to greatest, since the naturals are well-ordered).

	We construct a bijection $\phi \colon \N \to$
	the set of polynomials with integer coefficients.
	Specifically, if $n\in\N$ and $n = p_0^{\alpha_0} p_1^{\alpha_1} \cdots p_k^{\alpha_k}$
	with $\alpha_i \in \N \cup \{0\}, \alpha_k > 0$ is the unique prime factorization of $n$
	(where $p_k$ is the largest prime factor),
	then we say $\phi(n)$ is the $k$-degree polynomial
	with coefficients $\alpha_0, \alpha_1, \dots ,\alpha_k$,
	specifically
	\[
		\phi(n) = \alpha_0 + \alpha_1x  + \alpha_2x^2 + \cdots \alpha_kx^k
	\]
	We now prove that this is a bijection.
	Note that since a prime factorization is unique
	by the Fundamental Theorem of Arithmetic,
	the $\{\alpha_1,\dots,\alpha_k\}$ are unique,
	and so $\phi$ maps $n$ to a unique coefficients,
	and since the coefficients (with order) completely determine a polynomial,
	$n$ is mapped to a unique polynomial.
	Thus $\phi$ is injective.
	To show that $\phi$ is onto the set of polynomials with integer coefficients,
	let $y$ be a polynomial with integer coefficients, specifically
	$y = \beta_0 + \beta_1x + \beta_2x^2 + \cdots + \beta_kx^k$
	where $\beta_i \in \Z$ for all $i\in\N\cup\{0\}$.
	But then, by construction, $\phi(p_0^{\beta_0}p_1^{\beta_1}\cdots p_k^{\beta_k}) = y$,
	and $p_0^{\beta_0}p_1^{\beta_1}\cdots p_k^{\beta_k} \in \N$,
	so $\phi$ is surjective.
	Thus, $\phi$ is a bijection, so the polynomials with integer coefficients are countable.
\end{proof}
\item \begin{proof}[Solution]\let\qed\relax
	Let $\mathbb{A}$ denote the algebraic numbers.
	Note that by the factor theorem,
	the number of roots a polynomial has cannot be larger than
	the degree of that polynomial.
	Since a polynomial has finite terms,
	the associated set of roots of that polynomial is finite then.
	The union of all the sets of the roots for a polynomial
	is $\mathbb{A}$ (by definition).
	So if $a_i$ denotes the set of roots for a polynomial $p_i(x)$
	with integer coefficients (where $i\in\N$, since the polynomials are countable by 6(a)),
	then we have
	\[
		\mathbb{A} = \bigcup_{i\in\N}a_i
	\]
	But note that this is a countable union of finite sets,
	which from the Corollary to Theorem 2.12 from Rudin,
	means that $\mathbb{A}$ must be finite or countable as well.
	In other words, $\mathbb{A} \leq \N$.
	But $\N \subset \mathbb{A}$ (consider the polynomials $p(x) = x-a$,
	$a \in \N$),
	and so there is a simple injection from $\N to \mathbb{A}$ ($\phi(x) = x$)
	thus $\N \leq \mathbb{A}$.
	Then, by the Schroeder Bernstein Theorem, $|\mathbb{A}| = |\N|$,
	which means $\mathbb{A}$ is countable.
\end{proof}
\item \begin{proof}[Solution]\let\qed\relax
	For the sake of contradiction, assume the set of transcendental numbers is empty.
	Since the transcendental numbers are $\R \setminus \mathbb{A}$,
	we have $\R = \mathbb{A} \cup \left(\R \setminus \mathbb{A}\right)$.
	Thus, if $\R \setminus \mathbb{A} = \emptyset$,
	then we have $\R = \mathbb{A}$.
	But the real numbers are uncountable and from 6(b),
	the algebraic numbers are countable, which is a contradiction
	(equal sets must have the same cardinality).
	Thus, $\R \setminus \mathbb{A} \neq \emptyset$.
\end{proof}
\item \begin{proof}[Solution]\let\qed\relax
	We claim that the transcendental numbers are uncountable.
	For the sake of contradiction, assume this is not the case.
	Then $\R \setminus \mathbb{A}$ is either finite or countable.
	From 6(b), recall the algebraic numbers are countable,
	and from lecture, and we know that a union of a countable set
	with a finite or countable set is also countable,
	thus $\mathbb{A} \cup (\R \setminus \mathbb{A})$ is countable.
	But the real numbers are uncountable,
	and we have $\R = \mathbb{A} \cup \left(\R \setminus \mathbb{A}\right)$,
	so we have a contradiction (equal sets must have the same cardinality),
	thus $\R \setminus \mathbb{A}$ must be uncountable
	(then $\mathbb{A} \cup (\R \setminus \mathbb{A})$ would be uncountable as well).
\end{proof}
\end{enumerate}
\clearpage

\subsection*{Problem 7}
\begin{enumerate}
\item \begin{proof}[Solution]\let\qed\relax
	See if $\lvert A \cup B\rvert = \lvert C \cup D \rvert$, then
	\[
		\lvert A \rvert + \lvert B \rvert
		= \lvert A \cup B\rvert
		= \lvert C \cup D\rvert
		= \lvert C \rvert + \lvert D \rvert
	\]
	So it suffices to prove $\lvert A \cup B\rvert = \lvert C \cup D \rvert$.
	Since $|A| = |C|$, there exists a bijection $\phi_A \colon A  \to C$,
	and since $|B| = |D|$, there exists a bijection $\phi_B \colon B \to D$.
	We can define another bijection $\phi \colon A \cup B \to C \cup D$ as follows
	\[
		\phi(x) = \begin{cases} \phi_A(x), & x\in A\\ \phi_B(x), & x\in B\end{cases}
	\]
	And since $A$ and $B$ are disjoint, each element $x \in A \cup B$
	is either in $A$ or $B$ but not both,
	so is only going to be mapped to a single element in $C \cup D$.
	Since $\phi_A$ maps $1$-$1$ onto $C$ and $\phi_B$ maps $1$-$1$ onto $D$,
	$\phi$ maps $1$-$1$ onto $C \cup D$ and so is bijective.
	Thus, $\lvert A \cup B\rvert = \lvert C \cup D \rvert$.
\end{proof}
\item \begin{proof}[Solution]\let\qed\relax
	We will let our $A_t$'s be a subset of $A \times \R$.
	Define $A_t = (a, t)$, where $t\in\R$ and $a \in A$.
	Note that $t$ is fixed for each $A_t$.
	See that $A_s \cap A_t = \emptyset$ if $s \neq t$,
	since all elements in $A_s$ have $s$ in the second spot in the tuple,
	and all elements in $A_t$ have $t$ in the second spot,
	and $s \neq t \implies (a,s) \neq (a,t)$ for all $a \in A$.
	Now, we can construct a $1$-$1$ map from $A$ to $A_t$
	($\phi(a) = (a,t)$, which is clearly injective)
	so $|A| \leq |A_t|$,
	and a $1$-$1$ map from $A_t$ to $A$
	($\psi((a,t) = a$ which is clearly injective)
	so $|A_t| \leq |A|$.
	Thus, by Schroeder-Bernstein Theorem, $|A_t| = |A|$.
	Therefore, our constructed $A_t$'s have the desired properties.
\end{proof}
\item \begin{proof}[Solution]\let\qed\relax
	Let $A,B$ be arbitrary sets such that
	$|A| = n$ and $B = \aleph_0$.
	First, note that there exists a $1$-$1$ map from $\N$ to $A \cup B$.
	There exists a bijection $\phi \colon \N \to B$ since $B$ is countable.
	Note $\phi$ is also a $1$-$1$ map from $\N$ to $A\cup B$,
	since it maps all elements in $\N$ to a unique element in $B$,
	and $B \subset A \cup B$.
	Thus $|\N| \leq |A\cup B| = |A| + |B|$.
	
	Now to show the other direction, we will show there is a $1$-$1$ map, call it $\psi$,
	from $|A| + |B|$ to $\N$.
	We can enumerate each element in $A$ with $1,2,\dots,n$, so $A = \{a_1,a_2,\dots,a_n\}$.
	We give the bijection $\psi \colon A \cup B \to \N$ as follows
	\[
		\psi(x) =
		\begin{cases}
			n & x = a_n \in A\\
			\phi^{-1}(x) & x \in B
		\end{cases}
	\]
	Since the range of $n$ is $\{1,2,\dots,n\}$ and the range of $\phi^{-1} + n$ is $\{m \in \N \mid m > n\}$
	on their respective domains,
	and all the component functions are $1$-$1$
	(since $\phi^{-1}$ is an injective map onto $\N$, since $\phi$ is a bijection),
	no two elements are mapped to the same value in $\N$.
	Thus, $\psi$ is $1$-$1$, and so $|A| + |B| = |A \cup B| \leq |\N|$.
	Then by Schroeder-Bernstein Theorem, we have $|A| + |B| = |\N|$,
	and since $A,B$ were arbitrary given their cardinality,
	we have in general $n + \aleph_0 = \aleph_0$.
\end{proof}
\item \begin{proof}[Solution]\let\qed\relax
	The corollary of Theorem 2.12 of Rudin states that the union of at most countably many at most countable sets
	is also at most countable.
	So if $A,B$ are arbitrary countable sets,
	$A \cup B$ is at most countable, thus
	$|A| + |B| = |A \cup B| \leq |\N|$.
	Now, since $A$ is countable, there exists a bijection from $\N$ to $A$,
	which is also a $1$-$1$ map from $\N$ to $A \cup B$
	(since every element in $A$ is also in $A \cup B$).
	Thus $|\N| \leq |A \cup B| = |A| + |B|$.
	So by Schroeder-Berstein Theorem,
	since $A,B$ were arbitrary sets such that $|A|=|B| = \aleph_0$,
	$\aleph_0 + \aleph_0 = \aleph_0$.
\end{proof}
\item \begin{proof}[Solution]\let\qed\relax
	Let $A,B$ be arbitrary sets such that $|A|=\aleph_0$, $|B|=c$,
	and $A\cap B = \emptyset$.
	First note that since $B \sim \R$,
	there exists a bijection from $B$.
	Thus, there is a $1$-$1$ map from $\R$ to $B$,
	and since $B$ is a subset of $A \cup B$,
	there is a $1$-$1$ from $\R$ to $A \cup B$ (just maps to the elements in $B$).
	Thus, $|\R| \leq |A\cup B| = |A| + |B|$.

	To show the other direction, let us denote the bijection from $B$ to $\R$
	by $\phi \colon B \to \R$,
	and the bijection from $A$ to $\N$ by $\psi \colon A \to \N$.
	We give a $1-1$ map $\Omega \colon A \cup B \to \R$ by
	\[
		\Omega(x) = \begin{cases}
			\psi(x) & x \in A\\
			\frac{1}{1+\phi(x)} & x \in B, \; \phi(x) > 0\\
			\frac{1}{\phi(x)-1} & x \in B, \; \phi(x) < 0\\
			0.1 & x \in B, \; \phi(x) = 0
		\end{cases}
	\]
	Since the range of $\psi(x)$ is $\N$, the range of $\frac{1}{1+\phi(x)}$ is $(0,1)$,
	the range of $\frac{1}{\phi(x)-1}$ is $(-1,0)$, and the range of the last function is $\{0,1\}$,
	on their respective domains,
	and all the component functions are themselves $1$-$1$ on these domains,
	no two elements are mapped to the same value in $\R$.
	Thus $\Omega$ is $1$-$1$,
	and so $|A| + |B| = |A \cup B| \leq |\R|$.
	Therefore, by Schroeder-Bernstein theorem, we have $|A| + |B| = |\R|$,
	and since $A$ and $B$ were arbitrary given their cardinalities,
	$\aleph_0 + c = c$.
\end{proof}
\item \begin{proof}[Solution]\let\qed\relax
	Let $A,B$ be arbitrary sets such that $|A|=|B|=c$,
	and $A\cap B = \emptyset$.
	First note that since $A \sim \R$,
	there exists a bijection from $A$.
	Thus, there is a $1$-$1$ map from $\R$ to $A$,
	and since $A$ is a subset of $A \cup B$,
	there is a $1$-$1$ from $\R$ to $A \cup B$ (just maps to the elements in $A$).
	Thus, $|\R| \leq |A\cup B| = |A| + |B|$.

	To show the other direction, let us denote the bijection from $B$ to $\R$
	by $\phi \colon B \to \R$,
	and the bijection from $A$ to $\R$ by $\psi \colon A \to \R$.
	We give a $1-1$ map $\Omega \colon A \cup B \to \R$ by
	\[
		\Omega(x) = \begin{cases}
			\frac{1}{1+\psi(x)} & x \in A, \; \psi(x) \geq 0\\
			\frac{1}{\psi(x)-1} & x \in A, \; \psi(x) < 0\\
			\frac{1}{1+\phi(x)} + 1& x \in B, \; \phi(x) \geq 0\\
			\frac{1}{\phi(x)-1} - 1& x \in B, \; \phi(x) < 0\\
		\end{cases}
	\]
	Since the range of $\frac{1}{1+\psi(x)}$ is $(0,1]$,
	the range of $\frac{1}{\psi(x)-1}$ is $-1,0)$,
	the range of $\frac{1}{1+\phi(x)}$ is $(1,2]$,
	and the range of $\frac{1}{\phi(x) - 1}$ is $(-2,-1)$
	on their respective domains,
	and all the component functions are themselves $1$-$1$ on these domains,
	no two elements are mapped to the same value in $\R$.
	Thus $\Omega$ is $1$-$1$,
	and so $|A| + |B| = |A \cup B| \leq |\R|$.
	Therefore, by Schroeder-Bernstein theorem, we have $|A| + |B| = |\R|$,
	and since $A$ and $B$ were arbitrary given their cardinalities,
	$c + c = c$.
\end{proof}

\end{enumerate}
\clearpage

\subsection*{Problem 8}
{\it For each family of real intervals $\mathcal{A}$ shown below,
find $\bigcap \mathcal{A}$ and $\bigcup\mathcal{A}$:
\begin{enumerate}
	\item $\mathcal{A} = \left\{\left[\dfrac{1}{n}, 1 - \dfrac{1}{n}\right]
	\colon n \in \N, n \geq 2\right\}$,
	\item $\mathcal{A} = \left\{\left(-1-\dfrac{1}{n}, 1 + \dfrac{1}{n}\right)
	\colon n \in \N,\right\}$.
\end{enumerate}
[Presentation: For each subproblem, define a set $S$
and verify that it equals the given combination by proving two inclusions.
You may assume the Archimedean Property.]}
\begin{enumerate}
\item \begin{proof}[Solution]\let\qed\relax
	We start with $\bigcap \mathcal{A}$.
	We claim $\bigcap \mathcal{A} = S$ where $S = \{\frac{1}{2}\}$.
	First, let $s \in S$.
	For all $2\leq n \in\N$,
	we have $\frac{1}{2} \geq \frac{1}{n}$ and $\frac{1}{2} \leq 1-\frac{1}{n}$.
	Thus $s = \frac{1}{2} \in [\frac{1}{n}, 1-\frac{1}{n}]$.
	So $s$ is in every set in $\mathcal{A}$, which means $s \in \bigcap\mathcal{A}$.
	Thus $S \subseteq \bigcap \mathcal{A}$.
	Let $\alpha \in \bigcap\mathcal{A}$.
	Then for all $2 \leq n \in \N$, we have $\alpha \in [\frac{1}{n}, 1-\frac{1}{n}]$.
	In other words, for any $n$, $\frac{1}{n} \leq \frac{1}{2} \leq \alpha$
	(it is to the right of the left bound of every set),
	and $1-\frac{1}{n} \geq \frac{1}{2} \geq \alpha$
	(it is to the left of the right bound of every set).
	But then $\alpha = \frac{1}{2}$.
	So $\alpha \in S$.
	Thus $\bigcap \mathcal{A} \subseteq S0$.
	Therefore, $S = \bigcap \mathcal{A}$.

	Now we look at $\bigcup \mathcal{A}$.
	We claim $\bigcup \mathcal{A} = S$ where $S = (0,1)$.
	We make use of the corollary of the Archimedian property:
	$\forall \ep > 0$, $\exists n\in\N$ such that $\frac{1}{n} < \ep$
	(from week 3 notes).
	Let $s \in S$.
	Then by the corollary, $\exists n_1\in\N$ such that $s > \frac{1}{n_1}$ (since $s$ is positive).
	Now note that $s < 1$, so $1-s > 0$.
	Thus, by the corollary,
	there exists $n_2$ such that $\frac{1}{n_2} < 1 - s$ or $s < 1 - \frac{1}{n_2}$.
	Let $n = \max\{n_1,n_2,2\}$.
	Note that the inequalities still hold,
	since if $n > n'$, then $s > \frac{1}{n'} \geq \frac{1}{n}$
	and $s < 1-\frac{1}{n'} \leq 1-\frac{1}{n}$.
	Thus, since we have some $2 \leq n \in \N$ such that $\frac{1}{n} < s < 1-\frac{1}{n}$.
	Then $s \in [\frac{1}{n}, 1-\frac{1}{n}]$, thus $s \in \bigcup \mathcal{A}$.
	Thus $S \subseteq \bigcup \mathcal{A}$.
	Now let $\alpha \in \bigcup \mathcal{A}$.
	Then for some $2 \leq n \in \N$, $\alpha \in [\frac{1}{n}, 1-\frac{1}{n}]$.
	Therefore $\alpha \geq \frac{1}{n} > 0$ and $\alpha \leq 1-\frac{1}{n} < 1$,
	thus $\alpha \in (0,1) = S$.
	Finally, this means $\bigcup \mathcal{A} \subseteq S$, so $S = \bigcup \mathcal{A}$.
\end{proof}
\item \begin{proof}[Solution]\let\qed\relax
	We start with $\bigcap \mathcal{A}$.
	We claim $\bigcap \mathcal{A} = S$ where $S = [-1,1]$.
	Let $s \in S$.
	Then for any $n \in \N$, $s \geq -1 > -1-\frac{1}{n}$ and $s \leq 1 <1+\frac{1}{n}$.
	But then $s$ is in every set in $\mathcal{A}$,
	so $s \in \bigcap \mathcal{A}$.
	Thus $S \subseteq \bigcap\mathcal{A}$.
	Now let $\alpha \in \bigcap \mathcal{A}$.
	Then for all $n \in \N$, $\alpha \in (-1-\frac{1}{n}, 1+\frac{1}{n})$.
	For the sake of contradiction, assume $\alpha > 1$.
	Then $\alpha - 1 > 0$, so by the corollary of Archimedes mentioned earlier,
	there exists $n \in \N$ such that $\frac{1}{n} < \alpha - 1 \implies
	1 + \frac{1}{n} < \alpha$,
	but this contradicts that $\alpha < 1 + \frac{1}{n}$ for all $n\in\N$.
	So we must have $\alpha \leq 1$.
	Again, for the sake of contradiction, assume $\alpha < -1$.
	Then $-\alpha - 1 > 0$, so by the corollary,
	there exists $n \in \N$ such that $\frac{1}{n} < -\alpha - 1
	\implies \alpha < -1 -\frac{1}{n}$,
	but this contradicts that $\alpha > -1-\frac{1}{n}$ for all $n \in \N$.
	So we must have $\alpha \geq -1$.
	Together, this gives us that $\alpha \in [-1,1] = S$.
	So $\bigcap\mathcal{A} \subseteq S$,
	therefore $\bigcap\mathcal{A} = S$.

	Now we look at $\bigcup \mathcal{A}$.
	We claim $\bigcup \mathcal{A} = S$ where $S = (-2, 2)$.
	Let $s \in S$.
	Any such $s$ is in one of the intervals in $\mathcal{A}$
	(namely when $n=1$: $(-1-1,1+1)$)
	and so $s$ is in the union of the intervals $\mathcal{A}$.
	Thus $S \subseteq \bigcup\mathcal{A}$.
	Now let $\alpha \in \bigcup \mathcal{A}$.
	Then for some $n\in\N$, $\alpha \in (-1-\frac{1}{n}, 1+\frac{1}{n})$.
	Therefore, $\alpha > -1-\frac{1}{n} \geq -2$ and $\alpha < 1 + \frac{1}{n} \leq 2$.
	Thus $\alpha \in (-2,2) = S$, and so $\bigcup \mathcal{A} \subseteq S$.
	Therefore, $\bigcup \mathcal{A} = S$.
\end{proof}
\end{enumerate}
\end{document}
