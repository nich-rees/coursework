\documentclass{article}
\usepackage{amsmath, amsfonts, amsthm, amssymb}
\usepackage{geometry}
\geometry{letterpaper, margin=2.0cm, includefoot, footskip=30pt}

\usepackage{fancyhdr}
\pagestyle{fancy}

\lhead{Math 321}
\chead{Notes}
\rhead{Nicholas Rees}
\cfoot{Page \thepage}

\newtheorem*{problem}{Problem}
\theoremstyle{plain}
\newtheorem{theorem}{Theorem}
\newtheorem{lemma}{Lemma}
\newtheorem{proposition}{Proposition}
\newtheorem{corollary}{Corollary}
\theoremstyle{remark}
\newtheorem{definition}{Definition}
\newtheorem{remark}{Remark}

\newcommand{\N}{{\mathbb N}}
\newcommand{\Z}{{\mathbb Z}}
\newcommand{\Q}{{\mathbb Q}}
\newcommand{\R}{{\mathbb R}}
\newcommand{\C}{{\mathbb C}}
\newcommand{\ep}{{\varepsilon}}
\newcommand{\SR}{{\mathcal R}}

\renewcommand{\theenumi}{(\alph{enumi})}

\begin{document}
\section{January 9}
\subsection{Complex Numbers}
Starts with $\N = \{1, 2, 3, \dots\}$.
We can solve $x + 2 = 5$ ($x = 3$), but we cannot solve $x + 5 = 2$.
So we introduce $\Z = \{ \dots, -2 , -1, 0 , 1 , 2, \dots\}$.
Now $x + a = b$ is always solvable in $\Z$ ($a,b \in \Z$), namely $x = b - a \in \Z$.
So consider $2x = 8$. This has the solution $x = 4 \in \Z$.
But it's easy to come up with equations like this that aren't solvable in $\Z$,
namely $8x = 2$.
So we enlarge our system of numbers to $\Q = \{\frac{p}{q} \mid p,q \in \Z, q \neq 0\}$.
Now we can solve $ax=b$ for $a,b \in \Q$ as long as $a \neq 0$.
\begin{remark}
	If we tried to add another number $\infty$ to $\Q$
	so that $\infty$ is a solution to $0x = 1$,
	this would lead to a breakdown of the rules of arithmetic because
	$0 \cdot a = 0$ for all $a$ by distributive law
	($0 \cdot a + 0 \cdot a = (0 + 0)\cdot a = 0\cdot a = 0 + 0\cdot a \implies 0\cdot a = 0$).
\end{remark}
We can now do linear algebra: in $\Q$, we can solve all linear equations and systems of linear equations.

From $\Q$ to $\R$: we want to do calculus.
Put in all limits of monotone increasing bounded sequences, e.g.
\[
	\lim_{n\to\infty} (1 + \frac{1}{n})^n = e \not\in \Q
\]
\[
	\lim_{n\to\infty} \sum_{i=1}^n \frac{1}{i^2}
	= \sum_{i=1}^\infty \frac{1}{i^2} = 1 + \frac{1}{4} + \frac{1}{9} + \cdots
	= \frac{pi^2}{6} \not\in \Q
\]
Actually, calculating the above limit is a highlight of this course.

As a consequence, we get the intermediate value theorem:
$f \colon [a,b] \to \R$ continuous, $f(a) < 0$, $f(b) > 0$,
then $\exists x \in (a,b)$ such that $f(x) = 0$.
Also the extremal value theorem:
$f \colon [a,b] \to \R$ continuous, then $\exists x \in [a,b]$ such that
$\forall y \in [a,b]$, $f(x) \geq f(y)$.
In particuluar, say $a > 0$, then $f(x) = x^2 - a$ on the interval $[0,1+a]$,
$f(0) = -a < 0$ and $f(1+a) = (1+a)^2 - a = 1+ a + a^2 > 2$,
so by the IVT: $\exists x \in \R$ such that $f(x) = 0$.
So $x^2 - a = 0$ has a solution in $\R$.
So we have a solution to this quadratic equation in $\R$.
The notation we use is $\sqrt{a}$.
Positive real numbers have square roots in $\R$.
So we can sole all quadratic equations $x^2 + bx + c = 0$ if $b^2 - 4c \geq 0$,
namely $x = -\frac{b}{2} \pm \frac{\sqrt{b^2 - 4c}}{2}$.

Now we go from $\R$ to $\C$: if $b^2 - 4c < 0$, we cannot solve $x^2+bx+c$ in $\R$.
\[
	x = -\frac{b}{2} \pm \frac{\sqrt{b^2 - 4c}}{2}
	= -\frac{b}{2} \pm \frac12 \sqrt{-1}\sqrt{4c-b^2}
\]
where $\sqrt{4c-b^2} \in \R$.
So we need to make sense of $\sqrt{-1}$,
and then we can solve all quadratic equations $x^2+bx+c=0$ where $b,c \in \R$.
We simply add the symbol $i := \sqrt{-1}$ to $\R$.
We then get the solutions $x = \alpha \pm i\beta$ where $\alpha = -\frac{b}{2}$
and $\beta = \frac12 \sqrt{4c - b^2}$ where $\alpha,\beta \in \R$.
We call $i$ the ``imaginary unit" and write numbers as $\alpha + i\beta$
where $\alpha,\beta \in \R$.
We do our calculations the usual way using the extra rule $i^2 = -1$.

Miracle: this leads to a coherent system of numbers $\C$,
the complex numbers,
where all quadratic equations can be solved, and we can do calculus (the contents of this course).

Some definitions of the operations:
\begin{align*}
	+ &\colon (a + ib) + (c + id) = (a+c) + i(b + d)\\
	\times &\colon (a+ib)(c+id) = (ac - bd) + i(ad + bc)
\end{align*}

Now, this was somewhat informal.
So formally, we define $\C = \R^2$ (assuming $\R$ is given).
Addition is the same as vector addition.
The multiplication is $(a,b)(c,d) = (ac-bd,ad+bc)$.
One can check that this multiplication is commutative, associative,
satisifies the distributive law,
there is a multiplicative unit $(1,0)$,
and every nonzero complex number has a multiplicative inverse:
$(a,b)^{-1} = \left(\frac{a}{a^2+b^2}, \frac{-b}{a^2+b^2}\right)$.
Hence, we can freely divide (multiplying by the multiplcative inverse)
by nonzero complex numbers.
So $\C$ is a field (see [BMPS]).

We can map $\R$ to $\C$ by $a \mapsto (a,0)$.
So geometrically, $\C$ is the plane and $\R$ is the $x$-axis.
This is a ``field morphism",
i.e. it respects addition and multiplication and sends the multiplication unit
to the multiplication unit
(so $(a\cdot b, 0) = (a,0)\cdot (b,0)$
and $(a+b,0) = (a,0) + (b,0)$).
We have $\alpha \in \R$, $(a,b) \in \C$,
scalar multiplication: $\alpha(a,b) = (\alpha a, \alpha b)$
and complex multiplication: $(\alpha,0)\cdot(a,b) = (\alpha a, \alpha \beta)$.
So we identify $\R$ with its image in $\C$.
Standard basis of $\C = \R^2$: $(1,0), (0,1)$.
We can abbreviat $1 = (1,0)$ and $i = (0,1)$.
Write $(a,b) = a(1,0) + b(0,1) = a1 + bi = a + ib$.
We can check that $i^2 = -1$: $(0,1)\cdot(0,1) = (-1,0) = -1$.

We write $z \in \C$ as $z = a+ib$, $a,b \in \R$.
We call $a$ the real part and $b$ the imaginary part,
and write $a = \mathrm{Re}(z), b = \mathrm{Im}(z)$.
$|a+ib| = \sqrt{a^2 = b^2}$ as the norm / absolute value / modulus of $z = a + ib$.

\subsubsection{Polar form}
It is often convenient to write complex numbers in a different form.
Imagining $z$ as a point on the Cartesian plane,
we let $r$ be the distance from the origin and $\theta$ the angle $z$ sweeps out.
We can compute $a = r\cos\theta$ and $b = r\sin\theta$.
So $a+ib = r\cos\theta + ir\sin\theta = r(\cos\theta + i\sin\theta)$.
$r$ is the modulus of $a + ib$ and we call $\theta$ the argument.
The argument is ambiguous, but we can restrict $\theta \in (-\pi,\pi]$,
which is called the principal value of the argument.
$r(\cos\theta + i\sin\theta) = s(\cos\phi + i\sin\phi)$ if and only if
$r = s$ and $\phi - \theta \in 2\pi\Z$.

With this, we can get a geometric meaning of multiplication.
Fix $z = r(\cos\theta +i \sin\theta)$.
Consider the ``multiplying by $z$" map $\C \to \C$ where $w \mapsto zw$.
Write $(x,y)$ as $\binom{x}{y}$.
\begin{align*}
	w = \binom{x}{y} &\mapsto r(\cos\theta + i\sin\theta)\cdot\binom{x}{y}\\
	&= r(\cos\theta + i\sin\theta)\cdot(x+iy)\\
	&= rx\cos\theta - ry\sin\theta + i(yr\cos\theta + xr\sin\theta)\\
	&= \binom{rx\cos\theta - ry\sin\theta}{ry\cos\theta+rx\sin\theta}\\
	&= \begin{pmatrix} r\cos\theta & -r\sin\theta\\ r\sin\theta & r\cos\theta \end{pmatrix}
	\binom{x}{y}\\
	&= r\begin{pmatrix} \cos\theta & -\sin\theta\\ \sin\theta & \cos\theta \end{pmatrix}
	\binom{x}{y}
\end{align*}
where we have the rotation matrix.
So we are scaling $w$ by the modulus $r$ and rotating it by the argument $\theta$.

\section{January 11}
More about the argument:
let $z = 1(\cos{\pi/3} + i\sin{\pi/3})$.
The possible values of $\mathrm{arg}(z)$ are
$\dots, \pi/3 - 2\pi, \pi/3, \pi/3 + 2\pi, \pi/3+4\pi, \dots = \pi/3 + 2\pi\Z$
where $2\pi\Z = 2\pi\{\dots, -1,0,1,2,\dots\} = \{\dots, -2\pi, 0, 2\pi, 4\pi, \dots\}$.
Then $\mathrm{Arg}(z) = \pi/3 + 2\pi\Z
= \{\dots, \pi/3 - 2\pi, \pi/3, \pi/3 + 2\pi, \pi/3 + 4\pi, \dots\}$.
Hence, $\mathrm{Arg}(z) :=$ the multivalued argument of $z$.
It is an example of a multifunction which associates to each $z \in \C$
a \emph{set} of complex numbers.
So $\mathrm{Arg}(\frac12 + \frac12i\sqrt{3}) = \frac{\pi}{3} + 2\pi\Z$.
The \emph{principal argument} of $z$ is the unique $\pi \in \mathrm{Arg}(z)$
such that $-\pi < \theta \leq \pi$.
Notation: $\mathrm{arg}(z)$ is the principal argument.
(Warning: other sources use different notation.)
E.g. $\mathrm{arg}(-1) = \pi$ but $\mathrm{Arg}(-1) = \pi + 2\pi\Z$.

\subsection{Complex Conjugation}
\begin{definition}[Complex Conjugate]
	$\overline{a+ib} := a-ib$ (reflection across the real axis).
\end{definition}
Some properties of the conjugate:
\begin{itemize}
	\item $\overline{z+w} = \overline{z} + \overline{w}$
	\item $\overline{zw} + \overline{w}\overline{z}$
	\item $\lvert z \rvert^2 = z\overline{z}$
\end{itemize}
See that if $z = a + ib$, then
\[
	z\overline{z} = (a+ib)(a-ib) = a^2 - i^2b + aib - aib = a^2 + b^2 = |a+ib|^2
\]

\subsubsection{The standard way to divide complex numbers}
We have previously defined the multiplicative inverse of a complex number,
but the standard way to divide is actually using the conjugate.
We have
\[
	\frac{a+ib}{c+id} = \frac{(a+ib)(c-id)}{(c+id)(c-id)}
	= \frac{ac + bd + i(bc-ad)}{c^2 + d^2}
	= \frac{ac + bd}{c^2 + d^2} + i \frac{bc - ad}{c^2 + d^2}
\]

\subsection{$n$th roots of complex numbers}
\begin{proposition}[De Moivre's formula]
	\[
		z  = r(\cos\theta + i\sin\theta)
	\]
	\[
		z^n = r^n(\cos(n\theta) + i\sin(n\theta))
	\]
\end{proposition}
To find a third root of $z = r(\cos\theta + i\sin\theta)$,
divide $\theta$ by $3$ and extract a third root of $r$
(can always find a real $n$th root because $r \geq 0$):
\[
	w_1 = \sqrt[3]{r}\left(\cos\frac{\theta}{3} + i\sin\frac{\theta}{3}\right)
\]
then $w_1^3 = z$.
But there are in fact $3$ 3rd roots of $z$.
The others are found by dividing the circle with radius $\sqrt[3]{r}$ equally:
\[
	w_2 = \sqrt[3]{r}\left(\cos\left(\frac{\theta}{3} + \frac{2\pi}{3}\right)
		+ i\sin\left(\frac{\theta}{3} + \frac{2\pi}{3}\right)\right)
\]
\[
	w_3 = \sqrt[3]{r}\left(\cos\left(\frac{\theta}{3} + \frac{4\pi}{3}\right)
		+ i\sin\left(\frac{\theta}{3} + \frac{4\pi}{3}\right)\right)
\]

Another example $(1+i)^8 = 1 + 8i + \binom{8}{2}i^2 + \binom{8}{3}i^4 + \cdots$
which isn't something we want to work with.
Since $1+i = \sqrt{2}\left(\cos\frac{\pi}{4} + i\sin\frac{\pi}{4}\right)$,
we can actually write (and compute) it much easier:
\[
	(1+i)^8 = \sqrt{2}^8\left(\cos\left(8\frac{\pi}{4}\right) +
		i\sin\left(8\frac{\pi}{4}\right)\right)
		= 16 \left(\cos(2\pi) + i\sin(2\pi)\right) = 16
\]
The fact that this is a real number is because $1+i$
is a vertex of the octagon that has a vertex along the $x$-axis.

\subsection{Phase}
The \emph{phase} of $z \neq 0$ is $\frac{z}{|z|}$.
The phase of $z$ is the complex number of modulus $1$ with the same argument.
The phase keeps track of the ``angle" without the ambiguity in the argument.

A phase portraint is used for visualization.
We associate colours to the phases.
Red is associated with the positive real numbers,
green with $\theta = 2\pi/3$ and blue with $-2\pi/3$.
Then yellow is $\pi/3$, magenta is $-\pi/3$,
and cyan is the negative real numbers.

\subsection{The complex exponential}
Recall for $x \in \R$,
\[
	e^x = 1 + x + \frac12x^2 + \frac16x^3 + \cdots = \sum_{n=0}^\infty \frac{1}{n!}x^n
\]
This also works for complex numbers. If $z \in \C$:
\[
	e^z = 1 + z + \frac12z^2 + \cdots = \sum_{n=0}^\infty \frac{1}{n!}z^n
\]
For $z = i\theta$ where $\theta \in \R$ (purely imaginary) we get
\begin{align*}
	e^z &= e^{i\theta} = 1 + i\theta - \frac12 \theta^2 -
	\frac{1}{3!}i\theta^3 + \frac{1}{4!}\theta^4 + \frac{1}{5!}i\theta^5 +- \cdots\\
		&= 1 - \frac12\theta^2 + \frac{1}{4!}\theta^4 - \frac{1}{6!}\theta^6
	+ \cdots +
	i\left(\theta - \frac{1}{3!}\theta^3 + \frac{1}{5!}\theta^5 + \cdots\right)\\
		&= \cos\theta + i\sin\theta
\end{align*}
So if all these infinte sums behave proerply,
we deduce from this Euler's formula:
\[
	e^{i\theta} = \cos\theta + i\sin\theta
\]
From now on, $z$ in polar coordinates will be written
$z = r(\cos\theta + i\sin\theta) = re^{i\theta}$
where $r$ is the modulus and $\theta$ is the argument from before.
We get the famous identity with this formula:
\[
	e^{2\pi i} = 1
\]
In fact, $e^{2\pi i \Z} = \cos(2\pi\Z) + i\sin(2\pi\Z) = 1$.
The complex exponential function ha speriod $2\pi i$:
\[
	e^{z + 2\pi i} = e^ze^{2\pi i} = e^z\cdot 1 = e^z
\]

We can also introduce the complex $\sin$ and $\cos$ functions by their power series:
\begin{align*}
	\cos(z) &= 1 - \frac{1}{2}z^2 + \frac{1}{4!}z^4 -+ \cdots\\
	\sin(z) &= z - \frac{1}{3!}z^3 +- \cdots
\end{align*}
Then $e^{iz} = \cos{z} + i\sin{z}$.

Also $\overline{e^z} = \overline{sum_{n=0}^\infty \frac{1}{n!}z^n}
= \sum_{n=0}^\infty \frac{1}{n!}\overline{z}^n = e^{\overline{z}}$
so $e^{-iz} = \cos{z} - i\sin{z}$.
From these, we get
\begin{align*}
	\cos(z) &= \frac12\left(e^{iz} + e^{-iz}\right)\\
	\sin(z) &= \frac{1}{2i}\left(e^{iz} - e^{-iz}\right)
\end{align*}

\section{January 16}
\subsection{$n$th roots of unity}
The $n$th roots of the unity are the $n$ complex numbers $\omega_0,\dots,\omega_{n-1}$
such that $\omega_0^n = 1, \dots, \omega_{n-1}^n = 1$ are the $n$th roots of unity.
We let $\omega_0 = 1$. $\omega_0,\dots,\omega_{n-1}$ form a regular $n$-gon
with vertices on the unit circle.
For example, when $n = 3$, we have
$\omega_1 = e^{2\pi i/2}, \omega_2 = e^{4\pi i/2}, \omega_0 = 1$.

If $z$ is a complex number and $w_0^n = z$ so $w_0$ is one $n$th root of $z$,
then the others are $w_1 = \omega_1w_0, \dots, w_{n-1} = \omega_{n-1}w_0$,
obtained by multiplying $w_0$ by the $n$th roots of unity.

Example:
\[
	\sum_{n=0}^{100} i^n =  i^0 + i^1 + \cdots _ i^{100}
	= 1 + i + (-1) + (-i) + \cdots + 1
	= 26\cdot 1 + 25\cdot i + 25(-1) + 25(-i)
	= 26 - 25 = 1
\]
or we can use the geometric series formula
$1 + z + z^2 + \cdots + z^n = \frac{1-z^{n+1}}{1-z}$ to get
$\sum_{n=0}^{100}i^n = \frac{1-i^{101}}{1-i} = 1$.

Example: Write $-1 + 2i$ in polar form (with principle value of $\theta$).
Compute $r = \lvert -1 + 2i \rvert = \sqrt{1 + 4} = \sqrt{5}$.
So $-1 + 2i = \sqrt{5}\left(\cos\theta + i\sin\theta\right)$
so $-1 = \sqrt{5}\cos\theta$ and $2 = \sqrt{5}\sin\theta$.
So $\tan \theta = \frac{\sin\theta}{\cos\theta} = -2$,
which gives $\theta = \tan^{-1}(-2) + \pi$
(since $\tan^{-1}$ gives angles between $-\frac{\pi}{2}$ and $\frac{\pi}{2}$
and $-1 + 2i$ is in the second quadrant).
If it had been in thee third quadrant, we would subtract $\pi$.
This because the principle argument of a complex number is between $\pi$ and $-\pi$.

\subsection{Complex Functions}
There are two types:
\begin{enumerate}
	\item[(1).] Domain $[a,b] \subset \R$ is an interval in $\R$
		(codomain is $\C$).
		``A path in $\C$".
		$\gamma \colon [a,b] \to \C$, or $t \mapsto \gamma(t)$.
		Example: $\gamma \colon [0,1] \to \C$ by $t \mapsto 1 + it^2$.
	\item[(2).] Complex functions: the domain is a subset of $\C$,
		so $f \colon D \to \C$ where $D \subset \C$, or $z \mapsto f(z)$
		(and codomain is $\C$).
		Example: $\sin \colon \C \to \C$ by $z \mapsto \sin{z} = z - \frac{1}{3!}z^3 +- \cdots$.
\end{enumerate}
Visualising complex functions $f \colon \C \to \C$: analytic landscapes.
Consider the absolute value function $|f| \colon \C \to \R_{\geq 0}$
by $z \mapsto |f(z)|$ which we can draw with a $3$-dimensional graph;
then we colour by the phase/argument of $f(z)$.
Example: $f(z) = z^2$ where $z = x + iy$.
Then $|f(z)| = |z|^2 = x^2 + y^2$.
Then paraboloid $z = x^2 + y^2$ is the analytic landscape of $f(z) = z^2$.
As $z$ goes around the origin once, the phase of $f(z) = z^2$ goes around the origin twice:
so there is a double rainbow in the phase portrait of $f(z) = z^2$.
This is the phase portrait:
for $f \colon \C \to \C$, colour each $z \in \C$ with the phase of $f(z)$
(so $\frac{f(z)}{|f(z)|} \leftrightarrow$ colour).

\subsection{Limits}
We want to make sense of $\lim_{z\to z_0}f(z) + L$
($f\colon D \to \C$ is a complex function and $D \subset \C$)
where $z_0 \in \C$ but not necessarily in $D$, and $L \in \C$.
\begin{definition}
	$\lim_{z\to z_0} f(z) = L$ means that for all $\ep > 0$,
	there exists a $\delta > 0$ such that
	\[
		0 < |z - z_0| < \delta \implies |f(z) - L| < \ep
	\]
\end{definition}
So for every disc $D(L,\ep)$ with centre $L$, there exists a disc $D(z_0,\delta)$
with centre $z_0$ such that $f$ maps the punctured disc $D'(z_0,\delta)$ into $D(L,\ep)$.

\begin{definition}
	If $z_0 \in D$ and $\lim_{z\to z_0}f(z) = f(z_0)$ then $f$ is \emph{continuous} at $z_0$.
\end{definition}

Example: $\lim_{z\to 0}\frac{\overline{z}}{z}$ does not exist.
In this case, our $D = \C \setminus \{0\}$.
If $z \neq 0$ is real, $\overline{z} = z$ so $f(z) = 1$.
If $z$ is purely imaginary, $z = iy, y\in\R$, $\overline{z} = -iy$
so $\frac{\overline{z}}{z} = \frac{-iy}{iy} = -1$.
For example, $\lim_{z\to0}\overline{z}/z \neq 0$,
since no $\delta$ exists: if $\ep = \frac12$,
will always have $f(z)$ when $z$ is real outside of $D(0,\delta)$.

\section{January 23}
Recall we were looking at complex functions $f \colon \C \to \C$
or $f \colon D \to \C$ where $D \subset \C$
(both inputs and outputs are complex numbers).
We can write the input and output in terms of real and imaginary parts:
$f(z) = w = u + iv$ where $z = x + iy$.
We can define $u,v \colon \R^2 \to \R$ to be functions
that give the real and imaginary parts of $f$, respectively,
i.e. $f(x+iy) = u + iv = u(x+iy) + iv(x+iy) = u(x,y) + iv(x,y)$.
Then $f(z) = \langle u(x,y), v(x,y) \rangle$ can be viewed as a planar vector field.

Example: $f(z) = \sin{z}$.
When dealgin with complex trig. functions, it is good to convert to exponentials.
\begin{align*}
	\sin{z} &= \frac{1}{2i}\left(e^{iz} - e^{-iz}\right)\\
			&= \frac{1}{2i} \left(e^{i(x+iy)} - e^{-i(x+iy)}\right)\\
			&= \frac{1}{2i}\left(e^{ix - y} - e^{-ix + y}\right)\\
			&= \frac{1}{2i}\left(e^{ix}e^{-y} - e^{-ix}e^y\right)\\
			&= \frac{1}{2i}\left(e^{-y}(\cos{x}+i\sin{x})
				- e^y(\cos{x} - i\sin{x})\right)\\
			&= \frac{1}{2i}\left(\left(e^{-y}-e^y\right)\cos{x}
				+ i\left(e^{-y}+e^y\right)\sin{x}\right)\\
			&= \frac{-i}{2}\left(\left(e^{-y}-e^y\right)\cos{x}
				+ i \left(e^{-y}+e^y\right)\sin{x}\right)\\
			&= \frac{e^y + e^{-y}}{2}\sin{x} + i\frac{e^y-e^{-y}}{2}\cos{x}
\end{align*}
So we have $u(x,y) = \frac{e^y + e^{-y}}{2}\sin{x} = \cosh{y}\sin{x}$
and $v(x,y) = \frac{e^y-e^{-y}}{2}\cos{x} = \sinh{y}\cos{x}$
(we won't ever use these function names in this class).
As a vector field, $\sin(z) =
\langle \frac{e^y+e^{-y}}{2}\sin{x}, \frac{e^y-e^{-y}}{2}\cos{x}\rangle$.

\subsection{Infinite Limits}
Recall $\lim_{z\to z_0}f(z) = L$ means that
$\forall \ep > 0$, $\exists \delta > 0$ such that
$0 < \lvert z-z_0\rvert < \delta \implies \lvert f(z) - L \rvert < \ep$.
What does this condition mean for sequences:
 suppose we have a sequence of complex numbers $z_1,z_2,z_3,\dots$
 converging to $z_0$ (but avoiding $z_0$).
 it will eventually have to be within $\delta$ of $z_0$, say from $N$ onward.
 So $0 < \lvert z_n - z_0 \rvert < \delta$ for all $n \geq N$,
 then we must have $\lvert f(z_n) - L \rvert < \ep$ for all $n \geq N$.
 But this is true for all $\ep > 0$, no matter how small,
 so the sequence $f(z_1),f(z_2),\dots$ converges to $L$.
 \emph{This is true no matter in wchi direction $z_n \to z_0$.}

 The \emph{extended complex plane} is $\C \cup \{\infty\}$,
 usually denoted $\hat{\C}$.
 It can be identified with the Riemann sphere by stereograhpic pojection:
 for every point on the sphere, we have a $1$-$1$ correspondance
 with a point on the complex plane by taking the place where
 the ray from the top of the sphere that intersects the desired point of the sphere
 intersects the complex plan
 (his sphere has $0$ as its origin; so the points inside the unit circle
 correspond to points in the southern hemisphere).
 The only point on the sphere that doesn't have a corresponding point in $\C$
 is the north pole $N$, but points ``very close" to $N$ on the sphere
 correspond to points outside a circle of large radius in $\C$.
 This leads to the following definitions:
 $\lim_{z \to z_0}f(z) = \infty$ means $\forall M > 0$, $\exists \delta > 0$
 such that $0 < \lvert z - z_0 \rvert < \delta \implies \lvert f(z) \rvert > M$.

 Example: $\lim_{z \to 0}\frac{1}{z} = \infty$ where
 $z = re^{i\theta}$ so $\frac{1}{z} = \frac{1}{r}e^{-i\theta}$.
 Algebraic proof: given $M > 0$, let $\delta = \frac{1}{M} > 0$.
 Then $0 < \lvert z \rvert < \delta \implies 0 < \lvert z \rvert < M
 \implies \frac{1}{|z|} > M \implies \left\lvert \frac{1}{z} \right\rvert > M$.

 Some other definitions:
 $\lim_{z \to \infty} f(z) = L$ means $\forall \ep > 0$, $\exists M > 0$
 such that $\lvert z \rvert > M \implies \lvert f(z) - L \rvert < \ep$;
 and $\lim_{z \to \infty} f(z) = \infty$ means $\forall M > )$, $\exists N > 0$
 such that $\lvert z \rvert > N \implies \lvert f(z) \rvert > M$.

\begin{remark}
	All the rules for computing limits of sums, products, quotients remain true
	if we stipulate:
	$\infty + a = \infty$ for all $a \in \C$ ($\infty+\infty$ is not defined),
	$\infty \cdot a = \infty$ for all $a \in \C\setminus\{0\}$
	($\infty\cdot0$ is not defined),
	$\frac{a}{\infty} = 0, \frac{a}{0} = \infty$ for ff,
	and $\frac{\infty}{0} = \infty$, $\frac{0}{\infty}=0$, $\infty\infty = \infty$.
\end{remark}

Example 1: $\frac{0}{\infty} = 0$ means if $\lim_{z \to z_0}f(z) = 0$ and
$\lim_{z\to z_0}g(z) = \infty$,
then $\lim_{z\to z_0}\frac{f(z)}{g(z)} =
\frac{\lim_{z\to z_0}f(z)}{\lim_{z\to z_0}g(z)} = \frac{0}{\infty} = 0$.

Example 2: $f(z) = \frac{1}{z}, g(z) = -\frac{1}{z}$.
$\lim_{z \to 0} f(z) = \lim_{z \to 0} \frac{1}{z}
= \frac{\lim_{z \to 0} 1}{\lim_{z \to 0} z} = \frac{1}{0} = \infty$.
Similarly, $\lim_{z \to 0} g(z) = \lim_{z \to 0} -\frac{1}{z}
= \frac{-1}{0} = \infty$.
And then
\[
	0 = \lim_{z \to 0}0 = \lim_{z \to 0}\left(\frac{1}{z}-\frac{1}{z}\right)
	\lim_{z\to0}\left(f(z) + g(z)\right) =
	\lim_{z \to 0}f(z) + \lim_{z \to 0}g(z) = \infty + \infty = \text{ not defined}
\]
which is why $\infty + \infty$ is not defined,
because $\infty + \infty = \lim_{z\to\infty}\left(f(z) + f(z)\right) = \infty$ as well.

\subsection{M\"{o}bius transformations}
\begin{definition}[M\"{o}bius transformation]
	A \emph{M\"{o}bius transformation} is a linear fractional transformation
	$f \colon \hat{\C} \to \hat{\C}$
	\[
		f(z) = \frac{az + b}{cz + d}
	\]
	$a,b,c,d \in \C$, $ad - bc \neq 0$.
\end{definition}
Why do we require $ad - bc \neq 0$?
Well, if $ad = bc$, then
$c \neq 0$ implies $b = \frac{ad}{c}$ and
$\frac{az+b}{cz+d} = \frac{caz + ad}{c^2z+cd} = \frac{a}{c}\frac{cz+d}{cz+d} = \frac{a}{c}$
so $f(z)$ is constant;
$c = 0$ implies that if $d \neq 0$ then $a = 0$ and so
$f(z) = \frac{az+b}{cz+d} = \frac{b}{d}$ a constant again,
or if $d = 0$ then $f(z) = \frac{az+b}{0}$ which is not defined (or constant $\infty$).
So $ad - bc \neq 0$ assures that $f$ is not constant.

Define $f(z) = \frac{az+b}{cz+d}$ as a function $\hat{\C} \to \hat{\C}$:
\begin{enumerate}
	\item $cz + d = 0$.
		When $c \neq 0$ then the denominator vanishes at $z = -d/c$,
		so
		\[
			\lim_{z \to -d/c} \frac{az+b}{cz+d}
			= \frac{\displaystyle\lim_{z\to-d/c}(az+b)}{\displaystyle\lim_{z\to-d/c}(cz+d)}
			= \frac{-\frac{ad}{c}+b}{0}
			= \frac{\frac{-ad+bc}{c}}{0} = \infty
		\]
		since $\frac{-ad+bc}{c} \neq 0$.
		When $c = 0$ then $f(z) = \frac{az+b}{d}$ with $d\neq0$ so no problem with denominator.
	\item $f(\infty) =$ ?.
\end{enumerate}
ff I think I'm done for today, I will just fill in examples later.

ff

In fact, every rotation of the Riemann sphere is a M\"{o}bius transformation.
We can iterate $f$ and draw trajectories.
Here the trajectories are circles (the second picture in the assigned reading).
But there are other classes of transformations than just iteration of one???
\end{document}
