\documentclass{article}
\usepackage{amsmath, amsfonts, amsthm, amssymb}
\usepackage{geometry}
\geometry{letterpaper, margin=2.0cm, includefoot, footskip=30pt}

\usepackage{fancyhdr}
\pagestyle{fancy}

\lhead{Math 321}
\chead{Notes}
\rhead{Nicholas Rees}
\cfoot{Page \thepage}

\newtheorem*{problem}{Problem}
\theoremstyle{plain}
\newtheorem{theorem}{Theorem}
\newtheorem{lemma}{Lemma}
\newtheorem{proposition}{Proposition}
\newtheorem{corollary}{Corollary}
\theoremstyle{remark}
\newtheorem{definition}{Definition}
\newtheorem{remark}{Remark}

\newcommand{\N}{{\mathbb N}}
\newcommand{\Z}{{\mathbb Z}}
\newcommand{\Q}{{\mathbb Q}}
\newcommand{\R}{{\mathbb R}}
\newcommand{\C}{{\mathbb C}}
\newcommand{\ep}{{\varepsilon}}
\newcommand{\SR}{{\mathcal R}}

\renewcommand{\theenumi}{(\alph{enumi})}

\begin{document}
\section{January 9}
\subsection{Complex Numbers}
Starts with $\N = \{1, 2, 3, \dots\}$.
We can solve $x + 2 = 5$ ($x = 3$), but we cannot solve $x + 5 = 2$.
So we introduce $\Z = \{ \dots, -2 , -1, 0 , 1 , 2, \dots\}$.
Now $x + a = b$ is always solvable in $\Z$ ($a,b \in \Z$), namely $x = b - a \in \Z$.
So consider $2x = 8$. This has the solution $x = 4 \in \Z$.
But it's easy to come up with equations like this that aren't solvable in $\Z$,
namely $8x = 2$.
So we enlarge our system of numbers to $\Q = \{\frac{p}{q} \mid p,q \in \Z, q \neq 0\}$.
Now we can solve $ax=b$ for $a,b \in \Q$ as long as $a \neq 0$.
\begin{remark}
	If we tried to add another number $\infty$ to $\Q$
	so that $\infty$ is a solution to $0x = 1$,
	this would lead to a breakdown of the rules of arithmetic because
	$0 \cdot a = 0$ for all $a$ by distributive law
	($0 \cdot a + 0 \cdot a = (0 + 0)\cdot a = 0\cdot a = 0 + 0\cdot a \implies 0\cdot a = 0$).
\end{remark}
We can now do linear algebra: in $\Q$, we can solve all linear equations and systems of linear equations.

From $\Q$ to $\R$: we want to do calculus.
Put in all limits of monotone increasing bounded sequences, e.g.
\[
	\lim_{n\to\infty} (1 + \frac{1}{n})^n = e \not\in \Q
\]
\[
	\lim_{n\to\infty} \sum_{i=1}^n \frac{1}{i^2}
	= \sum_{i=1}^\infty \frac{1}{i^2} = 1 + \frac{1}{4} + \frac{1}{9} + \cdots
	= \frac{pi^2}{6} \not\in \Q
\]
Actually, calculating the above limit is a highlight of this course.

As a consequence, we get the intermediate value theorem:
$f \colon [a,b] \to \R$ continuous, $f(a) < 0$, $f(b) > 0$,
then $\exists x \in (a,b)$ such that $f(x) = 0$.
Also the extremal value theorem:
$f \colon [a,b] \to \R$ continuous, then $\exists x \in [a,b]$ such that
$\forall y \in [a,b]$, $f(x) \geq f(y)$.
In particuluar, say $a > 0$, then $f(x) = x^2 - a$ on the interval $[0,1+a]$,
$f(0) = -a < 0$ and $f(1+a) = (1+a)^2 - a = 1+ a + a^2 > 2$,
so by the IVT: $\exists x \in \R$ such that $f(x) = 0$.
So $x^2 - a = 0$ has a solution in $\R$.
So we have a solution to this quadratic equation in $\R$.
The notation we use is $\sqrt{a}$.
Positive real numbers have square roots in $\R$.
So we can sole all quadratic equations $x^2 + bx + c = 0$ if $b^2 - 4c \geq 0$,
namely $x = -\frac{b}{2} \pm \frac{\sqrt{b^2 - 4c}}{2}$.

Now we go from $\R$ to $\C$: if $b^2 - 4c < 0$, we cannot solve $x^2+bx+c$ in $\R$.
\[
	x = -\frac{b}{2} \pm \frac{\sqrt{b^2 - 4c}}{2}
	= -\frac{b}{2} \pm \frac12 \sqrt{-1}\sqrt{4c-b^2}
\]
where $\sqrt{4c-b^2} \in \R$.
So we need to make sense of $\sqrt{-1}$,
and then we can solve all quadratic equations $x^2+bx+c=0$ where $b,c \in \R$.
We simply add the symbol $i := \sqrt{-1}$ to $\R$.
We then get the solutions $x = \alpha \pm i\beta$ where $\alpha = -\frac{b}{2}$
and $\beta = \frac12 \sqrt{4c - b^2}$ where $\alpha,\beta \in \R$.
We call $i$ the ``imaginary unit" and write numbers as $\alpha + i\beta$
where $\alpha,\beta \in \R$.
We do our calculations the usual way using the extra rule $i^2 = -1$.

Miracle: this leads to a coherent system of numbers $\C$,
the complex numbers,
where all quadratic equations can be solved, and we can do calculus (the contents of this course).

Some definitions of the operations:
\begin{align*}
	+ &\colon (a + ib) + (c + id) = (a+c) + i(b + d)\\
	\times &\colon (a+ib)(c+id) = (ac - bd) + i(ad + bc)
\end{align*}

Now, this was somewhat informal.
So formally, we define $\C = \R^2$ (assuming $\R$ is given).
Addition is the same as vector addition.
The multiplication is $(a,b)(c,d) = (ac-bd,ad+bc)$.
One can check that this multiplication is commutative, associative,
satisifies the distributive law,
there is a multiplicative unit $(1,0)$,
and every nonzero complex number has a multiplicative inverse:
$(a,b)^{-1} = \left(\frac{a}{a^2+b^2}, \frac{-b}{a^2+b^2}\right)$.
Hence, we can freely divide (multiplying by the multiplcative inverse)
by nonzero complex numbers.
So $\C$ is a field (see [BMPS]).

We can map $\R$ to $\C$ by $a \mapsto (a,0)$.
So geometrically, $\C$ is the plane and $\R$ is the $x$-axis.
This is a ``field morphism",
i.e. it respects addition and multiplication and sends the multiplication unit
to the multiplication unit
(so $(a\cdot b, 0) = (a,0)\cdot (b,0)$
and $(a+b,0) = (a,0) + (b,0)$).
We have $\alpha \in \R$, $(a,b) \in \C$,
scalar multiplication: $\alpha(a,b) = (\alpha a, \alpha b)$
and complex multiplication: $(\alpha,0)\cdot(a,b) = (\alpha a, \alpha \beta)$.
So we identify $\R$ with its image in $\C$.
Standard basis of $\C = \R^2$: $(1,0), (0,1)$.
We can abbreviat $1 = (1,0)$ and $i = (0,1)$.
Write $(a,b) = a(1,0) + b(0,1) = a1 + bi = a + ib$.
We can check that $i^2 = -1$: $(0,1)\cdot(0,1) = (-1,0) = -1$.

We write $z \in \C$ as $z = a+ib$, $a,b \in \R$.
We call $a$ the real part and $b$ the imaginary part,
and write $a = \mathrm{Re}(z), b = \mathrm{Im}(z)$.
$|a+ib| = \sqrt{a^2 = b^2}$ as the norm / absolute value / modulus of $z = a + ib$.

\subsubsection{Polar form}
It is often convenient to write complex numbers in a different form.
Imagining $z$ as a point on the Cartesian plane,
we let $r$ be the distance from the origin and $\theta$ the angle $z$ sweeps out.
We can compute $a = r\cos\theta$ and $b = r\sin\theta$.
So $a+ib = r\cos\theta + ir\sin\theta = r(\cos\theta + i\sin\theta)$.
$r$ is the modulus of $a + ib$ and we call $\theta$ the argument.
The argument is ambiguous, but we can restrict $\theta \in (-\pi,\pi]$,
which is called the principal value of the argument.
$r(\cos\theta + i\sin\theta) = s(\cos\phi + i\sin\phi)$ if and only if
$r = s$ and $\phi - \theta \in 2\pi\Z$.

With this, we can get a geometric meaning of multiplication.
Fix $z = r(\cos\theta +i \sin\theta)$.
Consider the ``multiplying by $z$" map $\C \to \C$ where $w \mapsto zw$.
Write $(x,y)$ as $\binom{x}{y}$.
\begin{align*}
	w = \binom{x}{y} &\mapsto r(\cos\theta + i\sin\theta)\cdot\binom{x}{y}\\
	&= r(\cos\theta + i\sin\theta)\cdot(x+iy)\\
	&= rx\cos\theta - ry\sin\theta + i(yr\cos\theta + xr\sin\theta)\\
	&= \binom{rx\cos\theta - ry\sin\theta}{ry\cos\theta+rx\sin\theta}\\
	&= \begin{pmatrix} r\cos\theta & -r\sin\theta\\ r\sin\theta & r\cos\theta \end{pmatrix}
	\binom{x}{y}\\
	&= r\begin{pmatrix} \cos\theta & -\sin\theta\\ \sin\theta & \cos\theta \end{pmatrix}
	\binom{x}{y}
\end{align*}
where we have the rotation matrix.
So we are scaling $w$ by the modulus $r$ and rotating it by the argument $\theta$.

\section{January 11}
More about the argument:
let $z = 1(\cos{\pi/3} + i\sin{\pi/3})$.
The possible values of $\mathrm{arg}(z)$ are
$\dots, \pi/3 - 2\pi, \pi/3, \pi/3 + 2\pi, \pi/3+4\pi, \dots = \pi/3 + 2\pi\Z$
where $2\pi\Z = 2\pi\{\dots, -1,0,1,2,\dots\} = \{\dots, -2\pi, 0, 2\pi, 4\pi, \dots\}$.
Then $\mathrm{Arg}(z) = \pi/3 + 2\pi\Z
= \{\dots, \pi/3 - 2\pi, \pi/3, \pi/3 + 2\pi, \pi/3 + 4\pi, \dots\}$.
Hence, $\mathrm{Arg}(z) :=$ the multivalued argument of $z$.
It is an example of a multifunction which associates to each $z \in \C$
a \emph{set} of complex numbers.
So $\mathrm{Arg}(\frac12 + \frac12i\sqrt{3}) = \frac{\pi}{3} + 2\pi\Z$.
The \emph{principal argument} of $z$ is the unique $\pi \in \mathrm{Arg}(z)$
such that $-\pi < \theta \leq \pi$.
Notation: $\mathrm{arg}(z)$ is the principal argument.
(Warning: other sources use different notation.)
E.g. $\mathrm{arg}(-1) = \pi$ but $\mathrm{Arg}(-1) = \pi + 2\pi\Z$.

\subsection{Complex Conjugation}
\begin{definition}[Complex Conjugate]
	$\overline{a+ib} := a-ib$ (reflection across the real axis).
\end{definition}
Some properties of the conjugate:
\begin{itemize}
	\item $\overline{z+w} = \overline{z} + \overline{w}$
	\item $\overline{zw} + \overline{w}\overline{z}$
	\item $\lvert z \rvert^2 = z\overline{z}$
\end{itemize}
See that if $z = a + ib$, then
\[
	z\overline{z} = (a+ib)(a-ib) = a^2 - i^2b + aib - aib = a^2 + b^2 = |a+ib|^2
\]

\subsubsection{The standard way to divide complex numbers}
We have previously defined the multiplicative inverse of a complex number,
but the standard way to divide is actually using the conjugate.
We have
\[
	\frac{a+ib}{c+id} = \frac{(a+ib)(c-id)}{(c+id)(c-id)}
	= \frac{ac + bd + i(bc-ad)}{c^2 + d^2}
	= \frac{ac + bd}{c^2 + d^2} + i \frac{bc - ad}{c^2 + d^2}
\]

\subsection{$n$th roots of complex numbers}
\begin{proposition}[De Moivre's formula]
	\[
		z  = r(\cos\theta + i\sin\theta)
	\]
	\[
		z^n = r^n(\cos(n\theta) + i\sin(n\theta))
	\]
\end{proposition}
To find a third root of $z = r(\cos\theta + i\sin\theta)$,
divide $\theta$ by $3$ and extract a third root of $r$
(can always find a real $n$th root because $r \geq 0$):
\[
	w_1 = \sqrt[3]{r}\left(\cos\frac{\theta}{3} + i\sin\frac{\theta}{3}\right)
\]
then $w_1^3 = z$.
But there are in fact $3$ 3rd roots of $z$.
The others are found by dividing the circle with radius $\sqrt[3]{r}$ equally:
\[
	w_2 = \sqrt[3]{r}\left(\cos\left(\frac{\theta}{3} + \frac{2\pi}{3}\right)
		+ i\sin\left(\frac{\theta}{3} + \frac{2\pi}{3}\right)\right)
\]
\[
	w_3 = \sqrt[3]{r}\left(\cos\left(\frac{\theta}{3} + \frac{4\pi}{3}\right)
		+ i\sin\left(\frac{\theta}{3} + \frac{4\pi}{3}\right)\right)
\]

Another example $(1+i)^8 = 1 + 8i + \binom{8}{2}i^2 + \binom{8}{3}i^4 + \cdots$
which isn't something we want to work with.
Since $1+i = \sqrt{2}\left(\cos\frac{\pi}{4} + i\sin\frac{\pi}{4}\right)$,
we can actually write (and compute) it much easier:
\[
	(1+i)^8 = \sqrt{2}^8\left(\cos\left(8\frac{\pi}{4}\right) +
		i\sin\left(8\frac{\pi}{4}\right)\right)
		= 16 \left(\cos(2\pi) + i\sin(2\pi)\right) = 16
\]
The fact that this is a real number is because $1+i$
is a vertex of the octagon that has a vertex along the $x$-axis.

\subsection{Phase}
The \emph{phase} of $z \neq 0$ is $\frac{z}{|z|}$.
The phase of $z$ is the complex number of modulus $1$ with the same argument.
The phase keeps track of the ``angle" without the ambiguity in the argument.

A phase portraint is used for visualization.
We associate colours to the phases.
Red is associated with the positive real numbers,
green with $\theta = 2\pi/3$ and blue with $-2\pi/3$.
Then yellow is $\pi/3$, magenta is $-\pi/3$,
and cyan is the negative real numbers.

\subsection{The complex exponential}
Recall for $x \in \R$,
\[
	e^x = 1 + x + \frac12x^2 + \frac16x^3 + \cdots = \sum_{n=0}^\infty \frac{1}{n!}x^n
\]
This also works for complex numbers. If $z \in \C$:
\[
	e^z = 1 + z + \frac12z^2 + \cdots = \sum_{n=0}^\infty \frac{1}{n!}z^n
\]
For $z = i\theta$ where $\theta \in \R$ (purely imaginary) we get
\begin{align*}
	e^z &= e^{i\theta} = 1 + i\theta - \frac12 \theta^2 -
	\frac{1}{3!}i\theta^3 + \frac{1}{4!}\theta^4 + \frac{1}{5!}i\theta^5 +- \cdots\\
		&= 1 - \frac12\theta^2 + \frac{1}{4!}\theta^4 - \frac{1}{6!}\theta^6
	+ \cdots +
	i\left(\theta - \frac{1}{3!}\theta^3 + \frac{1}{5!}\theta^5 + \cdots\right)\\
		&= \cos\theta + i\sin\theta
\end{align*}
So if all these infinte sums behave proerply,
we deduce from this Euler's formula:
\[
	e^{i\theta} = \cos\theta + i\sin\theta
\]
From now on, $z$ in polar coordinates will be written
$z = r(\cos\theta + i\sin\theta) = re^{i\theta}$
where $r$ is the modulus and $\theta$ is the argument from before.
We get the famous identity with this formula:
\[
	e^{2\pi i} = 1
\]
In fact, $e^{2\pi i \Z} = \cos(2\pi\Z) + i\sin(2\pi\Z) = 1$.
The complex exponential function ha speriod $2\pi i$:
\[
	e^{z + 2\pi i} = e^ze^{2\pi i} = e^z\cdot 1 = e^z
\]

We can also introduce the complex $\sin$ and $\cos$ functions by their power series:
\begin{align*}
	\cos(z) &= 1 - \frac{1}{2}z^2 + \frac{1}{4!}z^4 -+ \cdots\\
	\sin(z) &= z - \frac{1}{3!}z^3 +- \cdots
\end{align*}
Then $e^{iz} = \cos{z} + i\sin{z}$.

Also $\overline{e^z} = \overline{sum_{n=0}^\infty \frac{1}{n!}z^n}
= \sum_{n=0}^\infty \frac{1}{n!}\overline{z}^n = e^{\overline{z}}$
so $e^{-iz} = \cos{z} - i\sin{z}$.
From these, we get
\begin{align*}
	\cos(z) &= \frac12\left(e^{iz} + e^{-iz}\right)\\
	\sin(z) &= \frac{1}{2i}\left(e^{iz} - e^{-iz}\right)
\end{align*}

\section{January 16}
\subsection{$n$th roots of unity}
The $n$th roots of the unity are the $n$ complex numbers $\omega_0,\dots,\omega_{n-1}$
such that $\omega_0^n = 1, \dots, \omega_{n-1}^n = 1$ are the $n$th roots of unity.
We let $\omega_0 = 1$. $\omega_0,\dots,\omega_{n-1}$ form a regular $n$-gon
with vertices on the unit circle.
For example, when $n = 3$, we have
$\omega_1 = e^{2\pi i/2}, \omega_2 = e^{4\pi i/2}, \omega_0 = 1$.

If $z$ is a complex number and $w_0^n = z$ so $w_0$ is one $n$th root of $z$,
then the others are $w_1 = \omega_1w_0, \dots, w_{n-1} = \omega_{n-1}w_0$,
obtained by multiplying $w_0$ by the $n$th roots of unity.

Example:
\[
	\sum_{n=0}^{100} i^n =  i^0 + i^1 + \cdots _ i^{100}
	= 1 + i + (-1) + (-i) + \cdots + 1
	= 26\cdot 1 + 25\cdot i + 25(-1) + 25(-i)
	= 26 - 25 = 1
\]
or we can use the geometric series formula
$1 + z + z^2 + \cdots + z^n = \frac{1-z^{n+1}}{1-z}$ to get
$\sum_{n=0}^{100}i^n = \frac{1-i^{101}}{1-i} = 1$.

Example: Write $-1 + 2i$ in polar form (with principle value of $\theta$).
Compute $r = \lvert -1 + 2i \rvert = \sqrt{1 + 4} = \sqrt{5}$.
So $-1 + 2i = \sqrt{5}\left(\cos\theta + i\sin\theta\right)$
so $-1 = \sqrt{5}\cos\theta$ and $2 = \sqrt{5}\sin\theta$.
So $\tan \theta = \frac{\sin\theta}{\cos\theta} = -2$,
which gives $\theta = \tan^{-1}(-2) + \pi$
(since $\tan^{-1}$ gives angles between $-\frac{\pi}{2}$ and $\frac{\pi}{2}$
and $-1 + 2i$ is in the second quadrant).
If it had been in thee third quadrant, we would subtract $\pi$.
This because the principle argument of a complex number is between $\pi$ and $-\pi$.

\subsection{Complex Functions}
There are two types:
\begin{enumerate}
	\item[(1).] Domain $[a,b] \subset \R$ is an interval in $\R$
		(codomain is $\C$).
		``A path in $\C$".
		$\gamma \colon [a,b] \to \C$, or $t \mapsto \gamma(t)$.
		Example: $\gamma \colon [0,1] \to \C$ by $t \mapsto 1 + it^2$.
	\item[(2).] Complex functions: the domain is a subset of $\C$,
		so $f \colon D \to \C$ where $D \subset \C$, or $z \mapsto f(z)$
		(and codomain is $\C$).
		Example: $\sin \colon \C \to \C$ by $z \mapsto \sin{z} = z - \frac{1}{3!}z^3 +- \cdots$.
\end{enumerate}
Visualising complex functions $f \colon \C \to \C$: analytic landscapes.
Consider the absolute value function $|f| \colon \C \to \R_{\geq 0}$
by $z \mapsto |f(z)|$ which we can draw with a $3$-dimensional graph;
then we colour by the phase/argument of $f(z)$.
Example: $f(z) = z^2$ where $z = x + iy$.
Then $|f(z)| = |z|^2 = x^2 + y^2$.
Then paraboloid $z = x^2 + y^2$ is the analytic landscape of $f(z) = z^2$.
As $z$ goes around the origin once, the phase of $f(z) = z^2$ goes around the origin twice:
so there is a double rainbow in the phase portrait of $f(z) = z^2$.
This is the phase portrait:
for $f \colon \C \to \C$, colour each $z \in \C$ with the phase of $f(z)$
(so $\frac{f(z)}{|f(z)|} \leftrightarrow$ colour).

\subsection{Limits}
We want to make sense of $\lim_{z\to z_0}f(z) + L$
($f\colon D \to \C$ is a complex function and $D \subset \C$)
where $z_0 \in \C$ but not necessarily in $D$, and $L \in \C$.
\begin{definition}
	$\lim_{z\to z_0} f(z) = L$ means that for all $\ep > 0$,
	there exists a $\delta > 0$ such that
	\[
		0 < |z - z_0| < \delta \implies |f(z) - L| < \ep
	\]
\end{definition}
So for every disc $D(L,\ep)$ with centre $L$, there exists a disc $D(z_0,\delta)$
with centre $z_0$ such that $f$ maps the punctured disc $D'(z_0,\delta)$ into $D(L,\ep)$.

\begin{definition}
	If $z_0 \in D$ and $\lim_{z\to z_0}f(z) = f(z_0)$ then $f$ is \emph{continuous} at $z_0$.
\end{definition}

Example: $\lim_{z\to 0}\frac{\overline{z}}{z}$ does not exist.
In this case, our $D = \C \setminus \{0\}$.
If $z \neq 0$ is real, $\overline{z} = z$ so $f(z) = 1$.
If $z$ is purely imaginary, $z = iy, y\in\R$, $\overline{z} = -iy$
so $\frac{\overline{z}}{z} = \frac{-iy}{iy} = -1$.
For example, $\lim_{z\to0}\overline{z}/z \neq 0$,
since no $\delta$ exists: if $\ep = \frac12$,
will always have $f(z)$ when $z$ is real outside of $D(0,\delta)$.

\section{January 23}
Recall we were looking at complex functions $f \colon \C \to \C$
or $f \colon D \to \C$ where $D \subset \C$
(both inputs and outputs are complex numbers).
We can write the input and output in terms of real and imaginary parts:
$f(z) = w = u + iv$ where $z = x + iy$.
We can define $u,v \colon \R^2 \to \R$ to be functions
that give the real and imaginary parts of $f$, respectively,
i.e. $f(x+iy) = u + iv = u(x+iy) + iv(x+iy) = u(x,y) + iv(x,y)$.
Then $f(z) = \langle u(x,y), v(x,y) \rangle$ can be viewed as a planar vector field.

Example: $f(z) = \sin{z}$.
When dealgin with complex trig. functions, it is good to convert to exponentials.
\begin{align*}
	\sin{z} &= \frac{1}{2i}\left(e^{iz} - e^{-iz}\right)\\
			&= \frac{1}{2i} \left(e^{i(x+iy)} - e^{-i(x+iy)}\right)\\
			&= \frac{1}{2i}\left(e^{ix - y} - e^{-ix + y}\right)\\
			&= \frac{1}{2i}\left(e^{ix}e^{-y} - e^{-ix}e^y\right)\\
			&= \frac{1}{2i}\left(e^{-y}(\cos{x}+i\sin{x})
				- e^y(\cos{x} - i\sin{x})\right)\\
			&= \frac{1}{2i}\left(\left(e^{-y}-e^y\right)\cos{x}
				+ i\left(e^{-y}+e^y\right)\sin{x}\right)\\
			&= \frac{-i}{2}\left(\left(e^{-y}-e^y\right)\cos{x}
				+ i \left(e^{-y}+e^y\right)\sin{x}\right)\\
			&= \frac{e^y + e^{-y}}{2}\sin{x} + i\frac{e^y-e^{-y}}{2}\cos{x}
\end{align*}
So we have $u(x,y) = \frac{e^y + e^{-y}}{2}\sin{x} = \cosh{y}\sin{x}$
and $v(x,y) = \frac{e^y-e^{-y}}{2}\cos{x} = \sinh{y}\cos{x}$
(we won't ever use these function names in this class).
As a vector field, $\sin(z) =
\langle \frac{e^y+e^{-y}}{2}\sin{x}, \frac{e^y-e^{-y}}{2}\cos{x}\rangle$.

\subsection{Infinite Limits}
Recall $\lim_{z\to z_0}f(z) = L$ means that
$\forall \ep > 0$, $\exists \delta > 0$ such that
$0 < \lvert z-z_0\rvert < \delta \implies \lvert f(z) - L \rvert < \ep$.
What does this condition mean for sequences:
 suppose we have a sequence of complex numbers $z_1,z_2,z_3,\dots$
 converging to $z_0$ (but avoiding $z_0$).
 it will eventually have to be within $\delta$ of $z_0$, say from $N$ onward.
 So $0 < \lvert z_n - z_0 \rvert < \delta$ for all $n \geq N$,
 then we must have $\lvert f(z_n) - L \rvert < \ep$ for all $n \geq N$.
 But this is true for all $\ep > 0$, no matter how small,
 so the sequence $f(z_1),f(z_2),\dots$ converges to $L$.
 \emph{This is true no matter in wchi direction $z_n \to z_0$.}

 The \emph{extended complex plane} is $\C \cup \{\infty\}$,
 usually denoted $\hat{\C}$.
 It can be identified with the Riemann sphere by stereograhpic pojection:
 for every point on the sphere, we have a $1$-$1$ correspondance
 with a point on the complex plane by taking the place where
 the ray from the top of the sphere that intersects the desired point of the sphere
 intersects the complex plan
 (his sphere has $0$ as its origin; so the points inside the unit circle
 correspond to points in the southern hemisphere).
 The only point on the sphere that doesn't have a corresponding point in $\C$
 is the north pole $N$, but points ``very close" to $N$ on the sphere
 correspond to points outside a circle of large radius in $\C$.
 This leads to the following definitions:
 $\lim_{z \to z_0}f(z) = \infty$ means $\forall M > 0$, $\exists \delta > 0$
 such that $0 < \lvert z - z_0 \rvert < \delta \implies \lvert f(z) \rvert > M$.

 Example: $\lim_{z \to 0}\frac{1}{z} = \infty$ where
 $z = re^{i\theta}$ so $\frac{1}{z} = \frac{1}{r}e^{-i\theta}$.
 Algebraic proof: given $M > 0$, let $\delta = \frac{1}{M} > 0$.
 Then $0 < \lvert z \rvert < \delta \implies 0 < \lvert z \rvert < M
 \implies \frac{1}{|z|} > M \implies \left\lvert \frac{1}{z} \right\rvert > M$.

 Some other definitions:
 $\lim_{z \to \infty} f(z) = L$ means $\forall \ep > 0$, $\exists M > 0$
 such that $\lvert z \rvert > M \implies \lvert f(z) - L \rvert < \ep$;
 and $\lim_{z \to \infty} f(z) = \infty$ means $\forall M > )$, $\exists N > 0$
 such that $\lvert z \rvert > N \implies \lvert f(z) \rvert > M$.

\begin{remark}
	All the rules for computing limits of sums, products, quotients remain true
	if we stipulate:
	$\infty + a = \infty$ for all $a \in \C$ ($\infty+\infty$ is not defined),
	$\infty \cdot a = \infty$ for all $a \in \C\setminus\{0\}$
	($\infty\cdot0$ is not defined),
	$\frac{a}{\infty} = 0, \frac{a}{0} = \infty$ for ff,
	and $\frac{\infty}{0} = \infty$, $\frac{0}{\infty}=0$, $\infty\infty = \infty$.
\end{remark}

Example 1: $\frac{0}{\infty} = 0$ means if $\lim_{z \to z_0}f(z) = 0$ and
$\lim_{z\to z_0}g(z) = \infty$,
then $\lim_{z\to z_0}\frac{f(z)}{g(z)} =
\frac{\lim_{z\to z_0}f(z)}{\lim_{z\to z_0}g(z)} = \frac{0}{\infty} = 0$.

Example 2: $f(z) = \frac{1}{z}, g(z) = -\frac{1}{z}$.
$\lim_{z \to 0} f(z) = \lim_{z \to 0} \frac{1}{z}
= \frac{\lim_{z \to 0} 1}{\lim_{z \to 0} z} = \frac{1}{0} = \infty$.
Similarly, $\lim_{z \to 0} g(z) = \lim_{z \to 0} -\frac{1}{z}
= \frac{-1}{0} = \infty$.
And then
\[
	0 = \lim_{z \to 0}0 = \lim_{z \to 0}\left(\frac{1}{z}-\frac{1}{z}\right)
	\lim_{z\to0}\left(f(z) + g(z)\right) =
	\lim_{z \to 0}f(z) + \lim_{z \to 0}g(z) = \infty + \infty = \text{ not defined}
\]
which is why $\infty + \infty$ is not defined,
because $\infty + \infty = \lim_{z\to\infty}\left(f(z) + f(z)\right) = \infty$ as well.

\subsection{M\"{o}bius transformations}
\begin{definition}[M\"{o}bius transformation]
	A \emph{M\"{o}bius transformation} is a linear fractional transformation
	$f \colon \hat{\C} \to \hat{\C}$
	\[
		f(z) = \frac{az + b}{cz + d}
	\]
	$a,b,c,d \in \C$, $ad - bc \neq 0$.
\end{definition}
Why do we require $ad - bc \neq 0$?
Well, if $ad = bc$, then
$c \neq 0$ implies $b = \frac{ad}{c}$ and
$\frac{az+b}{cz+d} = \frac{caz + ad}{c^2z+cd} = \frac{a}{c}\frac{cz+d}{cz+d} = \frac{a}{c}$
so $f(z)$ is constant;
$c = 0$ implies that if $d \neq 0$ then $a = 0$ and so
$f(z) = \frac{az+b}{cz+d} = \frac{b}{d}$ a constant again,
or if $d = 0$ then $f(z) = \frac{az+b}{0}$ which is not defined (or constant $\infty$).
So $ad - bc \neq 0$ assures that $f$ is not constant.

Define $f(z) = \frac{az+b}{cz+d}$ as a function $\hat{\C} \to \hat{\C}$:
\begin{enumerate}
	\item $cz + d = 0$.
		When $c \neq 0$ then the denominator vanishes at $z = -d/c$,
		so
		\[
			\lim_{z \to -d/c} \frac{az+b}{cz+d}
			= \frac{\displaystyle\lim_{z\to-d/c}(az+b)}{\displaystyle\lim_{z\to-d/c}(cz+d)}
			= \frac{-\frac{ad}{c}+b}{0}
			= \frac{\frac{-ad+bc}{c}}{0} = \infty
		\]
		since $\frac{-ad+bc}{c} \neq 0$.
		When $c = 0$ then $f(z) = \frac{az+b}{d}$ with $d\neq0$ so no problem with denominator.
	\item $f(\infty) =$ ?.
\end{enumerate}
ff I think I'm done for today, I will just fill in examples later.

ff

In fact, every rotation of the Riemann sphere is a M\"{o}bius transformation.
We can iterate $f$ and draw trajectories.
Here the trajectories are circles (the second picture in the assigned reading).
But there are other classes of transformations than just iteration of one???


\section{January 25}
\subsection{M\"{o}bius Transformations}
More about $f(z) = \frac{1}{z}$.
Geometrically as a transformation of the plane $\C$,
this is the composition of two geometric transformations:
\begin{enumerate}
	\item[(1)] Circle inversion in the unit circle:
		if $z$ is a point in the plane, $z^*$ is its inversion.
		Then $\arg(z) = \arg(z^*)$ and $|z^*||z| = 1$.
	\item[(2)] Reflection in the $x$-axis $z \mapsto \bar{z}$.
\end{enumerate}
So $\frac{1}{z} = \overline{z^*}$.
We then get that $\left\lvert \frac{1}{z} \right\rvert = \frac{1}{|z|}$
and $\arg\frac{1}{z} = - \arg{z}$ (from these facts).

The circle inversion and the reflection both have the property: circles or lines
map to circles or lines.
Therefore, also the M\"{o}bius transformation $f(z) = 1/z$ has this property.
For example, $f(\text{circle in }\C \text{ not through }0)
= f(\text{circle in }\hat{\C}\text{ through }\infty \text{ not }0) =$
circle in $\hat{C}$ through $0$ not through $\infty = $
circle through $0$.
A line is a circle through the north pole on the Riemann sphere.

Two more examples:
$g(z) = z + b$, a translation.
This can be written as $\frac{az + b}{cz + d}$ with $a =1, c= 0, d = 1$,
so the matrix $\begin{bmatrix} 1 & b \\ 0 & 1 \end{bmatrix}$.
This takes lines to lines and circles to circles.
$h(z) = az$, a rotation-dilation (rotostretch).
This can be written $\begin{bmatrix} a & 0 \\ 0 & 1 \end{bmatrix}$,
and required $a \neq 0$.
$h$ preserves lines and circles.

Fact: Every M\"{o}bius transformation is a composition of
transformations of these three types: $f,g,h$.
\[
	\frac{az + b}{cz + d} = \frac{a}{(ad-bc)\frac{1}{az+b} + c}
\]
(if $a \neq 0$).
Consequence: All M\"{o}bius transformatoins map lines and circles to lines or circles.
On the Riemann sphere: all MTs map circles to circles.

Consider the transformations $g$ and $h$ we were looking at before.
First, $g(z) = z + b$.
In the plane, trajectories are parallel lines:
$z, z+ b, z+ 2b, \dots$.
But on the sphere, there is one fixed point, $\infty$:
$\infty + b = \infty$.
$g(z)$ just maps a circles to a circle that is closer to $\infty$.
This is a parabolic transformation.

Second, $h(z) = az$.
When $|a| = 1$, the trajectories are circles.
This is an elliptic MT.
When $a$ is real, $h(z)$ is a scaling transformation.
The trajectory goes from $0$ to $\infty$, or vice versa (the two fixed points).
this is a hyperbolic MT.
If $a$ is not real, $|a| \neq 1$, then trajectories are spirals,
or loxodromes.
These are called loxodromic MTs.
Same fixed points.

In general, every M\"{o}bius transformation (not the identity)
has either $1$ or $2$ fixed points.
1 fixed point: parabolic (ex. translation).
2 fixed points: hyperbolic, elliptic, loxodromic.
There are pictures in Wegert.

Example calculation:
Let $(p,q,0)$ in $\C$ correspond to $(x,y,z)$ on $\hat{C} = \mathbb{S}^2$.
ff I'm not doing this shit stupid ass projection.

Back to M\"{o}bius transformations:
Every $2 \times 2$ matrix with complex entires, $\det \neq 0$ gives rise to a MT,
i.e. $\begin{bmatrix} a & b \\ c & d \end{bmatrix} \to \frac{az + b}{cz + d}$.
Multiply the matrix by $\lambda \in \C$, $\lambda \neq 0$, we get the same MT.
Every M\"{o}bius transformatoin, which is also a MT.
Recall $\begin{bmatrix} a & b \\ c & d \end{bmatrix}^{-1}
= \frac{1}{ad-bc}\begin{bmatrix} d & -b \\ -c & a \end{bmatrix}$.
If $f(z) = \frac{az + b}{cz + d}$,
then $g(z) = \frac{dz - b}{-cz + a}$ is the inverse!
One can compute and verify this, but I'm not gonna.

If the matrix $A$ defines the MT $f$, then $A^{-1}$ defines the $f^{-1}$.
Also, if $A$ defines $f$, and $B$ defines $g$, then $AB$ defines $f \circ g$,
i.e. composition of MT are again a MT.
M\"{o}bius transformations are one-to-one and onto maps $\hat{C} \to \hat{C}$.

\section{January 30}
Fact: Every non-identity M\"{o}bius transformation has either 
$1$ fixed point (the matrix as a repeated eignevalue and is not diagonliazble,
e.g. $\begin{pmatrix} 1 & b \\ 0 & 1 \end{pmatrix}$, $z \mapsto z + b$,
only fixed point is $\infty$), the parabolic case,
or $2$ fixed points (the matrix has $2$ distinct eigenvalues
and is therefore diagonlizable,
e.g. $\begin{pmatrix} a & 0 \\ 0 & d \end{pmatrix}$, $z \mapsto \frac{a}{d}z$).
If $\lambda_1,\lambda_2$ are the eigenvalues, then $\frac{\lambda_1}{\lambda_2}\in\R$
means the transformation is hyperbolic,
$\lvert \frac{\lambda_1}{\lambda_2}\rvert = 1$ means the transformation is elliptic,
and otherwise poxochrome??? case ff.

Fact: ff there exists a unique M\"{o}bius transformation
$f \colon \hat{\C} \to \hat{\C}$ such that $f(A) = A'$
and $f(B) = B'$ and $f(C) = C'$.
Example: Find the M\"{o}bius transform such that
$f(0) = 1, f(i) = i, f(\infty) = -1$.
We have $f(z) = \frac{az+b}{cz+d}$.
Then $f(0) = \frac{b}{d} = 1$,
$f(i) = \frac{ai + b}{ci + d} = i$, and $f(\infty) = \frac{a}{c} = -1$.
So we get three conditions: $b = d$, $ai + b = -c + id$, and $a = -c$.
We get a homogenous system of linear equations with $3$ equations, $4$ indeterminants:
\[
	\begin{cases}
		b - d = 0\\
		ia + b + c -id = 0\\
		a + c = 0
	\end{cases}
\]
This will always have exactly $1$ free variable,
hence we get exactly $1$ M\"{o}bius transformation
(since the numerator and denomonator of $\frac{az+b}{cz+d}$ are
homogenous for any of their variables,
so fixing any will result in cancellation since the other variables
will be a linear scalar multiple of the fixed variable, see below).
Suppose $d$ is free, so we can let $d = 1 \implies b = 1$,
then using $ai + 1 = -c + i$ and $a = -c$, we get $a = 1, c = -1$,
so $f(z) = \frac{z+1}{-z+1}$ satisfies our conditions
(could have also let $d = 2$ to get $f(z) = \frac{z+1}{-2z+2}$ which is the same as before...
how do we gaurantee that this will happen, i.e. that we get unique transform?).

Three distinct points in $\hat{\C}$ determine a circle or a line (circle through $\infty$).
$0,i,\infty$ determine the imaginary axis:
makes it into the unit circle, since it passes through $-1,i,1$.
$f$ maps the imaginary axis to the unit circle.

A common M\"{o}bius transform question would be to find a MT mapping
a specific line to a specific circle.
Do this as above.
Could also be given a question that asks to find a MT mapping
the upper half plane to the unit disc.
Then, we have to map $-1,0,1$ ($\mathrm{Im}(z)\geq0$), to $-1,i,1$.
The order of our points is important to determine
if the transformation will map the upper half plane to the inside
of the circle or the outside.
If we walk from $-1$ to $0$ to $1$, the region we have is to our left.
If we walk from $1$ to $i$ to $-1$, the region we want is also to our left.
So we choose
\[\begin{cases}
	-1 \mapsto 1\\
	0 \mapsto  i\\
	1 \mapsto -1
\end{cases}\]

Example: Consider $z \mapsto \frac{1}{z}$, $f(z) = \frac{1}{z}$.
What does this M.T. do to the veritcal lines in $\C$?
Consider the vertical line $L$ that passes through $a \in \R$ ($a\neq0$).
An equation for $L$ is $\mathrm{Re}(z) = a$.
Using $\mathrm{Re}(z) = \frac{1}{2}(z+\bar{z})$, rewrite as
$2a = z + \bar{z}$.
We can now find an equation for $f(L)$.
Write $w$ for $f(z)$, the image of $z$ under $f$.
Want an equation for $w$ but we have one for $z$.
$w = f(z), w = \frac{1}{z}$.
Solve for $z$ in terms of $w$, i.e. $z = \frac{1}{w}$.
Then plug into our equation for $z$ to get $(1/w) + \overline{(1/w)} = 2a$,
or $1/w + 1/\bar{w} = 2a$, which is the equation for the image $f(L)$.
Rewrite $\bar{w} + w = 2aw\bar{w}$.
We have $\mathrm{Re}(w) = a|w|^2$.
Since $a \in \R$, we can rewrite this as $|w|^2 - 2\mathrm{Re}(\frac{1}{2a}w) = 0$.
We now complete the ``absolute value squared":
a general formula for any $u,v \in \C$ is
\begin{align*}
	|u-v|^2
	&= (u-v)(\bar{u}-\bar{v})\\
	&= u\bar{u} - u\bar{v} - v\bar{u} + v\bar{v}\\
	&= |u|^2 - (u\bar{v} + \overline{u\bar{v}}) + |v|^2\\
	&= |u|^2 - 2\mathrm{Re}(u\bar{v}) + |v|^2
\end{align*}
(a formula in our formula sheet).
Hence, $\left\lvert w - \frac{1}{2a}\right\rvert^2 = \left\lvert \frac{1}{2a}\right\rvert^2$.
This is the equation for the circle of radius $\left\lvert \frac{1}{2a}\right\rvert$
centered at $\frac{1}{2a}$ (this is a circle through $0$).

\subsection{Complex Differentiation (follow Howie)}
\begin{definition}
	Let $f \colon D \to \C, D \subset \C$ be a complex function.
	$f$ is (complex) differentiable at $c \in D$ if
	\[
		\lim_{z \to c} \frac{f(z) - f(c)}{z-c}
	\]
	exists.
	This limit is called the derivative of $f$ at $c$,
	notation is $f'(c)$.
\end{definition}

Standard results from calculus apply (sums, products, quotient, chain).
Obviously, $f(z) = z$ is differentiable at all $c \in \C$:
\[
	\lim_{z\to c} \frac{z-c}{z-c} = 1
\]
so $f'(c) = 1$ for all $c \in \C$.
So polynomials are differentiable: $f(z) = 2z^2 + 3iz + 1$
gives $f'(z) = 4z + 3i$.
Less simple: $g(z) = \frac{z^2 + i}{z-i}$ is differentiable for all $c \neq i$.

We can write out what it means for the real/imaginary parts.
If $z = x + iy$, then $f(z) = f(x+iy) = u(x+iy) + iv(x + iy)
= u(x,y) + iv(x,y)$,
where $u,v \colon \R^2 \to \R$.
We will look at the derivative limit as $z$ approaches some point $c$.
If $c = a + ib$, $a,b \in \R$ is a constant,
and fix $z$ be on the horizontal line through $c$,
i.e. $z = x + iy$ where $y = b$, then
\begin{align*}
	\lim_{x \to a} \frac{f(x+iy) - f(a + ib)}{x + iy - (a + ib)}
	&= \lim_{x \to a} \frac{u(x,y) + iv(x,y) - u(a,b) - iv(a,b)}{x+iy - a - ib}\\
	&= \lim_{x\to a} \frac{u(x,b) + iv(x,b) - u(a,b) - iv(a,b)}{x - a}\\
	&= \lim_{x \to a} \frac{u(x,b) - u(a,b)}{x-a} +
	i\lim_{x \to a} \frac{v(x,b) - v(a,b)}{x-a}
\end{align*}
Hence $f'(a+ib) = \frac{\partial u}{\partial x} \big\vert_{(a,b)}
+ i \frac{\partial v}{\partial x}\big\vert_{(a,b)}$.
Now we fix $x = a$ and let $y$ vary:
\begin{align*}
	f'(a+ib) &= \lim_{y \to b} \frac{u(a,y) + iv(a,y) - (u(a,b) + iv(a,b))}{(a+iy)-(a+ib)}\\
			 &= lim_{y \to b} \frac{v(a,y) - v(a,b)}{y-b} - i\lim_{y\to b}
			 \frac{u(a,y) - u(a,b)}{y-b}\\
			 &= \frac{\partial v}{\partial y} \bigg\vert_{(a,b)}
			 - i \frac{\partial u}{\partial y}\bigg\vert_{(a,b)}
\end{align*}

So if $f$ is complex differentiable at $c = (a+ib) = (a,b)$, then
\[
	\frac{\partial u}{\partial x}\bigg\vert_{(a,b)} = \frac{\partial v}{\partial y}\bigg\vert_{(a,b)}
	\qquad
	\frac{\partial v}{\partial x}\bigg\vert_{(a,b)} = -\frac{\partial u}{\partial y}\bigg\vert_{(a,b)}
\]
(The Cauchy-Riemann equations).
This is only from considering approaching along vertical and horizontal lines,
so not a sufficient condition.

\section{February 1}
\subsection{Differentiation (cont.)}
Example: $f(z) = \bar{z}$.
$f(x+iy) = \overline{x+iy} = x -iy = u(x,y) + iv(x,y)$.
So $u(x,y) = x$ and $v(x,y) = -y$.
We have $\frac{\partial u}{\partial x} = 1$,
$\frac{\partial u}{\partial y} = 1$, $\frac{\partial v}{\partial x} = 0$,
and $\frac{\partial v}{\partial y} = -1$.
So our second criteria is satisfied, i.e.
$\frac{\partial u}{\partial y} = -\frac{\partial v}{\partial x}$,
but our first criteria is not, since $1 = -1$ holds nowhere in $\C$,
so $\frac{\partial u}{\partial x} \neq \frac{\partial v}{\partial y}$.
Hence, $f(z) = \bar{z}$ is not complex differentiable.

Example: $f(z) = |z|^2 = z\bar{z}$.
$f(x+iy) = (x+iy)(x-iy) = x^2 + y^2$
so $u(x,y) = x^2 + y^2$ and $v(x,y) = 0$.
We have $\frac{\partial u}{\partial x} = 2x$, $\frac{\partial u}{\partial y} = 2y$,
$\frac{\partial v}{\partial x} = 0$, and $\frac{\partial v}{\partial y} = 0$.
The two criteria give that $2x = 0 = 2y$, so the only solution
to both equations is $(x,y) = (0,0)$, or $z = 0$.
So $f(z) = |z|^2$ is not (complex) differentiable anywhere,
except for maybe at the origin.
Then compute the limit at the origin:
\[
	\lim_{z \to 0} \frac{f(z) - f(0)}{z - 0}
	= \lim_{z \to 0}\frac{|z|^2 - 0}{z}
	= \lim_{z \to 0}\bar{z} = \overline{\lim_{z \to 0}z} = 0
\]
Since this limit exists, $f(z) = |z|^2$ is complex differentiable at $z = 0$,
and $f'(0) = 0$.

Warning: just because the C.R. equations hold at some point $z = c$,
doesn't mean that $f(z)$ is complex differentiable at $z = c$.
Counterexample: $f(x + iy) = \sqrt{|xy|}$ at the origin $c = 0$.
We have $u(x,y) = \sqrt{|xy|}, v(x,y) = 0$.
On the real axis ($y = 0$) and on the imaginary axis ($x = 0$),
we have $u(x,y) = 0$.
So $\frac{\partial u}{\partial x} \big \vert_0$ and
$\frac{\partial u}{\partial y} \big \vert_0$ exist and are $0$
(also $\frac{\partial v}{\partial x} \big \vert_0 =
\frac{\partial v}{\partial x} \big \vert_0 = 0$).
So the C.R. equations hold at $c = 0$.
BUT: $f$ is not (complex) differentiable at $c = 0$.
Consider $f$ restricted to the line $x = y$.
Then $f(x + ix) = \sqrt{|x^2|} = |x|$,
and $\lim_{x \to 0^+} \frac{|x| - |0|}{x-0} = 1$
but $\lim_{x \to 0^-} \frac{|x| - |0|}{x-0} = -1$
so $\lim_{z\to 0} \frac{f(z) - f(0)}{z-0}$ does not exist,
and hence $f$ is not complex differentiable at $c = 0$.

\begin{theorem}
	Let $D = D(c,r)$ the disk centered at $c \in \C$, radius $r \in \R$, $r > 0$.
	Suppose that $D \subset$ the domain of the complex function $f = u + iv$.
	Suppose that $\frac{\partial u}{\partial x},\frac{\partial u}{\partial y},
	\frac{\partial v}{\partial x},\frac{\partial v}{\partial y}$
	exists and are continuous at all points of $D$.
	Then if the C.R. equations hold at $c$, then $f$ is complex differentiable at $c$.
	(If the C.R. equations hold at all points of $D$,
	then $f$ is holomorphic at all points of $D$).
\end{theorem}

\begin{definition}
	A subset $U \subset \C$ is open if
	for every $z_0 \in U$, there exists an $r > 0$ such that $D(z_0,r) \subset U$.
\end{definition}
E.g. $\{z \mid |z - z_0| < r\}$ is open, but
$\{z \mid |z - z_0| \leq r\}$ is not open.

\begin{definition}
	Let $U \subset \C$ is open, $f \colon U \to \C$ a function.
	If $f$ complex differentiable at every point $c \in U$, then $f$ is called
	holomorphic or analytic.
	If $U = \C$, then $f$ is entire.
\end{definition}
So $f(z) = |z|^2$ is complex differentiable at $0$ but not holomorphic at $0$
(holomorphic requires differentiable on some open set).

\subsection{Power Series}
How do we construct holomorphic functions
(beyond polynomials, rational functions)?
The most important way is by power series.

Recall the geometric series,
$\sum_{n = 0}^\infty r^n = 1 + r + r^2 + \cdots$, where $0 \leq r < 1$.
The limit exists by the ratio test: $\lim_{n\to\infty} \frac{r^{n+1}}{r^n}
= \lim_{n\to\infty} r = r < 1$.
The sum is $\sum_{n=0}^\infty r^n = \frac{1}{1-r}$.

Now, recall what it means for a sequence of complex numbers to converge:
$(z_n)_{n \geq 0}$ converges to $L \in \C$, $\lim_{n\to\infty} z_n = L$
if $\forall \ep > 0, \exists N \colon |z_n - L| < \ep$
(the same definition as for sequences of real numbers,
except the interval $|x_n - L| < \ep$ is replaced by the disc $|z_n - L| < \ep$).
A series $\sum_{n = 0}^\infty z_n$ converges if
$\lim_{k \to \infty}\left(\sum_{n=0}^k z_n \right)$ converges
(if the limit of the sequence of partial sums exists)
and $\sum_{n = 0}^\infty z_n = \lim_{k \to \infty} \sum_{n = 0}^k z_n$.
Fact: If $\sum_{n=0}^\infty z_n$ converges, then $\lim_{n \to \infty} z_n = 0$.
\begin{definition}
	If $\sum_{n=0}^\infty |z_n|$ converges then $\sum_{n=0}^\infty z_n$
	is called \emph{absolutely convergent} (absolute convergence $\implies$ convergence).
\end{definition}
\begin{remark}
	$\sum_{n=0}^\infty|z_n|$ is a series of real numbers,
	so all standard convergence tests apply
	(ratio test, root test, comparison test, integral test, etc.).
\end{remark}

We also have complex geometric series when $r = z \in \C$.
We might ask, when is $\sum_{n = 0}^\infty z^n$ is absolutely convergent?
$\sum_{n=0}^\infty |z^n| = \sum_{n=0}^\infty |z|^n$,
which converges if $|z| < 1$.
So $\sum_{n=0}^\infty z^n$ is absolutely convergent for all
$z \in D(0,1) = \{z \mid |z| < 1\}$ and the limit is
\[
	\sum_{n=0}^\infty z^n = 1 + \sum_{n=1}^\infty z^n = 1+z\sum_{n=1}^\infty z^{n-1}
	= 1 + z\sum_{n=0}^\infty z^n
	\implies (1-z)\sum_{n=0}^\infty z^n = 1 \implies
	\sum_{n=0}^\infty z^n = \frac{1}{1-z}
\]

\begin{definition}[Power series]
	A series of the form $\sum_{n=0}^\infty c_n(z-a)^n$,
	$(c_0,c_1,\dots)$ a sequence of complex numbers,
	is called a $\emph{power series}$,
	where $c_n$ are the coefficients, $a\in \C$ is the center.
\end{definition}
When $a = 0$, we get the special case $\sum_{n=0}^\infty c_nz^n$,
e.g. $\sum_{n=0}^\infty z^n$ is the geometric series,
or $\sum_{n=0}^\infty \frac{1}{n!}z^n$ is the exponential series.

\begin{theorem}
	Suppose $\sum_{n=0}^\infty c_n(z-a)^n$ converges at $z = z_0$.
	Then $\sum_{n=0}^\infty c_nz^n$ converges absolutely at all $z$
	such that $|z - a| < |z_0 - a|$.
\end{theorem}
\begin{theorem}
	The power series $\sum_{n=0}^\infty c_n(z-a)^n$ satisfies one (and only one)
	of the conditions
	\begin{enumerate}
		\item[(1)] converges for all $z \in \C$
		\item[(2)] converges only for $z = a$
		\item[(3)] there exists a positive real number $R > 0$
			(the radius of convergence) such that $\sum_{n=0}^\infty c_n(z-a)^n$
			converges absolutely for all $z \colon |z-a| < R$
			(and diverges for all $|z-a| > R$).
	\end{enumerate}
\end{theorem}


\section{February 6}
\subsection{Power series (cont.)}
\begin{theorem}
	Suppose $\sum_{n=0}^\infty c_nz^n$ converges at $z = z_0$.
	Then $\sum_{n=0}^\infty c_nz^n$ converges absolutely
	foor all $z$ such that $|z| < |z_0|$
\end{theorem}
\begin{proof}
	$c_n z_0^n \to 0$ so there exists $R > 0$ such that
	$|c_nz_0^n| < R$ for all $n \geq 0$.
	Let $z$ be such that $|z| < |z_0|$,
	then $\frac{|z|}{|z_0|} < 1$ so the geometric series
	$\sum_{n=0}^\infty \left(\frac{|z|}{|z_0|}\right)^n$ converges;
	then also $\sum_{n=0}^\infty R\left(\frac{|z|}{|z_0|}\right)^n$ converges.
	\[
		|c_nz^n| = \left|c_n z_0^n \left(\frac{z}{z_0}\right)^n\right|
		= |c_nz_0^n|\left(\frac{|z|}{|z_0|}\right)^n < R\left(\frac{|z|}{|z_0|}\right)^n
	\]
	so by the comparison test, $\sum_{n=0}^\infty |c_nz^n|$ converges.
\end{proof}
\begin{remark}
	This holds with arbitrary centre: if $\sum c_n(z-a)^n$ converges
	at $z_0$, then it converges absolutely for all $z$ such that
	$|z-a| < |z_0 - a|$.
\end{remark}

\begin{theorem}
	Let $\sum_{n=0}^\infty c_n(z-a)^n$ be a power series with radius of convergence $R$.
	\begin{enumerate}
		\item[(1)] If $\lim_{n\to\infty} \frac{|c_{n+1}|}{|c_n|}$ exists
			and equal to $\lambda$, then $R = 1/\lambda$.
		\item[(2)] If $\lim_{n\to\infty} \sqrt[n]{|c_n|}$ exists
			and equal to $\lambda$, then $R = 1/\lambda$.
	\end{enumerate}
	($R = 0,\infty$ included).
\end{theorem}

Example: $\sum_{n=0}^\infty 2^n(z-i)^n$.
Note $c_n = 2^n$ and $a = i$.
We can compute $\lim_{n\to\infty} \frac{|c_{n+1}|}{|c_n|}
= \lim_{n\to\infty} \frac{2^{n+1}}{2^n} = \lim_{n\to\infty} 2= 2$,
so $R = 1/2$.
(Directly, this is a geometric series $\sum_{n=0}^\infty (2(z-i))^n$ which converges
when $|2(z-i)| < 1 \implies |z-i| < \frac12$ so $R = 1/2$.)

Example: $\sum_{n=0}^\infty \frac{1}{n!} z^n$.
Using ratio test, we get $\lim_{n\to\infty}\frac{1}{(n+1)!}/\frac{1}{n!}
= \lim_{n\to\infty} \frac{n!}{(n+1)!} = \lim_{n \to \infty} \frac{1}{n+1} = 0$,
so $R = \infty$.
Hence the exponential series converges for all $z \in \C$.

\begin{theorem}
	Let $\sum_{n=0}^\infty c_n(z-a)^n$ be a power series
	with radius of convergence $R > 0$. Then define
	\[
		f(z) = \sum_{n=0}^\infty c_n(z-a)
	\]
	with domain $D(a,R) = \{z \mid |z-a| < R\}$.
	Then $f(z)$ is holomorphic in $D(a,R)$ with
	\[
		f'(z) = \sum_{n=1}^\infty nc_n(z-a)^{n-1}
	\]
\end{theorem}

Example: $\exp(z) = \sum_{n=0}^\infty \frac{1}{n!} z^n$ is an entire function
with derivative
\[
	\exp'(z) = \sum_{n=1}^\infty n\frac{1}{n!} z^{n-1}
	= \sum_{n=1}^\infty \frac{1}{(n-1)!} z^{n-1} = \sum_{n=0}^\infty \frac{1}{n!}z^n = \exp(z)
\]

Other properties of $\exp(z)$:
\begin{enumerate}
	\item $\exp(z) \neq 0$ for all $z \in \C$
	\item $\exp(z_1 + z_2) = \exp(z_1)\exp(z_2)$
	\item $\exp(2\pi i) = \cos(2\pi) + i\sin(2\pi) = 1$.
		More generally, $\exp(2\pi i k) = 1$ for all $k \in \Z$
		and $\exp(z + 2\pi i) = \exp(z)\exp(2\pi i) = \exp(z)$.
		So $\exp$ is periodic with period $2 \pi i$.
		In fact, $\exp(z_1) = \exp(z_2)$ if and only if
		there exists a $k \in \Z$ such that $z_2 + z_1 + 2\pi i k$.
\end{enumerate}

Example: $\sum_{n=0}^\infty (n+1)^2 z^n$.
Try to use the ratio test to find $R$:
\[
	\lim_{n\to \infty} \frac{(n+2)^2}{(n+1)^2}
	= \lim_{n\to\infty} \left(\frac{1+2/n}{1+1/n}\right)^2
	= \left(\lim_{n\to\infty} \frac{1+2/n}{1+1/n}\right)^2
	= \left(\frac{1 + \lim 2/n}{1 + \lim 1/n}\right)^2
	= \left(\frac{1 + 0}{1 + 0}\right)^2 = 1
\]
So $R = 1/1 = 1$.
(The ratio test for $\sum_{n=0}^\infty (n+1)^2$ is inconclusive though.)

Example: $\sum_{n=0}^\infty z^{n!} = z^1 + z^1 + z^2 + z^6 + z^{24} + \cdots
= 2z + z^2 + z^6 + z^{24} + \cdots$.
This is a power series with coefficients
$0,2,1,0,0,0,1,0,\dots$.
Ratoi test does not work, $\lim_{n\to\infty} \sqrt[n]{|c_n|}$ does not exist.
But there does exist a general formula:
\[
	\frac{1}{R} = \limsup \sqrt[n]{|c_n|}
\]
We have $\sqrt[n]{|c_n|} = 0, \sqrt[n]{2}, 1, 0, 0, 0, 1, 0, \dots$,
so in this case $\limsup \sqrt[n]{|c_n|} = 1$,
so $R = 1$.

\subsection{Logarithms}
\begin{definition}
	$z = \exp(w) \iff w$ is a logarithm of $z$
\end{definition}
If $w$ is a lograithm of $z$ then $w + 2\pi i$ is another lograithm of $z$
($z = \exp(w) \implies z = \exp(w + 2\pi i)$).
So the logarithm of $z$ is multivalued!

Say $w = u + iv$ ($u,v \in \R$),
then $z = \exp(w) \iff z = \exp(u + iv)
\iff z = \exp(u)\exp(iv) = e^u(\cos{v} + i\sin{v})
\iff |z| = e^u$ and $v = \arg(z)$ ($v$ is an argument of $z$)
$\iff u = \ln |z|$ and $v = \arg(z)$.
(We write $\ln$ for the usual real logarithm $\ln \colon \R_{>0} \to \R$.)
So $w = \log{z} \iff \mathrm{Re}(w) = \ln|z|$ and $\mathrm{Im}(w) = \arg(z)$.

\begin{definition}
	The \emph{principal logarithm} of $z \in \C$, $z \neq 0$ is
	\[
		\log(z) := \ln|z| + i \arg(z)
	\]
	where $\arg(z)$ is the principal argument of $z$,
	so $- \pi < \mathrm{Im}(\log(z)) \leq \pi$.
\end{definition}
Example: $\log(-2) = \ln|-2| + i\arg(-2) = \ln(2) + i\pi$.
All the logarithms of $-2$: $\mathrm{Log}(-2) = \ln(2) + i\pi + 2\pi i \Z$.

If we consider $z \mapsto \log(z)$,
a circle of points has fixed real component,
and the imaginary component ranges from $-\pi i$ to $\pi i$.
As the radius of the circle gets larger, the image (which is a line)
goes more to the right.

When $z = r \in \R$, $r > 0$: $\log(r) = \ln|r| + i\arg(r) = \ln(r)$.

We have $\log(e^{i\theta}) = \ln\left|e^{i\theta}\right| + i\arg(e^{i\theta})
= \ln(1) + i\theta = i\theta$ if $ - \pi < \theta < \pi$,
or $\log(\frac12 e^{i\theta}) = \ln(\frac12) + i\theta$.

We will now look at the multifunction.
$\mathrm{Arg}(z) = $ the set of all arguments of $z$,
which is $\arg(z) + 2\pi \Z = \{\arg(z) + 2\pi n \mid n \in \Z\}$.
The multifunction $\mathrm{Log}(z) = \ln|z| + i\mathrm{Arg}(z)$.

\subsubsection{Power functions}

We can use logarithms to define powers:
\[
	w^z = \exp(z\log(w))
\]
In this formula, the meaning of $w^z$ depends on the meaning of $\log(w)$.
The most common meaning is the principal value
(unless otherwise specified what we mean) and then the multifunction $\exp(z\mathrm{Log}(w))$.

A power function: fix a complex number $c \in \C$ and consider the function
$z \mapsto z^c = \exp(c \log(z))$.
E.g. $z \mapsto z^{1/3}$ ($c = 1/3$).
The principal value of $z^{1/3}$ is
$\exp(\frac13 \log(z)) = \exp(\frac13(\ln|z| + i\arg(z)))
= \exp(\frac13\ln|z|) \exp(i\frac13 \arg(z))
= |z|^{1/3}\exp(\underbrace{i \frac13 \arg(z)}_{\text{divide arg. by }3})$.
This is the unique solution of $w^3 = z$ with $-\pi/3 < \arg \leq \pi/3$.

If $p$ is a positive integer,
the principal value of $z^{1/p}$ is the unique $p$th root of $z$
in the region with arguments $-\pi/p < \arg \leq \pi/p$.
The multifuction $z^{1/p}$ gives all $p$ $p$th roots of $z$
(it is multivalued with $p$ many values).

\subsubsection{Exponential functions}
We define the exponential function $z \mapsto c^z$ with base $c$ as
\[
	c^z = \exp(z \log(c))
\]
and actually not multivalued,
once we make the choice of $\log{e}$,
e.g. $z \mapsto e^z = \exp(z\log{e})$
the usual choice $\log(e) = \ln(e) = 1$,
then $e^z = \exp(z \log(e)) = \exp(z)$.
(If we made the choice of $1 + 2\pi i$, then
$z \mapsto e^z = \exp(z(1+2\pi i)) = \exp(z)\exp(2\pi i z)$.)

We have $\exp(\log(z)) = z$.

\subsubsection{Properties of exponential, log, and power functions
(from formula sheet)}
\textbf{Exponential function} $\exp(z) = e^z$, if $e^z$ is defined
in terms of the principal value of $\log(e) = 1$, which is \textbf{emph} assumed.
\newline $e^{z_1 + z_2} = e^{z_1}e^{z_2}$.
\newline $(e^z)^n = e^{nz}$, if $n$ is an integer.
\newline $c^{z_1 + z_2} = c^{z_1}c^{z_2}$,
if all exponential are defined in terms of the same value of $\log(c)$.
\newline $(c^z)^n = c^{nz}$, if $n$ is an integer,
and if all exponentials are defined in terms of the same value of $\log(c)$.

\textbf{Arguments} the principal branch of the argument satisfies $-\pi < \arg(z) \leq \pi$,
the branch $\arg_\alpha$ of $\mathrm{Arg}$ with values
$-\pi + \alpha < \arg_\alpha(z) \leq \pi + \alpha$ is
$\arg_\alpha(z) = \arg(ze^{-i\alpha}) + \alpha$
(what??? is this just not doing anything?).

\textbf{Logarithms} Every branch $w = f(z)$ of the logarithm satsifies $\exp(w) = z$.
\newline The derivatie of any branch of the lograithm is $\frac{1}{z}$.
\newline The principal branch of the logarithm is $\log(z) = \ln|z| + i\arg(z)$.
\newline $\log(\exp(z)) = z$, if $-\pi < \mathrm{Im}(z) \leq \pi$.
\newline $\mathrm{Log}(\exp(z)) = z + 2\pi i \Z$.
\newline $\log(z_1z_2) = \log{z_1} + \log{z_2}$,
if $- \pi < \arg(z_1 + \arg(z_2) \leq \pi$.
\newline $\log(rz) = \ln(r) + \log{z}$, if $r \in \R_{>0}$.
\newline $\mathrm{Log}(z_1z_2) = \mathrm{Log}z_1 + \mathrm{Log}z_2$.
\newline the branch $\log_\alpha$ of $\mathrm{Log}$ with
$-\pi + \alpha < \mathrm{Im}(\log_\alpha(z)) \leq \pi + \alpha$
is $\log_\alpha(z) = \log(ze^{-i\alpha}) + i\alpha$.

\textbf{Power functions} $z^{c_1}z^{c_2} = z^{c_1 + c_2}$,
if all three power functions are defined in terms of the same branch
of the logarithm.
\newline $(z^c)^n = z^{nc}$, for all $n \in \Z$,
if both power functions are defined in terms of the same branch of the logarithm.
\newline $z^{c_1 + c_2} \subseteq z^{c_1}z^{c_2}$ for the power multifunctions.
\newline $(z_1 z_2)^c = z_1^cz_2^c$ for the principal branches if
$- \pi < \arg(z_1) + \arg(z_2) \leq \pi$.
\newline $(z_1z_2)^c = z_1^cz_2^c$ for the multifunctions.
\newline the derivative of $z^c$ is $cz^{c-1}$,
as long as both power functions are defined in terms of the same branch of the logarithm.
\newline every branch of $w = z^{m/n}$ ($m,n$ integers, $n \neq 0$) is
a solution of $w^n = z^m$.

Example: The rule $z^{c_1 + c_2} \subseteq z^{c_1}z^{c_2}$
for the power multifunctions:
\begin{align*}
	z^{c_1 + c_2}
	&= \exp((c_1 + c_2)\log(z))\\
	&= \exp((c_1 + c_2)(\ln|z| + i\mathrm{Arg}(z)))\\
	&= \exp((c_1 + c_2)(\ln|z| + i\arg(z) + 2\pi i \Z))\\
	&= \exp((c_1 + c_2)\ln|z| + i(c_1 + c_2)\arg(z) + (c_1+c_2)2\pi i \Z)\\
	z^{c_1}z^{c_2}
	&= \exp(c_1(\ln|z| + i\arg(z) + 2\pi i \Z))\exp(c_2(\ln|z| _ i\arg(z) + 2\pi i \Z))\\
	&= \exp(c_1(\ln|z| + i\arg(z) + 2\pi i \Z) + c_2(\ln|z| + i\arg(z) + 2\pi i \Z))\\
	&= \exp((c_1 + c_2) \ln|z| + (c_1 + c_2)i\arg(z)
	+ c_1 2\pi i \Z + c_2 2\pi i \Z)
\end{align*}
so the claim follows from $(c_1 + c_2)2\pi i \Z \subseteq c_1 2\pi i \Z + c_2 2 \pi i \Z$,
because if $k \in \Z$, $(c_1 + c_2)2\pi i k 
= c_1 2\pi i k + c_2 2 \pi i k \in c_1 2\pi i \Z + c_2 2\pi i \Z$,
but in general $c_1 2\pi i \Z + c_2 2\pi i \Z$ is larger than $(c_1 + c_2) 2\pi i \Z$.


\section{February 8 and 15}
Midterm 1: PLOMS (??? what the hell is this acronym).

\subsection{The cross-ratio}
\begin{definition}
	Given $4$ complex numbers (including $\infty$) $z,a,b,c$, at least $3$ of them distinct.
	Then the cross-ratio of $z,a,b,c$ is determined to be
	\[
		(z,a \colon b,c) := \frac{z-b}{z-c} \colon \frac{a-b}{a-c}
		= \frac{(z-b)(a-c)}{(z-c)(a-b)} \in \hat{\C}
	\]
\end{definition}
\begin{theorem}
	Fix three distinct points $a,b,c$ in $\hat{\C}$.
	Then the mapping $\hat{\C} \to \hat{\C}$ given by
	$z \mapsto (z,a \colon b,c) = \frac{(z-b)(a-c)}{(z-c)(a-b)}
	= \frac{(a-c)z + (c-a)b}{(a-b)z + (b-a)c}$
	is a M\"{o}bius transformation.
	In fact, it's the unique M.T. such that
	\begin{align*}
		a &\mapsto \frac{(a-b)(a-c)}{(a-c)(a-b)} = 1\\
		b &\mapsto \frac{(b-b)(a-c)}{(b-c)(a-b)} = 0\\
		c &\mapsto \frac{c-b}{c-c}\frac{a-c}{a-b} = \infty
	\end{align*}
\end{theorem}
You can use the cross-ratio to find the M.T. sending
$a \mapsto a'$, $b \mapsto b'$, and $c \mapsto c'$,
Since if $f$ is $(z,a\colon b,c)$ and $g$ is $(z,a'\colon b',c')$,
then $g^{-1}\circ f$ does what we want and is M.T.
(Note: BMPS uses a different convention.
There are $6$ ways to define the cross-ratio. Here, he followed wikipedia.)

\begin{theorem}
	Let $f \colon \hat{\C} \to \hat{\C}$ be a M\"{o}bius transformation.
	Then for any $4$ points in $\hat{\C}$ we have
	$(z,a \colon b,c) = (f(z),f(a) \colon f(b),f(c))$.
	So M\"{o}bius transformations preserve the cross-ratio.
\end{theorem}

\subsection{An application of Cauchy-Riemann equations}
Find a holomorphic function $f \colon \C \to \C$ with real part $u(x,y) = y - 2xy$.
We have $\frac{\partial u}{\partial x} = -2y$ and
$\frac{\partial u}{\partial y} = 1-2x$.
From the C.R. equations, this gives us $\frac{\partial v}{\partial y} = -2y$
and $\frac{\partial v}{\partial x} = -1 + 2x$.
These are 2 partial differential equations for the unknown function $v(x,y)$.
Examining the first, we guess $v(x,y) = -y^2 + h(x)$,
which gives $\frac{\partial v}{\partial x} = 0 + h'(x)$.
So using the second, we have $-1 + 2x = h'(x)$,
and so $h(x) = x^2 -x + C$,
so $v(x,y) = -y^2 + x^2 - x + C$.

The most general holomorphic function with real part $u(x,y) = y - 2xy$ is
\[
	f(x,y) = y - 2xy + i(-y^2 + x^2 - x + C)
\]
For example, $f(x,y) = y-2xy + i(-y^2 + x^2 - x)$.

\subsection{Geometric meaning of complex derivaties}
Suppose $f(z)$ is holomorphic at$z = z_0$.
What is the meaning of $f'(z_0)$?
Let $\gamma \colon \R \to \C$ be a path (differentiable),
or we can think of $\gamma \colon \R \to \R^2$.
We have $\gamma(t) = (x(t),y(t))$ and $\gamma'(t) = (x'(t),y'(t))$.
Note $\gamma'(t_0)$ is the tangent vector to the path
$\gamma$ at the point $\gamma(t_0)$.

Suppose that $\gamma(0) = z_0$.
Consider $\R \overset{\gamma}{\to}\C \overset{f}{\to} \C$ the composition.
The chain rule applies to this composition:
$(f\circ \gamma)'(0) = f'(\gamma(0))\gamma'(0) = f'(z_0) \gamma'(0)$.
Note the tangent vector to $\gamma$ at $z_0$ gets mapped by $f$
to the tangent vector of $f \circ g$ at $f(z_0)$.
We have shown that this is done by multypling by $f'(z_0)$.
So tangent vectors are rotated by $\arg(f'(z_0))$,
and are scaled by the factor $|f'(z_0)|$.
In particular, if $\gamma_1,\gamma_2$ both pass through $z_0$,
both tangent vectors are rotated by the same angle $\arg(f'(z_)))$,
i.e. the angle between the paths is preserved under $f$.
So holomorphic maps are angle preserving (what we call \emph{conformal})
and $|f'(z_0)|$ is the infinitesimal scaling factor.

So the derivative describes how paths in $\C$ change after getting mapped by $f$.

\subsection{Branches of logarithms and powers}
Ex. Find all values of $i^i$ and the principal value.
\begin{align*}
	i^i &= \exp(i \mathrm{Log}(i))\\
		&= \exp(i(\ln|i| + i\mathrm{Arg}(i)))\\
		&= \exp(-\mathrm{Arg}(i))\\
		&= \exp(-(\arg(i) + 2\pi\Z))\\
		&= \exp(-\pi/2 + 2\pi \Z)
\end{align*}
So the princiapl value of $i^i$ is $e^{-\pi/2}$.
All values of $i^i$ are $e^{-\pi/2 + 2\pi k}$ for $k \in \Z$ (they're all real).

\begin{definition}[Branch of the logarithm]
	A \emph{branch} of the logarithm is any holomorphic function
	$f \colon U \to \C$ where $U \in \C$ is open,
	and $\exp(f(z)) = z$ for all $z \in U$.
\end{definition}
For example, the principal branch (as a holomorphic function)
is defined on the open subset $U = \C \setminus \R_{\leq 0}$.
$\R_{\leq 0}$ is called the \emph{branch cut}.

Other branches include $\log(z) + 10 \pi i$ defined on $z \in \C \setminus\R_{\leq 0}$.
Check: $\log(z)$ is holomorphic for $z \in \C \setminus \R_{\leq 0}$
and $\exp(\log(z) + 10\pi i) = \exp(\log{z})\exp(10\pi i) = z \cdot 1 = z$.

Next, we will consider branches with branch cuts that are other rays through $0$.
Define $\arg_\alpha(z)$ to be the value of $\mathrm{Arg}(z)$ such that
$-\pi + \alpha < \arg_\alpha(z) \leq \pi + \alpha$.
Formula for $\arg_\alpha(z)$ in terms of $\arg(z)$:
start with $\mathrm{Arg}(z) - \alpha = \mathrm{Arg}(ze^{-i\alpha})
\implies \arg_\alpha(z) - \alpha = \arg(ze^{-i\alpha})$ (by definition of $\arg_\alpha$) hence
\[
	\arg_\alpha(z) = \arg(ze^{-i\alpha}) + \alpha
\]

This branch of $\mathrm{Arg}(z)$ gives rise to a branch of $\mathrm{Log}(z)$:
\begin{align*}
	\log_\alpha(z) &= \ln|z| + i\arg_\alpha(z)\\
				   &= \ln|z| + i\arg(ze^{-i\alpha}) + i\alpha\\
				   &= \ln|ze^{-i\alpha}| + i\arg(ze^{-i\alpha}) + i\alpha
\end{align*}
And so
\[
	\log_\alpha(z) = \log(ze^{-i\alpha}) + i\alpha
\]
(this is the $\log$ where the branch cut is the ray pointing in the direction of
$\pi + \alpha$ from the origin:
$\log_\alpha(z)$ is not holomorphic when $ze^{-i\alpha} = -r$
or $z = - re^{i\alpha}$ ($r \geq 0$)).

\subsection{Singularities}
Say $f$ is holomorphic on $U \setminus \{a\}$, $U$ open and $a \in U$.
\begin{itemize}
	\item $a$ is a \emph{removable singularity} if $f$ can be extended to $U$ holomorphically,
		e.g. $f(z) = \frac{z^2-1}{z+1}= z-1$ on $\C\setminus \{-1\}$,
		so we can extend $f$ by $f(-1) = -2$.
	\item $a$ is a \emph{pole} if $(z-a)^Nf(z)$ has a removable singularity
		for some $N > 0$,
		e.g. $f(z) = \frac{1}{(z+1)^5}$.
		Here $a = -1$.
		Then $(z-a)^5f(z) = (z+1)^5\frac{1}{(z+1)^5} = 1$.
		The smallest possible such $N$ is the order of the pole.
		So we say $f(z) = \frac{1}{(z+1)^5}$ has a pole of order $5$ at $z = -1$.
	\item $f$ has a \emph{zero} of order $n$ if $\frac{f(z)}{(z-a)^n}$
		has a removable singularity,
		and $n$ is the largest such value,
		e.g. $f(z) = (z+1)^2(z-2)$.
		Here $z = -1$ is a zero.
		$\frac{f(z)}{(z+1)^2} = z - 2$.
		so we say that it has a zero of order $2$ at $z = -1$.
\end{itemize}

Fact: If $f(z) = (z-a)^ng(z)$ where $g(z)$ is holomorphic at $z = a$,
then if $n > 0$, $f$ has a zero of order $n$ at $z = a$,
and if $n < 0$, $f$ has a pole of order $n$ at $z = a$.

\begin{definition}
	$f(z)$ is \emph{meromorphic} in $U \subset \C$ open
	if the only type of singularities it has are poles (or removable singularities).
\end{definition}
If a singularity is neither removable, nor a pole, it is an \emph{essential} singularity.
E.g. $e^{1/z}$ at $z = 0$ is an essential singularity.

\begin{remark}
	Meromorphic functions $U \to \C$ can be extended to continuous functions
	$U \to \hat{\C}$ by sending the poles to $\infty$.
\end{remark}
To see this, for continuity, need $f(a) = \lim_{z \to a} f(z)$,
and so if $a$ is a pole of $f$,
$\lim_{z \to a} f(z) = \lim_{z \to a} \frac{1}{(z-a)^n}g(z)
= \frac{1}{0}g(a) = \frac{\neq 0}{0} = \infty = f(a)$
(where we know $g(a) \neq 0$ and $n > 0$).
In fact, meromorphic functions can be thought of as holomorphic functions $U \to \hat{\C}$.

To extend meromorphic functions defined in $U = \C$ to $\hat{\C}$:
\begin{definition}
	The type of singularity of $f(z)$ at $z - \infty$ is the
	type of singularity of $f(\frac{1}{z})$ at $z = 0$.
\end{definition}
Ex. $h(z) = \frac{2z^2 + 3z + 1}{5z^5 + z}$.
Write $w = \frac{1}{z}$ (so $z = \infty \leftrightarrow w= 0$)
and substitute for $z$.
\[
	\frac{2\frac{1}{w^2} + 3\frac{1}{w} + 1}{5\frac{1}{w^5} + \frac{1}{w}}
	= \frac{2w^3 + 3w^4 + w^5}{5 + w^4} = w^3 \underbrace{\frac{2 + 3w + w^2}
	{5 + w^4}}_{g(w), g(0)\neq0}
\]
this has a zero of order $3$ at $w = 0$.
So $h(z)$ has a zero of order $3$ at $z = \infty$.
So we can extend $h \colon \hat{\C} \to \hat{\C}$ continuous where $\infty \mapsto 0$.
(Again, in a more advanced course,
we can consider meromorphic functions to be holomorphic functions $\hat{\C} \to \hat{\C}$.
Examples: M\"{o}bius transformations are the unique meromorphic functions
$\hat{\C} \to \hat{\C}$ which are one-to-one and onto.)
\end{document}
