\documentclass{article}
\usepackage{amsmath, amsfonts, amsthm, amssymb}
\usepackage{geometry}
\geometry{letterpaper, margin=2.0cm, includefoot, footskip=30pt}

\usepackage{fancyhdr}
\pagestyle{fancy}

\lhead{Math 437}
\chead{Homework 2}
\rhead{Nicholas Rees, 11848363}
\cfoot{Page \thepage}

\newtheorem*{problem}{Problem}

\newcommand{\N}{{\mathbb N}}
\newcommand{\Z}{{\mathbb Z}}
\newcommand{\Q}{{\mathbb Q}}
\newcommand{\R}{{\mathbb R}}
\newcommand{\C}{{\mathbb C}}
\newcommand{\ep}{{\varepsilon}}

\renewcommand{\theenumi}{(\alph{enumi})}

\begin{document}
\section{Problem 1}
{\it Find all integers $n>1$ with the property that for each positive
divisor $d$ of $n$, we also have that
\[
	(d+2)\mid(n+2)
\]
}
\begin{proof}[Solution]
	We first show no even integers satisfy our property.
	If $n$ is even, there exists some $k \in \N$ such that $2k = n$,
	so $k \mid n$.
	Furthermore, $n+2$ is even as well,
	so $\frac{n+2}{2}$ is an integer,
	specifically $\frac{n+2}{2} = \frac{n}{2}+1 = k+1$.
	So $2(k+1) = n+2 \implies k+1 \mid n+2$.
	Furthermore, $k+1$ must be the greatest possible divisor of $n+2$
	that is not equal to $n+2$,
	since if a divisor $d$ was greater than $k+1$,
	there is some integer $1 < m < 2$ where $dm = n+2$,
	but no such integer $m$ exists.
	But $k + 2 > k + 1$, and $k + 2 = \frac{n}{2} + 2 \neq n + 2$
	when $n > 1$,
	so $k + 2 \nmid n + 2$.
	So there exists a divisor of $n$ such that two more than it
	is not a divisor of $n + 2$.

	Now, we consider odd composite integers $n$.
	Then there exists an integers $1 < d,q < n$ (not necessarily distinct)
	such that $qd = n$ (so $d \mid n$).
	Without loss of generality, let $d \geq q$.
	Then $n + 2 = qd + 2$.
	Note that both $q$ and $d$ must be odd,
	otherwise $n$ would be divisible by $2$ and would be even,
	which would be against our assumption.
	We can write $n + 2 = qd + 2 + 2q - 2q = q(d+2) - 2(q-1)$,
	so
	\[
		(n+2)/(d+2) = q - 2(q-1)/(d+2)
	\]
	But since $d \geq q$, $d+2 > q - 1$ so $d+2 \nmid q-1$.
	Furthermore, $q-1$ is even so $2(q-1)$ is even,
	but $d+2$ is odd, so $d+2 \nmid 2(q-1)$,
	so $2(q-1)/(d+2)$ is not an integer,
	thus our term on the right is not an integer.
	But then $d + 2 \nmid n+2$.
	Thus odd composite integers do not satisfy our property either.

	Finally, consider the only remaining possibility,
	when $n$ is an odd prime number.
	The only such divisors of this is $n$ and $1$.
	$n + 2 \mid n + 2$ trivially.
	If and only if $1 + 2 = 3 \nmid n + 2$, we have our desired property then.
	So if $n$ is prime and of the form $n = 3j - 2$ for some $j \in \N$,
	then $n$ must satisfiy our property.
	We have shown that no other such $n$ can satisfy our property,
	thus this is all the possible solutions.
\end{proof}
\clearpage

\section{Problem 2}
{\it Find all positive integers $m$ and $n$ such that
\[
	2^m - 3^n = 7
\]
}
\begin{proof}[Solution]
	We can rearrange our equation to get
	\begin{equation}\label{237}
		2^m = 7 + 3^n
	\end{equation}
	Obviously, any $m$ that satisfies the above equation
	will also satisfy $2^m = 2\cdot2^{m-1} \equiv 7 \; (\mathrm{mod} \; 3)$
	(since $3 \mid 3^n$ for any $n$).
	That is to say, the set of solutions $M_1$ to equation (\ref{237})
	(elements in $M_1$ are of the form $2^m$)
	is a subset of the set of solutions $M_2$ to our subsequent relation,
	$M_1 \subset M_2$.
	If $X$ is the set of solutions $x\in\Z$ to $2x \equiv 7 \; (\mathrm{mod} \; 3)$,
	then clearly $M_2 \subset X$.

	Recall proposition 7.2 (B) from the course notes: if $a,b,m \in \Z$ with $m\neq0$
	and $d = gcd(a,m)$,
	then if $d \mid b$, the congruence equation $ax \equiv b \; (\mathrm{mod} \; m)$
	has exactly $d$ solutions.
	Since $2$ and $3$ are coprime, we have that $d = 1$,
	so $d \mid 7$,
	thus $2x \equiv 7 \; (\mathrm{mod} \; 3)$
	has exactly one solution.
	Thus, $X$ has exactly one element.
	Therefore, since $M_1 \subset X$,
	equation (\ref{237}) has at most one solution.

	We can verify that there does exist such a solution,
	namely when $m = 4$ and $n = 2$,
	then we have $2^4 - 3^2 = 16 - 9 = 7$.
\end{proof}
\clearpage

\section{Problem 3}
{\it Let $k \in \N$. Show that there exists $k$ consecutive positive integers
with the property that no integer from this set is of the form
$a^2 + b^2$ for some $a,b \in \Z$.}
\begin{proof}[Solution]
	Let $k \in \N$ be arbitrary.
	Let $m$ be the product of the squares of the first
	$k$ primes $q$ of the form $q = 4j + 3$
	(for some $j \in \N$),
	i.e. $m = \prod_{i=1}^k q^2_i$.
	
	Note that we can always $k$-many $q_i$.
	To prove this, for the sake of contradiction,
	assume there are only $r < k$ many such primes of this form,
	$q_1,q_2,\dots,q_r$
	(where $q_1 < q_2 < \cdots < q_r$).
	Note that $n = 4q_1q_2\cdots q_r -1$ is of the form $4j+3$,
	but $n > q_r$, so by assumption,
	$n$ cannot be prime,
	so is composite.
	Further, note that none of the $q_i$ and $2$ divide $n$,
	thus all the primes in the prime factor decomposition of $n$
	is of the form $4j + 1$, thus $n \equiv 1\, (\mathrm{mod}\, 4)$
	which is a contradiction.
	Thus $n$ is a prime greater than $q_r$ and $n = 4j - 1$.
	We can do this indefinitely to get $k$ many primes of the form $4j + 3$.

	Now, consider the system
	\begin{align*}
		x &\equiv q_1 - 1 \, (\mathrm{mod}\, q_1^2)\\
		x &\equiv q_2 - 2 \, (\mathrm{mod}\, q_2^2)\\
		  &\vdots\\
		x &\equiv q_k - k \, (\mathrm{mod}\, q_k^2)\\
	\end{align*}
	By the Chinese Remainder Theorem,
	there exists a unique solution to the system modulo $m$,
	which we'll call $x_0$.
	Then, let $x_i = x_0 + i$ for $1 \leq i \leq k$.
	Note that $x_i = q_i + nq_i^2$,
	thus $q_i \mid x_i$ but $q_i^2 \nmid x_i$
	(and so $q_i^s \nmid x_i$ for all $s \geq 2$).
	Thus, there are $k$ consecutive integers $x_1,x_2,\dots,x_k$
	whose prime number decomposition that contain
	a prime of the form $4j+3$ with exponent $1$.
	But by Theorem 13.4 from the notes,
	for all $a,b \in \Z$, $a^2 + b^2 \neq x_i$ for all $1 \leq i \leq k$,
	since the exponent of $q_i$ is $\gamma_i = 1$,
	which is not even.
\end{proof}
\clearpage

\section{Problem 4}
{\it As always, for each positive integer $m$,
we have that $d(m)$ is the number of the positive divisors of $m$;
also, we let $\phi(m)$ be the corresponding value of
the Euler $\phi$-function.
Then compute the following limits:
\[
	\lim_{n\to\infty}\frac{n!}{d(n!)\phi(n!)}
\]
\[
	\lim_{n\to\infty} \frac{n!}{2^{d(n!)}}
\]}

\begin{proof}[Solution]
	\hfill\break
	\textbf{Limit 1:}
	Note that the prime factor decomposition of $n!$
	contains every prime that came before it,
	since if $p$ is prime and $p < n$,
	$p \mid n!$ by definition of factorial.
	Thus, if we enumerate the primes in order (ie. $p_1 = 2$, $p_2 = 3$, etc.),
	let $p_r$ be the greatest prime less than or equal to $n$.
	Then we can write $n! = \prod_{i=1}^r p_i^{\alpha_i}$ thus
	\[
		\frac{n!}{d(n!)\phi(n!)} =
		\prod_{i=1}^r \frac{p_i^{\alpha_i}}{d(p_i^{\alpha_i})\phi(p_i^{\alpha_i})}
		= \prod_{i=1}^r \frac{p_i^{\alpha_i}}{(\alpha_i+1)p_i^{\alpha_i-1}(p-1)}
		= \prod_{i=1}^r \frac{p_i}{(\alpha_i+1)(p_i-1)}
	\]
	Note that for any 

	\hfill \break
	\noindent\textbf{Limit 2:}
	We are proving $\frac{a_{n+1}}{a_n} = (n+1)2^{d(n!) - d((n+1)!)} < 1$.
	If $n+1$ is prime,
	then ff see photo.

	Now if $n+1$ is not prime,
	we claim the inequality $n+1 < 2^{\prod_{\beta_i \neq 0}(\alpha_i + \beta_i + 1)}$.
	If $n+1$ is a prime squared,
	something something $48^2$ has $6$ factors of $7$...
	we show that it works.
	Somehow, this links to when $n+1$ contains any prime squared.

	Now, our remaining case is when $n+1$ is a sequence of primes with exponent $1$.

	Current strategy:
	if $d((n+1)!) - d(n!) > n \implies d((n+1)!) > n + d(n!)$,
	we are done since $(n+1)2^{-n}$ is definitely less than $1$.
	Asymptotoically, this looks like $d(n!) > n^2$.
\end{proof}
\end{document}
