\documentclass{article}
\usepackage{amsmath, amsfonts, amsthm, amssymb}
\usepackage{geometry}
\geometry{letterpaper, margin=2.0cm, includefoot, footskip=30pt}

\usepackage{fancyhdr}
\pagestyle{fancy}

\lhead{Math 437}
\chead{Homework 2}
\rhead{Nicholas Rees, 11848363}
\cfoot{Page \thepage}

\newtheorem*{problem}{Problem}

\newcommand{\N}{{\mathbb N}}
\newcommand{\Z}{{\mathbb Z}}
\newcommand{\Q}{{\mathbb Q}}
\newcommand{\R}{{\mathbb R}}
\newcommand{\C}{{\mathbb C}}
\newcommand{\ep}{{\varepsilon}}

\renewcommand{\theenumi}{(\alph{enumi})}

\begin{document}
\section{Problem 1}
{\it Find all integers $n>1$ with the property that for each positive
divisor $d$ of $n$, we also have that
\[
	(d+2)\mid(n+2)
\]
}
\begin{proof}[Solution]
	ff
\end{proof}
\clearpage

\section{Problem 2}
{\it Find all positive integers $m$ and $n$ such that
\[
	2^m - 3^n = 7
\]
}
\begin{proof}[Solution]
	We can rearrange our equation to get
	\begin{equation}\label{237}
		2^m = 7 + 3^n
	\end{equation}
	Obviously, any $m$ that satisfies the above equation
	will also satisfy $2^m = 2\cdot2^{m-1} \equiv 7 \; (\mathrm{mod} \; 3)$
	(since $3 \mid 3^n$ for any $n$).
	That is to say, the set of solutions $M_1$ to equation (\ref{237})
	(elements in $M_1$ are of the form $2^m$)
	is a subset of the set of solutions $M_2$ to our subsequent relation,
	$M_1 \subset M_2$.
	If $X$ is the set of solutions $x\in\Z$ to $2x \equiv 7 \; (\mathrm{mod} \; 3)$,
	then clearly $M_2 \subset X$.

	Recall proposition 7.2 (B) from the course notes: if $a,b,m \in \Z$ with $m\neq0$
	and $d = gcd(a,m)$,
	then if $d \mid b$, the congruence equation $ax \equiv b \; (\mathrm{mod} \; m)$
	has exactly $d$ solutions.
	Since $2$ and $3$ are coprime, we have that $d = 1$,
	so $d \mid 7$,
	thus $2x \equiv 7 \; (\mathrm{mod} \; 3)$
	has exactly one solution.
	Thus, $X$ has exactly one element.
	Therefore, since $M_1 \subset X$,
	equation (\ref{237}) has at most one solution.

	We can verify that there does exist such a solution,
	namely when $m = 4$ and $n = 2$,
	then we have $2^4 - 3^2 = 16 - 9 = 7$.
\end{proof}
\clearpage

\section{Problem 3}
{\it Let $k \in \N$. Show that there exists $k$ consecutive positive integers
with the property that no integer from this set is of the form
$a^2 + b^2$ for some $a,b \in \Z$.}
\begin{proof}[Solution]
	ff
\end{proof}
\clearpage

\section{Problem 4}
{\it As always, for each positive integer $m$,
we have that $d(m)$ is the number of the positive divisors of $m$;
also, we let $\phi(m)$ be the corresponding value of
the Euler $\phi$-function.
Then compute the following limits:
\[
	\lim_{n\to\infty}\frac{n!}{d(n!)\phi(n!)}
\]
\[
	\lim_{n\to\infty} \frac{n!}{2^{d(n!)}}
\]}

\begin{proof}[Solution]
	\hfill\break
	\textbf{Limit 1:}
	$d(mn) = d(m)d(n), \phi(mn)=\phi(m)\phi(n)$ when $m,n$ coprime.
	We can see that $d(n!) = d(p_1^{e_1})d(\prod_{i=2}^kp_i^{e_i})
	= \prod (e_i + 1)$.
	And can also check $\phi(n!) = \prod (p_j^{e_j} - p_{j-1}^{e_{j-1}}$
	or something from $\phi(p_1\cdots p_k) = \prod\phi(p_i) = \prod(p_i-1)$.
	Simpler case: $d(p_1\cdots p_k) = \prod_i^k d(p_i) = \prod_i^k 2 = 2^k$
	and $\phi(p_1\cdots p_k) = ff$.

	\hfill \break
	\noindent\textbf{Limit 2:} ff
\end{proof}
\end{document}
