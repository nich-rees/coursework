\documentclass{article}
\usepackage{amsmath, amsfonts, amsthm, amssymb}
\usepackage{geometry}
\geometry{letterpaper, margin=2.0cm, includefoot, footskip=30pt}

\usepackage{fancyhdr}
\pagestyle{fancy}

\lhead{Math 437}
\chead{Homework 2}
\rhead{Nicholas Rees, 11848363}
\cfoot{Page \thepage}

\newtheorem*{problem}{Problem}

\newcommand{\N}{{\mathbb N}}
\newcommand{\Z}{{\mathbb Z}}
\newcommand{\Q}{{\mathbb Q}}
\newcommand{\R}{{\mathbb R}}
\newcommand{\C}{{\mathbb C}}
\newcommand{\ep}{{\varepsilon}}

\renewcommand{\theenumi}{(\alph{enumi})}

\begin{document}
	$d(mn) = d(m)d(n), \phi(mn)=\phi(m)\phi(n)$ when $m,n$ coprime.
	We can see that $d(n!) = d(p_1^{e_1})d(\prod_{i=2}^kp_i^{e_i})
	= \prod (e_i + 1)$.
	And can also check $\phi(n!) = \prod (p_j^{e_j} - p_{j-1}^{e_{j-1}}$
	or something from $\phi(p_1\cdots p_k) = \prod\phi(p_i) = \prod(p_i-1)$.
	Simpler case: $\d(p_1\cdots p_k) = \prod_i^k d(p_i) = \prod_i^k 2 = 2^k$
	and $\phi(p_1\cdots p_k) = ff$.
\end{document}
