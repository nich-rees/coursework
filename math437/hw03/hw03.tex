\documentclass{article}
\usepackage{amsmath, amsfonts, amsthm, amssymb}
\usepackage{geometry, hyperref}
\geometry{letterpaper, margin=2.0cm, includefoot, footskip=30pt}

\usepackage{fancyhdr}
\pagestyle{fancy}

\lhead{Math 437}
\chead{Homework 2}
\rhead{Nicholas Rees, 11848363}
\cfoot{Page \thepage}

\newtheorem*{problem}{Problem}

\newcommand{\N}{{\mathbb N}}
\newcommand{\Z}{{\mathbb Z}}
\newcommand{\Q}{{\mathbb Q}}
\newcommand{\R}{{\mathbb R}}
\newcommand{\C}{{\mathbb C}}
\newcommand{\ep}{{\varepsilon}}

\renewcommand{\theenumi}{(\alph{enumi})}

\begin{document}
\section{Problem 1}
{\it Let $\{a_n\}_{n\geq 0}$ be a sequence defined as follows:
\[
	a_0 = 0; a_1 = 1; a_2 = 2 \text{ and}
\]
\[
	a_{n+3} = 5^n\cdot a_{n+2} + n^2\cdot a_{n+1} + 11a_n
	\text{ for }n\geq 0
\]
Prove that there exist infinitely many $n \in \N$
such that $2023 \mid a_n$.
}
\begin{proof}[Solution]\let\qed\relax
	Try to find periodic behaviour: let $a_k = a_j$,
	then try to find information.
\end{proof}
\clearpage

\section{Problem 2}
{\it Let $n \in \N$.
Find the number of solutions for the congruence equation:
\[
	x^3 \equiv 1 \; (\mathrm{mod}\, n)
\]
}
\begin{proof}[Solution]\let\qed\relax
	ff
\end{proof}
\clearpage

\section{Problem 3}
\textbf{As always, $\phi(\cdot)$ is the Euler-$\phi$ function.}\newline
{\it Let $\alpha$ be any real number in the interval $[0,1]$.
Prove that there exists an infinite sequence $\{n_k\}_{k \geq 1} \subset \N$ such that
\[
	\lim_{k\to\infty} \frac{\phi(n_k)}{n_k} = \alpha
\]
}
\begin{proof}[Solution]\let\qed\relax
If $n = \prod_{i=1}^{\pi(n)} p_i^{\alpha_i}$, then
\[
	\phi(n) = \prod_{i=1}^{\pi(n)}(p_i^{\alpha_i} - p_i^{\alpha_i - 1})
	= n \prod_{\substack{p\mid n \\ p \text{ prime}}}\left(1 - \frac{1}{p}\right)
\]
Thus
\[
	\frac{\phi(n)}{n} = \prod_{\substack{p\mid n \\ p \text{ prime}}}\left(1 - \frac{1}{p}\right)
\]
To show that $\{\frac{\phi(n)}{n}\}$ is dense in $(0,1)$,
let $x,y \in (0,1)$ where $x < y$,
then we claim there exists $n$ such that $x < \frac{\phi(n)}{n} < y$.
It is sufficient to consider when $x,y \in \Q \cap (0,1)$.
Let $x = p'_1/q_1$ and $y = p'_2/q_2$ where $p'_i < q_i$ and $\gcd(p'_i,q_i) = 1$.
We can rewrite them to have the same denomonator:
$x = p_1/q$ and $y = p_2/q$, where $p_1 < p_2 < q$.
Note $1 - \frac{1}{p} = \frac{p-1}{p}$.
Choose the $p$ such that $p \mid q$
(I want something more, like the product of all the primes is $q$).
So let's just assume that $q$'s prime decomposition
only has exponents $1$ (and then could show this is dense)
so then we want $p_1 < \prod(p-1) < p_2$.
Perhaps if it is too big, then we find more $p$ later and multiply it down.
If it is too small, ff

This is the problem: we can't write every rational number
as a product of $(p-1)/p$,
or even every rational with denominator whose prime number decomposition
do not have extra exponents.
But somehow we achieve density.

Note that $\prod(1-\frac{1}{p})$ converges iff $\sum -\frac{1}{p}$ converges
(\href{https://math.stackexchange.com/questions/2966559/convergence-of-infinite-product-prod1a-n-where-a-n-changes-sign}{stack exchange link in source code}).
But I don't think we really care.

Maybe useful: the rationals in simplified form do not
contain any shared primes between the numerator and denominator.

Maybe we do a completely different approach to density.
What if $n = q$ or something.
\end{proof}
\end{document}
