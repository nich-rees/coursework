\documentclass{article}
\usepackage{amsmath, amsfonts, amsthm, amssymb}
\usepackage{geometry, hyperref}
\geometry{letterpaper, margin=2.0cm, includefoot, footskip=30pt}

\usepackage{fancyhdr}
\pagestyle{fancy}

\lhead{Math 437}
\chead{Homework 3}
\rhead{Nicholas Rees, 11848363}
\cfoot{Page \thepage}

\newtheorem*{problem}{Problem}

\newcommand{\N}{{\mathbb N}}
\newcommand{\Z}{{\mathbb Z}}
\newcommand{\Q}{{\mathbb Q}}
\newcommand{\R}{{\mathbb R}}
\newcommand{\C}{{\mathbb C}}
\newcommand{\ep}{{\varepsilon}}

\renewcommand{\theenumi}{(\alph{enumi})}

\begin{document}
\section{Problem 1}
{\it Let $\{a_n\}_{n\geq 0}$ be a sequence defined as follows:
\[
	a_0 = 0; a_1 = 1; a_2 = 2 \text{ and}
\]
\[
	a_{n+3} = 5^n\cdot a_{n+2} + n^2\cdot a_{n+1} + 11a_n
	\text{ for }n\geq 0
\]
Prove that there exist infinitely many $n \in \N$
such that $2023 \mid a_n$.
}
\begin{proof}[Solution]\let\qed\relax
	Let
	\[
		V_n = (\overline{a_{n+2}}, \overline{a_{n+1}}, \overline{a_{n}},
		\overline{5^n}, \overline{n})
	\]
	where $\overline{m}$ is the equivalence class of $m$ modulo $2023$.
	Note that there are only $2023^5$ permutations of
	\newline$(a_{n+2}, a_{n+1}, a_{n}, 5^n, n)$ when each element
	is considered modulo $2023$,
	thus there are only $2023^5$ possible values of $V_n$.
	Furthermore, $V_n$ determines uniquely $V_{n+1}$:
	if $V_n =
	(\overline{v_1},\overline{v_2},\overline{v_3},\overline{v_4},\overline{v_5})$, then
	\begin{align*}
		n+1 &\equiv v_5 + 1 \,(\mathrm{mod}\, 2023)\\
		5^{n+1} &\equiv 5v_4 \,(\mathrm{mod}\, 2023)\\
		a_{n+1} &\equiv v_2 \,(\mathrm{mod}\, 2023)\\
		a_{n+2} &\equiv v_1 \,(\mathrm{mod}\, 2023)\\
		a_{n+3} &\equiv v_4v_1 + v_5^2v_2 + 11v_3 \,(\mathrm{mod}\, 2023)
	\end{align*}
	which determines $V_{n+1}$.
	Hence $V_n$ determines uniquely $V_{n+i}$ for all $i \in \N$
	(since $V_{n+i}$ is determined by $V_{n+i-1}$,
	and $V_{n+i-1}$ is determined by $V_{n+i-2}$, etc.
	until we get that it is determined by $V_{n}$).
	Hence, if $V_n = V_m$, we must have $V_{n+i} = V_{m+i}$
	for all $i \in \N$.

	Let $k = 2023^5 + 1$, and consider $V_k$.
	By the pigeon-hole principle, there must exist some $m \leq 2023^5$ such that
	$V_k = V_m$ and thus we must have that
	$V_{k+i} = V_{m+i}$ for all $i \in \N$, as we proved before.
	Hence, $a_{k+i} \equiv a_{m+i}\, (\mathrm{mod} \, 2023)$ for all $i \in \N$,
	since these are just the third element of our $V$'s,
	which must be equal for equality of $V_{k+i}$ and $V_{m+i}$.
	Thus, $(a_n)$ is periodic with period $p = k - m$.

	Note that $a_0 = 0$, thus, it is sufficient to show that
	$\overline{a_0} = \overline{a_{0+p}}$.
	To prove this, assume for the sake of contradiction that there
	is some least $j > 0$ where $\overline{a_{j+p}} = \overline{a_j}$
	but $\overline{a_{j+p-1}} \neq \overline{a_{j-1}}$
	($a_{j-1}$ will always be defined since $j-1 \geq 0$).
	Then $V_j = V_{j+p}$ and $V_{j-1} \neq V_{j + p - 1}$.
	See that
	\begin{itemize}
		\item $a_{j+1} \equiv a_{j+p+1} \, (\mathrm{mod}\, 2023)$
		\item $a_{j} \equiv a_{j+p} \, (\mathrm{mod}\, 2023)$
		\item $5^{j} \equiv 5^{j+p} \, (\mathrm{mod}\, 2023)$
			implies $5^{j-1} \equiv 5^{j+p-1} \, (\mathrm{mod}\, 2023)$
			since $5$ is coprime with $2023$ and so we can divide it out
		\item $j \equiv j+p \, (\mathrm{mod}\, 2023)$
			implies that  $j-1 \equiv j+p-1 \, (\mathrm{mod}\, 2023)$
			(and also $j^2 \equiv (j+p)^2 \,(\mathrm{mod}\, 2023)$,
			which we will make use of later).
	\end{itemize}
	Thus, since $V_{j-1} \neq V_{j + p - 1}$,
	since all the other elements are the same,
	we must have that $a_{j-1} \not\equiv a_{j+p-1}\,(\mathrm{mod}\,2023)$.
	But since we have $a_{j+2} \equiv a_{j+p+2} \, (\mathrm{mod}\, 2023)$,
	we have
	\[
		5^j\cdot a_{j+1} + j^2\cdot a_j + 11a_{j-1}
		\equiv 5^{j+p}\cdot a_{j+p+1} + (j+p)^2\cdot a_{j+p} + 11a_{j+p-1}
		\;(\mathrm{mod}\,2023)
	\]
	\[
		\implies 11a_{j-1} \equiv 11a_{j+p-1}\;(\mathrm{mod}\,2023)
	\]
	\[
		\implies a_{j-1} \equiv a_{j+p-1}\;(\mathrm{mod}\,2023)
	\]
	since $11$ is coprime with $2023$ so we can divide out by it.
	But this contradicts that $V_{j-1} \neq V_{j + p - 1}$.
	Therefore, $j\not>0$,
	so $\overline{a_0} = \overline{a_{0+p}} = \overline{0}$,
	and this infinitly repeats every $p$,
	thus there are infinitely many $n$ such that $2023 \mid a_n$.
\end{proof}
\clearpage

\section{Problem 2}
{\it Let $n \in \N$.
Find the number of solutions for the congruence equation:
\[
	x^3 \equiv 1 \; (\mathrm{mod}\, n)
\]
}
\begin{proof}[Solution]\let\qed\relax
	Consider the unique prime factors of $n$,
	specifically $n = p_1^{\alpha_1}p_2^{\alpha_2} \cdots p_r^{\alpha_r}$
	(where $\alpha_i \geq 1$).

	If we have $p_i = 2$, we seek to find the number of solutions to
	$x^3 - 1 \equiv 0 \, (\mathrm{mod}\, 2^{\alpha_i})$.
	Notice that we must have $2^{\alpha_i} \mid x^3 - 1$,
	so $x^3 - 1$ is even, hence $x^3$ is odd, which implies $x$ is odd.
	Then, $\frac{d}{dx}(x^3-1) = 3x^2$ is not even,
	so $\frac{d}{dx}(x^3-1) \neq 0 \,(\mathrm{mod}\,2^{\alpha_i})$.
	Therefore, we can invoke Hensel's lemma:
	if $x^3 - 1 \equiv 0 \, (\mathrm{mod}\, 2)$,
	the only solution modulo $2$ is $x = 1$,
	thus we know that $x^3 - 1 \equiv 0 \,(\mathrm{mod}\, 2^{\alpha_i})$
	has only one solution as well.
	Thus, there is only one solution to $x^3 \equiv 1 \, (\mathrm{mod}\, p_i^{\alpha_i})$.

	If $p_i \neq 2$,
	note that since $p \nmid 1$, and $3 \in \Z^+$,
	by theorem 18.2, we have the number of solutions to
	$x^3 \equiv 1 \,(\mathrm{mod}\,p_i^{\alpha_i})$
	is $d_i = \gcd(3,p_i^{\alpha_i})$
	(note that we never have the $0$ solutions case,
	because $1^{\phi(p_i^{\alpha_i})/d} \equiv 1 \, (\mathrm{mod}\,p_i^{\alpha_i})$ always).
	We can now compute $d_i$:
	\[
		d_i = \gcd(3,\phi(p_i^{\alpha_i})) = \gcd(3, p^{\alpha_i-1}(p_i-1))
	\]
	We can have $p_i \equiv 0\, (\mathrm{mod}\, 3)$,
	$p_i \equiv 1\, (\mathrm{mod}\, 3)$,
	or $p_i \equiv 2\, (\mathrm{mod}\, 3)$.

	In the $0\, (\mathrm{mod}\, 3)$, this says that $3 \mid p_i$,
	which is only true when $p_i = 3$ (by the definition of a prime).
	Then if $\alpha_i = 1$, we have $\gcd(3,2) = 1$.
	If $\alpha_i > 1$, we have $\gcd(3,3^{\alpha_i}2) = 3$.

	If $p_i \equiv 1\, (\mathrm{mod}\, 3)$, then
	$\gcd(3,p_i^{\alpha_i}(p_i - 1)) = 3$ since $3 \mid p_i - 1$
	and $p_i^{\alpha_i} \geq 3+1$.
	
	If $p_i \equiv 2\, (\mathrm{mod}\, 3)$,
	then $\gcd(3,p_i^{\alpha_i}(p_i-1) = 1$,
	since $3 \nmid p_i^{\alpha_i}$ (by definition of $p_i$ being prime and not $3$)
	and $3 \nmid p_i - 1 = 3k + 1$
	by definition of $p_i$ being $2\, (\mathrm{mod}\, 3)$.

	Let $N_P(m)$ denote the number of solutions to
	$x^3 - 1 \equiv 0\, (\mathrm{mod}\,m)$.
	From Theorem 8.2, since
	$p_i^{\alpha_i}$ is coprime with $p_j^{\alpha_j}$ when $i \neq j$,
	we have $N_P(n) = \prod N_P(p_i^{\alpha_i})$.
	We can rewrite $n$ as
	\[
		n = 2^l3^k\prod_{i=1}^r p_i^{\alpha_i} \prod_{j=1}^s q_i^{\beta_j}
	\]
	where $l,k \in \N \cup \{0\}$, $p_i,q_j$ are prime,
	$p_i \equiv 1 \, (\mathrm{mod}\,3)$,
	$q_j \equiv 2 \, (\mathrm{mod}\,3)$ not $2$,
	and $r$ and $s$ are the number of such primes where $\alpha_i,\beta_j \geq 1$.

	Thus,
	\[
		N_P(n) = N_P(2^l)N_P(3^k)
		\prod_{i=1}^r N_P(p_i^{\alpha_i}) \prod_{j=1}^s N_P(q_i^{\beta_j})
		= N_P(3^k)3^r
	\]
	Hence, we have
	\[
		N_P(n) = \begin{cases}
			3^{r+1} & \text{if }k>1\\
			3^{r} & \text{otherwise}
		\end{cases}
	\]
\end{proof}
\clearpage

\section{Problem 3}
\textbf{As always, $\phi(\cdot)$ is the Euler-$\phi$ function.}\newline
{\it Let $\alpha$ be any real number in the interval $[0,1]$.
Prove that there exists an infinite sequence $\{n_k\}_{k \geq 1} \subset \N$ such that
\[
	\lim_{k\to\infty} \frac{\phi(n_k)}{n_k} = \alpha
\]
}
\begin{proof}[Solution]\let\qed\relax
For any $n \in \N$, we can take the prime factor decomposition
$n = p_1^{\alpha_1} \cdots p_r^{\alpha_r}$ where all $\alpha_i \geq 1$.
Furthermore, recall that $\phi(n) = \prod_{i=1}^r p_i^{\alpha_i-1}(p_i-1)$, hence
\[
	\frac{\phi(n)}{n} =
	\frac{\prod_{i=1}^r p_i^{\alpha_i-1}(p_i-1)}{\prod_{i=1}^r p_i^{\alpha_i}}
	= \prod_{i=1}^r \frac{p_i-1}{p_i}
\]

We let $\ep > 0$ be arbitrary.
By the Archimedean principle, there exists some $N \in \N$ such that
$N > \ep > 0$, so $\frac{1}{N} < \ep$.
By the infinitude of primes, there is some $J \in \N$ such that
$p_J > N$ so $\frac{1}{p_J} < \ep$ as well.
If $j > J$, and we assume that we indexed the primes so that they were increasing,
we have that $0 < \frac{1}{p_j} < \ep$ as well.

We now provide the sequence defined by
\[
	x_n = \prod_{i=0}^n \frac{p_{i+J}-1}{p_{i+J}}
\]
Note that if $m = \prod_{i=0}^n p_{i+J}$, we showed above then that
\[
	x_n = \frac{\phi(m)}{m}
\]
We have $1 > \frac{p_J-1}{p_J} = x_0 = 1 - \frac{1}{p_J} > 1- \ep$.
Furthermore, for any $n \in \N$, $x_n > x_{n+1}$ since
we are multiplying by a value less than one,
hencee
\begin{align*}
	0 < x_n - x_{n+1}
	&= \prod_{i=0}^n\frac{p_{i+J} - 1}{p_{i+J}} -
	\prod_{i=0}^{n+1}\frac{p_{i+J} - 1}{p_{i+J}} \\
	&= \prod_{i=0}^n\left(\frac{p_{i+J} - 1}{p_{i+J}}\left(1-1+\frac{1}{p_{n+J+1}}\right)\right)\\
	&= \frac{1}{p_{n+J+1}}\prod_{i=0}^n\frac{p_{i+J}-1}{p_{i+J}}\\
	&< \ep\prod_{i=0}^n\frac{p_{i+J}-1}{p_{i+J}} < \ep
\end{align*}

Now we can invoke the Euler product formula to get
\[
	\prod_{p\text{ prime}}\left(\frac{1}{1-\frac{1}{p^s}}\right)
	= \sum_{n=1}^\infty \frac{1}{n^s}
\]
therefore
\[
	\lim_{n\to\infty} \prod_{i=1}^n \frac{p_i}{p_i-1}
	= \lim_{n\to\infty} \prod_{i=1}^n \frac{1}{1-\frac{1}{p_i}}
	= \sum_{k=1}^\infty \frac{1}{k} = +\infty
\]
which gives that
\[
	\lim_{n\to\infty}\prod_{i=1}^n \frac{p_i-1}{p_i} = 0
\]
Now, by the definition of the limit, there exists $N \in \N$ such that for all $n > N$, we get
\[
	\prod_{i=1}^n \frac{p_i-1}{p_i} < \left(\frac{1}{\prod_{i=1}^{J-1}\frac{p_i}{p_i-1}}\right)\ep
\]
Thus
\[
	x_{N} = \prod_{i=0}^J\frac{p_{i+J}-1}{p_{i+J}}
	= \left(\prod_{i=1}^{N+J}\frac{p_i-1}{p_i}\right)
	\left(\prod_{i=1}^{J-1}\frac{p_i}{p_i-1}\right) < \ep
\]
Thus, $0 < x_N < \ep$.

Now, we know that after some $J$, $x_n$ is within $\ep$ of $1$,
the sequence is monotonically decreasing and within $\ep$ of each other as well,
and for large enough $N$, we are also within $\ep$ of $0$.
So for any $\alpha \in [0,1]$, we have that $(x_n)\cap (\alpha - \ep, \alpha + \ep) \neq \emptyset$,
and there are infinitly points in the intersection.
So $\{x_n\}$ is dense in $[0,1]$, hence we must always have that
$\alpha$ is a limit point for some subsequence of $(x_{n_k})$ of $(x_n)$.
Thus, we can always find a subsequence $(n_k)$ such that
\[
	\lim_{k\to\infty}x_{n_k} = \lim_{k\to\infty}\frac{\phi(n_k)}{n_k} = \alpha
\]
\end{proof}
\end{document}
