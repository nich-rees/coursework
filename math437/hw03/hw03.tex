\documentclass{article}
\usepackage{amsmath, amsfonts, amsthm, amssymb}
\usepackage{geometry, hyperref}
\geometry{letterpaper, margin=2.0cm, includefoot, footskip=30pt}

\usepackage{fancyhdr}
\pagestyle{fancy}

\lhead{Math 437}
\chead{Homework 2}
\rhead{Nicholas Rees, 11848363}
\cfoot{Page \thepage}

\newtheorem*{problem}{Problem}

\newcommand{\N}{{\mathbb N}}
\newcommand{\Z}{{\mathbb Z}}
\newcommand{\Q}{{\mathbb Q}}
\newcommand{\R}{{\mathbb R}}
\newcommand{\C}{{\mathbb C}}
\newcommand{\ep}{{\varepsilon}}

\renewcommand{\theenumi}{(\alph{enumi})}

\begin{document}
\section{Problem 1}
{\it Let $\{a_n\}_{n\geq 0}$ be a sequence defined as follows:
\[
	a_0 = 0; a_1 = 1; a_2 = 2 \text{ and}
\]
\[
	a_{n+3} = 5^n\cdot a_{n+2} + n^2\cdot a_{n+1} + 11a_n
	\text{ for }n\geq 0
\]
Prove that there exist infinitely many $n \in \N$
such that $2023 \mid a_n$.
}
\begin{proof}[Solution]\let\qed\relax
	Note that there are only $2023^5$ permutations of
	$(a_{n+2}, a_{n+1}, a_{n}, 5^n, n)$ when each element
	is considered modulo $2023$.
	Furthermore, $a_{n+2}\,(\mathrm{mod}\,2023), a_{n+1}\,(\mathrm{mod}\,2023),
	a_{n}\,(\mathrm{mod}\,2023), 5^n\,(\mathrm{mod}\,2023), n\,(\mathrm{mod}\,2023)$
	determines the value $a_{n+3}\,(\mathrm{mod}\,2023)$,
	since ff (need to consider $n \implies n^2$).

	Let $k = 2023^5 + 1$, and consider the $a_k$.
	By the pigeon-hole principle, there must exist some $m$ such that
	$(a_{k+2},a_{k+1},a_{k},5^k,k) = (a_{m+2},a_{m+1},a_{m},5^m,m)$
	(recall that these are all modulo $2023$),
	and thus we must have that $a_{k+i} = a_{m+i}$ for all $i \in \N$,
	as we proved before.
	Thus, $(a_n)$ is periodic with period $p = k - m$.

	Note that $a_0 = 0$, thus, it is sufficient to show that
	$a_0 = a_{0+p}$.
	To prove this, assume for the sake of contradiction that there
	is some least $j > 0$ where $a_{j+p} = a_j$ but $a_{j+p-1} \neq a_{j-1}$.
	Then $(a_{j+2}, a_{j+1}, a_{j}, 5^j, j) =
	(a_{j+p+2}, a_{j+p+1}, a_{j+p}, 5^{j+p}, j+p)$
	and $(a_{j+1}, a_{j}, a_{j-1}, 5^{j-1}, j-1) \neq
	(a_{j+p+1}, a_{j+p}, a_{j+p-1}, 5^{j+p-1}, j+p-1)$.
	But then we have one of $a_{j-1} \neq a_{j+p-1} \,(\mathrm{mod}\,2023)$,
	$5^{j-1} \neq 5^{j+p-1} \,(\mathrm{mod}\,2023)$,
	or $j-1 \neq n+p-1 \,(\mathrm{mod}\,2023)$.
	ff

	ff we can repeat this process with $j-1$, until we get to $j=0$.
\end{proof}
\clearpage

\section{Problem 2}
{\it Let $n \in \N$.
Find the number of solutions for the congruence equation:
\[
	x^3 \equiv 1 \; (\mathrm{mod}\, n)
\]
}
\begin{proof}[Solution]\let\qed\relax
	Consider the unique prime factors of $n$,
	specifically $n = p_1^{\alpha_1}p_2^{\alpha_2} \cdots p_r^{\alpha_r}$
	(where $\alpha_i \geq 1$).
	If $p_i = 2$, then $x^3 \equiv 1 \, (\mathrm{mod}\,2)$
	is solved whenever $x^3$ is odd,
	which has one solution $\mathrm{mod}\,2$, namely $x = 1$.
	If $p_i \neq 2$,
	note that since $p \nmid 1$, and $3 \in \Z^+$,
	by theorem 18.2, we have the number of solutions to
	$x^3 \equiv 1 \,(\mathrm{mod}\,p_i^{\alpha_i})$
	is $d_i = \gcd(3,p_i^{\alpha_i})$
	(note that we never have the $0$ solutions case,
	because $1^{\phi(p_i^{\alpha_i})/d} \equiv 1 \, (\mathrm{mod}\,p_i^{\alpha_i})$ always).
	We can now compute $d_i$:
	\[
		d_i = \gcd(3,\phi(p_i^{\alpha_i})) = \gcd(3, p^{\alpha_i-1}(p_i-1))
	\]
	We can have $p_i \equiv 0\, (\mathrm{mod}\, 3)$,
	$p_i \equiv 1\, (\mathrm{mod}\, 3)$,
	or $p_i \equiv 2\, (\mathrm{mod}\, 3)$.

	In the $0\, (\mathrm{mod}\, 3)$, this says that $3 \mid p_i$,
	which is only true when $p_i = 3$ (by the definition of a prime).
	Then if $\alpha_i = 1$, we have $\gcd(3,2) = 1$.
	If $\alpha_i > 1$, we have $\gcd(3,3^{\alpha_i}2) = 3$.

	If $p_i \equiv 1\, (\mathrm{mod}\, 3)$, then
	$\gcd(3,p_i^{\alpha_i}(p_i - 1)) = 3$ since $3 \mid p_i - 1$
	and $p_i^{\alpha_i} \geq 3+1$.
	
	If $p_i \equiv 2\, (\mathrm{mod}\, 3)$,
	then $\gcd(3,p_i^{\alpha_i}(p_i-1) = 1$,
	since $3 \nmid p_i^{\alpha_i}$ (by definition of $p_i$ being prime and not $3$)
	and $3 \nmid p_i - 1 = 3k + 1$
	by definition of $p_i$ being $2\, (\mathrm{mod}\, 3)$.

	Let $N_P(m)$ denote the number of solutions to
	$x^3 - 1 \equiv 0\, (\mathrm{mod}\,m)$.
	From Theorem 8.2, since
	$p_i^{\alpha_i}$ is coprime with $p_j^{\alpha_j}$ when $i \neq j$,
	we have $N_P(n) = \prod N_P(p_i^{\alpha_i})$.
	We can rewrite $n$ as
	\[
		n = 2^l3^k\prod_{i=1}^r p_i^{\alpha_i} \prod_{j=1}^s q_i^{\beta_j}
	\]
	where $l,k \in \N \cup \{0\}$, $p_i,q_j$ are prime,
	$p_i \equiv 1 \, (\mathrm{mod}\,3)$,
	$q_j \equiv 2 \, (\mathrm{mod}\,3)$ not $2$,
	and $r$ and $s$ are the number of such primes where $\alpha_i,\beta_j \geq 1$.

	Thus,
	\[
		N_P(n) = N_P(2^l)N_P(3^k)
		\prod_{i=1}^r N_P(p_i^{\alpha_i}) \prod_{j=1}^s N_P(q_i^{\beta_j})
		= N_P(3^k)3^r
	\]
	Hence, we have
	\[
		N_P(n) = \begin{cases}
			3^{r+1} & \text{if }k>1\\
			3^{r} & \text{otherwise}
		\end{cases}
	\]
\end{proof}
\clearpage

\section{Problem 3}
\textbf{As always, $\phi(\cdot)$ is the Euler-$\phi$ function.}\newline
{\it Let $\alpha$ be any real number in the interval $[0,1]$.
Prove that there exists an infinite sequence $\{n_k\}_{k \geq 1} \subset \N$ such that
\[
	\lim_{k\to\infty} \frac{\phi(n_k)}{n_k} = \alpha
\]
}
\begin{proof}[Solution]\let\qed\relax
If $n = \prod_{i=1}^{\pi(n)} p_i^{\alpha_i}$, then
\[
	\phi(n) = \prod_{i=1}^{\pi(n)}(p_i^{\alpha_i} - p_i^{\alpha_i - 1})
	= n \prod_{\substack{p\mid n \\ p \text{ prime}}}\left(1 - \frac{1}{p}\right)
\]
Thus
\[
	\frac{\phi(n)}{n} = \prod_{\substack{p\mid n \\ p \text{ prime}}}\left(1 - \frac{1}{p}\right)
\]
To show that $\{\frac{\phi(n)}{n}\}$ is dense in $(0,1)$,
let $x,y \in (0,1)$ where $x < y$,
then we claim there exists $n$ such that $x < \frac{\phi(n)}{n} < y$.
It is sufficient to consider when $x,y \in \Q \cap (0,1)$.
Let $x = p'_1/q_1$ and $y = p'_2/q_2$ where $p'_i < q_i$ and $\gcd(p'_i,q_i) = 1$.
We can rewrite them to have the same denomonator:
$x = p_1/q$ and $y = p_2/q$, where $p_1 < p_2 < q$.
Note $1 - \frac{1}{p} = \frac{p-1}{p}$.
Choose the $p$ such that $p \mid q$
(I want something more, like the product of all the primes is $q$).
So let's just assume that $q$'s prime decomposition
only has exponents $1$ (and then could show this is dense)
so then we want $p_1 < \prod(p-1) < p_2$.
Perhaps if it is too big, then we find more $p$ later and multiply it down.
If it is too small, ff

This is the problem: we can't write every rational number
as a product of $(p-1)/p$,
or even every rational with denominator whose prime number decomposition
do not have extra exponents.
But somehow we achieve density.

Note that $\prod(1-\frac{1}{p})$ converges iff $\sum -\frac{1}{p}$ converges
(\href{https://math.stackexchange.com/questions/2966559/convergence-of-infinite-product-prod1a-n-where-a-n-changes-sign}{stack exchange link in source code}).
But I don't think we really care.

Maybe useful: the rationals in simplified form do not
contain any shared primes between the numerator and denominator.

Maybe we do a completely different approach to density.
What if $n = q$ or something.
\end{proof}
\end{document}
