\documentclass{article}
\usepackage{amsmath, amsfonts, amsthm, amssymb}
\usepackage{geometry}
\geometry{letterpaper, margin=1.7cm, includefoot, footskip=30pt}

\usepackage{fancyhdr}
\pagestyle{fancy}

\lhead{Math 437}
\chead{Homework 1}
\rhead{Nicholas Rees, 11848363}
\cfoot{Page \thepage}

\newtheorem*{problem}{Problem}

\newcommand{\N}{{\mathbb N}}
\newcommand{\Z}{{\mathbb Z}}
\newcommand{\Q}{{\mathbb Q}}
\newcommand{\R}{{\mathbb R}}
\newcommand{\C}{{\mathbb C}}
\newcommand{\ep}{{\varepsilon}}

\renewcommand{\theenumi}{(\alph{enumi})}

\begin{document}
\subsection*{Problem 1}
{\it For each real number $x$, we let $[x]$ be the integer part of $x$,
i.e., the largest integer less than or equal to $x$;
for example, $[2.3] = 2$, $[5] = 5$, and $[-3.6] = 4$.

For each $m \in \N$, prove that there exists $n \in \N$ such that
\[m = \left[\frac{n}{\sqrt{n+1}}\right].\]}

\begin{proof}
	For all $n \in \N$, we have that
	\[
		\frac{(n+1)}{\sqrt{(n+1)+1}} - \frac{n}{\sqrt{n+1}}
		= \frac{n+1-n}{\sqrt{(n+1)(n+2)}}
		= \frac{1}{\sqrt{(n+1)(n+2)}}
	\]
	But note that
	\begin{equation}
		0 < \frac{1}{\sqrt{(n+1)(n+2)}} < 1
	\end{equation}
	since $(n+1)(n+2) > 1$ for all $n \in \N$, then $\sqrt{(n+1)(n+2)} > 1$,
	and so its reciprocal is less than $1$;
	and all values are positive so it is greater than $0$.
	This inequality means that each consecutive term in $\left(\frac{n}{\sqrt{n+1}}\right)_{n\in\N}$
	has a positive difference less than $1$.
	Thus, at most, the difference between $\left[\frac{(n+1)}{\sqrt{(n+1)+1}}\right]$
	and $\left[\frac{n}{\sqrt{n+1}}\right]$ is at most $1$,
	since consecutive terms of $\left(\frac{n}{\sqrt{n+1}}\right)_{n\in\N}$
	can never jump more than $1$ (the distance between consecutive integers),
	thus the floor of consecutive integers will never ``skip" an integer.
	So we have
	\begin{equation}
		 \left[\frac{(n+1)}{\sqrt{(n+1)+1}}\right] \leq \left[\frac{n}{\sqrt{n+1}}\right] + 1
	\end{equation}

	Also, for all $m \in \N$, there exists $n' \in \N$, such that
	\[
		m \leq \left[ \frac{n'}{\sqrt{n'+1}} \right]
	\]
	(and since the sequence is nondecreasing,
	this means all $n\in \N$ where $n\geq n'$ also satisfies this inequality).
	We simply choose $n' = m^2 + 2m$, and one can see that
	\begin{align*}
		\frac{m^2 + 2m}{\sqrt{m^2+2m+1}}
		&= \frac{(m+1) - 1}{m+1}\\
		&= m + 1 - \frac{1}{m+1}\\
		&> m \qquad(\text{since } \forall m\in\N, 1 > \frac{1}{m+1})\\
		\implies \left[\frac{m^2 + 2m}{\sqrt{m^2+2m+1}}\right] &\geq m
	\end{align*}
	where the implication at the end is because $m$ is an integer,
	and $[x]$ will map to the integer directly below it.

	Now, consider the subset of the integers
	$\{n \mid n\in\N, m \leq \left[ \frac{n}{\sqrt{n+1}} \right]\}$.
	By the well-ordering of the naturals, this set has a least element;
	we will denote this $n_2$,
	and let $n_1 = n_2 - 1$.
	Since $n_2$ was the smallest element such that
	$\left[\frac{n_1}{\sqrt{n_1 + 1}}\right] \leq m$,
	we have $\left[\frac{n_1}{\sqrt{n_1 + 1}}\right] \leq m$.
	Apply (2), we have
	\[
		\left[\frac{n_1}{\sqrt{n_1+1}}\right] \leq m
		\leq \left[\frac{n_2}{\sqrt{n_2+1}}\right] \leq \left[\frac{n_1}{\sqrt{n_1+1}}\right] + 1
	\]
	But since $\left[\frac{n_2}{\sqrt{n_2+1}}\right]$ is an integer,
	it cannot be between two consecutive integers,
	thus either $\left[\frac{n_2}{\sqrt{n_2+1}}\right] = \left[\frac{n_1}{\sqrt{n_1+1}}\right]$,
	or $\left[\frac{n_2}{\sqrt{n_2+1}}\right] = \left[\frac{n_1}{\sqrt{n_1+1}}\right] + 1$.
	In the first case, we then have $m = \left[\frac{n_2}{\sqrt{n_2+1}}\right]$ and we are done.
	In the second case,
	since $m$ is also an integer, it cannot be between two consecutive integers,
	thus either $m = \left[\frac{n_1}{\sqrt{n_1+1}}\right]$,
	or $m = \left[\frac{n_2}{\sqrt{n_2+1}}\right]$,
	in either case, we are done.
\end{proof}
\clearpage

\subsection*{Problem 2}
{\it An (infinite) arithmetic progression in $\N$ is a set
of the form $\{an+b\}_{n\geq0}$ for some given $a,b\in\N$.

If a set $S$ is the complement in $\N$ of a union of finitely many arithmetic progressions,
then prove that $S$ is a union of finitely many arithmetic progressions along with a finite set.}

\begin{proof}
	We will use the notation $a\N_0 + b$ to represent $\{an+b\}_{n\geq 0}$.
	Let us have $l$ many arithmetic progressions (finitely many).
	By De Morgan's laws, we have
	\[
		S = \left(\bigcup_{i=1}^l(a_i\N_0 + b_i)\right)^C =
		\bigcap_{i=1}^l\left(a_i\N_0 + b_i\right)^C
	\]

	Consider the complement of an arbitrary arithmetic progression
	$(a_i \N_0 + b_i)^C$.
	We claim that this is just the union of a finite set
	and $a_i-1$ many arithmetic progressions (finitely many).
	Specifically
	\[
		(a_i \N_0 + b_i)^C = \{1,2,\dots,b_i-1\} \cup
		\bigcup_{j=1}^{a_i-1}(a_i\N_0 + b_i + j)
	\]
	To prove this, we prove set inclusion in both directions.
	Let $k \in (a_i\N_0 + b_i)^C$.
	Then $k \neq a_in + b_i$ or $k - b_i \neq a_in$ for all $n \in \N_0$.
	If $k < b_i$, then $k \in \{1,2,\dots,b_i-1\}$ and we are done.
	Now assume $k \geq b_i$.
	By the division algorithm, there exists unique integers $q,r$
	where and $0 \leq r \leq a_i - 1$ such that
	$k - b_i = a_iq + r \implies k = a_iq + b_i + r$.
	But note that $q \geq 0$ since $k - b_i \geq 0$ and all the other values are nonnegative,
	and $0 < r$ otherwise $k$ would be able to be represented as $a_in + b_i$.
	But then, this is precisely to say that
	$k \in a_i\N_0 + b_i + j$ (as above) since
	$0 \leq j \leq a_i-1$ like $r$, and $q \in \N_0$.
	And so $k \in \bigcup_{j=1}^{a_i - 1}(a_i\N_0 + b_i + j)$.
	This is sufficient to show that
	\[
		(a_i \N_0 + b_i)^C \subseteq \{1,2,\dots,b_i-1\} \cup
		\bigcup_{j=1}^{a_i-1}(a_i\N_0 + b_i + j)
	\]
	Now let $k \in \{1,2,\dots,b_i-1\} \cup \bigcup_{j=1}^{a_i-1}(a_i\N_0 + b_i + j)$.
	If $k \in \{1,2,\dots,b_i-1\}$,
	then $k$ is smaller than the smallest element in $a_i\N_0 + b_i$,
	specifically $b_i$,
	thus $k \in (a_i\N_0 + b_i)^C$.
	If $k \not\in\{1,2,\dots,b_i-1\}$,
	then $k = a_in + b_i + j$ for $n\in\N_0$ and $1 \leq j \leq a_i-1$.
	For the sake of contradiction, assume that there exists $n'\in\N_0$
	such that $k = a_in' + b_i$.
	So $a_in + b_i + j = a_in' + b_i \implies a_in + j = a_in' \implies j= a_i(n' - n)$.
	But then, since the right side of the equality is divisible by $a_i$,
	we must have that $a_i \mid j$.
	So $a_i < |j|$ or $j = 0$.
	But recall $j$ is positive, so this is saying
	$a_i < j$.
	But recall that $j \leq a_i - 1 < a_i$,
	and so we arrive at a contradiction.
	Thus, there does not exist $n' \in \N_0$ such that $k = a_in' + b_i$,
	thus $k \in (a_i\N_0 + b_i)^C$.
	This is sufficient to show
	\[
		(a_i \N_0 + b_i)^C \supseteq \{1,2,\dots,b_i-1\}
		\cup \bigcup_{j=1}^{a_i-1}(a_i\N_0 + b_i + j)
	\]
	therefore we have shown equality of the two sets.
	
	Thus, we can rewrite $S$ as
	\[
		S = \bigcap_{i=1}^{l}\left(\{1,2,\dots,b_i-1\}\cup
		\bigcup_{j=1}^{a_i-1}(a_i\N_0 + b_i + j)\right)
	\]

	We now induct on $l$ to show $S$ is the union of a finite set
	and finitely many arithmetic progressions.
	When $l = 1$, it is obvious to see that this is true
	($\{1,\dots,b_1 - 1\}$ is finite and $\bigcup_{j=1}^{a_1-1}(a_1\N_0 + (b_i + j))$
	is a union of $a_1 - 1$ many arithmetic prograssions).
	Now, let our property be true for $l = m \in \N$,
	we want to show that this implies for $l = m + 1$,
	$S$ is a union of a finite set and finitly many arithmetic progressions.
	We have
	\begin{align*}
		S &= \bigcap_{i=1}^{m+1}\left(\{1,2,\dots,b_i-1\}\cup
		\bigcup_{j=1}^{a_i-1}(a_i\N_0 + b_i + j)\right)\\
		  &= \bigcap_{i=1}^m\left(\{1,2,\dots,b_i-1\} \cup
		\bigcup_{j=1}^{a_i-1}(a_i\N_0+b_i+j)\right)
		\cap \left( \{1,2,\dots,b_{m+1}-1\} \cup
		\bigcup_{j=1}^{a_{m+1}-1}(a_{m+1}\N_0+b_{m+1}+j)\right)
	\end{align*}
	But by assumption, our set on the left of the intersection is a union
	of a finite set and finitely many arithmetic progressions.
	Let us denote that finite set by $F$,
	and each finitely many arithmetic progressions as $A_1,A_2,\dots,A_k$ where $k\in\N$.
	Then we have
	\[
		S = \left(F \cup \bigcup_{i=1}^k A_i\right)
		\cap \left( \{1,2,\dots,b_{m+1}-1\} \cup
		\bigcup_{j=1}^{a_{m+1}-1}(a_{m+1}\N_0+b_{m+1}+j)\right)
	\]
	Now, using the distributive property of unions and intersections, we have
	\begin{align*}
		S =& \left( F \cap \{1,2,\dots,b_{m+1}-1\}\right)
		\cup \left(F \cap \bigcup_{j=1}^{a_{m+1}-1}(a_{m+1}\N_0+b_{m+1}+j)\right)\\
		  &\cup \left(\bigcup_{i=1}^kA_i \cap \{1,2,\dots,b_{m+1}-1\}\right)
		\cup \left(\bigcup_{i=1}^k A_i \cap \bigcup_{j=1}^{a_{m+1}-1}(a_{m+1}\N_0+b_{m+1}+j)\right)
	\end{align*}
	But note that the first three terms in the union are all some set
	intersected with a finite set,
	and since the largest an intersection between two sets can be
	is the size of the smaller set,
	we must have that the first three terms are finite sets.
	Unioning them all together would yield another finite set,
	since the largest a union can be is the sum of the sizes of the component sets,
	and so we can denote this as another finite set $F'$, giving us
	\[
		S = F' \cup
		\left(\bigcup_{i=1}^k A_i \cap \bigcup_{j=1}^{a_{m+1}-1}(a_{m+1}\N_0+b_{m+1}+j)\right)
	\]
	We can now look at the intersection of unions of arithmetic progressions.
	By the distributive property of unions and intersections,
	we get the union of the pairwise intersections between them, specifically
	\begin{align*}
		S = F'
		&\cup (A_1 \cap (a_{m+1}\N_0 + b_{m+1} + 1))
		\cup (A_1 \cap (a_{m+1}\N_0 + b_{m+1} + 2)) \cup \cdots \cup
		(A_1 \cap (a_{m+1}\N_0 + b_{m+1} + a_{m+1}-1 - 1))\\
		&\cup (A_2 \cap (a_{m+1}\N_0 + b_{m+1} + 1))
		\cup (A_1 \cap (a_{m+1}\N_0 + b_{m+1} + 2))
		\cup \cdots \cup (A_2 \cap (a_{m+1}\N_0 + b_{m+1} + a_{m+1}-1 - 1))\\
		&\cup \cdots \cup (A_k \cap (a_{m+1}\N_0 + b_{m+1} + 1))\cup \cdots
		\cup (A_k \cap (a_{m+1}\N_0 + b_{m+1} + a_{m+1}-1 - 1))
	\end{align*}
	Note that each term (not $F'$) is the intersection of two arithmetic progressions.
	We claim that any intersection of two arithmetic progressions will either be
	an empty set, or will be another arithmetic progression.
	To show this, consider two arbitrary arithmetic progressions,
	$a'\N_0 + b'$ and $a^*\N_0 + b^*$ where $a',a^*,b',b^* \in \N$.
	If there does not exist $m,n \in \N$ such that $a'm + b' = a^*n+b^*$,
	then their intersection is the empty set, as desired.
	Now, let their intersection be nonempty.
	Since this is a subset of the naturals,
	by the well-ordering property, there is a least element of the intersection, $\eta$.
	We claim that the intersection is then the set $\mathrm{lcm}[a',a^*]\N_0 + \eta$,
	which is an arithemetic progression.
	To show this claim, $\alpha \in \N$
	is in the intersection if and only if
	there exists $m,n\in\N$ such that $\alpha = a'm + \eta = a^*n + \eta$,
	since $\alpha$ is $\eta$ plus some multiple of $a'$ and $a^*$
	in order to be within both arithmetic progressions
	(since every point in an arithmetic progression is seperated
	by some multiple of the gap length $a$).
	Then we have $a'm = a^*n$, which, by definition,
	is a common multiple of $a'$ and $a^*$.
	Thus, $\alpha$ is in the intersection
	if and only if $\alpha = \gamma + \eta$
	where $\gamma$ is a common multiple of $a'$ and $a^*$.
	But recall from Proposition 3.10 from the course notes
	(lcm divides all common multiples),
	we have $\mathrm{lcm}[a',a^*] \mid \gamma$,
	thus $\gamma = n\mathrm{lcm}[a',a^*]$ for some $n \in \N$.
	Thus, $\alpha \in (a'\N_0 + b') \cup (a^*\N_0 + b^*)$
	if and only if $\alpha \in \mathrm{lcm}[a',a^*]\N_0 + \eta$.
	But then the intersect is also a arithmetic progression.

	Since we were working for arbitrary intersections of arithmetic progressions,
	all of our intersections in the equation for $S$
	must either be empty or an arithmetic progression.
	Thus, since there are finitely many intersections of arithmetic progressions,
	we have that $S$ is the union of a finite set and finitely many arithmetic progressions.
\end{proof}
\clearpage

\subsection*{Problem 3}
{\it Find all odd positive integers $m$ and $n$ with the property that
\[n \mid (3m+1) \text{ and } m \mid (n^2 + 3).\]}

\begin{proof}
	First, we can rewrite our divisibility statements into equalities.
	Namely, there exists $k,l \in \N$ such that
	\begin{equation}\label{n=}
		kn = 3m + 1
	\end{equation}
	\begin{equation}\label{m=}
		lm = n^2 + 3
	\end{equation}
	If we multiply (\ref{n=}) by $l$, we have $lkn = 3lm + l$
	and by substituting in (\ref{m=}), we get
	$lkn = 3n^2 + 9 + l \implies lkn - 3n^2 = l + 9$.
	It is easy to see the left hand side of the equality is divisible by $n$,
	thus $n \mid l + 9$ as well.
	But since divisors are always less than or equal to a value they divide,
	we get
	\begin{equation}\label{n<l}
		n \leq l + 9
	\end{equation}
	
	We now consider multiplying (\ref{n=}) by $n$,
	which gives us $n^2k = 3mn + n \implies n^2k+3k = 3mn + n + 3k$.
	Now subbing in (\ref{m=}) on the left, we get $mlk = 3mn + n + 3k$.
	Multiplying everything by $k$, we get
	$mlk^2 = 3mnk + nk + 3k^2 \implies mlk^2 = 3mnk + 3m + 1 + 3k^2$ (again by (\ref{n=})).
	But then this equality can become $mlk^2 - 3m - 3mnk = 3k^2 + 1$,
	and the left hand side of the equality is divisible by $m$,
	thus $m \mid 3k^2 + 1$, which gives
	\begin{equation}\label{m<k}
		m \leq 3k^2 + 1
	\end{equation}
	
	Note that we have $n \leq 3m + 1$ and $k \leq 3m + 1$
	(which is true by the fact $n \mid (3m+1)$ and $k \mid (3m+1)$ from (\ref{n=})).
	Now, plugging in (\ref{m<k}) into the first inequality, we have
	\begin{equation}\label{n<k}
		n \leq 9k^2 + 4
	\end{equation}
	And multiplying the second inequality by $l$ we have $kl \leq 3ml + l$,
	and subbing in (\ref{m=}) we have $kl \leq 3n^2 + 9 + l$.
	But using (\ref{n<l}), we get that $kl \leq 3(l+9)^2 + 9 + l
	\implies kl \leq 3l^2 + 54l + l + 243 + 9
	\implies k \leq 3l + 55 + \frac{252}{l}$.
	Note that $l$ is always positive, so dividing by it always decreases the value.
	Thus, using our last inequality,
	\begin{equation}\label{k<l}
		k \leq 3l + 55 + 252 = 3l + 307
	\end{equation}
	
	Finally, consider subbing in (\ref{n=}) into (\ref{m=}),
	which gives us
	\begin{align*}
		ml &= \left(\frac{3m+1}{k}\right)^2 + 3\\
		   &= \frac{9m^2}{k^2} + \frac{6m}{k^2} + \frac{1}{k^2} + 3\\
		l &= \frac{9m}{k^2} + \frac{6}{k^2} + \frac{1}{mk^2} + \frac{3}{m}\\
		3l + 307 &= \frac{27m}{k^2} + \frac{18}{k^2} + \frac{3}{mk^2} + \frac{9}{m} + 307
	\end{align*}
	But we can substitute in (\ref{k<l}) to get
	\begin{align*}
		k &\leq \frac{27m}{k^2} + \frac{18}{k^2} + \frac{3}{mk^2} + \frac{9}{m} + 307\\
		k^3 &\leq 27m + 18 + \frac{3}{m} + \frac{9k^2}{m} + 307k^2
	\end{align*}
	\begin{equation*}
		k^3 - \left(\frac{9}{m} + 307\right)k^2 - \left(\frac{3}{m} + 18 + 27m\right) \leq 0
	\end{equation*}
	We can simplify this further, by noting the largest value $\frac{9}{m}, \frac{3}{m}$
	can obtain is when $m = 1$,
	and also using (\ref{m<k}), we get
	$k^3 - \left(\frac{9}{m} + 307\right)k^2 - \left(\frac{3}{m} + 18 + 27m\right)
	\geq k^3 - 316k^2 - \left(21 + 27(3k^2 + 1)\right)
	= k^3 - 397k^2 - 48$.
	So we have
	\begin{equation}
		k^3 - 397k^2 - 48 \leq 0
	\end{equation}
	Recall from Calc 1 that a cubic function of $k$ with a positive leading coefficient
	will always be positive when $k$ is larger than the largest root,
	and it can be nonpositive when $k$ is smaller,
	thus we can put a bound on $k$ in order for
	the polynomial in the inequality to be nonnegative,
	namely the largest root of the polynomial.
	One can verify that the last root happens between $k = 397$ and $398$.
	We can verify that this polynomial is positive for all $k \geq 398$
	with the first derivative test:
	$3k^2 - 794k = k(3k - 794)$ which is positive when $k > \frac{794}{3} \approx 265$;
	we can see that this polynomial is convex up for all $k \geq 398$
	with the second derivative test:
	$6k - 794$ which is positive when $k > \frac{794}{6} \approx 132$.
	Thus, $k \leq 398$ whenever our polynomial is nonpositive
	(and it definitely attains a nonpositive value, see $k = 0$).
	Thus we get the bound
	\begin{equation}\label{k<}
		k < 398
	\end{equation}

	We can use this bound to get a bound on $n$ and $m$,
	and then test a finite number of inputs to see which $n,m$ satisfy our conditions.
	Substituting (\ref{k<}) into (\ref{m<k}) and (\ref{n<k}), we get
	\begin{equation}
		m < 475213
	\end{equation}
	\begin{equation}
		n < 1425640
	\end{equation}
	
	We can verify using a computer system in a reasonable time
	the just over $338$ billion possibilities for $(n,m)$
	(for reference, my program took less than $4$ minutes).
	We get the following pairs for $(n,m)$:
	\[
		(n,m) = (1,1), (13,43), (37,49)
	\]
\end{proof}
\end{document}
