\documentclass{article}
\usepackage{amsmath, amsfonts, amsthm, amssymb}
\usepackage{geometry}
\geometry{letterpaper, margin=2.0cm, includefoot, footskip=30pt}

\usepackage{fancyhdr}
\pagestyle{fancy}

\lhead{Math 321}
\chead{Homework 1}
\rhead{Nicholas Rees, 11848363}
\cfoot{Page \thepage}

\newcommand{\N}{{\mathbb N}}
\newcommand{\Z}{{\mathbb Z}}
\newcommand{\Q}{{\mathbb Q}}
\newcommand{\R}{{\mathbb R}}
\newcommand{\C}{{\mathbb C}}
\newcommand{\ep}{{\varepsilon}}

\newcommand{\problem}[1]{
	\begin{center}\fbox{
		\begin{minipage}{17.0 cm}
			\setlength{\parindent}{1.5em}
			{\it \noindent#1}
		\end{minipage}}
	\end{center}}

\newtheorem{lemma}{Lemma}

\renewcommand{\theenumi}{(\alph{enumi})}

\begin{document}
\begin{center}
	{\bf Math 321 Homework 1}
\end{center}

In this homework, we will need several definitions. Let $I=[a,b]$ be an interval and $k\geq 0$ be an integer. If $f\colon I\to\R$ is a function that is $k$-times differentiable on $I$, then we define
\[
\Vert f \Vert_{C^k(I)}=\sum_{j=0}^k \sup_{x\in I} |f^{(j)}(x)|.
\]
This quantity is called the ``$C^k$ norm of $f$.''
We define $C^k(I)$ to be the set of functions $f\colon I\to\R$ that satisfy the following two properties.
(i): $f$ is $k$-times differentiable on $I$, and
(ii): $f^{(k)}$ is continuous on $I$.
We define a metric on $C^k(I)$ as follows: $d(f,g)=\Vert f-g \Vert_{C^k(I)}$, i.e.
\begin{equation}\label{CkNorm}
	d(f,g) = \sum_{j=0}^k \sup_{x\in I} |f^{(j)}(x)-g^{(j)}(x)|.
\end{equation}
It is straightforward to verify that this is indeed a metric, but you do not have to do so for this homework.

\subsection*{Problem 1}
\problem{
Let $f(t) = e^t$; recall that $f$ is monotone increasing, $f'(t) = f(t)$, and $f(0)=1$.
Let $P_n(t)$ be the $n$-th order Taylor polynomial of $f$ at the point $x_0=0$,
as discussed in lecture.
Let $I=[-1,1]$ and let $k\geq 1$ be an integer.

Using Taylor's theorem, prove that the sequence $\{P_n\}$ converges to
$f$ in the metric space $C^k(I)$. 

\emph{Hints} (i) Compute the Taylor polynomial $P_n(t)$. (ii) What is the derivative of $P_n$? (iii) What are the higher derivatives of $P_n$? (iv) How can you estimate each term in \eqref{CkNorm}?
}
\begin{proof}[Solution]\let\qed\relax
	Recall that for $f(t)$, the $n$-th ordered Taylor polynomial at $x_0 = 0$ is
	\[
		P_n(t) = \sum_{i=0}^n \frac{x^i}{i!}
	\]
	Furthermore, note that the $j$-th derivative of $P_n(t)$
	is $0$ if $j > n$ and
	\[
		\frac{d^j}{dx^j}P_n(t) =
		\frac{d^j}{dx^j}\sum_{i=0}^{j-1}\frac{x^i}{i!}
		+ \sum_{i=j}^n\frac{d^j}{dx^j}\frac{x^i}{i!}
		= 0 + \sum_{i=j}^n \frac{1}{i!}\frac{i!}{(i-j)!}x^{i-j}
		= \sum_{i=j}^n \frac{x^{i-j}}{(i-j)!}
		= \sum_{i=0}^{n-j}\frac{x^i}{i!} = P_{n-j}(t)
	\]
	when $j \geq n$.

	Recall from Taylor's theorem that there exists $c$ between $t$ and $0$ such that
	\[
		e^t = P_n(t) + \frac{f^{(n+1)}(c)}{(n+1)!}t^{n+1}
		= P_n(t) + \frac{e^c}{(n+1)!}t^{n+1}
	\]

	Then we make some argument about how $e^c$ is maximal at $1$ in $I$,
	but this will decrease arbitrarily, so $\{P_n\} \to f$. ff
\end{proof}


\subsection*{Problem 2}
\problem{
Let $f(t) = e^t$. Let $P_n(t)$ be the $n$-th order Taylor polynomial of $f$ at the point $x_0=0$.
\begin{enumerate}
	\item Let $n\geq 1$. Prove that $n!P_n(1)$ is an integer.
	\item Using part (a) and Taylor's theorem, prove that Euler's number $e$ is irrational.
		You may use the fact that $e^t$ is strictly monotone increasing, and $0<e<3$.
\end{enumerate}
\emph{Hint}: if $e$ were rational, then we could write $e=m/n$....
}

\begin{enumerate}
	\item \begin{proof}[Solution]\let\qed\relax
			See
			\[
				n!P_n(t) = n!\sum_{i=0}^n\frac{x^i}{i!}
				= \sum_{i=0}(n-i)!x^i
			\]
			Now when $x = 1$, each term is an integer,
			and the integers are closed under addition,
			so $n!P_n(1)$ is an integer as well.
		\end{proof}
	\item \begin{proof}[Solution]\let\qed\relax
			Assume, for the sake of contradiction, that $e \in \Q$,
			that is to say,
			$e = m/n$, for some $m \in \Z, n \in \N$.
			Note $n!e = m(n-1)! \in \Z$.
			Also recall by Taylor's theorem that
			\[
				e = P_n(1) + \frac{f^{(n+1)}(x)}{(n+1)!}
				= P_n(1) + \frac{e^x}{(n+1)!}
			\]
			for some $x \in (0,1)$.
			Then
			\[
				n!e = n!P_n(t) + \frac{e^x}{n+1}
			\]
			Let $n \geq 2$.
			Since $0 < e < 3$ and $0 < x < 1$,
			we have $0 < e^x < 3$, and so when $n \geq 2$,
			$n+1 > e^x \implies \frac{e^x}{n+1} \not\in \Z$.
			However, by part (a), we know that $n!P_n(t) \in \Z$
			and so $\frac{e^x}{n+1} = n!e - n!P_n(t) \in \Z$,
			which is a contradiction.
		\end{proof}
\end{enumerate}


\subsection*{Problem 3}
The next problem concerns monotone increasing functions,
and will help prepare us for the Riemann–Stieltjes integral.
Let $\alpha\colon [0,1]\to\R$ be increasing.
Recall from last term that for every $c\in[0,1]$,
$\lim_{x\searrow c}\alpha(x)$ and $\lim_{x\nearrow c}\alpha(x)$ always exist.
Thus $\alpha$ is continuous at $c$ if and only if
$\lim_{x\searrow c}\alpha(x)=\lim_{x\nearrow c}\alpha(x)$.
If $\alpha$ is not continuous at $c$,
then $\lim_{x\nearrow c}\alpha(x) < \lim_{x\searrow c}\alpha(x)$,
and we say $\alpha$ has a \emph{jump discontinuity} at $c$.
\problem{
Let $\alpha\colon [0,1]\to\R$ be montone increasing.
Prove that the set of points $c\in [0,1]$ where $\alpha$ is not continuous is
either finite (possibly empty), or countably infinite.
}

\begin{proof}[Solution]\let\qed\relax
	For the sake of contradiction, assume that the set of discontinuous
	points is uncountably infinite.
	ff
	Recall that $\lim_{x\searrow c}\alpha(x) = a$ when
	for all $\ep > 0$, there exists some $\delta > 0$
	such that for all $x \in (c,c+\delta)$, $|\alpha(x) - a| < \ep$.
	Sequences: if $\alpha(x_n) \to a$ as $n \to \infty$
	for all sequences $\{x_n\}$ in $(c,1]$ such that $x_n \to c$.
\end{proof}
\end{document}
