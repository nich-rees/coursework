\documentclass{article}
\usepackage{amsmath, amsfonts, amsthm, amssymb}
\usepackage{geometry}
\geometry{letterpaper, margin=2.0cm, includefoot, footskip=30pt}

\usepackage{fancyhdr}
\pagestyle{fancy}

\lhead{Math 321}
\chead{Homework 1}
\rhead{Nicholas Rees, 11848363}
\cfoot{Page \thepage}

\newcommand{\N}{{\mathbb N}}
\newcommand{\Z}{{\mathbb Z}}
\newcommand{\Q}{{\mathbb Q}}
\newcommand{\R}{{\mathbb R}}
\newcommand{\C}{{\mathbb C}}
\newcommand{\ep}{{\varepsilon}}

\newcommand{\problem}[1]{
	\begin{center}\fbox{
		\begin{minipage}{17.0 cm}
			\setlength{\parindent}{1.5em}
			{\it \noindent#1}
		\end{minipage}}
	\end{center}}

\newtheorem{lemma}{Lemma}

\renewcommand{\theenumi}{(\alph{enumi})}

\begin{document}
\begin{center}
	{\bf Math 321 Homework 1}
\end{center}

In this homework, we will need several definitions. Let $I=[a,b]$ be an interval and $k\geq 0$ be an integer. If $f\colon I\to\R$ is a function that is $k$-times differentiable on $I$, then we define
\[
\Vert f \Vert_{C^k(I)}=\sum_{j=0}^k \sup_{x\in I} |f^{(j)}(x)|.
\]
This quantity is called the ``$C^k$ norm of $f$.''
We define $C^k(I)$ to be the set of functions $f\colon I\to\R$ that satisfy the following two properties.
(i): $f$ is $k$-times differentiable on $I$, and
(ii): $f^{(k)}$ is continuous on $I$.
We define a metric on $C^k(I)$ as follows: $d(f,g)=\Vert f-g \Vert_{C^k(I)}$, i.e.
\begin{equation}\label{CkNorm}
	d(f,g) = \sum_{j=0}^k \sup_{x\in I} |f^{(j)}(x)-g^{(j)}(x)|.
\end{equation}
It is straightforward to verify that this is indeed a metric, but you do not have to do so for this homework.

\subsection*{Problem 1}
\problem{
Let $f(t) = e^t$; recall that $f$ is monotone increasing, $f'(t) = f(t)$, and $f(0)=1$.
Let $P_n(t)$ be the $n$-th order Taylor polynomial of $f$ at the point $x_0=0$,
as discussed in lecture.
Let $I=[-1,1]$ and let $k\geq 1$ be an integer.

Using Taylor's theorem, prove that the sequence $\{P_n\}$ converges to
$f$ in the metric space $C^k(I)$. 

\emph{Hints} (i) Compute the Taylor polynomial $P_n(t)$. (ii) What is the derivative of $P_n$? (iii) What are the higher derivatives of $P_n$? (iv) How can you estimate each term in \eqref{CkNorm}?
}
\begin{proof}[Solution]\let\qed\relax
	Recall that for $f(t)$, the $n$-th ordered Taylor polynomial at $x_0 = 0$ is
	\[
		P_n(t) = \sum_{i=0}^n \frac{x^i}{i!}
	\]
	Furthermore, note that the $j$-th derivative of $P_n(t)$
	is $0$ if $j > n$ and
	\begin{equation}\label{deriv}
		\frac{d^j}{dx^j}P_n(t) =
		\frac{d^j}{dx^j}\sum_{i=0}^{j-1}\frac{x^i}{i!}
		+ \sum_{i=j}^n\frac{d^j}{dx^j}\frac{x^i}{i!}
		= 0 + \sum_{i=j}^n \frac{1}{i!}\frac{i!}{(i-j)!}x^{i-j}
		= \sum_{i=j}^n \frac{x^{i-j}}{(i-j)!}
		= \sum_{i=0}^{n-j}\frac{x^i}{i!} = P_{n-j}(t)
	\end{equation}
	when $j \leq n$.

	Recall from Taylor's theorem that,
	since $e^t$ is continuous and it's $(n+1)$ derivative always exists
	(simple induction, since $f'(t) = f(t)$),
	there exists $c_n$ between $t$ and $0$ such that
	\begin{equation}\label{eappx}
		e^t = P_n(t) + \frac{f^{(n+1)}(c_n)}{(n+1)!}t^{n+1}
		= P_n(t) + \frac{e^{c_n}}{(n+1)!}t^{n+1}
	\end{equation}

	We remark that since $e^t$ is $k$-times differentiable for any $k$,
	$e^t \in C^k(I)$.
	So we now only need to show $d(e^t,P_n) < \ep$ for arbitrary $\ep > 0$.
	Consider $d(e^t,P_n)$ in $C^k(I)$ when we fix $n \geq k$.
	\begin{align*}
		d(e^t,P_n)
		&= \sum_{j=0}^k \sup_{t \in I} \lvert e^t - P_n^{(j)}(t)\rvert\\
		&= \sum_{j=0}^k \sup_{t \in I}\lvert e^t - P_{n-j}(t)\rvert
		& \text{applying (\ref{deriv})}\\
		&= \sum_{j=0}^k \sup_{t \in I}
		\left\lvert P_{n-j}(t) + \frac{e^{c_{n-j}}}{(n-j+1)!}t^{n-j+1} - P_{n-j}(t)\right\rvert
		& \text{applying (\ref{eappx})}\\
		&= \sum_{j=0}^k \sup_{t \in I}
		\left\lvert \frac{e^{c_{n-j}}}{(n-j+1)!}t^{n-j+1}\right\rvert\\
		&= \sum_{j=0}^k \frac{1}{(n-j+1)!}
		\sup_{t \in I}\left\lvert e^{c_{n-j}}t^{n-j+1}\right\rvert\\
		&\leq \sum_{j=0}^k \frac{e}{(n-j+1)!}\\
		&\leq \frac{ke}{(n-k+1)!}\\
		&\leq \frac{ke}{n-k}
	\end{align*}
	where the 6th line is done by the following reasoning:
	since $c_{n-j} \leq 1$ always,
	and since $e^t$ is monotonically increasing,
	$e^{c_{n-j}} \leq e^1$ and so
	$|e^{c_{n-j}}t^{n-j+1}| \leq |et^{n-j+1}|$
	and taking the supremum of both sides preserves weak inequalities, giving
	\[
		\sup_{t\in I}|e^{c_{n-j}}t^{n-j+1}|
		\leq \sup_{t\in I}|et^{n-j+1}|
	\]
	Now, since the maximum value $|t^x|$ can obtain when $t \in I$
	(and $x>0$, since $n-j+1 \geq n-k+1 \geq 1$) is $1$, we have
	\[
		|et^{n-j+1}| = |e||t^{n-j+1}| < e
		\implies \sup_{t\in I}|et^{n-j+1}| \leq e
	\]
	giving us the desired inequality.

	Let $\ep > 0$. Let $N = \mathrm{max}\{k,\lceil 2ke/\ep + k\rceil\}$.
	Then for all $n \geq N$, we have
	\[
		d(e^t,P_n) \leq \frac{ke}{n-k} \leq \frac{ke}{N-k}
		\leq \frac{ke}{2ke/\ep + k - k} = \frac{\ep}{2} < \ep
	\]
	Hence, $\{P_n\}$ converges to $f = e^t$ in $C^k(I)$.
\end{proof}


\subsection*{Problem 2}
\problem{
Let $f(t) = e^t$. Let $P_n(t)$ be the $n$-th order Taylor polynomial of $f$ at the point $x_0=0$.
\begin{enumerate}
	\item Let $n\geq 1$. Prove that $n!P_n(1)$ is an integer.
	\item Using part (a) and Taylor's theorem, prove that Euler's number $e$ is irrational.
		You may use the fact that $e^t$ is strictly monotone increasing, and $0<e<3$.
\end{enumerate}
\emph{Hint}: if $e$ were rational, then we could write $e=m/n$....
}

\begin{enumerate}
	\item \begin{proof}[Solution]\let\qed\relax
			See
			\[
				n!P_n(t) = n!\sum_{i=0}^n\frac{t^i}{i!}
				= \sum^n_{i=0}\left(n(n-1)\cdots(i+1)t^i\right)
			\]
			Now when $t = 1$, each term is an integer
			since the integers are closed under multiplication.
			The sum will also be an integer since integers are closed under addition,
			so $n!P_n(1)$ is an integer as well.
		\end{proof}
	\item \begin{proof}[Solution]\let\qed\relax
			Assume, for the sake of contradiction, that $e \in \Q$,
			that is to say,
			$e = m/n$, for some $m \in \Z, n \in \N$.
			Note then $n!e = m(n-1)! \in \Z$.
			Let $n' = \max\{2,n\}$.
			We also have $n'!e \in \Z$
			(if $n' \neq n$ then $2 > n$, which means $n = 1 \implies e = m$,
			and $2e = 2m \in \Z$).
			Also recall by Taylor's theorem that
			\[
				e = P_{n'}(1) + \frac{f^{(n'+1)}(x)}{(n'+1)!}
				= P_{n'}(1) + \frac{e^x}{(n'+1)!}
			\]
			for some $x \in (0,1)$.
			Then
			\[
				n'!e = n'!P_{n'}(t) + \frac{e^x}{n'+1}
			\]
			Since $0 < e < 3$ and $0 < x < 1$,
			we have $0 < e^x < 3$, and so since $n' \geq 2$,
			$n'+1 > e^x \implies \frac{e^x}{n'+1} \not\in \Z$.
			However, by part (a), we know that $n'!P_{n'}(t) \in \Z$
			and so $\frac{e^x}{n'+1} = n'!e - n'!P_{n'}(t) \in \Z$
			(since this is the difference of two integers),
			which is a contradiction.
		\end{proof}
\end{enumerate}


\subsection*{Problem 3}
The next problem concerns monotone increasing functions,
and will help prepare us for the Riemann–Stieltjes integral.
Let $\alpha\colon [0,1]\to\R$ be increasing.
Recall from last term that for every $c\in[0,1]$,
$\lim_{x\searrow c}\alpha(x)$ and $\lim_{x\nearrow c}\alpha(x)$ always exist.
Thus $\alpha$ is continuous at $c$ if and only if
$\lim_{x\searrow c}\alpha(x)=\lim_{x\nearrow c}\alpha(x)$.
If $\alpha$ is not continuous at $c$,
then $\lim_{x\nearrow c}\alpha(x) < \lim_{x\searrow c}\alpha(x)$,
and we say $\alpha$ has a \emph{jump discontinuity} at $c$.
\problem{
Let $\alpha\colon [0,1]\to\R$ be montone increasing.
Prove that the set of points $c\in [0,1]$ where $\alpha$ is not continuous is
either finite (possibly empty), or countably infinite.
}

\begin{proof}[Solution]\let\qed\relax
	Consider the set $D\subset [0,1]$ where
	$D = \{c \mid \alpha \text{ has a jump discontinuity at }c\}$.
	We seek to show there exists a injective map from $D$ to $\Q$,
	and so the cardinality of $D$ is at most the cardinality of $\Q$,
	hence $D$ is at most countable.
	By the density of the rationals, since 
	$\lim_{x\nearrow c}\alpha(x) < \lim_{x\searrow c}\alpha(x)$ when $c \in D$,
	there is some rational $q$ such that
	$\lim_{x\nearrow c}\alpha(x) < q < \lim_{x\searrow c}\alpha(x)$.
	Let $\phi \colon D \to \Q$ be defined by $c \mapsto q_c$
	where $q_c$ is one such rational such that 
	$\lim_{x\nearrow c}\alpha(x) < q_c < \lim_{x\searrow c}\alpha(x)$.

	We now prove injectivity of $\phi$.
	Let $c_1,c_2 \in D$ such that $c_1 \neq c_2$.
	WLOG let $c_1 < c_2$.
	Let $m = \frac{c_1+c_2}{2}$.
	Since $\alpha$ is monotonically increasing, we have
	$\alpha(c_1) \leq \alpha(m) \leq \alpha(c_2)$.
	Clearly $\inf_{c_1<x<m}\alpha(x) \leq \alpha(m)$
	and $\sup_{m<x<c_2}\alpha(x) \geq \alpha(m)$,
	and $\lim_{x\searrow c_1}\alpha(x) = \inf_{c_1<x<m}\alpha(x)$
	and $\lim_{x\nearrow c_2}\alpha(x) = \sup_{m<x<c_2}\alpha(x)$ by Rudin Theorem 4.29,
	so
	\[
		q_{c_1} < \lim_{x\searrow c_1}\alpha(x) \leq \alpha(m) \leq \lim_{x\nearrow c_2}\alpha(x) < q_{c_2}
	\]
	Hence, $q_{c_1} \neq q_{c_2}$ which implies $\phi(c_1) \neq \phi(c_2)$,
	which shows the injectivity of the map.
	Hence, $D$ is at most countable (either finite or countably infinite).
\end{proof}
\end{document}
