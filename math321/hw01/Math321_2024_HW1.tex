\documentclass[12pt]{article}
\usepackage{amsmath,amssymb,amsthm,epsf, graphics,enumerate,fancyhdr}

\pagestyle{fancy}


\setlength{\parindent}{0pt}
\setlength{\textwidth}{6.5in}
\setlength{\oddsidemargin}{0in}
\addtolength{\textheight}{1in}

\renewcommand\theenumi{\alph{enumi}}
\renewcommand\labelenumi{(\theenumi)}


\newcommand{\ZZ}{\mathbb{Z}}
\newcommand{\RR}{\mathbb{R}}


\theoremstyle{definition}
\newtheorem{problem}{}
\newtheorem{solution}{}

\begin{document}
\fancyhf{} % sets both header and footer to nothing
\renewcommand{\headrulewidth}{0pt}
\fancyfoot[L]{Copyright 2024 The University of British Columbia}

%\pagestyle{empty}

\begin{center} 
{\bf{Math 321 Assignment 1}} \\ 
{\bf{Due Friday, January 19 at 9 am}} 
\end{center}  
\bigskip
\hrule
\begin{center}
{\underline{Instructions}} 
\end{center} 
\begin{enumerate}[(i)] 
\item Homework should be submitted using Canvas. Include your name and SID.
%
\item Solutions should be well-crafted, legible and written in complete English sentences. You will be graded both on accuracy as well as the quality of exposition.
%
\item Theorems stated in the text or proved in lecture do not need to be reproved. Any other statement should be justified rigorously.
% 
\end{enumerate} 
\hrule
\bigskip

In this homework, we will need several definitions. Let $I=[a,b]$ be an interval and $k\geq 0$ be an integer. If $f\colon I\to\RR$ is a function that is $k$-times differentiable on $I$, then we define
\[
\Vert f \Vert_{C^k(I)}=\sum_{j=0}^k \sup_{x\in I} |f^{(j)}(x)|.
\]
This quantity is called the ``$C^k$ norm of $f$.'' We define $C^k(I)$ to be the set of functions $f\colon I\to\RR$ that satisfy the following two properties. (i): $f$ is $k$-times differentiable on $I$, and (ii): $f^{(k)}$ is continuous on $I$. We define a metric on $C^k(I)$ as follows: $d(f,g)=\Vert f-g \Vert_{C^k(I)}$, i.e.
\begin{equation}\label{CkNorm}
d(f,g) = \sum_{j=0}^k \sup_{x\in I} |f^{(j)}(x)-g^{(j)}(x)|.
\end{equation}
It is straightforward to verify that this is indeed a metric, but you do not have to do so for this homework.



%%%
%%%

\begin{problem}
Let $f(t) = e^t$; recall that $f$ is monotone increasing, $f'(t) = f(t)$, and $f(0)=1$. Let $P_n(t)$ be the $n$-th order Taylor polynomial of $f$ at the point $x_0=0$, as discussed in lecture. Let $I=[-1,1]$ and let $k\geq 1$ be an integer. 

Using Taylor's theorem, prove that the sequence $\{P_n\}$ converges to $f$ in the metric space $C^k(I)$. 
\end{problem}

%%%
%%%
\medskip

%%%
%%%

\noindent\emph{Hints} (i) Compute the Taylor polynomial $P_n(t)$. (ii) What is the derivative of $P_n$? (iii) What are the higher derivatives of $P_n$? (iv) How can you estimate each term in \eqref{CkNorm}?

%%%
%%%

\medskip

\begin{problem}
Let $f(t) = e^t$. Let $P_n(t)$ be the $n$-th order Taylor polynomial of $f$ at the point $x_0=0$.

\medskip

\noindent a) Let $n\geq 1$. Prove that $n!P_n(1)$ is an integer.

\medskip

\noindent b) Using part a) and Taylor's theorem, prove that Euler's number $e$ is irrational. You may use the fact that $e^t$ is strictly monotone increasing, and $0<e<3$.\\
\emph{Hint}: if $e$ were rational, then we could write $e=m/n$....
\end{problem}


%%%
%%%


\medskip
The next problem concerns monotone increasing functions, and will help prepare us for the Riemann–Stieltjes integral. Let $\alpha\colon [0,1]\to\RR$ be increasing. Recall from last term that for every $c\in[0,1]$, $\lim_{x\searrow c}\alpha(x)$ and $\lim_{x\nearrow c}\alpha(x)$ always exist. Thus $\alpha$ is continuous at $c$ if and only if $\lim_{x\searrow c}\alpha(x)=\lim_{x\nearrow c}\alpha(x)$. If $\alpha$ is not continuous at $c$, then $\lim_{x\nearrow c}\alpha(x) < \lim_{x\searrow c}\alpha(x)$, and we say $\alpha$ has a \emph{jump discontinuity} at $c$.

\begin{problem}
Let $\alpha\colon [0,1]\to\RR$ be montone increasing. Prove that the set of points $c\in [0,1]$ where $\alpha$ is not continuous is either finite (possibly empty), or countably infinite.
\end{problem}

\end{document}
