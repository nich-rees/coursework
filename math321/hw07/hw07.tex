\documentclass{article}
\usepackage{amsmath, amsfonts, amsthm, amssymb}
\usepackage{geometry}
\geometry{letterpaper, margin=2.0cm, includefoot, footskip=30pt}

\usepackage{fancyhdr}
\pagestyle{fancy}

\lhead{Math 321}
\chead{Homework 7}
\rhead{Nicholas Rees, 11848363}
\cfoot{Page \thepage}

\newcommand{\N}{{\mathbb N}}
\newcommand{\Z}{{\mathbb Z}}
\newcommand{\Q}{{\mathbb Q}}
\newcommand{\R}{{\mathbb R}}
\newcommand{\C}{{\mathbb C}}
\newcommand{\ep}{{\varepsilon}}

\newcommand{\problem}[1]{
	\begin{center}\fbox{
		\begin{minipage}{17.0 cm}
			\setlength{\parindent}{1.5em}
			{\it \noindent#1}
		\end{minipage}}
	\end{center}}

\newtheorem{lemma}{Lemma}
\theoremstyle{remark}
\newtheorem{remark}{Remark}

\renewcommand{\theenumi}{(\alph{enumi})}

\begin{document}
\begin{center}
	{\bf Math 321 Homework 7}\\
	(Including work made in collaboration with Matthew Bull-Weizel)
\end{center}

\subsection*{Problem 1}
\problem{
	Let $\mathcal{F}$ be a family of equicontinuous functions,
	$f \colon [a,b] \to \R$ that is pointwise bounded.
	Prove that the function $g(x) = \sup_{f \in \mathcal{F}}f(x)$ is continuous on $[a,b]$.
}
\begin{proof}[Solution]\let\qed\relax
	Consider a point $x \in [a,b]$.
	Let $\ep > 0$.
	Since $g(x)$ is the supremum of $\{f(x)\}_{f \in \mathcal{F}}$,
	there exists a $f_1 \in \mathcal{F}$ such that
	$g(x) < \frac{\ep}{2} + f_1(x)$.
	Furthermore, since $f_1 \in \mathcal{F}$ is in an equicontinuous set,
	there is a $\delta > 0$ such that $|x-y| < \delta$ implies $|f(x) - f(y)| < \frac{\ep}{2}$
	for all $f \in \mathcal{F}$, and so certainly this inequality holds for $f_1$.
	This gives $-\frac{\ep}{2} < f_1(x) - f_1(y) < \frac{\ep}{2}$.
	Then $g(x) < \frac{\ep}{2} + \frac{\ep}{2} + f_1(y) \leq \ep + g(y)$,
	so $g(x) - g(y) < \ep$.
	For that same $y$, we get some $f_2 \in \mathcal{F}$ such that
	$g(y) < \frac{\ep}{2} + f_2(y)$.
	Since $f_2 \in \mathcal{F}$, we get that $f_2(y) - f_2(x) < \frac{\ep}{2}$
	since $|x-y| < \delta$, thus
	$g(y) < \frac{\ep}{2} + \frac{\ep}{2} + f_2(x) \leq \ep + g(x)$.
	Therefore, we have $g(y) - g(x) < \ep$.
	Hence, we have shown that when $|x-y| < \delta$, we have
	$|g(x) - g(y)| < \ep$.
\end{proof}


\subsection*{Problem 2}
\problem{
	For each $n \in \N$, let $f_n \colon \R \to \R$ be continuous.
	Suppose that the family $\{f_n\}_{n\in\N}$ is pointwise bounded and equicontinuous.
	Prove that $\{f_n\}_{n\in\N}$ has a subsequence that converges uniformly
	on every compact set $K \subset \R$ and converges on all of $\R$.
}
\begin{proof}[Solution]\let\qed\relax
	Define $K_m = [-m,m]$, for $m \in \N$.
	The $K_m$ are closed and bounded, and are subsets of $\R$,
	so the Heine Borel property gives us that these are compact.
	Identify $K_m$ with the metric space with the metric from $\R$ on the compact set $K_m$.
	Hence, $K_m$ is a compact metric space.

	Note that $\{f_n\} \subset \mathcal{C}(K_m)$ since $f_n \colon \R \to \R$,
	so surely $f_n\vert_{K_m}$ must be continuous as well.
	Furthermore, $\{f_n\}$ are equicontinuous and pointwise bounded on $K_m$,
	since if $\phi(x) > |f_n(x)|$ for all $x \in \R$,
	surely $\phi(x) > |f_n(x)|$ for all $x \in K_m \subset \R$;
	and if $\ep > 0$, there is a $\delta > 0$ such that
	$d(x,y) < \delta \implies |f_n(x) - f_n(y)| < \ep$ for all $x,y \in \R$ and $f_n \in \{f_n\}$,
	and so $d(x,y) < \delta \implies |f_n(x) - f_n(y)| < \ep$ for all $x,y \in K_m \subset \R$.
	Furthermore, note that all of these properties still hold if $\{f_n\}$
	is replaced by a subsequence of itself above.

	We can conlude by Arzel\`{a}-Ascoli (Rudin Theorem 7.25 (b)) that
	there is a uniformly convergent subsequence of $\{f_n\}$ on the compact set $K_1$.
	Let us denote the uniformly convergent subsequence on $K_1$
	by $\{f_n^{(1)}\}_{n\in\N}$.
	We can iterate this process:
	if $\{f_n^{(m)}\}_{n \in \N}$ is a uniformly convergent subsequence on $K_m$,
	and furthermore, $\{f_n^{(m)}\}$ satisfies the hypotheses of
	Arzel\'{a}-Ascoli as mentioned above, the theorem gives us
	a uniformly convergent subsequence of $\{f_n^{(m)}\}_{n\in\N}$ on $K_{m+1}$,
	which is our $\{f_n^{(m+1)}\}_{n\in\N}$.

	We now take the diagonal of all of these subsequences,
	i.e. $f_1^{(1)}, f_2^{(2)}, f_3^{(3)}, \dots$.
	Call this sequence $S$.
	By construction, after at most $n-1$ terms, $S$ is uniformly convergent on $K_n$,
	and so is uniformly convergent on $K_n$.
	Note that since every compact set $K\subset \R$ is bounded (Heine-Borel),
	there is some $K_n$ such that $K \subset K_n$.
	Hence, the sequence $S$ converges uniformly on every compact $K \subset \R$ as well,
	since it converges on a superset of $K$.

	Now we will show that $S$ converges on all of $\R$.
	Consider arbitrary $x \in \R$. Note that $\{x\}$ is compact,
	since it is closed and bounded (Heine-Borel)
	(closure is obvious vacuously, since there are no limit points,
	boundedness follows from the choice of any $M > 0$ and $d(x,x) < M$).
	Thus, as we mentioned previously,
	$S$ converges on $\{x\}$.
\end{proof}


\subsection*{Problem 3}
For the next problem, recall that a set $K$ in a metric space is compact
if and only if every infinite set $S \subset K$ has a limit point in $K$.
You may use this fact for the next problem.
\problem{
	Let $K$ be a compact metric space.
	Using Theorems 7.24 and 7.25, prove that a set $\mathcal{F} \subset \mathcal{C}(K)$
	(with the metric on $\mathcal{C}(K)$ determined by the supremum norm)
	is compact if and only if $\mathcal{F}$ is closed, bounded, and equicontinuous.
}
\begin{proof}[Solution]\let\qed\relax
	Assume that $\mathcal{F}$ is closed, bounded, and equicontinuous.
	If $\mathcal{F}$ is finite, then it is compact vacuously
	(there are no infinite sets $S$).
	Now assume that $\mathcal{F}$ is infinite.
	Since $\{f_n\}$ is bounded in $\mathcal{C}(K)$,
	we have that there exists some $f \in \{f_n\}$ are $M \in \R$
	such that $d(f,f_n) = \sup_x |f(x) - f_n(x)| < M$ for all $f_n \in \{f_n\}$.
	Then $\phi(x) = |f(x)| + M$ pointwise bounds $\{f_n\}$,
	since $|f_n(x)| < |f(x) + M| \leq |f(x)| + M$.
	Now let $S \subset K$ be an infinite subset.
	There must exist a countable subset of $S$, which we'll call $\{f_n\}$.
	Clearly, we must have that $\{f_n\} \subset \mathcal{C}(K)$ still,
	$\{f_n\}$ is pointwise bounded by the same $\phi(x)$ that pointwise bounds $\mathcal{F}$,
	and $\{f_n\}$ is equicontinuous since $\forall \ep > 0$, $\exists \delta$
	such that $d(x,y) < \delta \implies |f(x) - f(y)| < \ep$ for $x,y \in K$ and all
	$f \in \mathcal{F}$ and so must hold for all $f \in \{f_n\} \subset \mathcal{F}$.
	Then, by Theorem 7.25, we get that $\{f_n\}$ contains a uniformly convergent subsequence.
	Hence, $S$ contains a limit point,
	specifically the limit of the convergent subsequence of $\{f_n\}$.
	Finally, since $\mathcal{F}$ is closed, it contains all of its limit points,
	hence $S$ has a limit point in $\mathcal{F}$.
	Therefore, $\mathcal{F}$ is compact.

	Assume that $\mathcal{F} \subset \mathcal{C}(K)$ is compact.
	Then Rudin Theorem 2.34 says compact subsets of metric spaces are closed,
	hence $\mathcal{F}$ is closed.
	Furthermore, $\mathcal{F}$ is bounded:
	for the sake of contradiction, assume that $\mathcal{F}$ is unbounded.
	We must have that $\mathcal{F}$ is an infinite set, otherwise
	just take the distance set of all the possible distances between the elements of $\mathcal{F}$,
	and since this is finite, we can get a maximum distance; boundedness follows.
	Pick an element $s_0 \in \mathcal{F}$.
	Then, let $s_1$ be some elemenet in $\mathcal{F}$ such that $d_1 := d(s_0,s_1) \geq 1$,
	the existence of such is due to the unboudedness of $\mathcal{F}$,
	and $\mathcal{F}$ being infinite.
	Now choose $s_2$ such that $s_2 \in \mathcal{F}$ and $d_2 = d(s_0,s_1) \geq 2d_1$,
	again chosen because $\mathcal{F}$ is unbounded.
	Note that $s_2 \neq s_1$, since we require $s_2$ be further from
	$s_0$ than $s_1$ is.
	Generally, we choose $s_i \in \mathcal{F}$ and $d_i = d(s_0,s_i) \geq 2d_{i-1}$.
	But then $\bigcup_i s_i$ is an infinite set,
	but all points are at least distance $1$ apart
	(let $i > j$ then $d(s_0,s_j) + d(s_i,s_j) \geq d(s_0,s_i)
	\implies d(s_i,s_j) \geq d_i - d_j \geq 2d_{i-1} - d_{i-1} = d_{i-1} \geq 1$)
	and so all the points are isolated point, so $\bigcup_i s_i$ contains no limit points.
	But this means that $\mathcal{F}$ is not compact, a contradiction.
	
	It remains to show that $\mathcal{F}$ being compact implies that $\mathcal{F}$ is equicontinuous.
	We do so with the contrapositive,
	so assume that $\mathcal{F}$ is not equicontinuous.
	Let $\{f_n\}$ be an countably infinite subset of $\mathcal{F}$,
	that is not equictonintuous:
	we can find such a sequence, since we have that $\exists \ep$
	such that for all $\delta_n = \frac{1}{n} > 0$, there are $x,y \in K$
	and $f_n \in \mathcal{F}$ such that $|f_n(x) - f_n(y)| \geq \ep$.
	Note that $\{f_n\}$ cannot contain any convergent sequences of functions,
	since for every subsequence $\{f_{n_k}\}$ of $\{f_n\}$,
	since $\{f_{n_k}\}$ are not equicontinuous by construction,
	the contrapositive of Theoreem 2.24
	gives us that $\{f_{n_k}\}$ does not converge uniformly on $K$,
	but this is what it means to converge with the supremum norm metric,
	hence, $\{f_{n_k}\}$ does not converge.
	But this means that $\{f_n\} \subset \mathcal{F}$ is an infinite set
	which does not have a limit point in $\mathcal{F}$.
	Hence, $\mathcal{F}$ is not compact.
\end{proof}


\subsection*{Problem 4}
In the next problem, you will prove the converse to Theorem 7.25
\problem{
	Let $K$ be a compact metric space and let $\mathcal{F} \subset \mathcal{C}(K)$.
	Suppose that every sequence in $\mathcal{F}$ has a subsequence that
	converges uniformly on $K$.
	\begin{enumerate}
		\item Prove that $\mathcal{F}$ is uniformly bounded.
		\item Prove that $\mathcal{F}$ is equicontinuous.
	\end{enumerate}
}
\begin{enumerate}
	\item \begin{proof}[Solution]\let\qed\relax
		Hmm show closure is compact (sequential compactness).
		Out of time :(
	\end{proof}
	\item \begin{proof}[Solution]\let\qed\relax
		Out of time :/
	\end{proof}
\end{enumerate}


\subsection*{Problem 5}
\problem{
	For each $n \in \N$, let $f_n \colon \R \to \R$, and suppose that
	$\{f_n\}$ is an approximate identity.
	Prove that $\{f_n\}$ is not equicontinuous.
}
\begin{proof}[Solution]\let\qed\relax
	We prove the contrapositive.
	Assume that $\{f_n\}$ is equicontinuous.
	Then there is some $\delta'$ such that when $|x-y| < \delta'$,
	we have $|f_n(x) - f_n(y)| < \ep$ for all $n \in \N$.
	We can find some $N_1 \in \N$ such that for all $n \geq N_1$,
	$\int_{-\infty}^{-\delta'} |f_n(x)|dx < \ep'$.
	Similarly, we get an $N_2 \in \N$ such that for all $n \geq N_2$,
	$\int_{-\infty}^{\delta'} |f_n(x)|dx < \ep'$.
	Let $N = \max\{N_1,N_2\}$.
	Surely $|f_n(x)| < \ep'$ on $(-\infty,-\delta'] \cup [\delta',\infty)$.
	Rudin Theorem 6.13 also gives
	\[
		\int_{-\infty}^{-\delta'}f_n(x)dx \leq
		\left\lvert\int_{-\infty}^{-\delta'}f_n(x)dx\right\rvert < \ep'
	\]
	and
	\[
		\int_{\delta'}^\infty f_n(x)dx \leq
		\left\lvert\int_{\delta'}^{\infty}f_n(x)dx\right\rvert < \ep'
	\]
	(the fact that this is an indefinite integral does not ruin the theorem,
	since we know the integral exists, and the inequality holds for
	arbitrarily large values of the bounds of integration;
	I will use a similar generalization of 6.12 later as well).
	We can then conclude
	\[
		1 = \int_{-\infty}^{-\delta'}f_n(x)dx + \int_{-\delta'}^{\delta'} f_n(x) dx
		+ \int_{\delta'}^\infty f_n(x)dx < \int_{-\delta'}^{\delta'} f_n(x)dx + 2\ep'
	\]
	Therefore
	\[
		1 - 2\ep' < \int_{-\delta'}^{\delta'} f_n(x)dx
	\]
	This implies that for all $f_n \in \mathcal{F}$
	where $n \geq N$, there is some $x \in [-\delta',\delta']$ such that
	$f_n(x) > \frac{1}{2\delta'}(1-2\ep')$.
	Otherwise, Rudin Theorem 6.12(d) says the integral is less than or equal to $1-2\ep'$.
	Then, if $y \not\in [-\delta',\delta']$, since $|f(y)| < \ep'$,
	we have $|f_n(x) - f_n(y)| = f_n(x) - f_n(y) = \frac{1}{2\delta'}(1-2\ep') - \ep'$
	when $\ep' < \frac13$ and $\delta' < \frac12$ so that
	we get the equality of the absolute value
	(since $\frac{1}{2\delta'}(1-2\ep') > \ep'$).

	Now fix $\ep = 1$.
	For all $\delta > 0$, let $\delta' = \min\{\frac13\delta,\frac13\}$, and find $\ep' > 0$ such that
	$0 < \ep' < (\frac{1}{2\delta'} - 1)/(\frac{1}{\delta'} + 1)$
	and $\ep' < \frac13$.
	We have shown that there exists some $\ep$ such that for all $\delta > 0$,
	there are $x,y \in \R$ where $d(x,y) < \delta$ (namely, the $x \in [-\delta',\delta']$ from above
	and $y \not\in[-\delta',\delta']$, which we can choose
	because there are always points within $3\delta'$ of $x$ outside $[-\delta',\delta]$) and $f_n \in \{f_n\}$ where $n \geq N$ such that
	$|f_n(x) - f_n(y)| = \frac{1}{2\delta'}(1-2\ep') - \ep' \geq 1$.
	Hence, $\{f_n\}$ cannot be equicontinuous.
\end{proof}
\end{document}
