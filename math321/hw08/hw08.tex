\documentclass{article}
\usepackage{amsmath, amsfonts, amsthm, amssymb}
\usepackage{geometry}
\geometry{letterpaper, margin=2.0cm, includefoot, footskip=30pt}

\usepackage{fancyhdr}
\pagestyle{fancy}

\lhead{Math 321}
\chead{Homework 8}
\rhead{Nicholas Rees, 11848363}
\cfoot{Page \thepage}

\newcommand{\N}{{\mathbb N}}
\newcommand{\Z}{{\mathbb Z}}
\newcommand{\Q}{{\mathbb Q}}
\newcommand{\R}{{\mathbb R}}
\newcommand{\C}{{\mathbb C}}
\newcommand{\ep}{{\varepsilon}}

\newcommand{\problem}[1]{
	\begin{center}\fbox{
		\begin{minipage}{17.0 cm}
			\setlength{\parindent}{1.5em}
			{\it \noindent#1}
		\end{minipage}}
	\end{center}}

\newtheorem{lemma}{Lemma}
\theoremstyle{remark}
\newtheorem{remark}{Remark}

\renewcommand{\theenumi}{(\alph{enumi})}

\begin{document}
\begin{center}
	{\bf Math 321 Homework 8}\\
\end{center}

For the next problem, we will need the following definition.
Let $N \geq 1$ be an integer and let $s,t$ be integers.
We define the half-open square
\[
	S_{s,t;N} = \left[\frac{s}{N},\frac{s+1}{N}\right) \times
	\left[\frac{t}{n},\frac{t+1}{N}\right)
\]
Observe that for each fixed $N$, $\R^2$ is a disjoint union of the half-open squares
$\{S_{s,t;N} \colon s,t \in \Z\}$.
We say a function $f \colon \R^2 \to \C$ or $\R^2 \to \R$ is
``constant on half-open squares at resolution $N$"
if $f$ is constant on each $S_{s,t;N}$, i.e. $f$ can be written as
\[
	f(x,y) = \sum_{s,t \in \Z} a_{s,t}\chi_{s,t;N}(x,y)
\]
\subsection*{Problem 1}
\problem{
	\begin{enumerate}
		\item Let $N \geq 1$ be an integer and let $f \colon \R^2 \to \R$
			have compact support and be constant on half-open squares at resolution $N$.
			For each $x,y \in \R$, define
			\[
				g(x) = \int_{-\infty}^\infty f(x,y)dy, \quad
				h(y) = \int_{-\infty}^\infty f(x,y)dx, \quad
			\]
			Prove that $g$ and $h$ are integrable on $\R$, and that
			\[
				\int_{-\infty}^\infty g(x)dx = \int_{-\infty}^\infty h(y)dy
			\]
			(and that both integrals converge).
		\item Let $f \colon \R^2 \to \R$ be continuous and have compact support.
			For each $x,y \in \R$, define
			\[
				g(x) = \int_{-\infty}^\infty f(x,y)dy, \quad
				h(y) = \int_{-\infty}^\infty f(x,y)dx, \quad
			\]
			Prove that $g$ and $h$ are integrable on $\R$, and that
			\[
				\int_{-\infty}^\infty g(x)dx = \int_{-\infty}^\infty h(y)dy
			\]
	\end{enumerate}
	\begin{remark}
		You have just proved that for $f \colon \R^2 \to \R$ continuous
		with compact support, we have
		\[
			\int_{-\infty}^\infty \left(\int_{-\infty}^\infty f(x,y)dx\right)dy =
			\int_{-\infty}^\infty \left(\int_{-\infty}^\infty f(x,y)dy\right)dx
		\]
	\end{remark}
}
\begin{enumerate}
	\item \begin{proof}[Solution]\let\qed\relax
		Since $f$ has compact support, there exists some $s',t'$
		such that $a_{s,t} = 0$ when $|s| > s'$ or $|t| > t'$.
		Alternatively, $\mathrm{supp}(f) \subset [-M,M] \times [-M,M]$
		for some $M \in \R$. Thus, using Rudin Theorem 6.12(a)
		(our sum is finite, so there are no problems with using the theorem),
		\begin{align*}
			g(x) &= \int_{-\infty}^\infty f(x,y)dy\\
				 &= \int_{-M}^M \sum_{\substack{s,t\in\Z\\|s|\leq s',|t|\leq t'}} a_{s,t}
			\chi_{s,t;N}(x,y)dy\\
				 &= \sum_{{\substack{s,t\in\Z\\|s|\leq s',|t|\leq t'}}}
			\int_{t/N}^{(t+1)/N} a_{s,t}\chi_{s,t;N}(x,y)dy\\
				 &= \sum_{{\substack{s,t\in\Z\\|s|\leq s',|t|\leq t'}}}
				 a_{s,t}\int_{t/N}^{(t+1)/N}\chi_{s,t;N}(x,y)dy\\
				 &= \sum_{{\substack{s,t\in\Z\\|s|\leq s',|t|\leq t'}}}
				 \frac{a_{s,t}}{N}\chi_{s,t;N}(x)\\
		\end{align*}
		And the integral converges, via 6.12(a) as well.
		Similarly, we can do the same for $h(y)$ to get
		\[
			h(y) = \sum_{{\substack{s,t\in\Z\\|s|\leq s',|t|\leq t'}}}
			\frac{a_{s,t}}{N}\chi_{s,t;N}(y)\\
		\]
		Since our $s',t'$ are the same as before,
		we still have $\mathrm{supp}(g) \subset [-M,M]$
		and $\mathrm{supp}(h) \subset [-M,M]$.
		Hence, using Rudin Theorem 6.12(a) again, we see
		\begin{align*}
			\int_{-\infty}^\infty g(x)dx
			&= \int_{-M}^M \sum_{{\substack{s,t\in\Z\\|s|\leq s',|t|\leq t'}}}
				 \frac{a_{s,t}}{N}\chi_{s,t;N}(x)dx\\
			&= \sum_{{\substack{s,t\in\Z\\|s|\leq s',|t|\leq t'}}}
			\int_{s/N}^{(s+1)/N} \frac{a_{s,t}}{N}\chi_{s,t;N}(x)dx\\
			&= \sum_{{\substack{s,t\in\Z\\|s|\leq s',|t|\leq t'}}}
			\frac{a_{s,t}}{N}\int_{s/N}^{(s+1)/N}\chi_{s,t;N}(x)dx\\
			&= \sum_{{\substack{s,t\in\Z\\|s|\leq s',|t|\leq t'}}}
			\frac{a_{s,t}}{N^2}\\
		\end{align*}
		And again, the integral converges 6.12(a). Similarly,
		\[
			\int_{-\infty}^\infty h(y)dy
			= \sum_{{\substack{s,t\in\Z\\|s|\leq s',|t|\leq t'}}}
			\frac{a_{s,t}}{N^2}
		\]
		These are equal, hence we have shown
		\[
			\int_{-\infty}^\infty g(x)dx = \int_{-\infty}^\infty h(y)dy
		\]
	\end{proof}
	\item \begin{proof}[Solution]\let\qed\relax
		We first show that $g(x)$ is integrable.
		Since $f(x,y)$ has compact support, there is some
		$M \in \N$ such that $\mathrm{supp}(f) \subset [-M,M],[-M,M]$.
		Since $f(x,y)$ continuous in a compact set,
		it is uniformly continuous on its support.
		Let $\ep > 0$.
		Then for $\ep' = \ep/(4M)$, we have that there exists $\delta > 0$
		such that for all $(x_1,y),(x_2,y) \in [-M,M] \times [-M,M]$
		where $d((x_1,y),(x_2,y)) < \delta$ (the typical metric $d$),
		we have $|f(x_1,y) - f(x_2,y)| < \ep'$. Thus by Theorem 6.13
		\[
			|g(x_1) - g(x_2)|
			= \left\lvert \int_{-M}^M f(x_1,y) - f(x_2,y)dy\right\rvert
			\leq \int_{-M}^M |f(x_1,y) - f(x_2,y)|dy
			\leq \ep'\int_{-M}^Mdy = 2M\ep' = \ep/2 < \ep
		\]
		Therefore, $g(x)$ is continuous.
		Furthermore, when $x \not\in [-M,M]$, we have
		$\int_{-\infty}^\infty f(x,y)dy = \int_{-\infty}^\infty 0dy = 0$
		and so $\mathrm{supp}(g) \subset [-M,M]$,
		so $g(x)$ has compact support as well.
		Thus, $\int_{-\infty}^\infty g(x)dx = \int_{-M}^M g(x)dx$
		exists.
		In an identical manner, one can show that $h(y)$ is integrable
		by showing that it is continuous with compact support.

		Let $a_{s,t;N}$ be the value of $f(x,y)$ at the center of $S_{s,t;N}$. Define
		\[
			f_N(x,y) = \sum_{s,t\in\Z} a_{s,t}\chi_{s,t;N}(x,y)
		\]
		
		$f(x,y)$ is uniformly continuous,
		so for any $\ep > 0$, we get a $\delta > 0$ such that
		$|f(x_1,y_1) - f(x_2,y_2)| < \ep$ for all $(x_1,y_1),(x_2,y_2)
		\in [-M,M]\times[-M,M]$
		where $d((x_1,y_1),(x_2,y_2)) < \delta$.
		Note that we can choose $N$ large enough so that
		all points in $S_{s,t;N}$ are within $\delta$ of the center
		(the furthest points at the corners are $\sqrt{2}/(2N)$ from the center,
		which we can obviously make arbitrarily small).
		Let all $N$ written below be this $N$, or larger.

		Using the definitions we provided in part (a) (for $s',t'$), we have
		(with appropriate triangle inequality and Rudin 6.13):
		\begin{align*}
			\left\lvert g(x) - \int_{-M}^M f_N(x,y)dy \right\rvert
			&= \left\lvert \int_{-M}^M f(x,y)dy - \int_{-M}^M f_N(x,y)dy \right\rvert\\
			&= \left\lvert \int_{-M}^M f(x,y) - f_N(x,y)dy \right\rvert\\
			&= \left\lvert
				\sum_{{\substack{s,t\in\Z\\|s|\leq s',|t|\leq t'}}} 
				\int_{t/N}^{(t+1)/N} f(x,y) - a_{s,t;N}dy
				\right\rvert\\
			&= \sum_{{\substack{s,t\in\Z\\|s|\leq s',|t|\leq t'}}}
			\int_{t/N}^{(t+1)/N} |f(x,y) - a_{s,t;N}|dy\\
			&\leq \sum_{{\substack{s,t\in\Z\\|s|\leq s',|t|\leq t'}}}
			\ep \int_{t/N}^{(t+1)/N}dy\\
			&< \sum_{{\substack{t\in\Z\\|t|\leq t'}}}
			\frac{\ep}{N} = \frac{T}{N}\ep\\
		\end{align*}
		where $T$ is the number of $T = 2MN$, the number of squares
		along the $y$ direction that is inside $[-M,M]$.
		$T$ is also the number of squares along the $x$ direction inside $[-M,M]$, by symmetry.
		
		In the same method, we can get
		\[
			\left\lvert h(y) - \int_{-\infty}^\infty f_N(x,y)dx \right\rvert
			< \frac{T}{N}\ep
		\]

		Since $f_N(x,y)$ is defined from part (a),
		we can interchange the order of integration to get equal integrals,
		which gives us
		\begin{align*}
			\left\lvert \int_{-\infty}^\infty g(x)dx - \int_{-\infty}^\infty h(y)dx \right\rvert
			&= \left\lvert \int_{-M}^M g(x)dx - \int_{-M}^M h(y)dx \right\rvert\\
			&\leq \left\lvert \int_{-M}^M g(x)dx -
				\int_{-M}^M \int_{-M}^M f_N(x,y) dydx \right\rvert\\
			&+ \left\lvert \int_{-M}^M h(x)dx -
				\int_{-M}^M \int_{-M}^M f_N(x,y) dxdy \right\rvert\\
			&\leq \left\lvert \int_{-M}^M \left(g(x)dx -
				\int_{-M}^M f_N(x,y) dy\right)dx \right\rvert\\
			&+ \left\lvert \int_{-M}^M \left(h(x)dx -
				\int_{-M}^M f_N(x,y) dx\right)dy \right\rvert\\
			&< \left\lvert\int_{-M}^M \frac{T}{N}\ep dx \right\rvert
			+ \left\lvert\int_{-M}^M \frac{T}{N}\ep dy \right\rvert\\
			&= 4M\frac{2MN}{N}\ep
		\end{align*}
		and since $\ep > 0$ was abitrary, and the rest are constants
		determined before, we can make this bound arbitrarily small.
		Thus, the integrals are equal, i.e.
		\[
			\int_{-\infty}^\infty g(x)dx = \int_{-\infty}^\infty h(y)dy
		\]
	\end{proof}
\end{enumerate}


\subsection*{Problem 2}
\problem{
	Let $f,g \colon \R \to \R$ be continuous and have compact support. Prove that
	\[
		\int_{-\infty}^\infty |f * g(x)|dx \leq
		\left(\int_{-\infty}^\infty |f(x)|dx\right)
		\left(\int_{-\infty}^\infty |g(x)|dx\right)
	\]
	\emph{Hint}: Problem $1$ might be useful.
}
\begin{proof}[Solution]
	(Disclaimer: sorry for the $f',g'$ notation below,
	I originally proved it for $f,g$, but realized I needed it for
	$|f|,|g|$, and so I wanted to minimize the amount of extra typing I'd need.)

	Let $|f| = f'$ and $|g| = g'$.
	Since $f,g$ are continuous with compact support,
	$f',g'$ both also have compact support and are continuous.
	We now show that $f' * g'(x)$ is continuous and has compact support as well.
	Note that $f' * g'$ has compact support.
	We have compact sets $K_f,K_g$ such that
	$\mathrm{supp}(f') \subset K_f$ and $\mathrm{supp}(g') \subset K_g$.
	Heine-Borel says that both of these sets are bounded,
	and so there exists $M \in \R$ such that $K_f,K_g \subset [-M,M]$.
	See $f'*g'(x)$ is defined for all $x \in \R$, since
	$f'*g'(x) = \int_{-\infty}^\infty f'(t)g'(x-t)dt
	= \int_{-M}^Mf'(t)g'(x-t)dt$
	(since $f'(t) = 0$ when $t \not\in [-M,M]$),
	and $f'(t)g'(x-t)$ is the product of two continuous functions,
	and so is integrable and exists on $[-M,M]$ for any $x\in\R$.
	Furthermore, when $x \not\in [-2M,2M]$,
	we have that $g'(x-t) = 0$ since $t \in [-M,M]$,
	and so $\int_{-M}^Mf'(t)g'(x-t)dt = 0$ as well
	(since $f'(t)$ is bounded since it is continuous).
	Thus, $f'*g'(x)$ has support contained in $[-2M,2M]$, and so has compact support.
	
	We also note that $f' * g'(x)$ is continuous.
	Since $g'$ is continuous on a compact set $[-2M,2M]$, it is uniformly continuous on it.
	Let $\ep > 0$. Let $\ep' = \ep\left(2\int_{-2M}^{2M} f'(t)dt\right)^{-1}$.
	Then there exists $\delta > 0$ such that
	$|g'(x) - g'(y)| < \ep'$ when $|x-y| < \delta$.
	So if $|x-y| < \delta$ (and so $|x-t - (y-t)| < \delta$,
	using Rudin Theorem 6.12 and 6.13 gives us
	\begin{align*}
		|f'*g'(x) - f'*g'(y)|
		&\leq \left\lvert\int_{-2M}^{2M}f'(t)g'(x-t)dt
		- \int_{-2M}^{2M}f'(t)g'(y-t)dt\right\rvert\\
		&= \left\lvert \int_{-2M}^{2M}f'(t)(g'(x-t) - g'(y-t))dt \right\rvert\\
		&\leq \int_{-2M}^{2M} \lvert f'(t) \rvert \lvert g'(x-t) - g'(y-t) \rvert dt\\
		&\leq \ep'\int_{-2M}^{2M} \lvert f'(t) \rvert dt\\
		&= \frac{\ep}{2} < \ep
	\end{align*}
	Hence, $f' * g'(x)$ is continuous.

	Note that since $f * g(x)$, is continuous,
	it is Riemann integrable, particularly since $f * g(x)$ is compactly supported,
	we have $f * g(x) \in \mathcal{R}[-2M,2M]$.
	Furthermore, clearly this means
	$\mathrm{supp}(|f*g(x)|) = \mathrm{supp}(f*g(x)) \subset [-2M,2M]$ as well.
	Hence, Rudin Theorem 6.13 gives us
	\[
		|f * g(x)|
		= \left\lvert\int_{-\infty}^\infty f(t)g(x-t)dt\right\rvert
		= \left\lvert\int_{-2M}^{2M} f(t)g(x-t)dt\right\rvert
		\leq \int_{-2M}^{2M} |f(t)g(x-t)|dt
		= \int_{-\infty}^\infty |f(t)||g(x-t)|dt
		= f'*g'(x)
	\]

	Now, we can invoke Rudin Theorem 6.12(b) with the inequality from above,
	and then use the result proven in Problem 1(b),
	since $f'g'(x)$ is continuous and has compact support:
	\begin{align*}
		\int_{-\infty}^\infty |f * g(x)|dx
		&\leq \int_{-\infty}^\infty f'*g'(x)dx\\
		&= \int_{-\infty}^\infty \int_{-\infty}^\infty |f(t)| |g(x-t)|dxdt\\
		&= \int_{-\infty}^\infty |f(t)| \int_{-\infty}^\infty |g(x-t)|dxdt\\
		&= \int_{-\infty}^\infty |f(t)| \int_{-\infty}^\infty |g(x)|dxdt\\
		&= \int_{-\infty}^\infty |g(x)dx \int_{-\infty}^\infty |f(t)|dt
	\end{align*}
	as desired.
	Note that above, we have used the fact that for, fixed $t$,
	we have $\int_{-\infty}^\infty |g(x-t)|dx = \int_{-\infty}^\infty |g(x)|dx$:
	since $\mathrm{supp}(|g(x)|) \subset [-M,M]$,
	we have $\int_{-\infty}^\infty |g(x-t)|dx = \int_{-M+t}^{M+t}|g(x-t)|dx
	= \int_{-M}^M |g(x)|dx = \int_{-\infty}^\infty |g(x)|dx$
	(this technically uses Rudin Theorem 6.19, where $\phi(y) = y + t$,
	but this is also trivial to see as redefining $x = x - t$,
	and the integrator function $x-t$ acts the same as $x$).
\end{proof}


\subsection*{Problem 3}
\problem{
	Let $f \colon \R \to \R$, and suppose $f$ can be uniformly approximated by polynomials.
	Prove that $f$ must be a polynomial.

	\emph{Hint}: If $P_n \to f$ uniformly, consider the sequence $P_{n+1} - P_n$.
}
\begin{proof}[Solution]
	By the Cauchy Criterion (Rudin Theorem 7.8), we have that for any $\ep > 0$,
	there exists some $N$ such that for all $m,n \geq N$ and all $x \in \R$,
	$|P_m(x) - P_n(x)| < \ep$.
	Note that $P_m - P_n$ must be a constant.
	To see this, note that the difference of polynomials is also a polynomial,
	and if the difference were not a constant, then our polynomial has some degree $l \geq 1$,
	and so there is some $a\in\R^*$ such that
	$P_m(x) - P_n(x) = ax^l + q(x)$ where $q(x)$ is the rest of the polynomial,
	and has degree less than $l$;
	then as $x \to \infty$, we know that $ax^l + q(x) \to \pm\infty$
	(the sign is the same as $a$).
	(This is a standard result, if you really want a proof:
	let $A$ be the max of the coefficients of $q(x)$,
	then for $x > 1$, $ax^l + q(x) \geq ax^l - lAx^{l-1} = (ax^l - lA)x^{l-1} \geq ax^l - nA$,
	which very clearly diverges to $\pm \infty$ as $x \to \infty$,
	so the original polynomial does as well.)
	But this contradicts that $|P_m(x) - P_n(x)| = |ax^l + q(x)| < \ep$ for all $x \in \R$.
	Thus, $P_m(x) - P_n(x)$ is a constant.

	Let $c_n = P_n(x) - P_N(x)$ where $n > N$ from before.
	We then have for $m > N$, and all $x \in \R$ that
	\[
		|c_m - c_n| = |P_m - P_N - P_n + P_N| = | P_m - P_n| < \ep
	\]
	thus $\{c_n\}$ is a real valued sequence that is Cauchy.
	Hence, since $\R$ is complete, we get that $\{c_n\}$ converges,
	say to some $c \in \R$.

	Now, for all $\ep > 0$, there is some $N' \in \N$ such that for all $n \geq N_1$,
	we have $|c_n - c| < \ep/2$.
	Also, since $P_n \to f$ uniformly, we have that there exists some $N_2 \in \N$
	such that for all $n \geq N_2$, we have $|f(x) - P_n| < \ep/2$
	for all $x \in \R$.
	So when $n \geq \max\{N_1,N_2\}$, we have that for all $x \in \R$,
	\[
		|f(x) - P_N(x) - c |
		\leq |f(x) - P_n(x)| + |P_n(x) - P_N(x) - c|
		< \frac{\ep}{2} + |c_n - c|
		< \frac{\ep}{2} + \frac{\ep}{2} = \ep
	\]
	Hence, we get that $f(x) - P_N(x) - c = 0$ for all $x \in \R$,
	thus $f(x) = P_N(x) + c$, which is a polynomial,
	so $f(x)$ is a polynomial.
\end{proof}


For the next problem, we need the following definition.
Let $(X,d)$ be a metric space and let $\alpha > 0$.
We say that a function $f \in \mathcal{C}(X)$ is
``H\"{o}lder's continuous of exponent $\alpha$" if the quantity
\[
	N_\alpha(f) = \sup_{x \neq y}\frac{|f(x) - f(y)|}{d(x,y)^{\alpha}}
\]
is finite.
\subsection*{Problem 4}
\problem{
	\begin{enumerate}
		\item Prove that if $X$ is compact then
			$\{f \in \mathcal{C}(X) \colon \lVert f \rVert \leq 1
			\text{ and } N_\alpha(f) \leq 1\}$
			is a compact subset of $\mathcal{C}(X)$.
		\item Prove that $\{f \in C([0,1]) \colon \lVert f \rVert \leq 1\}$
			is \emph{not} a compact subset of $\mathcal{C}([0,1])$.
	\end{enumerate}
}
\begin{enumerate}
	\item \begin{proof}[Solution]\let\qed\relax
		Recall from Problem $3$ of Homework $7$ that it is sufficient to show that
		$\mathcal{F} = \{f \in \mathcal{C}(X) \colon \lvert f \rvert \leq 1
		\text{ and } N_\alpha(f) \leq 1 \}$ is closed, bounded, and equicontinuous.

		We have that $\mathcal{F}$ is bounded,
		since $0 \in \mathcal{F}$ and for all $f \in \mathcal{F}$,
		we have $d(0,f) \leq 1 < 2$, since $\lVert f \rVert \leq 1$.

		We show now that $\mathcal{F}$ is equicontinuous.
		Note that for $f \in \mathcal{F}$, since $N_\alpha(f) \leq 1$,
		\[
			|f(x) - f(y)| \leq N_\alpha(f)d(x,y)^\alpha \leq d(x,y)^\alpha
		\]
		Let $\ep > 0$. Then we can choose $\delta = \ep^{1/\alpha}$ to get that
		for any $f \in \mathcal{F}$ that when $d(x,y) < \delta$, we have
		\[
			|f(x) - f(y)| < \delta^\alpha = \ep
		\]
		hence $\mathcal{F}$ is equicontinuous.

		We now show that $\mathcal{F}$ is closed.
		That is, every limit point of $\mathcal{F}$ is also in $\mathcal{F}$.
		That is, for any $f$ such that there is some sequence
		$f_n \in \mathcal{F}$ where $f_n \to f$ uniformly (the metric of $C(\mathcal{X})$),
		then $f \in \mathcal{F}$ as well.
		Since $\mathcal{C}(X)$ with this metric
		is a complete metric space (Rudin Theorem 7.15),
		we have $f \in \mathcal{C}(X)$.
		Also, for all $\ep > 0$, there exists some $N_1 \in \N$ such that for all $n \geq N_1$,
		we have $\lVert f - f_n \rVert < \ep$. Thus
		\[
			\lVert f \rVert \leq \lVert f - f_n \rVert + \lVert f_n \rVert
			\leq 1 + \ep
		\]
		But since $\ep$ can be made arbitrarily small, we get the nonstrict inequality
		$\lVert f \rVert \leq 1$ as well.
		Finally, there exists $N_2 \in \N$ such that for all $n \geq N_2$,
		we have $\lVert f - f_n \rVert < \ep/2$,
		and for any $x,y \in X$ where $x \neq y$, we have
		\[
			|f(x) - f(y)|
			\leq |f(x) - f_n(x)| + |f_n(x) - f_n(y)| + |f_n(y) - f(y)|
			\leq 2\lVert f - f_n\rVert + d(x,y)^\alpha
			\leq \ep + d(x,y)^\alpha
		\]
		Since $\ep$ can be made arbitrarily small,
		we get the nonstrict inequality $|f(x) - f(y)| \leq d(x,y)^\alpha$.
		$d(x,y) \neq 0$, so we get $N_\alpha(f) = \frac{|f(x) - f(y)|}{d(x,y)^\alpha} \leq 1$.
		Thus, this verifies all the conditions on $\mathcal{F}$,
		thus $f \in \mathcal{F}$, and so $\mathcal{F}$ is closed.
		
		Finally, this shows that $\mathcal{F}$ is a compact subset of $\mathcal{C}(X)$.
	\end{proof}
	\item \begin{proof}[Solution]\let\qed\relax
		Define for $a \in \R^+$.
		\[
			f_a = \begin{cases}
				ax & x \in [0,1/a]\\
				1 & x \in (1/a,1]
			\end{cases}
		\]
		Note that for any $a$, $f_a \in \mathcal{C}(X)$.
		When $x \in [0,1/a)$ and $x \in (1/a,1]$, $f_a$ are functions which are
		known to be continuous.
		If $x = 1/a$, then $\lim_{x\nearrow \frac1a} f_a(x) = 1 = \lim_{x\searrow \frac1a}f_a(x)$,
		so it is continuous at $\frac{1}{a}$,
		and so continuous everywhere.
		Also, clearly $\lVert f_a \rVert = 1$,
		hence $f_a \in \{f \in \mathcal{C}([0,1]) \colon \lVert f \rVert \leq 1\}$
		for all $a \in \R^+$.

		We now show that $\{f \in \mathcal{C}([0,1]) \colon \lVert f \rVert \leq 1\}$
		is not compact.
		We do this by showing that the set is not equicontinuous,
		and so by Problem 3 of Homework 7, the set is not compact in $\mathcal{C}([0,1])$.
		Let $\ep = 1$. Then for all $\delta > 0$,
		let $x = \delta/2, y = 0$ so $d(x,y) < \delta$, and $f = f_{2/\delta}$, so we get
		\[
			|f(x) - f(y)| = |f_{2/\delta}(\delta/2)| = 1 \geq \ep
		\]
		This shows that $\{f \in \mathcal{C}([0,1]) \colon \lVert f \rVert \leq 1\}$
		is not equicontinuous, and hence not compact.
	\end{proof}
\end{enumerate}
\end{document}
