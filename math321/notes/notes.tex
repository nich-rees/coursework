\documentclass{article}
\usepackage{amsmath, amsfonts, amsthm, amssymb}
\usepackage{geometry}
\geometry{letterpaper, margin=2.0cm, includefoot, footskip=30pt}

\usepackage{fancyhdr}
\pagestyle{fancy}

\lhead{Math 321}
\chead{Notes}
\rhead{Nicholas Rees}
\cfoot{Page \thepage}

\newtheorem*{problem}{Problem}
\theoremstyle{plain}
\newtheorem{theorem}{Theorem}
\newtheorem{lemma}{Lemma}
\newtheorem{proposition}{Proposition}
\newtheorem{corollary}{Corollary}
\theoremstyle{remark}
\newtheorem{definition}{Definition}
\newtheorem{remark}{Remark}

\newcommand{\N}{{\mathbb N}}
\newcommand{\Z}{{\mathbb Z}}
\newcommand{\Q}{{\mathbb Q}}
\newcommand{\R}{{\mathbb R}}
\newcommand{\C}{{\mathbb C}}
\newcommand{\ep}{{\varepsilon}}
\newcommand{\SR}{{\mathcal R}}

\renewcommand{\theenumi}{(\alph{enumi})}

\begin{document}
\section{January 8}
``Sometimes MVT stands for `most-valuable theorem'."
\subsection{Logistics}
\begin{itemize}
	\item Homework: Due on Fridays at the start of class (Canvas), posted on Thursday or Friday.
		List name of people you worked with at the top
		\begin{itemize}
			\item Homework 0 is due this Friday, not due marks,
				but he is using to test automated test system to hand back
			\item Some help: watch Monty Python Holy Grail
		\end{itemize}
	\item Office hours: Zahl Wed. 1-2; TA TBD; TA2 TBD
	\item Weighting:
		\begin{itemize}
			\item HW 30\%
			\item MT 30\% Feb 14
			\item Final 40\%
		\end{itemize}
\end{itemize}

\subsection*{320 Addendum}
Typically, 321 picks up at integration after finishing with differentiation in 320.
But we will pick up some missed material at the end of 320.

Recall
\begin{definition}
	$f \colon [a,b] \to \R$, $c \in [a,b]$, we say that $f$
	is \emph{differentiable at $c$} if
	$\lim_{x \to c} \frac{f(x) - f(c)}{x-c}$ exists (as a real number).
	We denote this by $f'(c)$.
\end{definition}
This is nice, but could even do in high school.
But we can go up many levels of abstraction with our limit.
\begin{itemize}
	\item $c$ is a limit point in $[a,b]$
		(for every $\ep$, a point not $c$ exists inside the open ball)
	\item $g(x) = \frac{f(x) - f(c)}{x-c}$ is a function with
		domain $[a,b] \setminus \{c\}$ (thankfully, $c$ is still a limit point of this).
	\item If $c \in (a,b)$, the high school definition of the limit works.
		If $c = a,b$, then one-sided limit.
\end{itemize}
\begin{definition}
	If $f \colon [a,b] \to \R$ is differentiable at every point $c \in [a,b]$,
	then we say that $f$ is \emph{differentiable on $[a,b]$}
	and this gives us a new function $f' \colon [a,b] \to \R$.
\end{definition}
\begin{definition}
	If $f'$ is differentiable at $c \in [a,b]$, write $f''(c) = (f')'(c)$.
	Alternate notations:
	\[
		\begin{matrix}
			f(c), & f'(c), & f''(c), & f'''(c)\\
			f^{(0)}(c), & f^{(1)}(c), & f^{(2)}(c), & f^{(3)}(c), &
			\dots f^{(k)}(c) = \frac{d^k}{dx^k} f(x) |_{x=c}
		\end{matrix}
	\]
\end{definition}

Some questions to consider:
\begin{itemize}
	\item Why have codomain $\R$? Why not $\C$? Field $F$? General set / metric space?
	\item Why make domain a closed interval?? More general subset of $\R$? $\C$? Set / metric space?
\end{itemize}
The derivative is one of the most important concepts,
and so makes sense people have thought about making it more general.
Sometimes it works, sometimes it doesn't.

We've used the field structure of $\R$ in an important way
(not necessarily the order structure)...
more than just a metric space.
Seems ambitiuous to have a topological definition of a derivative,
because a derivative is a quantitative rate of change,
and we don't get that in a topology.
You can probably find people who have constructed a topological derivative,
but will have needed to give up desired properties.

\subsection{Taylor's Theorem}
Recall the special case to MVT that is used to prove MVT:
\begin{theorem}[Roll'es Theorem]
	Let $f \colon [a,b] \to \R$ be differentiable, with $f(a) = f(b)$.
	Then $\exists c \in (a,b)$ such that $f'(c) = 0$.
\end{theorem}
If you don't remember how to prove this, good exercise to go through
that uses a lot of material from Math 320.

We will use this to prove Taylor's theorem,
which seems really strong and handles most of our everyday functions,
but easy to step on landmines
(slightly rewording a true statement gives a really wrong one).
We are going to give sufficient hypotheses, could technically weaken, but not as clear.
\begin{theorem}[Taylor's Theorem (5.15 in Rudin)]
	Let $f \colon [a,b] \to \R$, let $n \geq 0$ be an integer.
	Suppose that $f$ is $(n+1)$ times differentiable on $[a,b]$.
	Let $x_0$ and $x$ be points in $[a,b]$ with $x_0 \neq x$.
	Then there exists a point $c$ strictly between $x_0$ and $x$ such that
	\begin{equation}
		f(x) = \sum_{k = 0}^n \frac{f^{(k)}(x_0)}{k!}(x-x_0)^k
		+ \frac{f^{(n+1)}(c)}{(n+1)!}(x-x_0)^{n+1}
	\end{equation}
\end{theorem}
We hope the error term is small so we can control it.
We won't prove this today because of the time.
But we will say $P_n(x) = \sum_{k = 0}^n \frac{f^{(k)}(x_0)}{k!}(x-x_0)^k$
is the ``degree n Taylor expansion of $f$ around $x_0$".
When you are choosing notation, there are competing goals:
don't want it to be flowery so that it becomes more complicated $P_n^{f,x_0}(x)$,
but will have to remember what we are hiding.
And also because what is of interest might be when we fix $f,x_0$,
and sending $n$ to infinity.

Is this equation helpful? Will depend on how small the ``error term" gets (could dominate!).
But polynomial if $x$ is close to $x_0$ is going to zero geometrically,
and factorial is going to zero faster than geometric,
so actually for most functions, we get to say something nice.

Question: let $f \colon \R \to \R$, infinitely differentiable.
Suppose $f^{(k)}(0) = 0$.
Is it true that $f$ must be the zero function?
Because then the function is just the error term, and so Taylor's theorem fails in the worst way.
\end{document}
