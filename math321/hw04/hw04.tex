\documentclass{article}
\usepackage{amsmath, amsfonts, amsthm, amssymb}
\usepackage{geometry}
\geometry{letterpaper, margin=2.0cm, includefoot, footskip=30pt}

\usepackage{fancyhdr}
\pagestyle{fancy}

\lhead{Math 321}
\chead{Homework 4}
\rhead{Nicholas Rees, 11848363}
\cfoot{Page \thepage}

\newcommand{\N}{{\mathbb N}}
\newcommand{\Z}{{\mathbb Z}}
\newcommand{\Q}{{\mathbb Q}}
\newcommand{\R}{{\mathbb R}}
\newcommand{\C}{{\mathbb C}}
\newcommand{\ep}{{\varepsilon}}

\newcommand{\problem}[1]{
	\begin{center}\fbox{
		\begin{minipage}{17.0 cm}
			\setlength{\parindent}{1.5em}
			{\it \noindent#1}
		\end{minipage}}
	\end{center}}

\newtheorem{lemma}{Lemma}
\theoremstyle{remark}
\newtheorem{remark}{Remark}

\renewcommand{\theenumi}{(\alph{enumi})}

\begin{document}
\begin{center}
	{\bf Math 321 Homework 3}\\
	(Including work made in collaboration with Tighe McAsey.)
\end{center}

\subsection*{Problem 1}
\problem{
	Let $f \in \mathcal{R}[a,b]$ and $0 < p < \infty$. Define
	\[
		\lVert f \rVert_p = \left( \int_a^b \lvert f \rvert^pdx\right)^{1/p}
	\]
	\begin{enumerate}
		\item Prove that for $0 < p < \infty$, $\lvert f \rvert^p \in \mathcal{R}[a,b]$
			(and hence the above definition makes sense).
		\item If $f$ is continuous, prove that
			\[
				\lim_{p \to \infty} \lVert f \rVert_p =
				\sup\{\lvert f(x) \rvert \colon x \in [a,b]\}.
			\]
		\item For $f$ fixed, define $\phi(p) = \lVert f \rVert_p^p$.
			Using Rudin Problem 6.10, prove that $p \mapsto \log{\phi(p)}$
			is convex on $(0 , \infty)$
			(recall Rudin problem 4.23 for the definition of convexity, and its consequences).
			Do not submit the proof of Rudin Problem 6.10
			(but I encourage you to do it; it is a good exercise).
	\end{enumerate}
	\begin{remark}
		Since convex functions are continuous (see Rudin Problem 4.23),
		you have just shown that $\phi$ and hence $p \mapsto \lVert f \rVert_p$ are continuous.
	\end{remark}
}
\begin{enumerate}
	\item \begin{proof}[Solution]\let\qed\relax
		By Rudin Theorem 6.13(b), since $f \in \mathcal{R}[a,b]$,
		we have $|f| \in \mathcal{R}[a,b]$.
		Then, since $p$ is finite,
		by applying Rudin Theorem 6.13(a) $p$ times, we get that
		$|f|^p \in \mathcal{R}[a,b]$ as well.
	\end{proof}
	\item \begin{proof}[Solution]\let\qed\relax
		Note that since $f$ is continuous, so is $|f|$
		since composition of continuous functions is also continuous.
		Similarly, $|f|^p$ is also continuous.
		Furthermore, the function is defined on the closed and bounded set $[a,b]$,
		which in $\R$ is compact,
		so $|f|$ attains its maximum value $M$ on $[a,b]$, say at point $e\in[a,b]$.
		So $|f| \leq M$ on $[a,b]$.
		Since $f(x) = x^p$ is a monotonically (strictly) increasing
		function on $[0,\infty)$,
		and $0 \leq |f|$,
		our inequality is preserved if we raise it to $p$, so
		$|f|^p \leq M^p$ on $[a,b]$.
		We then have $\int_a^b |f|^pdx \leq M^p(b-a)$ by Rudin Theorem 6.12(d)
		(and part (a), which says $|f|^p \in \mathcal{R}[a,b]$).
		Since $f(x) = x^{1/p}$ is a monotonically (strictly) increasing
		function on $[0,\infty)$,
		and $0 \leq |f|^p \implies 0 \leq \int_a^b |f|^pdx$,
		our inequality is preserved if we raise it to $1/p$, so
		\[
			\int_a^b |f|^pdx \leq M(b-a)^{1/p}
		\]
		Now when we take the limit $p \to \infty$,
		since limits preserve non-strict inequalities, we get
		\[
			\lim_{p\to\infty} \int_a^b |f|^pdx \leq M\lim_{p\to\infty} (b-a)^{1/p}
			= M
		\]
		where we use the fact $b - a > 0$
		(otherwise integral is $0$ and this fact actuall fails)
		and Rudin Theorem 3.20(b): $\lim_{n\to\infty} \sqrt[n]{p} = 1$ when $p > 0$.
		So $\lim_{p \to \infty} \lVert f \rVert_p$ has an upper bound,
		specifically, $M = \sup\{|f(x)| \colon x \in [a,b]\}$.
		So either $\lim_{p \to \infty} \lVert f \rVert_p = M$,
		or there exists some $L < M$ such that $\lim_{p \to \infty} \lVert f \rVert_p$.
		We proceed with showing such an $L$ cannot exist.

		We prove that for all $\ep > 0$, there exists some $N \in \N$
		such that for all $p \geq N$, we have $\lVert f \rVert_p \geq M - \ep$.
		Let $\ep > 0$.
		We may assume that $M - \ep > 0$,
		otherwise $\lVert f \rVert_p > 0 \geq M - \ep$ for all $p\in\N$, and we are done.
		Let $c,d$ be the points $a \leq c < d \leq b$
		defined as follows:
		\begin{itemize}
			\item $c$: If $f(x) \geq M - \ep/2$ for all $x \in [a,e]$,
				then let $c = a$.
				Otherwise, there is some $y \in [a,e]$ such that
				$f(y) < M - \ep/2$, and so by intermediate value theorem,
				since $f$ is continuous,
				there is some point $c' \in (y,e)$ such that $f(c') = M - \ep/2$.
				We let $c$ be the rightmost point, i.e.,
				$f(x) \in (M - \ep/2, M)$ when $x \in (c,e)$
				(and we can pick such a $c$,
				otherwise, if $f(x) = M - \ep/2$ arbitrarily close to $e$,
				we would break continuity since $f(e) = M$).
			\item $d$: Identical as above,
				i.e. $d = b$ when $f(x) \geq M - \ep/2$ for all $x \in [e,b]$;
				otherwise, $d$ is the leftmost point so that $f(d) = M - \ep/2$.
		\end{itemize}
		Note that we get the strict inequality on account of $c < e < d$.

		Since $|f|^p$ is positive, $|f|^p \geq |f_{[c,d]}|^p$
		where $f_{[c,d]} =
		\begin{cases} f(x) & x \in [c,d]\\ 0 & \text{otherwise}\end{cases}$,
		and $\int_c^d |f|^pdx = \int_a^b |f_{[c,d]}|^pdx$, so we have
		\[
			\int_a^b |f|^pdx \geq \int_a^b |f_{[c,d]}|^p =
			\int_c^d |f|^pdx \geq (M - \ep)^p(d - c) > 0
		\]
		Raising it to $1/p$ (which doesn't change inequalities) gives
		\[
			\lVert f \rVert_p \geq (M-\ep)(d-c)^{1/p}
		\]
		If $(d-c) \geq 1$, then we have $\lVert f \rVert_p \geq M - \ep$
		for all $p \in \N$
		(since $(d-c)^{1/p} \geq 1$ for all $p \in \N$),
		and we are done.
		Now, assume $(d-c) < 1$.
		Since $(d-c) > 0$, Rudin 3.20(b) gives the limit
		$\lim_{p\to\infty} \sqrt[p]{d-c} = 1$.
		Thus, there exists some $N \in \N$ such that for all $p \geq N$,
		we have $0 < 1 - (d-c)^{1/p} < \ep/(2M)$.
		Rearranging gives $(d-c)^{1/p} > 1 - \ep/(2M)$.
		Then for all $p \geq N$,
		\[
			\lVert f \rVert_p > (M - \ep/2)(1 - \ep/(2M)) =
			M -M(\ep/(2M)) - \ep/2 + \ep^2/(4M)
			\geq M - \ep + \ep^2/(4M)
			> M - \ep
		\]
		which is what we wanted to show.

		Hence, if $\lim_{p \to \infty} \lVert f \rVert_p = L<M$,
		there exists $N \in \N$ such that for all $p \geq N$,
		$\lVert f \rVert_p > M - (M-L)/2 > M - (M-L) = L$,
		which is a contradiction.
		This only leaves $\lim_{p \to \infty}\lVert f \rVert_p = M
		= \sup\{|f(x)| \colon x \in [a,b]\}$.
	\end{proof}
	\item \begin{proof}[Solution]\let\qed\relax
		Let $\lambda \in (0,1)$ and let $p,p' \in (0,\infty)$.
		Since $\lambda + (1 - \lambda) = 1$,
		and $|f|^{\lambda p}, |f|^{(1-\lambda)p'} \in \mathcal{R}[a,b]$ from part (a),
		we can use H\"{o}lder's inequality:
		\begin{align*}
			\log\phi(\lambda p + (1-\lambda)p')
			&= \log\left(\int_a^b|f|^{\lambda p + (1-\lambda) p'}dx\right)\\
			&= \log\left(\int_a^b(|f|^p)^\lambda(|f|^{p'})^{1-\lambda}dx\right)\\
			&\leq \log\left\lvert\int_a^b(|f|^p)^\lambda(|f|^{p'})^{1-\lambda}dx\right\rvert\\
			&\leq \log\left(
				\left(\int_a^b\left\lvert(|f|^p)^\lambda\right\rvert^{1/\lambda}
				dx\right)^\lambda
				\left(\int_a^b\left\lvert(|f|^{p'})^{1-\lambda}\right\rvert^{1/(1-\lambda)}
				dx\right)^{1-\lambda}
			\right)\\
			&= \log\left(\left(\int_a^b|f|^pdx\right)^\lambda
			\left(\int_a^b|f|^{p'}dx\right)^{1-\lambda}\right)\\
			&= \lambda\log\left(\int_a^b |f|^pdx\right)
			+ (1 - \lambda)\log\left(\int_a^b |f|^{p'}dx\right)\\
			&= \lambda \log\phi(p) + (1-\lambda)\log\phi(p')
		\end{align*}
		where we are using the fact that since $\log$ is a monotone increasing function,
		it preserves inequalities
		(and also eliminating some $|\cdot|$
		since $|f| > 0 \implies |f|^x > 0$).
	\end{proof}
\end{enumerate}


\subsection*{Problem 2}
\problem{
	Let $\{f_n\}$ and $\{g_n\}$ be sequences of functions from $\R \to \R$
	that converge pointwise.
	Must it be true that $\{f_n \circ g_n\}$ converges pointwise?
	If so, prove it.
	If not, give a counter-example and prove that your counter-example is correct.
}
\begin{proof}[Solution]\let\qed\relax
	It can be false.
	We provide the counter-example:
	define $g_n(x)$ on $0 < x \leq \frac{1}{n}$ as $g_n(x) = x$,
	and we periodically extend this function off of $(0,\frac{1}{n}]$
	so that $g_n(x+k) = g_n(x)$ for any $k \in \Z$;
	now let
	\[
		f_n(x) = \begin{cases} n, & 0 < x \leq \frac{1}{n} \\ 0, & \text{otherwise}\end{cases}
	\]
	as well.

	Clearly, both of these are functions from $\R \to \R$.
	Furthermore, we get that both of these sequences of functions
	converge pointwise to the $0$ function.
	To see this for $g$, note that $g_n$ attains its maximum value at $x = \frac{1}{n}$,
	since on $x \in (0,\frac{1}{n}]$, $g_n$ is monotone increasing and
	this is the right most value,
	and since $g_n$ is periodic, the largest value this function attains
	is the same as the largest value it attains on this interval.
	Furthermore, $g_n(\frac{1}{n}) = \frac{1}{n}$.
	Let $\ep > 0$ and $x \in \R$ be fixed,
	then Archimedean gives us some $N \in \N$ such that $\frac{1}{N} < \ep$,
	and so for any $n \geq N$, we have
	$\lvert g_n(x) - 0\rvert = g_n(x) \leq \frac{1}{n} \leq \frac{1}{N} < \ep$,
	which actually uniformly bounds $g_n$,
	and so we must have $f_n$ is pointwise convergent to $0$ for all $x \in \R$.

	To see this for $f$, fix some $\ep > 0$ and some $x \in \R$.
	Archimedean gives us some $N \in \N$ such that $\frac{1}{N} < x$.
	By the definition of $f_n$, when $n \geq N$,
	we have that $|f_n(x) - 0| = f_n(x) = 0 < \ep$
	(since $x > \frac{1}{N} \geq \frac{1}{n}$).
	Hence, $f_n$ is pointwise convergent to $0$ for all $x \in \R$.

	Now let us consider the composition, $f_n \circ g_n$.
	If $x \in \R$, then $g_n(x)$ maps $x$ to some value in $(0,\frac{1}{n})$,
	which then $f_n(x)$ would map to $n$.
	Hence, $f_n \circ g_n = n$.
	As $n \to \infty$, it is clear that for every $x$,
	this function diverges to infinity, and so obviously does not converge pointwise.
\end{proof}


\subsection*{Problem 3}
\problem{
	Let $E$ be a set and let $(M_1,d_1),(M_2,d_2)$ be metric spaces with
	the discrete metric (i.e. $d(x,y) = 0$ if $x= y$, and $d(x,y)=1$ if $x \neq y$).
	Let $\{g_n\}$ be a sequence of functions from $E \to M_1$,
	and let $\{f_n\}$ be a sequence of functions from $M_1 \to M_2$.
	Suppose that $\{f_n\}$ and $\{g_n\}$ converge pointwise.
	Must it be true that $\{f_n \circ g_n\}$ converges pointwise?
	If so, prove it.
	If not, give a counter-example and prove that your counter-example is correct.
}
\begin{proof}[Solution]\let\qed\relax
	We claim that this is true,
	specifically if $g \colon E \to M_1$ and $f \colon M_1 \to M_2$
	are functions such that $g_n \to g$ and $f_n \to f$ pointwise,
	$f_n \circ g_n \to f \circ g$ pointwise as well.

	Let $\ep > 0$ and $x \in E$.
	Since $g_n \to g$, for all $\ep' > 0$,
	there exists some $N_1$ such that $d(g_n(x),g(x)) < \ep'$
	for all $n > N_1$.
	If we let $\ep' = \frac{1}{2}$,
	since this is the discrete metric so
	$d(g_n(x),g(x))$ can only either be $1$ or $0$,
	this tells us that for all $n \geq N_1$, $d(g_n(x),g(x)) = 0$,
	i.e. $g_n(x) = g(x)$.
	
	Note $g(x) \in M_1$.
	Since $f_n \to f$, for all $\ep' > 0$,
	there exists some $N_2$ such that $d(f_n(g(x)), f(g(x))) < \ep'$
	for all $n > N_2$.
	If we let $\ep' = \frac{1}{2}$,
	since this is the discrete metric so
	$d(f_n(g(x)), f(g(x))$ can only either be $1$ or $0$,
	this tells us that for all $n \geq N_2$, $d(f_n(g(x)), f(g(x))) = 0$,
	i.e. $f_n(g(x)) = f(g(x))$.

	We now consider $\{f_n \circ g_n\}$.
	Let $N = \max\{N_1,N_2\}$.
	Then for all $n \geq N$, $f_n(g_n(x)) = f_n(g(x)) = f(g(x))$.
	Thus, $d(f_n(g_n(x)), f(g(x))) = d(f(g(x)), f(g(x))) = 0 < \ep$.

	$\ep,x$ were arbitrary,
	hence $\{f_n \circ g_n\}$ converges pointwise.
\end{proof}


\subsection*{Problem 4}
\problem{
	Let $\{f_n\}$ be a sequence of functions in $\mathcal{R}[a,b]$,
	let $f \in \mathcal{R}[a,b]$, let $f_n \to f$ pointwise,
	and suppose that $\{f_n(x)\}$ is monotone increasing for each $x \in [a,b]$.
	Prove that
	\[
		\lim_{n\to\infty} \int_a^b f_n(x)dx = \int_a^b f(x)dx
	\]
}
\begin{proof}[Solution]\let\qed\relax
	First, some definitions.
	If $P = \{x_0,\dots,x_n\}$ is a partition,
	define $D(P) = \max_i\{x_i - x_{i-1}\}$.
	Secondly, $|I|$ will denote the length of an interval
	(either open or closed or neither, e.g. $|(a,b]| = b - a$).

	We now collect some useful facts.
	\begin{lemma}
		If $f \in \mathcal{R}[a,b]$ and $\ep > 0$,
		then there exists some $\delta > 0$ such that
		if $P$ is some partition of $[a,b]$ where $D(P) < \delta$
		then $U(P,f) - L(P,f) < \ep$.
	\end{lemma}
	\begin{proof}
		Let $f \in \mathcal{R}[a,b]$, $\ep > 0$ be given.
		There must exist a partition, call it $P^*$ such that
		$U(P^*,f) - L(P^*,f) < \ep/3$.
		We claim $\delta = \min_i\{x_i - x_{i-1}\}$ is the value we want.
		Let $P$ be a partition of $[a,b]$ such that $D(P) < \delta$.
		Then an interval in $P$ is in at most two intervals in $P^*$.
		For all of the intervals of $P$ entirely contained within
		one interval of $P^*$, call it $[x_{i-1},x_i]$,
		their upper Riemann sum is bounded above by $M_i(x_i - x_{i-1})$
		(and likewise with their lower).
		For the intervals $[y_{i-1},y_i]$ of $P$ that are between two intervals of $P^*$
		(and there are at most $\#P^*$ of them, since there are only
		those many boundaries between intervals),
		say $[x_{i-2},x_{i-1}],[x_{i-1},x_i]$,
		then $\sup\{f(x) \colon x \in [y_{i-1},y_i]\}(y_{i} - y_{i-1})$
		is less than $M_{i-1}(x_{i-1}-x_{i-2}) + M_i(x_i - x_{i-1})$,
		since $\sup\{f(x) \colon x \in [y_{i-1},y_i]\} \leq \max\{M_{i-1},M_i\}$,
		and $y_{i} - y_{i-1} < \delta \leq (x_{i-1}-x_{i-2}),(x_i - x_{i-1})$
		(and likewise with the lower sum and infinum).
		Since there are $\#P^*$ of these points,
		an upper bound on the upper Riemann sum of these points
		is twice $U(P^*,f)$.
		Hence,
		\[
			U(P,f) - L(P,f) \leq 3U(P^*,f) - 3L(P^*,f) = 3(U(P^*,f) - L(P^*,f)) < \ep
		\]
	\end{proof}

	\begin{lemma}
		Let $f \in \mathcal{R}[a,b]$ be nonnegative, $\ep > 0$,
		and $\delta$ given by Lemma $1$ from $f$ and $\ep$.
		For any $a \leq u < v \leq b$ where $0 < (v-u) < \delta$
		and $s \in [u,v]$, we have
		\[
			\int_u^v fdx < \ep + f(s)(v-u)
		\]
	\end{lemma}
	\begin{proof}
		By Lemma 1, we have that $U(P,f) - L(P,f) < \ep$.
		Let $P^* = P \cup \{u,v\}$, a refinement of $P$.
		Hence, $U(P^*,f) - L(P^*,f) < \ep$ by Rudin Theorem 6.7.
		If $u$ and $v$ are entirely contained within the same interval of $P$,
		i.e. some $i$ such that $x_{i-1} \leq u < v \leq x_i$, then
		we have $\int_u^v fdx$ is less than some value in the upper sum,
		namely $M_{[u,v]}(v-u)$,
		and likewise, $f(s)(v-u)$ is greater than some value in the lower sum,
		namely $m_{[u,v]}(v-u)$.
		Hence, since all the terms are nonnegative since $f$ is nonnegative
		\[
			\int_u^v fdx - f(s)(v-u) \leq U(P^*,f) - L(P^*,f) < \ep
		\]
		The case where there is a a point from $P$ in between is handled
		identically by just removing that point:
		since we still have $D(P^*) < \delta$,
		we still keep our $\ep$ bound.
	\end{proof}

	Let $\ep > 0$, and define $\ep' = \frac{\ep}{4(b-a)+1}$.
	Define $g_n = f - f_n$.
	Clearly, $g_n$ is a nonnegative, monotone decreasing in $n$,
	and $g_n \to 0$ pointwise, by the properties of $f_n$.
	We seek to show that there exists some $K \in \N$ such that
	for all $n \geq K$, $\int_a^b g_n(x)dx < \ep$.

	Invoking Lemma 1, we define $\delta_n > 0$ such that
	for any partition $P$ of $[a,b]$ where $D(P) < \delta_n$,
	we have $U(P,g_n) - L(P,g_n) < 2^{-n}\ep'$.
	Define $k(s)$ be the least $k$ such that $g_k(s) < \ep'$
	for any $s \in [a,b]$,
	which must exist since $g_n \to g$ pointwise.
	Now define $I_s = (s - \delta_{k(s)}/2, s + \delta_{k(s)}/2)$.
	For all $s \in [a,b]$, $s \in I_s$,
	so $\{I_s\}_{s \in [a,b]}$ is an open cover of $[a,b]$.
	Since this interval is closed and bounded, it is compact in $\R$,
	so we can extract a finite subcover
	$I_{s_1},\dots,I_{s_N}$,
	particularly one that does not have any redudancy
	(i.e. $I_{s_i} \not\subset I_{s_j}$ when $i \neq j$),
	which we can do because removing a redundant open set doesn't
	change the union of the sets.
	Define $J_{s_i} = \overline{I_{s_i}} \cap [a,b]$,
	hence $\bigcup_{i=1}^N J_{s_i} = [a,b]$.

	Finally, for all $n \geq K = \max\{k(s_1),\dots,k(s_N)\}$, we have
	\begin{align}
		\int_a^b g_n(x)dx
		&\leq \sum_{i=1}^N \int_{J_{s_i}} g_n(x)dx\\
		&\leq \sum_{i=1}^N \int_{J_{s_i}} g_K(x)dx\\
		&\leq \sum_{i=1}^N \int_{J_{s_i}} g_{k(s_i)}(x)dx\\
		&\leq \sum_{i=1}^N \left(\ep'2^{-k(s_i)} + g_{k(s_i)}(s_i)|J_{s_i}|\right)\\
		&\leq \ep'\sum_{i=1}^N \left(2^{-k(s_i)} + |J_{s_i}|\right)\\
		&\leq \ep'(1 + 2(b-a)) = \ep
	\end{align}
	Where (1) is done by splitting up $[a,b]$ centered at the $s_i$ by Rudin Theorem 6.12(c),
	and then expanding the surrounding interval so that we are integrating
	over all of $J_{s_i}$ only increases the value since $g_n$ is nonnegative;
	(2) is because $g_n$ is monotone decreasing;
	(3) is by definition of $K$ and monotone decreasing;
	(4) is by Lemma 2;
	(5) is by definition of $k(s_i)$ so that $g_{k(s_i)}(s_i) < \ep'$; and
	(6) since the at any point in $[a,b]$, at most two intervals $J_{s_i}$ overlap,
	otherwise we would have a nested interval which we constructed to avoid,
	and so since the $J_{s_i}$ cover $[a,b]$, at most each $[a,b]$
	is inside two intervals, so $\sum_{i=1}^N |J_{s_i}| \leq 2(b-a)$;
	the dyadic terms are bounded above by $1$ by just taking our $k$ large enough, which we were allowed to do.

	Therefore, we have that for all $n \geq K$,
	$\int_a^b g_n(x)dx = \int_a^b (f(x) - f_n(x))dx
	= \int_a^b f(x)dx - \int_a^b f_ndx < \ep$,
	hence, $\lim_{n\to\infty} \int_a^b f_ndx = \int_a^b f(x)dx$.
\end{proof}
\end{document}
