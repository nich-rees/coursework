\documentclass{article}
\usepackage{amsmath, amsfonts, amsthm, amssymb}
\usepackage{geometry}
\geometry{letterpaper, margin=2.0cm, includefoot, footskip=30pt}

\usepackage{fancyhdr}
\pagestyle{fancy}

\lhead{Math 321}
\chead{Homework 4}
\rhead{Nicholas Rees, 11848363}
\cfoot{Page \thepage}

\newcommand{\N}{{\mathbb N}}
\newcommand{\Z}{{\mathbb Z}}
\newcommand{\Q}{{\mathbb Q}}
\newcommand{\R}{{\mathbb R}}
\newcommand{\C}{{\mathbb C}}
\newcommand{\ep}{{\varepsilon}}

\newcommand{\problem}[1]{
	\begin{center}\fbox{
		\begin{minipage}{17.0 cm}
			\setlength{\parindent}{1.5em}
			{\it \noindent#1}
		\end{minipage}}
	\end{center}}

\newtheorem{lemma}{Lemma}
\theoremstyle{remark}
\newtheorem{remark}{Remark}

\renewcommand{\theenumi}{(\alph{enumi})}

\begin{document}
\begin{center}
	{\bf Math 321 Homework 3}
\end{center}

\subsection*{Problem 1}
\problem{
	Let $f \in \mathcal{R}[a,b]$ and $0 < p < \infty$. Define
	\[
		\lVert f \rVert_p = \left( \int_a^b \lvert f \rvert^pdx\right)^{1/p}
	\]
	\begin{enumerate}
		\item Prove that for $0 < p < \infty$, $\lvert f \rvert^p \in \mathcal{R}[a,b]$
			(and hence the above definition makes sense).
		\item If $f$ is continuous, prove that
			\[
				\lim_{p \to \infty} \lVert f \rVert_p =
				\sup\{\lvert f(x) \rvert \colon x \in [a,b]\}.
			\]
		\item For $f$ fixed, define $\phi(p) = \lVert f \rVert_p^p$.
			Using Rudin Problem 6.10, prove that $p \mapsto \log{\phi(p)}$
			is convex on $(0 , \infty)$
			(recall Rudin problem 4.23 for the definition of convexity, and its consequences).
			Do not submit the proof of Rudin Problem 6.10
			(but I encourage you to do it; it is a good exercise).
	\end{enumerate}
	\begin{remark}
		Since convex functions are continuous (see Rudin Problem 4.23),
		you have just shown that $\phi$ and hence $p \mapsto \lVert f \rVert_p$ are continuous.
	\end{remark}
}
\begin{enumerate}
	\item \begin{proof}[Solution]\let\qed\relax
		ff follows from Rudin 6.12
	\end{proof}
	\item \begin{proof}[Solution]\let\qed\relax
		Note that since $f$ is continuous, so is $|f|$ (ff).
		Furthermore, the function is defined on the closed and bounded set $[a,b]$,
		which in $\R$ is compact,
		so $|f|$ attains its maximum value on $[a,b]$ (ff).
		6.21(d) gives us an upper bound.

		Hmm maybe like IVT.
		We have that sup is an upper bound.
		Prove monotonically increasing.
		Then assume some lower value is an upper bound,
		but $|f|$ attains this by IVT
		(or attains halfway between it and the sup),
		and somehow argue that the integral is larger.

		ff
	\end{proof}
	\item \begin{proof}[Solution]\let\qed\relax
		ff
	\end{proof}
\end{enumerate}


\subsection*{Problem 2}
\problem{
	Let $\{f_n\}$ and $\{g_n\}$ be sequences of functions from $\R \to \R$
	that converge pointwise.
	Must it be true that $\{f_n \circ g_n\}$ converges pointwise?
	If so, prove it.
	If not, give a counter-example and prove that your counter-example is correct.
}
\begin{proof}[Solution]\let\qed\relax
	It can be false.
	We provide the counter-example:
	define $f_n(x)$ on $0 < x \leq \frac{1}{n}$ as $f_n(x) = x$,
	and we periodically extend this function off of $(0,\frac{1}{n}]$
	so that $f_n(x+k) = f_n(x)$ for any $k \in \Z$;
	now let
	\[
		g_n(x) = \begin{cases} n, & 0 < x \leq \frac{1}{n} \\ 0, & \text{otherwise}\end{cases}
	\]
	as well.

	Clearly, both of these are functions from $\R \to \R$.
	Furthermore, we get that both of these sequences of functions
	converge pointwise to the $0$ function.
	To see this for $f$, note that $f_n$ attans its maximum value at $x = \frac{1}{n}$,
	since on $x \in (0,\frac{1}{n}]$, $f_n$ is monotone increasing and
	this is the right most value,
	and since $f_n$ is periodic, the largest value this function attains
	is the same as the largest value it attains on this interval.
	Furthermore, $f_n(\frac{1}{n}) = \frac{1}{n}$.
	Let $\ep > 0$ and $x \in \R$ be fixed,
	then Archimedean gives us some $N \in \N$ such that $\frac{1}{N} < \ep$,
	and so for any $n \geq N$, we have
	$\lvert f_n(x) - 0\rvert = f_n(x) \leq \frac{1}{n} \leq \frac{1}{N} < \ep$,
	which means that $f_n$ is pointwise convergent to $0$ for all $x \in \R$.

	To see this for $g$, fix some $\ep > 0$ and some $x \in \R$.
	Archimedean gives us some $N \in \N$ such that $\frac{1}{N} < x$.
	By the definition of $g_n$, when $n \geq N$,
	we have that $|g_n(x) - 0| = g_n(x) = 0 < \ep$
	(since $x > \frac{1}{N} \geq \frac{1}{n}$).
	Hence, $g_n$ is pointwise convergent to $0$ for all $x \in \R$.

	Now let us consider the composition, $\{f_n \circ g_n\}$.
	ff considering the values pointwise seem like a pain.
	Make some argument that $f_n \circ g_n = n$, which
	obviously converges pointwise nowhere.
\end{proof}


\subsection*{Problem 3}
\problem{
	Let $E$ be a set and let $(M_1,d_1),(M_2,d_2)$ be metric spaces with
	the discrete metric (i.e. $d(x,y) = 0$ if $x= y$, and $d(x,y)=1$ if $x \neq y$).
	Let $\{g_n\}$ be a sequence of functions from $E \to M_1$,
	and let $\{f_n\}$ be a sequence of functions from $M_1 \to M_2$.
	Suppose that $\{f_n\}$ and $\{g_n\}$ converge pointwise.
	Must it be true that $\{f_n \circ g_n\}$ converges pointwise?
	If so, prove it.
	If not, give a counter-example and prove that your counter-example is correct.
}
\begin{proof}[Solution]\let\qed\relax
	We claim that this is true,
	specifically if $g \colon E \to M_1$ and $f \colon M_1 \to M_2$
	are functions such that $g_n \to g$ and $f_n \to f$ pointwise,
	$f_n \circ g_n \to f \circ g$ pointwise as well.

	Let $\ep > 0$ and $x \in E$.
	Since $g_n \to g$, for all $\ep' > 0$,
	there exists some $N_1$ such that $d(g_n(x),g(x)) < \ep'$
	for all $n > N_1$.
	If we let $\ep' = \frac{1}{2}$,
	since this is the discrete metric so
	$d(g_n(x),g(x))$ can only either be $1$ or $0$,
	this tells us that for all $n \geq N_1$, $d(g_n(x),g(x)) = 0$,
	i.e. $g_n(x) = g(x)$.
	
	Note $g(x) \in M_1$.
	Since $f_n \to f$, for all $\ep' > 0$,
	there exists some $N_2$ such that $d(f_n(g(x)), f(g(x))) < \ep'$
	for all $n > N_2$.
	If we let $\ep' = \frac{1}{2}$,
	since this is the discrete metric so
	$d(f_n(g(x)), f(g(x))$ can only either be $1$ or $0$,
	this tells us that for all $n \geq N_2$, $d(f_n(g(x)), f(g(x))) = 0$,
	i.e. $f_n(g(x)) = f(g(x))$.

	We now consider $\{f_n \circ g_n\}$.
	Let $N = \max\{N_1,N_2\}$.
	Then for all $n \geq N$, $f_n(g_n(x)) = f_n(g(x)) = f(g(x))$.
	Thus, $d(f_n(g_n(x)), f(g(x))) = d(f(g(x)), f(g(x))) = 0 < \ep$.

	$\ep,x$ were arbitrary,
	hence $\{f_n \circ g_n\}$ converges pointwise.
\end{proof}


\subsection*{Problem 4}
\problem{
	Let $\{f_n\}$ be a sequence of functions in $\mathcal{R}[a,b]$,
	let $f \in \mathcal{R}[a,b]$, let $f_n \to f$ pointwise,
	and suppose that $\{f_n(x)\}$ is monotone increasing for each $x \in [a,b]$.
	Prove that
	\[
		\lim_{n\to\infty} \int_a^b f_n(x)dx = \int_a^b f(x)dx
	\]
}
\begin{proof}[Solution]\let\qed\relax
	We want $\inf_{P'} U(P',f) - \inf_P U(P,f_n)< \ep$.
	We know that for any $x \in [a,b]$, there exists some $N$
	such that for all $n \geq N$, $f(x) - f_n(x) < \ep$;
	also $f_n(x) \leq f_{n+1}(x)$.

	The integrals are a monotone sequence of functions,
	their supremum is $\int_a^b f(x)dx$.

	I think $f_n \to f$ uniformly because monotone.
	
	Once we have a uniform bound, then we get that for any
	$\int f_ndx > \int_a^b f - \ep dx = \int_a^b fdx - \ep(b-a)$.

	Probably prove that $f \geq f_n$ for all $n$.

	We first prove that $f_n$ converge uniformly.
	We do this by verifying the Cauchy criterion for uniform convergence (Rudin Theorem 7.8),
	which we can apply, since our functions are maps into $\R$
	(presumebly, since we have not defined the Riemann integral otherwise),
	which is a complete metric space.
	So let $\ep > 0$ be aribtrary.
	ff

	Note that it must converge to $f$ specifically,
	since ff (theorem doesn't specify what it converges to).

	For the sake of contradiction, assume that $f_n$ does not converge to $f$ uniformly.
	Then there exists some $\ep > 0$
	such that for all $N \in \N$,
	there is some $n \geq N$ and some $x \in [a,b]$
	such that $f(x) - f_n(x) \geq \ep$.
	But then $f_n(x) \not\in$

	Let $\ep > 0$ be arbitrary.
	Let $B$ be the lower bound on $f_1$, i.e. $B \leq f_1 \leq f_n \leq f$
	for all $n$ (by monotonicity).
	Let $\delta_0 = \sup_{x\in[a,b]} \{f(x) - B\}$
	Note $\delta_0 > 0$ for all $x \in [a,b]$,
	since $f(x) - B \geq f_1(x) - B \geq 0$
	(and we can ignore equality in the first inequality,
	because if that was the case, $f(x) = f_1(x) \implies f(x) = f_n(x)$
	and so uniform convergence is trivial... maybe make another paragraph for this ff),
	and must exist in $\$R$,
	since $f$ must be bounded above, say by $M$, and $0 < f(x) - B \leq M - B \in \R$.

	Now, define $\delta_n = \delta_0/2^n$.
	Note that since $f_n \to f$ pointwise,
	there exists some $N \in \N$ such that for all $n \geq N$,
	$f_n > f - \ep$,
	otherwise
	Since $f_n \in \mathcal{R}[a,b]$, $f_n$ is bounded on $[a,b]$.

	Consider $\ep_1 = \sup_{x\in[a,b]}\{f(x) - f_n(x)\}$.
	This satisfies uniformity for $\ep_1$.

	
	The Tighe Cook:
	Let $\ep > 0$, and define $\ep' = \frac{\ep}{2(b-a)}$.
	Consider the set of subintervals of $[a,b]$, denote it $X$.
	Define a set of tags for each subinterval,
	namely $S = \{s \colon s \in I, I \in X\}$.
	Conversely, let $I_s$ be the subinterval associated with $s\in S$,
	and $\delta_s$ is the length of $I_s$.
	For each $k \in \N$, let $L_k = \{s \colon g_k(s) < \ep', s \in S\}$.
	By pointwise convergence, there exists some $k_s$
	such that $s \in L_{k_s}$.
	Note that $\bigcup_{s \in S}I_s$ covers $[a,b]$ (sus with the boundrary),
	and since $[a,b]$ is compact, we can extract a finite subcover,
	so we have $G_1, G_2, \dots, G_N$ that covers $[a,b]$.
	Then there is a finite $k$ such that all the $s \in L_k$
	for all the $s$ centered in the $G_i$.

	Define $g_n = f - f_n$.
	Note that for any $I_s$ where $s \in L_k$,
	where we have
	\[
		\left\lvert \int_{I_s} g_n - g_n(s)\delta_s \right\rvert < \ep' \delta_s
	\]
	(For any $g_n$, $\exists \delta_n$ such that $P$ finr than $\delta_n$,
	so we have the inequality above).

	And then we're done:
	\[
		\int_a^bg_n = \sum_{I_s \in P}\int_{I_s} g_n \leq g_n\sum_{s} (g_n(s) + \ep')\delta)s \leq 2\ep'(b-a) = \ep
	\]
\end{proof}
\end{document}
