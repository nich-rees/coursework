\documentclass{article}
\usepackage{amsmath, amsfonts, amsthm, amssymb}
\usepackage{geometry}
\geometry{letterpaper, margin=2.0cm, includefoot, footskip=30pt}

\usepackage{fancyhdr}
\pagestyle{fancy}

\lhead{Math 321}
\chead{Homework 9}
\rhead{Nicholas Rees, 11848363}
\cfoot{Page \thepage}

\newcommand{\N}{{\mathbb N}}
\newcommand{\Z}{{\mathbb Z}}
\newcommand{\Q}{{\mathbb Q}}
\newcommand{\R}{{\mathbb R}}
\newcommand{\C}{{\mathbb C}}
\newcommand{\ep}{{\varepsilon}}

\newcommand{\problem}[1]{
	\begin{center}\fbox{
		\begin{minipage}{17.0 cm}
			\setlength{\parindent}{1.5em}
			{\it \noindent#1}
		\end{minipage}}
	\end{center}}

\newtheorem{lemma}{Lemma}
\theoremstyle{remark}
\newtheorem{remark}{Remark}

\renewcommand{\theenumi}{(\alph{enumi})}

\begin{document}
\begin{center}
	{\bf Math 321 Homework 9}\\
\end{center}

For the next problems, recall that if $(X,d)$ is a metric space
then $\mathcal{C}_\R(X)$ is the algebra of bounded continuous functions $f \colon X \to \R$.

\subsection*{Problem 1}
\problem{
	Let $K$ be a compact metric space and let $x_0 \in K$.
	Let $\mathcal{A} \subset \mathcal{C}_\R(K)$ be an algebra that
	separates points, and vanishes only at the point $x_0$, i.e.
	$f(x_0) = 0$ for all $f \in \mathcal{A}$, and for each $x \in K$ with $x \neq x_0$,
	there exists $g \in \mathcal{A}$ so that $g(x) \neq 0$.
	\begin{enumerate}
		\item Let $\mathcal{A}'$ be the set of functions of the form $f + c$,
			where $f \in \mathcal{A}$ and $c \in \R$.
			Prove that $\mathcal{A}'$ is an algebra, $\mathcal{A}'$ separates points,
			and $\mathcal{A}'$ vanishes at no point of $K$.
		\item Prove that
			\[
				\overline{\mathcal{A}} = \{f \in \mathcal{C}_\R(K) \colon f(x_0) = 0\}
			\]
	\end{enumerate}
}
\begin{enumerate}
	\item \begin{proof}[Solution]
		We show that $\mathcal{A}'$ is an algebra. Let $f,g \in \mathcal{A}$ and $a,b \in \R$.
		Note $f + a, g + b$ are arbitrary elements of $\mathcal{A}'$.
		\begin{enumerate}
			\item $(f + a) + (g + b) = (f + g) + (a + b)$.
				Note that $f + g \in \mathcal{A}$ and $a + b \in \R$,
				hence the sum is also in $\mathcal{A}'$.
			\item $(f+a)(g+b) = fg + ag + bf + ab$.
				Note that $fg, ag, bf \in \mathcal{A}$, hence $fg + ag + bf \in \mathcal{A}$,
				and $ab \in \R$, hence the product is also in $\mathcal{A}'$.
			\item $c(f+a) = cf + ca$.
				Note that $cf \in \mathcal{A}$ and $ca \in \R$,
				hence the scalar product is also in $\mathcal{A}'$.
				This confirms that $\mathcal{A}'$ is an algebra.
		\end{enumerate}

		To see that $\mathcal{A}'$ separates points, let $x,y \in X$
		such that $x \neq y$.
		Then there exists $f \in \mathcal{A}$ where $f(x) \neq f(y)$.
		Furthermore, $f = f + 0 \in \mathcal{A}'$,
		so $\mathcal{A}'$ separates the points $x,y$ as well,
		and these were arbitrary, so $\mathcal{A}'$ separates all points.

		Finally, we show that $\mathcal{A}'$ vanishes at no point.
		If $x \neq x_0$, then there exists $f \in \mathcal{A}$ such that $f(x) \neq 0$.
		Hence $f = f + 0 \in \mathcal{A}'$, and so $\mathcal{A}'$ does not
		vanish at $x$ as well.
		Now, if $x = x_0$, then for any $f \in \mathcal{A}$,
		we have $f + 1 \in \mathcal{A}'$ and $(f+1)(x_0) = f(x_0) + 1 = 1 \neq 0$,
		hence there is a function that doesn't vanish at $x_0$.
		This covers all possibilities for $x \in K$,
		hence $\mathcal{A}'$ vanishes at no point of $K$.
	\end{proof}
	\item \begin{proof}[Solution]
		ff I will come back to this when I am less hungry,
		but basically we are just copying the proof.
	\end{proof}
\end{enumerate}


\subsection*{Problem 2}
\problem{
	Let $K$ be a compact metric space.
	Let $\mathcal{A} \subset \mathcal{C}_\R(K)$ be an algebra that separates points.
	Prove that the closure $\overline{\mathcal{A}}$ consists of either:
	(i) $\mathcal{C}_\R(K)$, or
	(ii) all continuous functions $f$ on $K$ such that $f(x_0) = 0$
	for some fixed $x_0 \in K$.
}
\begin{proof}[Solution]
	We will show that if (i) is not true, then (ii) must be true,
	assuming that $\mathcal{A} \subset \mathcal{C}_\R(K)$ is an algebra
	that separates points.
	This follows from the contrapositive of Stone-Weierstrass.
	Assume that $\overline{\mathcal{A}} \neq \mathcal{C}_\R(K)$.
	Hence, $\mathcal{A}$ either fails to separate points or it vanishes at a point.
	We assume that it separates points,
	hence it vanishes at some point.
	This means that there is some $x_0 \in K$ where $f(x_0) = 0$ for all $f \in \mathcal{A}$.
	However, since it separates points, we cannot have some other $x_1 \in K$
	where $f(x_1) = 0$ for all $f \in \mathcal{A}$, otherwise it does not separat $x_0$ and $x_1$.
	This satisfies the hypothesis for Problem 1 above,
	and so by 1b, we have $\overline{\mathcal{A}} =
	\{f \in \mathcal{C}_\R(K) \colon f(x_0) = 0\}$, as desired.
\end{proof}


\subsection*{Problem 3}
\problem{
	Let $f \colon [0,1]^2 \to \R$ be continuous, let $\ep > 0$.
	Prove that there exists $n \in \N$ and continuous functions
	$g_1,\dots,g_n,h_1,\dots,h_n \colon [0,1] \to \R$ so that
	\begin{equation}
		\sup_{(x,y)\in[0,1]^2} \left\lvert f(x,y) - \sum_{i=1}^n g_i(x)h_i(y) \right\rvert < \ep
	\end{equation}
}
\begin{proof}[Solution]
	We claim that the set defined by
	\[
		\mathcal{A} = \{f(x,y) \in \mathcal{C}_\R([0,1]^2) \colon
		f(x,y) = \sum_{i=1}^n g_i(x)h_i(y), n \in \N, g_i,h_i \in \mathcal{C}([0,1])\} \subset \mathcal{C}_\R([0,1]^2)
	\]
	is an algebra. See if $g_i,h_i,g'_j,h'_j \in \mathcal{C}_\R([0,1])$
	for $1 \leq i \leq n$ and $1 \leq j \leq m$ for some $n,m \in N$,
	\begin{enumerate}
		\item To see closure under addition, we have
			\[
				\sum_{i=1}^n g_i(x)h_i(y) + \sum_{j=1}^m g'_j(x)h'_j(y)
				= \sum_{k=1}^{n+m}g''_k(x)h''_k(y)
			\]
			where $g''_k = g_k$ when $1 \leq k \leq n$ and $g''_k = g'_{k-n}$
			when $n + 1 \leq k \leq m$, and similarly for $h''_k$.
			Clearly, since $g''_k,h''_k \in \mathcal{C}_\R([0,1])$ as well,
			we have that the sum is in $\mathcal{A}$ as well.
		\item To see closure under multiplication, we have
			\[
				\left(\sum_{i=1}^n g_i(x)h_i(y)\right)\left(\sum_{j=1}^m g'_j(x)h'_j(y)\right)
				= \sum_{\substack{1 \leq i \leq n \\ 1 \leq j \leq m}}
				g_i(x)g'_j(x)h_i(y)h'_j(y)
			\]
			But note that the product of continuous functions is also continuous,
			and so $g_i(x)g'_j(x), h_i(y)h'_j(y) \in \mathcal{C}_\R([0,1])$ still,
			and we get a finite sum of $nm$ these elements.
			Hence, the product is in $\mathcal{A}$.
		\item Finally, we have for $c \in \R$,
			$c\sum_{i=1}^n g_i(x)h_i(y) = \sum_{i=1}^n cg_i(x)h_i(y)$
			and $cg_i(x) \in \mathcal{C}_\R([0,1])$ since a continuous function
			multiplied by a scalar is still continuous,
			hence any sclar multiple is in $\mathcal{A}$.
	\end{enumerate}
	Hence, we have that $\mathcal{A}$ is an algebra.

	To see that $\mathcal{A}$ separates points, consider $(x_1,y_1),(x_2,y_2) \in [0,1]^2$
	such that $(x_1,y_1) \neq (x_2,y_2)$.
	ff


	So Stone-Weierstrass gives us that there an algebra converges uniformly.
	This requirement is converging uniformly:
	$f_n = \sum_{i=1}^n g_i(x)h_i(y)$.
	We have to show that this is an algebra.
	I want to prove that functions of the form $\sum_{i=1}^n g_i(x)h_i(y)$
	where $g,h$ are continuous, form an algebra.
	Then we get a uniformly convergent sequence, eventually works.
	ff
\end{proof}


\subsection*{Problem 4}
\problem{
	Let $\alpha$ and $\beta$ be monotone non-decreasing continuous
	real-valued functions on $[0,1]$, with $\alpha(0) = \beta(0) = 0$.
	Suppose that, for all $n = 0,1,2,3,\dots$,
	\[
		\int_0^1 e^{-nx}d\alpha(x) = \int_0^1 e^{-nx}d\beta(x)
	\]
	\begin{enumerate}
		\item Prove that if $f \colon [0,1] \to \R$ is continuous
			then $\int_0^1 f(x)d\alpha(x) = \int_0^1 f(x)d\beta(x)$
		\item Does it follow that $\alpha(x) = \beta(x)$ for all $x \in [0,1]$?
			Prove it or give a counterexample.
	\end{enumerate}
}
\begin{enumerate}
	\item \begin{proof}[Solution]
		ff
		Something about how the finite sum is an algebra,
		and then theorem 6.12 and 7.16 for the win.
	\end{proof}
	\item \begin{proof}[Solution]
		Something that equals on the continuous, but not on the not continuous
		(implication doesn't go this way, but it is a necessary condition).
		Hmm... we require that $\alpha,\beta$ continuous.
		ff
	\end{proof}
\end{enumerate}
\end{document}
